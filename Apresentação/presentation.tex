%%%%%%%%%%%%%%%%%%%%%%%%%%%%%%%%%%%%%%%%%%%%%%%%%%%%%%%%%%%%%%%%%%%%%
% writeLaTeX Example: Molecular Chemistry Presentation
%
% Source: http://www.writelatex.com
%
% In these slides we show how writeLaTeX can be used with standard 
% chemistry packages to easily create professional presentations.
% 
% Feel free to distribute this example, but please keep the referral
% to writelatex.com
% 
%%%%%%%%%%%%%%%%%%%%%%%%%%%%%%%%%%%%%%%%%%%%%%%%%%%%%%%%%%%%%%%%%%%%%%
% How to use writeLaTeX: 
%
% You edit the source code here on the left, and the preview on the
% right shows you the result within a few seconds.
%
% Bookmark this page and share the URL with your co-authors. They can
% edit at the same time!
%
% You can upload figures, bibliographies, custom classes and
% styles using the files menu.
%
% If you're new to LaTeX, the wikibook is a great place to start:
% http://en.wikibooks.org/wiki/LaTeX
%
%%%%%%%%%%%%%%%%%%%%%%%%%%%%%%%%%%%%%%%%%%%%%%%%%%%%%%%%%%%%%%%%%%%%%%

\documentclass{beamer}

\nonstopmode

% For more themes, color themes and font themes, see:
% http://deic.uab.es/~iblanes/beamer_gallery/index_by_theme.html
%
\mode<presentation>
{
\usetheme{Madrid}       % or try default, Darmstadt, Warsaw, ...
\usecolortheme{default} % or try albatross, beaver, crane, ...
\usefonttheme{serif}    % or try default, structurebold, ...
\setbeamertemplate{navigation symbols}{}
\setbeamertemplate{caption}[numbered]
} 

\usepackage{etex}

\usepackage[brazil]{babel}  % pt_BR hifenization
\usepackage[utf8x]{inputenc}
\usepackage{chemfig}
\usepackage{chronology}
\usepackage{amssymb}
\usepackage{empheq}
\usepackage{mathpartir}
\usepackage[all]{xy}
\usepackage[color]{coqdoc}
\usepackage{pdfpcnotes}


\usepackage{amsmath}

\usepackage{color}
\usepackage{xcolor}
\usepackage{listings}


% On writeLaTeX, these lines give you sharper preview images.
% You might want to `comment them out before you export, though.
\usepackage{pgfpages}
\pgfpagesuselayout{resize to}[%
physical paper width=8in, physical paper height=6in]

% Here's where the presentation starts, with the info for the title slide
\title[Normalização no sistema $\lambda$ex]{Em direção à formalização das propriedades de normalização do sistema
    $\lambda ex$}
\author{
Lucas de M. Amaral
}
\institute[UnB]{Universidade de Brasilia}
\date{09/12/2016}

\begin{document}

\lstset{escapeinside={<@}{@>}}


\begin{frame}
\titlepage
\end{frame}

% These three lines create an automatically generated table of contents.
\begin{frame}{Visão geral}
\tableofcontents
\end{frame}

\section{Introdução}

\subsection{O assistente de provas Coq}

\frame{\tableofcontents[currentsection,currentsubsection]}

%-----------

\begin{frame}
\frametitle{Assistentes de prova}
\begin{itemize}
    \item O que são? \pause
        \pnote{1. Especificar teorias e provar propriedades no computador.
            Vantagem: verificar correção }
    \item Por que usá-los?  \pause
        \pnote{2. Falha humana, Teorema das
           quatro cores e do caso do Vladimir }
    \item Provas formais
        \pnote{3. Provas formais: menos erros;
                  Dificuldades;
                 Assistentes de prova auxiliam neste processo }
\end{itemize}
\end{frame}

%-----------

\begin{frame}
\frametitle{O Assistente de provas Coq}
\begin{itemize}
    \item Forte base teórica
    \item Boa quantidade de bibliotecas
\end{itemize}

\pause

\begin{block}{ Critério de de Bruijn }
        O critério de de Bruijn afirma que a correção de um sistema deve ser
        garantida por um verificador \emph{pequeno}. Em outras palavras, deve
        haver um \emph{kernel} pelo qual todas as fórmulas passam. Assim, se
        torna mais fácil estabelecer a confiabilidade do sistema, pois podemos
        verificar o kernel separadamente.
    
\end{block} 
\end{frame}

%-----------

\begin{frame}
\frametitle{O Assistente de provas Coq}
   \begin{block}{ Exemplo de uso }
       
    \coqdockw{Theorem} \coqdocvar{mult\_5\_is\_correct}:
    \coqdockw{\ensuremath{\forall}} \coqdocvar{n} : \coqdocvar{nat},
    (\coqdocvar{mult\_5} \coqdocvar{n}) = 5*\coqdocvar{n}.\coqdoceol \coqdocnoindent
    \coqdockw{Proof}.\coqdoceol \coqdocindent{2.00em} \coqdoctac{intros}.\coqdoceol
    \coqdocindent{2.00em} \coqdoctac{induction} \coqdocvar{n}.\coqdoceol
    \coqdocindent{2.00em} \coqdoctac{simpl}. \coqdoctac{reflexivity}.\coqdoceol
    \coqdocindent{2.00em} \coqdoctac{simpl} \coqdocvar{mult\_5}.\coqdoceol
    \coqdocindent{2.00em} \coqdoctac{rewrite} \coqdocvar{mult\_comm}.\coqdoceol
    \coqdocindent{2.00em} \coqdoctac{simpl}.\coqdoceol \coqdocindent{2.00em}
    \coqdoctac{rewrite} \coqdocvar{IHn}.\coqdoceol \coqdocindent{2.00em}
    \coqdoctac{rewrite} \coqdocvar{mult\_comm}.\coqdoceol \coqdocindent{2.00em}
    \coqdoctac{reflexivity}.\coqdoceol \coqdocnoindent \coqdockw{Qed}.\coqdoceol

   \end{block} 
\end{frame}


%===========

\subsection{O cálculo $\lambda$}

\frame{\tableofcontents[currentsection, currentsubsection]}


\begin{frame}
\frametitle{O Cálculo $\lambda$}
\begin{block}{Modelos teóricos de computação}
Estudar de uma maneira formal os limites da computação, entender quais funções são de fato computáveis, qual o poder computacional de certos tipos de operações, etc.
\end{block}
\begin{itemize}
\item Máquinas de Turing
\item \textbf{Cálculo $\lambda$}
\item Lógica Combinatória
\end{itemize}
\end{frame}

%-----------

\begin{frame}
\frametitle{Cálculo $\lambda$}

\begin{itemize}
\item Gramática mínima 
\end{itemize}

\begin{block}{Gramática}
\[ \tau := x\ |\ \lambda y.\tau\ |\ \tau \tau \]
\end{block}

\begin{block}{Exemplo}
    \[ (double(x) = 2*x) \leadsto (\lambda x.\ 2*x) \]
    \[ (\lambda x.\ 2*x)\ 3 \]
    \[ (\lambda n. \lambda f. (\lambda f. \lambda x.f (f\ x)) (n\ f))\ 
        (\lambda f. \lambda x. f (f (f\ x))) \]
\end{block}

\end{frame}

%-----------

\begin{frame}
\frametitle{Cálculo $\lambda$}

\begin{itemize}
\item Poucas regras de inferência
\end{itemize}

\begin{block}{Regras de inferência}
\[ (\lambda x.t)\ u \rightarrow_{\beta} t\{x/u\} \]
\end{block}

\pause

\begin{block}{Exemplo}
\[ (\lambda y .(\lambda x.x\ y))\ u\ v \rightarrow_{\beta} (\lambda x.x\ y)\{y/u\}\ v = (\lambda x.x\ u)\ v \]
\end{block}

\end{frame}

%-----------

\begin{frame}
\frametitle{Cálculo $\lambda$}

\begin{block}{ $\alpha$-equivalência }
    \[ f(x) = a*x;\ \ f(t) = a*t \]

    Dizemos que dois termos são $\alpha$-equivalentes se um pode ser obtido a
    partir do outro, através de renomeamento de variáveis ligadas.

    \[ \lambda x. x =_{\alpha} \lambda y. y \] 
    \[ \lambda x. z \neq_{\alpha} \lambda x. k \] 
    
\end{block}

\pause

\begin{block}{Justificativa}
    \[ ((\lambda x. y)\ z)\{y/x\} = ((\lambda x. x)\ z) \rightarrow_{\beta} z \] 
    \[ ((\lambda x. y)\ z)\{y/x\} =_{\alpha} ((\lambda k. y)\ z)\{y/x\} = ((\lambda k. x)\ z) \rightarrow_{\beta} x \] 
\end{block}
    
\end{frame}

%-----------

\begin{frame}
\frametitle{Cálculo $\lambda$}

\begin{block}{ Confluência }
    Um sistema de reescrita, tomando como exemplo o cálculo $\lambda$, é dito
    fracamente confluente se, para todo termo $t$, com $ t \rightarrow_\beta t' $
    e $ t \rightarrow_\beta t'' $, então deve existir um termo $u$ tal que $t'
    \rightarrow_\beta^* u$ e $t'' \rightarrow_\beta^* u$, onde
    $\rightarrow_\beta^*$ representa o fecho transitivo-reflexivo da
    $\beta$-redução.  

    O sistema é dito fortemente confluente se, para todo termo
    $t$ tal que $ t \rightarrow_\beta^* t' $ e $ t \rightarrow_\beta^* t'' $,
    existe um termo $u$ tal que $t' \rightarrow_\beta^* u$ e $t''
    \rightarrow_\beta^* u$. 

\begin{displaymath}
        \xymatrix{                 & t \ar[dl]_{\beta^*} \ar[dr]^{\beta^*} &    \\
                  t' \ar[dr]_{\beta^*} &                & t'' \ar[dl]^{\beta^*} \\
                                   &       u        &                   }
\end{displaymath}
\end{block}

\end{frame}

%-----------

\begin{frame}
\frametitle{Cálculo $\lambda$}

\begin{block}{ Normalização }
    Um termo $t$ é dito estar em forma normal quando não existe $t'$ tal que $ t
    \rightarrow t' $.  Um termo é dito \textbf{fracamente normalizável} quando
    existe uma estratégia de redução que leva a uma forma normal. O termo é
    \textbf{fortemente normalizável} se toda estratégia leva à forma normal.

    \pause
    \[ \omega := (\lambda x.\ x\ x) \]
    \[ \Omega := \omega\ \omega \]
    \[ \Omega = (\lambda x.\ x\ x)\ (\lambda x.\ x\ x) \rightarrow_{\beta} 
        (x\ x)\{x/\omega\} = \omega\ \omega = \Omega \]
    \pause
    \[ (\lambda x.\ z)\ \Omega \rightarrow_{\beta} (\lambda x.\ z)\ \Omega \]
    \pause
    \[ (\lambda x.\ z)\ \Omega \rightarrow_{\beta} z \]

\end{block}

\end{frame}

%===========

\subsection{O sistema $\lambda ex$}

\frame{\tableofcontents[currentsection, currentsubsection]}

\begin{frame}
\frametitle{Cálculos com substituições explícitas}
\begin{itemize}
    \item Formalizar a operação de substituição
        \begin{itemize}
            \item Termos da forma $t[x/u]$
        \end{itemize}
    \item Maior controle sobre a ordem de realização
\end{itemize}
\end{frame}

%-----------

\begin{frame}
\frametitle{O sistema $\lambda ex$}
\begin{block}{Sintaxe}
\[ \tau := x\ |\ \lambda x.\tau\ |\ \tau \tau\ |\ \tau[x/\tau]\ \]
\end{block}
\begin{table}[h]
\begin{empheq}[box=\fbox]{align*}
    (\lambda x.\ t)\ u\ \ \ &\rightarrow\ t[x/u] \\
    x[x/u]\ \ \             &\rightarrow\ u \\
    y[x/u]\ \ \             &\rightarrow\ y,\ se\ x\ \neq\ y \\
    (t\ u)[x/v]\ \ \        &\rightarrow\ t[x/v]\ u[x/v] \\
    t[x/u][y/v]\ \ \        &\rightarrow\ t[y/v][x/u[y/v]],\ se\ y\ \in fv(u) \\ 
    (\lambda y.\ u)[x/v]\ \ &\rightarrow\ (\lambda y.\ u[x/v])
\end{empheq}
\end{table}
\end{frame}

%-----------

\begin{frame}
\frametitle{O sistema $\lambda ex$}

Sejam $t, u, v$ termos e $x, y$ variáveis tais que $y \notin fv(u)$ e $x \notin
fv(v)$. 

\begin{table}[h]
\begin{displaymath}
        \xymatrix{ & ((\lambda x.\ t)\ u)[y/v] \ar[dl]_{B} \ar@{->>}[dr]_x &     \\
                  t[x/u][y/v] &               & (\lambda x.\ t[y/v])\ u \ar[d]_B \\
                              &               &  t[y/v][x/u]                    }
\end{displaymath}
\end{table}

\pause

\begin{block}{Classes de equivalência}
\[ t[x/u][y/v] =_C t[y/v][x/u] \ \ \ \ \ se\ y \notin fv(u)\ \&\ x \notin fv(v)\] 
\end{block}
    
\end{frame}


%-----------

\begin{frame}
    \frametitle{O sistema $\lambda ex$}
    \begin{block}{PSN}
        Seja $\lambda$z uma extensão do cálculo $\lambda$. Dizemos que $\lambda z$
        preserva a normalização forte se, para todo $\lambda$-termo fortemente
        normalizável, seu correspondente em $\lambda$z também é fortemente
        normalizável nesta extensão.
    \end{block}

\end{frame}

%-----------

\begin{frame}

    \begin{block}{ Estratégia de redução perpétua }
        Uma estratégia perpétua fornece uma cadeia de reduções \emph{infinita} para
        um termo, se uma existe. Caso não exista, fornece uma cadeia que termina em
        um termo em forma normal.
    \end{block}

    \pause
    
    \begin{block}{ Propriedade IE }
        Sejam $t,\ u$ termos. Seja $SN_{\lambda ex}$ o conjunto de termos fortemente
        normalizáveis do sistema $\lambda ex$. Se $u \in SN_{\lambda ex}$ e
        $t\{x/u\} \in SN_{\lambda ex}$, então $t[x/u] \in SN_{\lambda ex}$.
    \end{block}

\end{frame}

%===========

\section{Objetivos e motivação}
\frame{\tableofcontents[currentsection]}

\begin{frame}{Objetivos}
\begin{itemize}
\item Demonstrar a preservação das propriedades básicas do novo sistema pela
    relação de equivalência
\item Definir a noção de termos marcados e análogos para as propriedades
    anteriores
\item Iniciar a formalização da propriedade IE, dividindo a relação principal do
    sistema em duas e demonstrando a equivalencia destas
\end{itemize}
\end{frame}

\begin{frame}{Motivação}
\begin{itemize}
\item Um dos primeiros cálculos com substituições explícitas que preserva a
    maioria das propriedades esperadas do cálculo $\lambda$
\item Importância para o estudo de linguagens de programação
\item Assistentes de prova
\item Auxilia em provas de propriedade do próprio cálculo $\lambda$
\end{itemize}
\end{frame}

%\begin{frame}{Metodologia}
%\begin{itemize}
%\item Utilizar o assistente de provas \textit{Coq}
%\item Provar as propriedades básicas de um sistema derivado do Cálculo Lambda
%\item Tentar atacar os teoremas principais e observar os resultados auxiliares necessários
%\end{itemize}
%\end{frame}

%===========

\section{Formalização}
\subsection{Estruturas básicas}
\frame{\tableofcontents[currentsection, currentsubsection]}

%-----------

\begin{frame}{Gramática}


\begin{block}{Gramática}
\[ \tau := x\ |\ n\ |\ \lambda \tau\ |\ \tau \tau\ |\ \tau[\tau]\ \]
\end{block}

\begin{block}{Pré-termos}

\bigskip \coqdocnoindent \coqdockw{Inductive} \coqdef{LambdaES
    Defs.pterm}{pterm}{\coqdocinductive{pterm}} : \coqdockw{Set} :=\coqdoceol
\coqdocindent{1.00em} \ensuremath{|} \coqdef{LambdaES Defs.pterm
    bvar}{pterm\_bvar}{\coqdocconstructor{pterm\_bvar}} :
\coqexternalref{nat}{http://coq.inria.fr/distrib/8.4pl4/stdlib/Coq.Init.Datatypes}{\coqdocinductive{nat}}
\ensuremath{\rightarrow} \coqref{LambdaES
    Defs.pterm}{\coqdocinductive{pterm}}\coqdoceol \coqdocindent{1.00em}
\ensuremath{|} \coqdef{LambdaES Defs.pterm
    fvar}{pterm\_fvar}{\coqdocconstructor{pterm\_fvar}} : \coqdocdefinition{var}
\ensuremath{\rightarrow} \coqref{LambdaES
    Defs.pterm}{\coqdocinductive{pterm}}\coqdoceol \coqdocindent{1.00em}
\ensuremath{|} \coqdef{LambdaES Defs.pterm
    app}{pterm\_app}{\coqdocconstructor{pterm\_app}}  : \coqref{LambdaES
    Defs.pterm}{\coqdocinductive{pterm}} \ensuremath{\rightarrow}
\coqref{LambdaES Defs.pterm}{\coqdocinductive{pterm}} \ensuremath{\rightarrow}
\coqref{LambdaES Defs.pterm}{\coqdocinductive{pterm}}\coqdoceol
\coqdocindent{1.00em} \ensuremath{|} \coqdef{LambdaES Defs.pterm
    abs}{pterm\_abs}{\coqdocconstructor{pterm\_abs}}  : \coqref{LambdaES
    Defs.pterm}{\coqdocinductive{pterm}} \ensuremath{\rightarrow}
\coqref{LambdaES Defs.pterm}{\coqdocinductive{pterm}}\coqdoceol
\coqdocindent{1.00em} \ensuremath{|} \coqdef{LambdaES Defs.pterm
    sub}{pterm\_sub}{\coqdocconstructor{pterm\_sub}} : \coqref{LambdaES
    Defs.pterm}{\coqdocinductive{pterm}} \ensuremath{\rightarrow}
\coqref{LambdaES Defs.pterm}{\coqdocinductive{pterm}} \ensuremath{\rightarrow}
\coqref{LambdaES Defs.pterm}{\coqdocinductive{pterm}} \coqdoceol
%\coqdocindent{1.00em} \ensuremath{|} \coqdef{LambdaES Defs.pterm
%lsub}{pterm\_lsub}{\coqdocconstructor{pterm\_lsub}} : \coqref{LambdaES
%Defs.pterm}{\coqdocinductive{pterm}} \ensuremath{\rightarrow} \coqref{LambdaES
%Defs.pterm}{\coqdocinductive{pterm}} \ensuremath{\rightarrow} \coqref{LambdaES
%Defs.pterm}{\coqdocinductive{pterm}}.\coqdoceol 
\bigskip

\end{block}
    
\end{frame}

%-----------

\begin{frame}{Termos}

\begin{block}{Exemplos}
    \[ (\lambda \lambda\ 1\ 0)\ y\ x \] 
    \pause
    \[ (\lambda \lambda\ 2\ 0)\ y\ x \] 
    
\end{block}

\pause

\begin{block}{Termo bem formado}
    Dizemos que um pré-termo é um termo (bem formado) se toda ocorrência de
    um índice pertence a um número correspondente de abstrações. Em outras
    palavras, um termo não possui índices soltos.
\end{block}

\end{frame}

%-----------

\begin{frame}{Termos bem formados}

\begin{block}{Função recursiva}
\coqdockw{Fixpoint} \coqdocvar{lc\_at} (\coqdocvar{k}:\coqdocvar{nat})
(\coqdocvar{t}:\coqdocvar{pterm}) \{\coqdockw{struct} \coqdocvar{t}\} :
\coqdockw{Prop} := ...\coqdoceol

\coqdockw{Definition} \coqdocvar{term'} \coqdocvar{t} := \coqdocvar{lc\_at} 0 \coqdocvar{t}.
\end{block}

\begin{block}{Predicado}
\coqdocnoindent
\coqdockw{Inductive} \coqdef{LambdaES Defs.term}{term}{\coqdocinductive{term}} :
\coqref{LambdaES Defs.pterm}{\coqdocinductive{pterm}} \ensuremath{\rightarrow}
\coqdockw{Prop} := ...\coqdoceol
\end{block}

\bigskip
\coqdockw{Lemma} \coqdocvar{term\_eq\_term'} : \coqdockw{\ensuremath{\forall}}
\coqdocvar{t}, \coqdocvar{term} \coqdocvar{t} \ensuremath{\leftrightarrow}
\coqdocvar{term'} \coqdocvar{t}.\coqdoceol
\bigskip
    
\end{frame}

%-----------

\begin{frame}{Equação}

    \begin{block}{Definição de eqc}
        
    \coqdocnoindent \coqdockw{Inductive} \coqdef{Equation
        C.eqc}{eqc}{\coqdocinductive{eqc}} : \coqdocinductive{pterm}
    \ensuremath{\rightarrow} \coqdocinductive{pterm} \ensuremath{\rightarrow}
    \coqdockw{Prop} := \coqdoceol \coqdocindent{1.00em} \ensuremath{|}
    \coqdef{Equation C.eqc def}{eqc\_def}{\coqdocconstructor{eqc\_def}}:
    \coqdockw{\ensuremath{\forall}} \coqdocvar{t} \coqdocvar{u} \coqdocvar{v},
    \coqdocdefinition{lc\_at} 2 \coqdocvariable{t} \ensuremath{\rightarrow}
    \coqdocinductive{term} \coqdocvariable{u} \ensuremath{\rightarrow}
    \coqdocinductive{term} \coqdocvariable{v} \ensuremath{\rightarrow}
    \coqref{Equation C.eqc}{\coqdocinductive{eqc}}
    (\coqdocvariable{t}\coqdocnotation{[}\coqdocvariable{u}\coqdocnotation{][}\coqdocvariable{v}\coqdocnotation{]})
    (\coqdocnotation{(}\coqdocnotation{\&}
    \coqdocvariable{t}\coqdocnotation{)[}\coqdocvariable{v}\coqdocnotation{][}\coqdocvariable{u}\coqdocnotation{]}).\coqdoceol


    \bigskip \coqdockw{Definition} \coqdef{Equation C.eqc
        ctx}{eqc\_ctx}{\coqdocdefinition{eqc\_ctx}} (\coqdocvar{t}
    \coqdocvar{u}: \coqdocinductive{pterm}) :=
    \coqdocinductive{ES\_contextual\_closure} \coqref{Equation
        C.eqc}{\coqdocinductive{eqc}} \coqdocvariable{t}
    \coqdocvariable{u}.

    \coqdockw{Definition} \coqdef{Equation C.eqC}{eqC}{\coqdocdefinition{eqC}}
    (\coqdocvar{t} : \coqdocinductive{pterm}) (\coqdocvar{u} :
    \coqdocinductive{pterm}) := \coqdocinductive{star\_closure} \coqref{Equation
        C.eqc ctx}{\coqdocdefinition{eqc\_ctx}} \coqdocvariable{t}
    \coqdocvariable{u}.\coqdoceol \coqdockw{Notation} "t =e u" := (\coqref{Equation
    C.eqC}{\coqdocdefinition{eqC}} \coqdocvar{t} \coqdocvar{u}) (\coqdoctac{at}
\coqdockw{level} 66).\coqdoceol
    \end{block}

    \pause

\begin{itemize}
    \item Construtores de termos, abertura de termos, substituições e
        renomeamento de variáveis devem funcionar de maneira análoga para dois
        termos equivalentes.
    \item Boa formação de termos e corpos de abstrações, conjunto de variáveis
        livres e reduções dentro de fechos contextuais também devem ser
        preservadas pela equivalência.
\end{itemize}
    
\end{frame}

%-----------

\subsection{Equivalência das relações}
\frame{\tableofcontents[currentsection, currentsubsection]}

\begin{frame}{Cálculo com termos marcados}
    \begin{block}{Gramática}
    \[ \tau := x\ |\ \lambda x.\tau\ |\ \tau \tau\ |\ \tau[x/\tau]\ |\
        \tau[\![x/u]\!]\ \]
    \end{block}

\begin{block}{Termos marcados}
    \coqdocnoindent \coqdockw{Inductive} \coqdef{LambdaES
        Defs.pterm}{pterm}{\coqdocinductive{pterm}} : \coqdockw{Set} :=\coqdoceol
    \coqdocindent{1.00em} \ensuremath{|} ...\coqdoceol 
    \coqdocindent{1.00em} \ensuremath{|} \coqdef{LambdaES Defs.pterm
        lsub}{pterm\_lsub}{\coqdocconstructor{pterm\_lsub}} : \coqref{LambdaES
        Defs.pterm}{\coqdocinductive{pterm}} \ensuremath{\rightarrow}
    \coqref{LambdaES Defs.pterm}{\coqdocinductive{pterm}} \ensuremath{\rightarrow}
    \coqref{LambdaES Defs.pterm}{\coqdocinductive{pterm}}.\coqdoceol 
\end{block}

\end{frame}

%-----------

\begin{frame}{Cálculo com termos marcados}
    \begin{block}{Equação para termos marcados}
        
    \coqdocnoindent \coqdockw{Inductive} \coqdocvar{lab\_eqc}  :
    \coqdocvar{pterm} \ensuremath{\rightarrow} \coqdocvar{pterm}
    \ensuremath{\rightarrow} \coqdockw{Prop} := \coqdoceol \coqdocnoindent
    \coqdocindent{1.00em}\ensuremath{|} \coqdocvar{lab\_eqc\_rx1} :
    \coqdockw{\ensuremath{\forall}} \coqdocvar{t} \coqdocvar{u} \coqdocvar{v},
    \coqdoceol \coqdocindent{2.00em} \coqdocvar{lab\_term} \coqdocvar{u}
    \ensuremath{\rightarrow} \coqdocvar{term} \coqdocvar{v}
    \ensuremath{\rightarrow} \coqdocvar{lab\_eqc}
    (\coqdocvar{t}[\coqdocvar{u}][[\coqdocvar{v}]]) ((\&
    \coqdocvar{t})[[\coqdocvar{v}]][\coqdocvar{u}]) \coqdoceol \coqdocnoindent
    \coqdocindent{1.00em}\ensuremath{|} \coqdocvar{lab\_eqc\_rx2} :
    \coqdockw{\ensuremath{\forall}} \coqdocvar{t} \coqdocvar{u} \coqdocvar{v},
    \coqdoceol \coqdocindent{2.00em} \coqdocvar{term} \coqdocvar{u}
    \ensuremath{\rightarrow} \coqdocvar{lab\_term} \coqdocvar{v}
    \ensuremath{\rightarrow} \coqdocvar{lab\_eqc}
    (\coqdocvar{t}[[\coqdocvar{u}]][\coqdocvar{v}]) ((\&
    \coqdocvar{t})[\coqdocvar{v}][[\coqdocvar{u}]]) \coqdoceol \coqdocnoindent
    \coqdocindent{1.00em}\ensuremath{|} \coqdocvar{lab\_eqc\_rx3} :
    \coqdockw{\ensuremath{\forall}} \coqdocvar{t} \coqdocvar{u} \coqdocvar{v},
    \coqdoceol \coqdocindent{2.00em} \coqdocvar{term} \coqdocvar{u}
    \ensuremath{\rightarrow} \coqdocvar{term} \coqdocvar{v}
    \ensuremath{\rightarrow} \coqdocvar{lab\_eqc}
    (\coqdocvar{t}[[\coqdocvar{u}]][[\coqdocvar{v}]]) ((\&
    \coqdocvar{t})[[\coqdocvar{v}]][[\coqdocvar{u}]]).\coqdoceol

    \bigskip
    \coqdockw{Definition} \coqdocvar{lab\_eqC} (\coqdocvar{t}:
    \coqdocvar{pterm}) (\coqdocvar{u} : \coqdocvar{pterm}) :=
    \coqdocvar{trans\_closure} (\coqdocvar{lab\_contextual\_closure}
    \coqdocvar{lab\_eqc}) \coqdocvar{t} \coqdocvar{u} .\coqdoceol
    \coqdocnoindent \coqdockw{Notation} "t =\~{}e u" := (\coqdocvar{lab\_eqC}
    \coqdocvar{t} \coqdocvar{u}) (\coqdoctac{at} \coqdockw{level} 66).\coqdoceol
    \end{block}
\end{frame}

%-----------

\begin{frame}{Equivalência de reduções}


\begin{table}[h]
\begin{empheq}[box=\fbox]{align*}
    x[\![x/u]\!]\ \ \             &\rightarrow_{Var}\ u \\
    t[\![x/u]\!]\ \ \             &\rightarrow_{Gc}\ t    & se\ \text{\emph{x}} \notin fv(t)\\
    (t\ u)[\![x/v]\!]\ \ \        &\rightarrow_{App}\ t[\![x/v]\!]\ u[\![x/v]\!] \\
    (\lambda y.\ u)[\![x/v]\!]\ \ &\rightarrow_{Lamb}\ (\lambda y.\ u[\![x/v]\!])\\
    t[\![x/u]\!][\![y/v]\!]\ \ \        &\rightarrow_{Comp}\ t[\![y/v]\!][\![x/u[\![y/v]\!]]\!] & se\ y\ \in fv(u)  
\end{empheq}
    \caption{A redução $\rightarrow$$_{\underline{x}}$ }
\end{table}

\pause

\[ t \rightarrow_{\lambda \underline{ex}} t' \iff \exists s,\ s';\ t =_{e \cup
        \underline{e} \cup \alpha} s \rightarrow_{Bx \cup \underline{x}} s' =_{e \cup
        \underline{e} \cup \alpha} t' \] 

\pause

\[t \rightarrow_{\lambda \underline{ex}} t' \iff t \rightarrow_{\lambda
        \underline{ex}^i \cup \lambda \underline{ex}^e} t'\] 

\end{frame}

%-----------

\begin{frame}{Equivalência de relações}

    \begin{block}{Relação interna}
    \begin{itemize}
        \item Se $u \rightarrow_{Bx} u'$ e $u,\ u'$ são termos, então $t[\![x/u]\!]
            \rightarrow_{\lambda \underline{x}^i} t[\![x/u']\!]$ 
        \item Se $t
            \rightarrow_{\underline{x}} t'$, então $t \rightarrow_{\lambda
                \underline{x}^i} t'$
        \item Se $t \rightarrow_{\lambda \underline{x}^i} t'$, então vale 
            $t\ u \rightarrow_{\lambda \underline{x}^i} t'\ u$,  
            $u\ t \rightarrow_{\lambda \underline{x}^i} u\ t'$, 
            $\lambda x. t \rightarrow_{\lambda \underline{x}^i} \lambda x. t'$, 
            $t[x/u] \rightarrow_{\lambda \underline{x}^i} t'[x/u]$, 
            $u[x/t] \rightarrow_{\lambda \underline{x}^i} u[x/t']$ e 
            $t[\![x/u]\!] \rightarrow_{\lambda \underline{x}^i} t'[\![x/u]\!]$.
    \end{itemize}
    \end{block}

    \begin{block}{Relação externa}
    \begin{itemize}
        \item Se $t \rightarrow_{Bx} t'$ ocorre fora de uma substituição marcada, então 
            $t \rightarrow_{\lambda \underline{x}^e} t$ 
        \item Se $t \rightarrow_{\lambda \underline{x}^e} t'$, então vale 
            $t\ u \rightarrow_{\lambda \underline{x}^e} t'\ u$,  
            $u\ t \rightarrow_{\lambda \underline{x}^e} u\ t'$, 
            $\lambda x. t \rightarrow_{\lambda \underline{x}^e} \lambda x. t'$, 
            $t[x/u] \rightarrow_{\lambda \underline{x}^e} t'[x/u]$, 
            $u[x/t] \rightarrow_{\lambda \underline{x}^e} u[x/t']$ e 
            $t[\![x/u]\!] \rightarrow_{\lambda \underline{x}^e} t'[\![x/u]\!]$.
    \end{itemize}
    \end{block}
    
\end{frame}

%-----------

\begin{frame}{Equivalência de relações}
    \begin{block}{Formalização}
    \coqdocnoindent \coqdockw{Notation} ``t -{}-$>$[lx\_i]\ u" :=
    (\coqdocvar{ext\_lab\_EE\_ctx\_red}\ \coqdocvar{lab\_x\_i}\ \coqdocvar{t}
    \coqdocvar{u}) .

    \bigskip

    \coqdocnoindent \coqdockw{Notation} ``t -{}-$>$[lx\_e]\ u" :=
    (\coqdocvar{ext\_lab\_EE\_ctx\_red}\ \coqdocvar{sys\_Bx}\ \coqdocvar{t}
    \coqdocvar{u}) .  \coqdoceol

    \bigskip

    \coqdockw{Theorem} \coqdocvar{lab\_ex\_eq\_i\_e}:
    \coqdockw{\ensuremath{\forall}} \coqdocvar{t} \coqdocvar{t'},
    \coqdocvar{lab\_term} \coqdocvar{t} \ensuremath{\rightarrow} (\coqdocvar{t}
    -{}-$>$[\coqdocvar{lex}] \coqdocvar{t'} \ensuremath{\leftrightarrow} (\coqdocvar{t}
    -{}-$>$[\coqdocvar{lx\_i}] \coqdocvar{t'} \ensuremath{\lor} \coqdocvar{t}
    -{}-$>$[\coqdocvar{lx\_e}] \coqdocvar{t'})).\coqdoceol

    \end{block}
\end{frame}

%===========


\begin{frame}{Referências}
\begin{enumerate}
\item A theory of explicit substitutions with safe and full composition - Delia Kesner. Logical methods in computer science, 2009.
\item Formalizing a Named Explicit Substitutions Calculus in Coq - Washington L. R. de C. Segundo, Flávio L. C. de Moura, Daniel L. Ventura. Conferences on Intelligent Computer Mathematics, 2014.
\end{enumerate}
\end{frame}

\end{document}

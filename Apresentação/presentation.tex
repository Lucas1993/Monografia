%%%%%%%%%%%%%%%%%%%%%%%%%%%%%%%%%%%%%%%%%%%%%%%%%%%%%%%%%%%%%%%%%%%%%
% writeLaTeX Example: Molecular Chemistry Presentation
%
% Source: http://www.writelatex.com
%
% In these slides we show how writeLaTeX can be used with standard 
% chemistry packages to easily create professional presentations.
% 
% Feel free to distribute this example, but please keep the referral
% to writelatex.com
% 
%%%%%%%%%%%%%%%%%%%%%%%%%%%%%%%%%%%%%%%%%%%%%%%%%%%%%%%%%%%%%%%%%%%%%%
% How to use writeLaTeX: 
%
% You edit the source code here on the left, and the preview on the
% right shows you the result within a few seconds.
%
% Bookmark this page and share the URL with your co-authors. They can
% edit at the same time!
%
% You can upload figures, bibliographies, custom classes and
% styles using the files menu.
%
% If you're new to LaTeX, the wikibook is a great place to start:
% http://en.wikibooks.org/wiki/LaTeX
%
%%%%%%%%%%%%%%%%%%%%%%%%%%%%%%%%%%%%%%%%%%%%%%%%%%%%%%%%%%%%%%%%%%%%%%

\documentclass{beamer}

\nonstopmode

% For more themes, color themes and font themes, see:
% http://deic.uab.es/~iblanes/beamer_gallery/index_by_theme.html
%
\mode<presentation>
{
\usetheme{Madrid}       % or try default, Darmstadt, Warsaw, ...
\usecolortheme{default} % or try albatross, beaver, crane, ...
\usefonttheme{serif}    % or try default, structurebold, ...
\setbeamertemplate{navigation symbols}{}
\setbeamertemplate{caption}[numbered]
} 

\usepackage{etex}

\usepackage[brazil]{babel}  % pt_BR hifenization
\usepackage[utf8x]{inputenc}
\usepackage{chemfig}
\usepackage{chronology}
\usepackage{amssymb}
\usepackage{empheq}
\usepackage{mathpartir}
\usepackage[all]{xy}
\usepackage[color]{coqdoc}
\usepackage{pdfpcnotes}


\usepackage{amsmath}

\usepackage{color}
\usepackage{xcolor}
\usepackage{listings}


% On writeLaTeX, these lines give you sharper preview images.
% You might want to `comment them out before you export, though.
\usepackage{pgfpages}
\pgfpagesuselayout{resize to}[%
physical paper width=8in, physical paper height=6in]

% Here's where the presentation starts, with the info for the title slide
\title[Normalização no sistema $\lambda$ex]{Em direção à formalização das propriedades de normalização do sistema
    $\lambda ex$}
\author{
Lucas de M. Amaral
}
\institute[UnB]{Universidade de Brasilia}
\date{09/12/2016}

\begin{document}

\lstset{escapeinside={<@}{@>}}


\begin{frame}
\titlepage
\end{frame}

% These three lines create an automatically generated table of contents.
\begin{frame}{Visão geral}
\tableofcontents
\end{frame}

\section{Introdução}


\subsection{O cálculo $\lambda$}

\frame{\tableofcontents[currentsection, currentsubsection]}


\begin{frame}
\frametitle{O Cálculo $\lambda$}
\begin{block}{Modelos teóricos de computação}
Estudar de uma maneira formal os limites da computação, entender quais funções são de fato computáveis, qual o poder computacional de certos tipos de operações, etc.
\end{block}
\begin{itemize}
\item Máquinas de Turing
\item \textbf{Cálculo $\lambda$}
\item Lógica Combinatória
\end{itemize}
\end{frame}

%-----------

\begin{frame}
\frametitle{Cálculo $\lambda$}

\begin{itemize}
\item Gramática mínima 
\end{itemize}

\pnote{Sistema formal}
\pnote{Interessante pelo foco em funções e gramatica simples}

\begin{block}{Gramática}
\[ \tau := x\ |\ \lambda y.\tau\ |\ \tau \tau \]
\end{block}

\begin{itemize}
\item Poucas regras de inferência
\end{itemize}

\begin{block}{ $\alpha$-equivalência }
    Dizemos que dois termos são $\alpha$-equivalentes se um pode ser obtido a
    partir do outro, através de renomeamento de variáveis ligadas.

    \[ \lambda x. x =_{\alpha} \lambda y. y \] 
\end{block}

\begin{block}{$\beta$-redução}
\[ (\lambda x.t)\ u \rightarrow_{\beta} t\{x/u\} \]
\end{block}

\end{frame}

%-----------


\begin{frame}
\frametitle{Cálculo $\lambda$}

\begin{block}{ Normalização }
\begin{itemize}
    \item Um termo $t$ é dito estar em \textbf{forma normal} quando não existe
        $t'$ tal que $ t \rightarrow_{\beta} t' $.
    \item Um termo é dito \textbf{fracamente normalizável} quando
    existe uma estratégia de redução que leva a uma forma normal.
    \item O termo é \textbf{fortemente normalizável} se toda estratégia leva à
        forma normal.
\end{itemize}

\end{block}

\end{frame}

%===========

\subsection{O sistema $\lambda ex$}

\frame{\tableofcontents[currentsection, currentsubsection]}


\begin{frame}
\frametitle{O sistema $\lambda ex$}
\begin{block}{Sintaxe}
\[ \tau := x\ |\ \lambda x.\tau\ |\ \tau \tau\ |\ \tau[x/\tau]\ \]
\end{block}

\pause

\begin{table}[h]
\begin{empheq}[box=\fbox]{align*}
    (\lambda x.\ t)\ u\ \ \ &\rightarrow_B\ t[x/u] \\
    x[x/u]\ \ \             &\rightarrow\ u \\
    y[x/u]\ \ \             &\rightarrow\ y,\ se\ x\ \neq\ y \\
    (t\ u)[x/v]\ \ \        &\rightarrow\ t[x/v]\ u[x/v] \\
    t[x/u][y/v]\ \ \        &\rightarrow\ t[y/v][x/u[y/v]],\ se\ y\ \in fv(u) \\ 
    (\lambda y.\ u)[x/v]\ \ &\rightarrow\ (\lambda y.\ u[x/v])
\end{empheq}
\end{table}
\pnote{ Por que adicionar subst explic? }
\pnote{ Motivar: framework pra linguagens e assistentes, ajuda em provas sobre o
    calc lambda }
\pnote{ Pq este? Simples, permite composição de substituições e preserva PSN }
\end{frame}

%-----------

\begin{frame}
\frametitle{O sistema $\lambda ex$}

\pnote{O que diferencia o lex?  }

\begin{block}{PSN}
    Seja $\lambda$z uma extensão do cálculo $\lambda$. Dizemos que $\lambda z$
    preserva a normalização forte se, para todo $\lambda$-termo fortemente
    normalizável, seu correspondente em $\lambda$z também é fortemente
    normalizável nesta extensão.
\end{block}

\pnote{Em geral limita-se interação de substituições para preservar o PSN}
\pnote{Motivar com prova construtiva}

\pause

\begin{block}{Classes de equivalência}
\[ t[x/u][y/v] =_C t[y/v][x/u] \ \ \ \ \ se\ y \notin fv(u)\ \&\ x \notin fv(v)\] 
\end{block}

\begin{itemize}
    \item Como deve ser demonstrada a PSN?
\end{itemize}

\pnote{ Definir uma estratégia de redução}
\pnote{Usar esta estratégia para definir indutivamente os termos fortemente
    normalizáveis do sistema $\lambda ex$}
\pnote{Mostrar que os termos normalizáveis do cálculo $\lambda$ estão
    neste novo conjunto}
    
\end{frame}


%-----------

\begin{frame}
\frametitle{O sistema $\lambda ex$}
    
    \begin{block}{ Propriedade IE }
        Sejam $t,\ u$ termos. Seja $SN_{\lambda ex}$ o conjunto de termos fortemente
        normalizáveis do sistema $\lambda ex$. Se $u \in SN_{\lambda ex}$ e
        $t\{x/u\} \in SN_{\lambda ex}$, então $t[x/u] \in SN_{\lambda ex}$.
    \end{block}
    \pnote{Falar que o lex pode ser simulado pela estrategia}
    \pnote{Se a estratégia leva a um termo em forma normal, o original também
    era}
    \pnote{A perpetualidade é essencial para mostrar a PSN}
    \pnote{A propriedade IE é um caso particular, pois leva a subst na outra} 

\end{frame}


%===========

\section{Formalização}
\subsection{Estruturas básicas}
\frame{\tableofcontents[currentsection, currentsubsection]}

%-----------

\begin{frame}{Gramática}


\begin{block}{Gramática}
\[ \tau := x\ |\ n\ |\ \lambda \tau\ |\ \tau \tau\ |\ \tau[\tau]\ \]
\end{block}

\begin{block}{Pré-termos}

\bigskip \coqdocnoindent \coqdockw{Inductive} \coqdef{LambdaES
    Defs.pterm}{pterm}{\coqdocinductive{pterm}} : \coqdockw{Set} :=\coqdoceol
\coqdocindent{1.00em} \ensuremath{|} \coqdef{LambdaES Defs.pterm
    bvar}{pterm\_bvar}{\coqdocconstructor{pterm\_bvar}} :
\coqexternalref{nat}{http://coq.inria.fr/distrib/8.4pl4/stdlib/Coq.Init.Datatypes}{\coqdocinductive{nat}}
\ensuremath{\rightarrow} \coqref{LambdaES
    Defs.pterm}{\coqdocinductive{pterm}}\coqdoceol \coqdocindent{1.00em}
\ensuremath{|} \coqdef{LambdaES Defs.pterm
    fvar}{pterm\_fvar}{\coqdocconstructor{pterm\_fvar}} : \coqdocdefinition{var}
\ensuremath{\rightarrow} \coqref{LambdaES
    Defs.pterm}{\coqdocinductive{pterm}}\coqdoceol \coqdocindent{1.00em}
\ensuremath{|} \coqdef{LambdaES Defs.pterm
    app}{pterm\_app}{\coqdocconstructor{pterm\_app}}  : \coqref{LambdaES
    Defs.pterm}{\coqdocinductive{pterm}} \ensuremath{\rightarrow}
\coqref{LambdaES Defs.pterm}{\coqdocinductive{pterm}} \ensuremath{\rightarrow}
\coqref{LambdaES Defs.pterm}{\coqdocinductive{pterm}}\coqdoceol
\coqdocindent{1.00em} \ensuremath{|} \coqdef{LambdaES Defs.pterm
    abs}{pterm\_abs}{\coqdocconstructor{pterm\_abs}}  : \coqref{LambdaES
    Defs.pterm}{\coqdocinductive{pterm}} \ensuremath{\rightarrow}
\coqref{LambdaES Defs.pterm}{\coqdocinductive{pterm}}\coqdoceol
\coqdocindent{1.00em} \ensuremath{|} \coqdef{LambdaES Defs.pterm
    sub}{pterm\_sub}{\coqdocconstructor{pterm\_sub}} : \coqref{LambdaES
    Defs.pterm}{\coqdocinductive{pterm}} \ensuremath{\rightarrow}
\coqref{LambdaES Defs.pterm}{\coqdocinductive{pterm}} \ensuremath{\rightarrow}
\coqref{LambdaES Defs.pterm}{\coqdocinductive{pterm}} \coqdoceol
%\coqdocindent{1.00em} \ensuremath{|} \coqdef{LambdaES Defs.pterm
%lsub}{pterm\_lsub}{\coqdocconstructor{pterm\_lsub}} : \coqref{LambdaES
%Defs.pterm}{\coqdocinductive{pterm}} \ensuremath{\rightarrow} \coqref{LambdaES
%Defs.pterm}{\coqdocinductive{pterm}} \ensuremath{\rightarrow} \coqref{LambdaES
%Defs.pterm}{\coqdocinductive{pterm}}.\coqdoceol 
\bigskip

\end{block}
    
\end{frame}

%-----------

\begin{frame}{Termos}

\begin{block}{Exemplo}
    \[ (\lambda (\lambda \ 0\ 1)) \] 
\end{block}
    \pause
\begin{alertblock}{}
    \[ (\lambda\ 2\ x)  \] 
    
\end{alertblock}


\end{frame}

%-----------

\begin{frame}{Termos bem formados}

\begin{columns}[T]
\begin{column}{.48\textwidth}
\small{

\coqdocnoindent \coqdockw{Fixpoint} \coqdocvar{lc\_at}
(\coqdocvar{k}:\coqdocvar{nat}) (\coqdocvar{t}:\coqdocvar{pterm})
\{\coqdockw{struct} \coqdocvar{t}\} : \coqdockw{Prop} :=\coqdoceol
\coqdocindent{1.00em} \coqdockw{match} \coqdocvar{t} \coqdockw{with} \coqdoceol
\coqdocindent{1.00em} \ensuremath{|} \coqdocvar{pterm\_bvar} \coqdocvar{i}
\ensuremath{\Rightarrow} \coqdocvar{i} $<$ \coqdocvar{k}\coqdoceol
\coqdocindent{1.00em} \ensuremath{|} \coqdocvar{pterm\_fvar} \coqdocvar{x}
\ensuremath{\Rightarrow} \coqdocvar{True}\coqdoceol \coqdocindent{1.00em}
\ensuremath{|} \coqdocvar{pterm\_app} \coqdocvar{t1} \coqdocvar{t2}
\ensuremath{\Rightarrow} \coqdocvar{lc\_at} \coqdocvar{k} \coqdocvar{t1}
\ensuremath{\land} \coqdocvar{lc\_at} \coqdocvar{k} \coqdocvar{t2}\coqdoceol
\coqdocindent{1.00em} \ensuremath{|} \coqdocvar{pterm\_abs} \coqdocvar{t1}
\ensuremath{\Rightarrow} \coqdocvar{lc\_at} (\coqdocvar{S} \coqdocvar{k})
\coqdocvar{t1}\coqdoceol \coqdocindent{1.00em} \ensuremath{|}
\coqdocvar{pterm\_sub} \coqdocvar{t1} \coqdocvar{t2} \ensuremath{\Rightarrow}
(\coqdocvar{lc\_at} (\coqdocvar{S} \coqdocvar{k}) \coqdocvar{t1})
\ensuremath{\land} \coqdocvar{lc\_at} \coqdocvar{k} \coqdocvar{t2}\coqdoceol
\coqdocindent{1.00em} \coqdockw{end}.\coqdoceol

\coqdocemptyline
\coqdocnoindent\coqdockw{Definition} \coqdocvar{term'} \coqdocvar{t} := \coqdocvar{lc\_at} 0 \coqdocvar{t}.\coqdoceol

}

\end{column}

\vrule
\hspace{10pt}

\begin{column}{.48\textwidth}
\small{

\coqdocnoindent \coqdockw{Inductive} \coqdef{LambdaES
    Defs.term}{term}{\coqdocinductive{term}} : \coqref{LambdaES
    Defs.pterm}{\coqdocinductive{pterm}} \ensuremath{\rightarrow}
\coqdockw{Prop} :=\coqdoceol  \ensuremath{|}
\coqdef{LambdaES Defs.term var}{term\_var}{\coqdocconstructor{term\_var}} :
\coqdockw{\ensuremath{\forall}} \coqdocvar{x},\coqdoceol 
\ \ \coqref{LambdaES Defs.term}{\coqdocinductive{term}} (\coqref{LambdaES Defs.pterm
    fvar}{\coqdocconstructor{pterm\_fvar}} \coqdocvariable{x})\coqdoceol
 \ensuremath{|} \coqdef{LambdaES Defs.term
    app}{term\_app}{\coqdocconstructor{term\_app}} :
\coqdockw{\ensuremath{\forall}} \coqdocvar{t1} \coqdocvar{t2},\coqdoceol
\ \ \coqref{LambdaES Defs.term}{\coqdocinductive{term}}
\coqdocvariable{t1} \ensuremath{\rightarrow} \coqdoceol 
\ \ \coqref{LambdaES Defs.term}{\coqdocinductive{term}} \coqdocvariable{t2}
\ensuremath{\rightarrow} \coqdoceol  \ \ \coqref{LambdaES
    Defs.term}{\coqdocinductive{term}} (\coqref{LambdaES Defs.pterm
    app}{\coqdocconstructor{pterm\_app}} \coqdocvariable{t1}
\coqdocvariable{t2})\coqdoceol  \ensuremath{|}
\coqdef{LambdaES Defs.term abs}{term\_abs}{\coqdocconstructor{term\_abs}} :
\ \ \coqdockw{\ensuremath{\forall}} \coqdocvar{L} \coqdocvar{t1},\coqdoceol
 (\coqdockw{\ensuremath{\forall}} \coqdocvar{x},
\coqdocvariable{x} \coqdocnotation{\symbol{92}}\coqdocnotation{notin}
\coqdocvariable{L} \ensuremath{\rightarrow} \coqref{LambdaES
    Defs.term}{\coqdocinductive{term}} (\coqdocvariable{t1} \coqref{LambdaES
    Defs.::x 'x5E' x}{\coqdocnotation{\^{}}} \coqdocvariable{x}))
\ensuremath{\rightarrow}\coqdoceol \ \  \coqref{LambdaES
    Defs.term}{\coqdocinductive{term}} (\coqref{LambdaES Defs.pterm
    abs}{\coqdocconstructor{pterm\_abs}} \coqdocvariable{t1})\coqdoceol
 \ensuremath{|} \coqdef{LambdaES Defs.term
    sub}{term\_sub}{\coqdocconstructor{term\_sub}} :
\ \ \coqdockw{\ensuremath{\forall}} \coqdocvar{L} \coqdocvar{t1}
\coqdocvar{t2},\coqdoceol  (\coqdockw{\ensuremath{\forall}}
\coqdocvar{x}, \coqdocvariable{x}
\coqdocnotation{\symbol{92}}\coqdocnotation{notin} \coqdocvariable{L}
\ensuremath{\rightarrow} \coqref{LambdaES Defs.term}{\coqdocinductive{term}}
(\coqdocvariable{t1} \coqref{LambdaES Defs.::x 'x5E' x}{\coqdocnotation{\^{}}}
\coqdocvariable{x})) \ensuremath{\rightarrow}\coqdoceol 
\ \ \coqref{LambdaES Defs.term}{\coqdocinductive{term}} \coqdocvariable{t2}
\ensuremath{\rightarrow} \coqdoceol \ \  \coqref{LambdaES
    Defs.term}{\coqdocinductive{term}} (\coqref{LambdaES Defs.pterm
    sub}{\coqdocconstructor{pterm\_sub}} \coqdocvariable{t1}
\coqdocvariable{t2}).\coqdoceol

}

\end{column}

\end{columns}

\bigskip
\bigskip
\coqdocindent{4.00em}
\coqdockw{Lemma} \coqdocvar{term\_eq\_term'} : \coqdockw{\ensuremath{\forall}}
\coqdocvar{t}, \coqdocvar{term} \coqdocvar{t} \ensuremath{\leftrightarrow}
\coqdocvar{term'} \coqdocvar{t}.\coqdoceol
\bigskip
    
\end{frame}

%-----------

\begin{frame}{Equação}

    \begin{block}{Definição de eqc}
        
    \coqdocnoindent \coqdockw{Inductive} \coqdef{Equation
        C.eqc}{eqc}{\coqdocinductive{eqc}} : \coqdocinductive{pterm}
    \ensuremath{\rightarrow} \coqdocinductive{pterm} \ensuremath{\rightarrow}
    \coqdockw{Prop} := \coqdoceol \coqdocindent{1.00em} \ensuremath{|}
    \coqdef{Equation C.eqc def}{eqc\_def}{\coqdocconstructor{eqc\_def}}:
    \coqdockw{\ensuremath{\forall}} \coqdocvar{t} \coqdocvar{u} \coqdocvar{v},
    \coqdocdefinition{lc\_at} 2 \coqdocvariable{t} \ensuremath{\rightarrow}
    \coqdocinductive{term} \coqdocvariable{u} \ensuremath{\rightarrow}
    \coqdocinductive{term} \coqdocvariable{v} \ensuremath{\rightarrow}
    \coqref{Equation C.eqc}{\coqdocinductive{eqc}}
    (\coqdocvariable{t}\coqdocnotation{[}\coqdocvariable{u}\coqdocnotation{][}\coqdocvariable{v}\coqdocnotation{]})
    (\coqdocnotation{(}\coqdocnotation{\&}
    \coqdocvariable{t}\coqdocnotation{)[}\coqdocvariable{v}\coqdocnotation{][}\coqdocvariable{u}\coqdocnotation{]}).\coqdoceol


    \bigskip \coqdockw{Definition} \coqdef{Equation C.eqc
        ctx}{eqc\_ctx}{\coqdocdefinition{eqc\_ctx}} (\coqdocvar{t}
    \coqdocvar{u}: \coqdocinductive{pterm}) :=
    \coqdocinductive{ES\_contextual\_closure} \coqref{Equation
        C.eqc}{\coqdocinductive{eqc}} \coqdocvariable{t}
    \coqdocvariable{u}.

    \coqdockw{Definition} \coqdef{Equation C.eqC}{eqC}{\coqdocdefinition{eqC}}
    (\coqdocvar{t} : \coqdocinductive{pterm}) (\coqdocvar{u} :
    \coqdocinductive{pterm}) := \coqdocinductive{star\_closure} \coqref{Equation
        C.eqc ctx}{\coqdocdefinition{eqc\_ctx}} \coqdocvariable{t}
    \coqdocvariable{u}.\coqdoceol \coqdockw{Notation} "t =e u" := (\coqref{Equation
    C.eqC}{\coqdocdefinition{eqC}} \coqdocvar{t} \coqdocvar{u}) (\coqdoctac{at}
\coqdockw{level} 66).\coqdoceol
    \end{block}

    \pnote{Falar da preservação de propriedades pela equivalencia}

    %\pause

%\begin{itemize}
    %\item Construtores de termos, abertura de termos, substituições e
        %renomeamento de variáveis devem funcionar de maneira análoga para dois
        %termos equivalentes.
    %\item Boa formação de termos e corpos de abstrações, conjunto de variáveis
        %livres e reduções dentro de fechos contextuais também devem ser
        %preservadas pela equivalência.
%\end{itemize}
    
\end{frame}

%-----------

\subsection{Equivalência das relações}
\frame{\tableofcontents[currentsection, currentsubsection]}

\begin{frame}
\frametitle{O sistema $\lambda ex$}

\begin{itemize}
    \item Marcar algumas substituições seguras
    \item Definir reduções que permitem uma manipulação maior de termos marcados
    \item Demonstrar que a regra do sistema é simulada pelas novas reduções
    \item Estudar a terminação destas reduções
\end{itemize}
    
\end{frame}

%-----------

\begin{frame}{Cálculo com termos marcados}
    \begin{block}{Gramática de pré-termos}
    \[ \tau := x\ |\ \lambda x.\tau\ |\ \tau \tau\ |\ \tau[x/\tau]\ \]
    \end{block}
    \begin{block}{Gramática de pré-termos marcados}
    \[ \tau := x\ |\ \lambda x.\tau\ |\ \tau \tau\ |\ \tau[x/u]\ |\
        \tau[\![x/\tau]\!]\ \]
    \end{block}

\begin{block}{Termos marcados}
    \coqdocnoindent \coqdockw{Inductive} \coqdef{LambdaES
        Defs.pterm}{pterm}{\coqdocinductive{pterm}} : \coqdockw{Set} :=\coqdoceol
    \coqdocindent{1.00em} \ensuremath{|} ...\coqdoceol 
    \coqdocindent{1.00em} \ensuremath{|} \coqdef{LambdaES Defs.pterm
        lsub}{pterm\_lsub}{\coqdocconstructor{pterm\_lsub}} : \coqref{LambdaES
        Defs.pterm}{\coqdocinductive{pterm}} \ensuremath{\rightarrow}
    \coqref{LambdaES Defs.pterm}{\coqdocinductive{pterm}} \ensuremath{\rightarrow}
    \coqref{LambdaES Defs.pterm}{\coqdocinductive{pterm}}.\coqdoceol 
\end{block}

\pnote{ Estamos estendendo mais o sistema }

\end{frame}

%-----------


\begin{frame}{Equivalência de reduções}


\begin{table}[h]
\begin{empheq}[box=\fbox]{align*}
    x[\![x/u]\!]\ \ \             &\rightarrow_{Var}\ u \\
    t[\![x/u]\!]\ \ \             &\rightarrow_{Gc}\ t    & se\ \text{\emph{x}} \notin fv(t)\\
    (t\ u)[\![x/v]\!]\ \ \        &\rightarrow_{App}\ t[\![x/v]\!]\ u[\![x/v]\!] \\
    (\lambda y.\ u)[\![x/v]\!]\ \ &\rightarrow_{Lamb}\ (\lambda y.\ u[\![x/v]\!])\\
    t[\![x/u]\!][\![y/v]\!]\ \ \        &\rightarrow_{Comp}\ t[\![y/v]\!][\![x/u[\![y/v]\!]]\!] & se\ y\ \in fv(u)  
\end{empheq}
    \caption{A redução $\rightarrow$$_{\underline{x}}$ }
\end{table}

\pause

\[ t \rightarrow_{\lambda \underline{ex}} t' \iff \exists s,\ s';\ t =_{e \cup
        \underline{e} \cup \alpha} s \rightarrow_{Bx \cup \underline{x}} s' =_{e \cup
        \underline{e} \cup \alpha} t' \] 

\pause

\[t \rightarrow_{\lambda \underline{ex}} t' \iff t \rightarrow_{\lambda
        \underline{ex}^i \cup \lambda \underline{ex}^e} t'\] 

\end{frame}

%-----------

\begin{frame}{Equivalência de relações}

    \begin{block}{Relação interna}
    \begin{itemize}
        \item Se $u \rightarrow_{Bx} u'$ e $u,\ u'$ são termos, então $t[\![x/u]\!]
            \rightarrow_{\lambda \underline{x}^i} t[\![x/u']\!]$ 
        \item Se $t
            \rightarrow_{\underline{x}} t'$, então $t \rightarrow_{\lambda
                \underline{x}^i} t'$
        \item Se $t \rightarrow_{\lambda \underline{x}^i} t'$, então vale 
            $t\ u \rightarrow_{\lambda \underline{x}^i} t'\ u$,  
            $u\ t \rightarrow_{\lambda \underline{x}^i} u\ t'$, 
            $\lambda x. t \rightarrow_{\lambda \underline{x}^i} \lambda x. t'$, 
            $t[x/u] \rightarrow_{\lambda \underline{x}^i} t'[x/u]$, 
            $u[x/t] \rightarrow_{\lambda \underline{x}^i} u[x/t']$ e 
            $t[\![x/u]\!] \rightarrow_{\lambda \underline{x}^i} t'[\![x/u]\!]$.
    \end{itemize}
    \end{block}

    \begin{block}{Relação externa}
    \begin{itemize}
        \item Se $t \rightarrow_{Bx} t'$ ocorre fora de uma substituição marcada, então 
            $t \rightarrow_{\lambda \underline{x}^e} t$ 
        \item Se $t \rightarrow_{\lambda \underline{x}^e} t'$, então vale 
            $t\ u \rightarrow_{\lambda \underline{x}^e} t'\ u$,  
            $u\ t \rightarrow_{\lambda \underline{x}^e} u\ t'$, 
            $\lambda x. t \rightarrow_{\lambda \underline{x}^e} \lambda x. t'$, 
            $t[x/u] \rightarrow_{\lambda \underline{x}^e} t'[x/u]$, 
            $u[x/t] \rightarrow_{\lambda \underline{x}^e} u[x/t']$ e 
            $t[\![x/u]\!] \rightarrow_{\lambda \underline{x}^e} t'[\![x/u]\!]$.
    \end{itemize}
    \end{block}
    
\end{frame}

%-----------

\begin{frame}{Equivalência de relações}
    \begin{block}{Formalização}

    \coqdocnoindent \coqdockw{Notation} ``t -{}-$>$[lex]\ u" :=
    (\coqdocvar{lab\_EE\_ctx\_red}\ \coqdocvar{lab\_sys\_lx}\ \coqdocvar{t}
    \coqdocvar{u}) .

    \bigskip

    \coqdocnoindent \coqdockw{Notation} ``t -{}-$>$[lx\_i]\ u" :=
    (\coqdocvar{ext\_lab\_EE\_ctx\_red}\ \coqdocvar{lab\_x\_i}\ \coqdocvar{t}
    \coqdocvar{u}) .

    \bigskip

    \coqdocnoindent \coqdockw{Notation} ``t -{}-$>$[lx\_e]\ u" :=
    (\coqdocvar{ext\_lab\_EE\_ctx\_red}\ \coqdocvar{sys\_Bx}\ \coqdocvar{t}
    \coqdocvar{u}) .  \coqdoceol

    \bigskip

    \coqdockw{Theorem} \coqdocvar{lab\_ex\_eq\_i\_e}:
    \coqdockw{\ensuremath{\forall}} \coqdocvar{t} \coqdocvar{t'},
    \coqdocvar{lab\_term} \coqdocvar{t} \ensuremath{\rightarrow} (\coqdocvar{t}
    -{}-$>$[\coqdocvar{lex}] \coqdocvar{t'} \ensuremath{\leftrightarrow} (\coqdocvar{t}
    -{}-$>$[\coqdocvar{lx\_i}] \coqdocvar{t'} \ensuremath{\lor} \coqdocvar{t}
    -{}-$>$[\coqdocvar{lx\_e}] \coqdocvar{t'})).\coqdoceol

    \end{block}

    \pnote{Prova simples no papel, mas extensa no assistente}


    \pnote{ -> Redução interna quando vier dentro de marcada! }
    \pnote{ <- Análogo ao anterior, mas dividindo em duas. O caso base é feito
        com analise de casos na relação int e Bx }
    \pnote{ <- Não precisamos lidar com o caso da substituição marcada! Vem do
        caso base da interna }
\end{frame}

%-----------

\begin{frame}
\frametitle{O sistema $\lambda ex$}
    

    \coqdockw{Theorem} \coqdocvar{lab\_ex\_eq\_i\_e}:
    \coqdockw{\ensuremath{\forall}} \coqdocvar{t} \coqdocvar{t'},
    \coqdocvar{lab\_term} \coqdocvar{t} \ensuremath{\rightarrow} (\coqdocvar{t}
    -{}-$>$[\coqdocvar{lex}] \coqdocvar{t'} \ensuremath{\leftrightarrow} (\coqdocvar{t}
    -{}-$>$[\coqdocvar{lx\_i}] \coqdocvar{t'} \ensuremath{\lor} \coqdocvar{t}
    -{}-$>$[\coqdocvar{lx\_e}] \coqdocvar{t'})).


\begin{block}{Objetivo - Caso da abstração}
\center{

    \coqdocindent{1.2in}(\coqdocvar{pterm\_abs} \coqdocvar{t})
    -{}-$>$[\coqdocvar{lx\_i}](\coqdocvar{pterm\_abs} \coqdocvar{t'})
    \coqdoceol\ensuremath{\lor} (\coqdocvar{pterm\_abs} \coqdocvar{t})
    -{}-$>$[\coqdocvar{lx\_e}](\coqdocvar{pterm\_abs} \coqdocvar{t'}).

}
\end{block}



\begin{block}{Hipótese de indução - Caso da abstração}
\center{

    \coqdocindent{1.2in}\coqdockw{\ensuremath{\forall}} \coqdocvar{x} : \coqdocvar{VarSet.elt},
    \coqdocvar{x} \coqdocvar{\textbackslash notin} \coqdocvar{L}
    \ensuremath{\rightarrow} \coqdoceol  \coqdocvar{t} \^{} \coqdocvar{x}
    -{}-$>$[\coqdocvar{lx\_i}]\coqdocvar{t'} \^{} \coqdocvar{x} \ensuremath{\lor}
    \coqdocvar{t} \^{} \coqdocvar{x}-{}-$>$[\coqdocvar{lx\_e}]\coqdocvar{t'} \^{}
    \coqdocvar{x}.
}
\end{block}

\end{frame}

%-----------


\begin{frame}
\frametitle{O sistema $\lambda ex$}
    

    \coqdockw{Theorem} \coqdocvar{lab\_ex\_eq\_i\_e}:
    \coqdockw{\ensuremath{\forall}} \coqdocvar{t} \coqdocvar{t'},
    \coqdocvar{lab\_term} \coqdocvar{t} \ensuremath{\rightarrow} (\coqdocvar{t}
    -{}-$>$[\coqdocvar{lex}] \coqdocvar{t'} \ensuremath{\leftrightarrow} (\coqdocvar{t}
    -{}-$>$[\coqdocvar{lx\_i}] \coqdocvar{t'} \ensuremath{\lor} \coqdocvar{t}
    -{}-$>$[\coqdocvar{lx\_e}] \coqdocvar{t'})).


\begin{block}{Objetivo - Caso da substituição marcada (interno)}
\center{
    \coqdocvar{lab\_sys\_lx} \coqdocvar{u} \coqdocvar{u'}.
}
\end{block}



\begin{block}{Hipótese de indução - Caso da substituição marcada (interno)}
\center{

    \coqdocvar{t}[[\coqdocvar{u}]]
    -{}-$>$[\coqdocvar{lx\_i}] \coqdocvar{t}[[\coqdocvar{u'}]] \ensuremath{\lor}
    \coqdocvar{t}[[\coqdocvar{u}]]
    -{}-$>$[\coqdocvar{lx\_e}] \coqdocvar{t}[[\coqdocvar{u'}]]
}
\end{block}

\end{frame}


%===========

\section{Conclusão}
\frame{\tableofcontents[currentsection]}

\begin{frame}{Conclusão}

\begin{itemize}
    \item Continuamos a formalização do sistema $\lambda ex$ 
    \item Demonstramos preservação de propriedades pela equação no sistema puro
    \item Desenvolvemos mais a teoria no sistema com substituições marcadas
    \item Iniciamos a prova da propriedade IE, provando equivalência das reduções
    \item Como trabalho futuro, completar a prova da propriedade IE        
        
\end{itemize}
    
\end{frame}

\begin{frame}
    
\begin{thebibliography}{1}

\bibitem{initial}
Washington de~Carvalho~Segundo, Fl{\'{a}}vio L.~C. de~Moura, and Daniel
  Ventura.
\newblock Formalizing a named explicit substitutions calculus in coq.
\newblock In {\em Joint Proceedings of the MathUI, OpenMath and ThEdu Workshops
  and Work in Progress track at {CICM} co-located with Conferences on
  Intelligent Computer Mathematics {(CICM} 2014), Coimbra, Portugal, July 7-11,
  2014.}, volume 1186, 2014.

\bibitem{delia}
D.~Kesner.
\newblock {\em {A Theory of Explicit Substitutions with Safe and Full
  Composition}}, volume~5, pages 1--29.
\newblock 2009.

\end{thebibliography}
\end{frame}

\end{document}

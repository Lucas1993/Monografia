%%%%%%%%%%%%%%%%%%%%%%%%%%%%%%%%%%%%%%%%
% Classe do documento
%%%%%%%%%%%%%%%%%%%%%%%%%%%%%%%%%%%%%%%%

% Nós usamos a classe "unb-cic".  Deixe apenas uma das linhas
% abaixo não-comentada, dependendo se você for do bacharelado ou
% da licenciatura.

\documentclass[bacharelado]{unb-cic}
%\documentclass[licenciatura]{unb-cic}

\nonstopmode

%%%%%%%%%%%%%%%%%%%%%%%%%%%%%%%%%%%%%%%%
% Pacotes importados
%%%%%%%%%%%%%%%%%%%%%%%%%%%%%%%%%%%%%%%%

\usepackage[brazil,american]{babel}
\usepackage[T1]{fontenc}
\usepackage{indentfirst}
\usepackage{natbib}
\usepackage{xcolor,graphicx,url}
\usepackage{amsmath}
\usepackage{amssymb}
\usepackage{amsthm}
\usepackage{empheq}
\usepackage{mathpartir}
\usepackage{listings}
\usepackage{scrextend}
\usepackage[all]{xy}
\usepackage[color]{coqdoc}
\usepackage[utf8]{inputenc}


\lstset{frame = single, aboveskip=20pt, belowskip=20pt }

%%%%%%%%%%%%%%%%%%%%%%%%%%%%%%%%%%%%%%%%
% Definições
%%%%%%%%%%%%%%%%%%%%%%%%%%%%%%%%%%%%%%%%

\newtheorem{teorema}{Teorema}
\newtheorem{lema}{Lema}[chapter]
\newtheorem{definicao}{Definição}[chapter]

%%%%%%%%%%%%%%%%%%%%%%%%%%%%%%%%%%%%%%%%
% Cores dos links
%%%%%%%%%%%%%%%%%%%%%%%%%%%%%%%%%%%%%%%%

% Veja o arquivos cores.tex se quiser ver que outras cores estão
% pré-definidas.  Utilizando o comando \hypersetup abaixo nós
% evitamos aquelas caixas vermelhas feias em volta dos links.

%%%%%%%%%%%%%%%%%%%%%%%%%%%%%%%%%%%%%%%%
% Cores do estilo Tango
%%%%%%%%%%%%%%%%%%%%%%%%%%%%%%%%%%%%%%%%

\definecolor{LightButter}{rgb}{0.98,0.91,0.31}
\definecolor{LightOrange}{rgb}{0.98,0.68,0.24}
\definecolor{LightChocolate}{rgb}{0.91,0.72,0.43}
\definecolor{LightChameleon}{rgb}{0.54,0.88,0.20}
\definecolor{LightSkyBlue}{rgb}{0.45,0.62,0.81}
\definecolor{LightPlum}{rgb}{0.68,0.50,0.66}
\definecolor{LightScarletRed}{rgb}{0.93,0.16,0.16}
\definecolor{Butter}{rgb}{0.93,0.86,0.25}
\definecolor{Orange}{rgb}{0.96,0.47,0.00}
\definecolor{Chocolate}{rgb}{0.75,0.49,0.07}
\definecolor{Chameleon}{rgb}{0.45,0.82,0.09}
\definecolor{SkyBlue}{rgb}{0.20,0.39,0.64}
\definecolor{Plum}{rgb}{0.46,0.31,0.48}
\definecolor{ScarletRed}{rgb}{0.80,0.00,0.00}
\definecolor{DarkButter}{rgb}{0.77,0.62,0.00}
\definecolor{DarkOrange}{rgb}{0.80,0.36,0.00}
\definecolor{DarkChocolate}{rgb}{0.56,0.35,0.01}
\definecolor{DarkChameleon}{rgb}{0.30,0.60,0.02}
\definecolor{DarkSkyBlue}{rgb}{0.12,0.29,0.53}
\definecolor{DarkPlum}{rgb}{0.36,0.21,0.40}
\definecolor{DarkScarletRed}{rgb}{0.64,0.00,0.00}
\definecolor{Aluminium1}{rgb}{0.93,0.93,0.92}
\definecolor{Aluminium2}{rgb}{0.82,0.84,0.81}
\definecolor{Aluminium3}{rgb}{0.73,0.74,0.71}
\definecolor{Aluminium4}{rgb}{0.53,0.54,0.52}
\definecolor{Aluminium5}{rgb}{0.33,0.34,0.32}
\definecolor{Aluminium6}{rgb}{0.18,0.20,0.21}

\hypersetup{
  colorlinks=true,
  linkcolor=DarkScarletRed,
  citecolor=DarkScarletRed,
  filecolor=DarkScarletRed,
  urlcolor= DarkScarletRed
}



%%%%%%%%%%%%%%%%%%%%%%%%%%%%%%%%%%%%%%%%
% Informações sobre a monografia
%%%%%%%%%%%%%%%%%%%%%%%%%%%%%%%%%%%%%%%%

\title{Em direção à formalização das propriedades de normalização do sistema
    $\lambda ex$}%

\orientador{\prof \dr Flávio L. C. de Moura}{CIC/UnB}%
\coordenador{\prof \dr Rodrigo Bonifácio de Almeida}{Universidade de Brasília}%
\diamesano{09}{Dezembro}{2016}%

\membrobanca{\prof \dr Mauricio Ayala-Ríncon}{MAT/CIC/UnB}%
\membrobanca{\prof \dr Daniel Lima Ventura}{INF/UFG}%

\autor{Lucas de Moura}{Amaral}%

\CDU{004.4}

\palavraschave{cálculo lambda, verificação formal, substituições explícitas, Coq}%
\keywords{lambda calculus, formal verification, explicit substitutions, Coq}%



%%%%%%%%%%%%%%%%%%%%%%%%%%%%%%%%%%%%%%%%
% Texto
%%%%%%%%%%%%%%%%%%%%%%%%%%%%%%%%%%%%%%%%

\begin{document}
  \maketitle
  \pretextual

  \begin{dedicatoria}
      Dedico este trabalho à memória do meu pai, Gesley, que sempre fez de tudo
      para me dar uma boa educação e me incentivou a escolher o curso de Ciência
      de Computação. Também à minha família, que sempre me ajuda em tudo que
      preciso e não deixa de demonstrar interesse em minha vida: à minha
      mãe, Esther; meu padastro, Márcio; e a meus irmãos Thiago, Davi e
      Benjamin. Agradeço em especial à minha mãe, meu tio Donney e minha avó
      Nicinha, por nunca deixar faltar nada para que eu pudesse me concentrar em
      minha formação.

      Dedico também aos meus amigos, tanto da escola quanto da faculdade. Em
      especial, agradeço ao Calil, à Dani e à minha prima, Tephinha, por
      aguentarem minhas reclamações e me tranquilizarem nos momentos de
      ansiedade e cansaço.
  \end{dedicatoria}

  \begin{agradecimentos}
      Agradeço ao Departamento de Ciência da Computação e aos meus professores
      pela boa base técnica dada. Em especial, agradeço ao meu orientador,
      Flávio, pela disponibilidade e paciência para me ajudar no trabalho.
  \end{agradecimentos}

  \begin{resumo}
    O cálculo $\lambda$ é um sistema formal, capaz de expressar o processo
    computacional.  Pela sua simplicidade e expressividade, este cálculo é usado
    como modelo teórico para o paradigma de programação funcional. Em consequência
    disto, uma grande quantidade de extensões do cálculo foi proposta, com o
    objetivo de obter um sistema formal intermediário entre o cálculo $\lambda$ e suas
    implementações.  O objeto de estudo deste trabalho é uma destas variantes,
    chamada $\lambda$ex, um cálculo com substituições explicitas proposto por Delia
    Kesner. O objetivo foi continuar o trabalho de formalização deste cálculo, no
    assistente de prova Coq, iniciado em 2014, e que tem por objetivo fornecer
    uma prova mecância e construtiva da propriedade de formalização forte para o
    cálculo $\lambda ex$. Mais especificamente, iniciamos a prova da
    propriedade IE, chave para a prova da preservação da normalização forte do
    cálculo $\lambda ex$.
    \end{resumo}

  \selectlanguage{american}
  \begin{abstract}
    The $\lambda$-calculus is a formal system, capable of expressing the
    computational process.  Because of it's simplicity and expressiveness, this
    calculus is used as a theorical model for the paradigm of functional
    programming. Consequently, a great variety of extensions was proposed, with
    the goal of obtaining an intermediate formal system between the
    $\lambda$-calculus and its implementations.  The object of study of this
    work is one of these variants, called $\lambda$ex, a calculus with explicit
    subsititutions, proposed by Delia Kesner. The goal was to continue the work
    in the formalization of this calculus, in the Coq proof assistant, initiated
    in 2014, with the goal of providing a mechanical and constructive proof of
    the strong normalization property for the $\lambda ex$ calculus. More
    specifically, we began the proof of the IE property, key to the
    demonstration of the preservation of strong normalization of the $\lambda
    ex$-calculus.
\end{abstract}
  \selectlanguage{brazil}

  \tableofcontents
  \listoffigures
  \listoftables

  \textual
  \chapter{Introdução}

%%%%%%%%%%%%%%%%%%%%%%%%%%%%%%%%%%%%%%%%%%%%%%%%%%%%%%%%%%%%%%%%%%%%%%%%%%%%%%%%
%%%%%%%%%%%%%%%%%%%%%%%%%%%%%%%%%%%%%%%%%%%%%%%%%%%%%%%%%%%%%%%%%%%%%%%%%%%%%%%%
%%%%%%%%%%%%%%%%%%%%%%%%%%%%%%%%%%%%%%%%%%%%%%%%%%%%%%%%%%%%%%%%%%%%%%%%%%%%%%%%
\section{O assistente de provas Coq} 
\subsection{Motivação}

Assistentes de provas são sistemas computacionais que permitem aos usuários
realizar provas e definições em um computador. Neles, o usuário pode construir
toda sua teoria matemática em uma linguagem em que o sistema seja capaz de
verificar automaticamente. Ou seja, a principal motivação por trás de um
assistente de prova é verificar formalmente as provas de uma teoria. Apesar de
já existir um processo humano de verificação, muitas vezes ocorrem erros neste
processo, e provas que foram aceitas numa primeira avaliação são descobertas
problemáticas algum tempo depois. Como exemplo, podemos citar o teorema das
quatro cores \cite{four_colour}, que desafiou matemáticos por anos e foram
apresentadas falsas provas diversas vezes. Eventualmente, o teorema foi provado
em um assistente de prova e hoje sua corretude é aceita.

Mas, então, o que exatamente significa uma prova? Uma prova normalmente é
definida como o processo de se estabelecer a validade de alguma afirmação. Na
matemática, as provas costumam exigir uma clareza e rigor mais extremo, de modo
a se tornar indiscutível quando analizada com cuidado. Porém, matemáticos são
humanos e, infelizmente, cometem erros. Com isto em mente, foi definida uma
noção ainda mais forte de prova, chamada \emph{prova formal}.

Uma prova formal é uma sequência finita de sentenças tal que cada uma delas ou é
um axioma, uma suposição ou é derivada diretamente das sentenças anteriores
através das chamadas regras de inferência. A vantagem do rigor das provas
formais é que conferí-la se torna um processo muito mais simples, sendo
necessário apenas confirmar de onde vem cada uma das sentenças.

Por este motivo, provas formais são comumente construídas e verificadas pelos
assistentes de prova. Ao utilizar a linguagem do assistente, ele aos poucos
constrói a sequência de sentenças e simultaneamente checa sua validade. Porém,
isto gera outra dúvida: Por que confiar nos assistentes de prova?


\begin{description}
    \item[Lógica do assistente:] Os assistentes de prova em geral possuem uma
        teoria forte no qual são baseados. Em geral, existe um sistema
        matemático independente de implementação que pode ser estudado e
        verificado anteriormente.
    \item[Checagem do assitente:] O assistente em si é, também, um programa. Assim,
        podemos analizar seus algoritmos, demonstrar que só é possível provar
        teoremas derivados no sistema lógico interno e testar seu funcionamento
        como um programa normal.
    \item[Critério de De Bruijn:] Muitos assistentes de prova criam um
        \emph{objeto de prova}, uma prova formal independente do sistema que
        pode ser checada tanto por outro programa, quanto por um matemático
        interessado.  
\end{description}

Recentemente, existe um grande número de matemáticos interessados em assistentes
de prova, buscando construir uma teoria consistente para o uso destes e
produzindo o software necessário para facilitar seu uso. Em especial, um dos
assistentes com maior uso é o chamado Coq, a ser apresentado a seguir.

Para uma visão geral sobre o histórico e uso de assistentes de prova, veja
\cite{proof_assist}. 

\subsection{A ferramenta}

Neste trabalho, usaremos o Coq, um assistente de provas que está em
desenvolvimento desde 1983, em vários institutos de pesquisa franceses. O Coq
provê um rico ambiente para o desenvolvimento de um raciocínio formal checado
automaticamente. O núcleo do sistema é um checador de provas simples que garante
que apenas passos válidos de dedução são efetuados. Além deste núcleo, o
ambiente provê diversas táticas para facilitar a construções de provas, junto
com uma vasta biblioteca de definições e teoremas comuns.

A ferramenta vem acompanhada de uma linguagem de programação funcional, com
tipos dependentes. É através desta linguagem que podemos criar as definições e
provar os lemas de nossa teoria. Ela é baseada no Cálculo de Construções
Indutivas, uma extensão do cálculo lambda que serve como modelo teórico para o
sistema. O processo de se verificar a corretude de uma prova em Coq se reduz ao
problema de \emph{checagem de tipos}. A seguir, será feita uma introdução à
sintaxe e o funcionamento da ferramenta, baseada em tutoriais disponibilizados
na página oficial do sistema, em \cite{coq} e \cite{coq2}.


Os objetos de Coq podem ser divididos em duas categorias, \emph{Prop} e
\emph{Type}. A categoria \emph{Prop} é a das proposições bem formadas. Um
exemplo de proposição na linguagem seria:

\begin{lstlisting}[basicstyle=\small]
    forall A B : Prop, A -> (A \/ B).
\end{lstlisting}

Predicados podem ser definidos indutivamente, como a seguir.

\begin{lstlisting}[basicstyle=\small]
    Inductive even : N -> Prop :=
    | even_0 : even 0
    | even_S n : odd n -> even (n + 1)
    with odd : N -> Prop :=
    | odd_S n : even n -> odd (n + 1).
\end{lstlisting}

Predicados também podem ser feitos como definições diretas, como:

\begin{lstlisting}[basicstyle=\small]
    Definition sqr(x : N) := exists z, z*z = x.
\end{lstlisting}

Assim, podemos utilizar estes predicados como propriedades de algum objeto,
provando algo como \emph{even(2)} ou \emph{sqr(4)}.

\emph{Type} é a categoria de estruturas matemáticas e estruturas de dados.
Alguns exemplos de tipos são:

\begin{lstlisting}[basicstyle=\small]
    Z x Z -> Z
\end{lstlisting}

Tipos também podem ser definidos indutivamente:

\begin{lstlisting}[basicstyle=\small]
    Inductive nat : Set :=
    | 0 : nat
    | S : nat -> nat.
\end{lstlisting}

Neste caso, elementos do tipo \emph{nat} são: 0, S ( 0 ), S ( S ( 0 ) ), etc.

O desenvolvimento de provas em Coq é feito através de uma linguagem de provas,
que permite um processo guiado pelo usuário. Ao utilizar uma tática, o usuário
está construindo os objetos de prova. Por exemplo, a tática
\texttt{intro n}, onde \emph{n} é do tipo \texttt{nat}, constrói o termo (com um
buraco):

\begin{lstlisting}[basicstyle=\small]
    fun (n:nat) => _
\end{lstlisting}

Onde \_ representa um termo que irá ser construído futuramente, utilizando
outras táticas. Um exemplo de proposição simples que podemos querer provar é
\texttt{forall A : Prop, A -> A}. Provamos isto da seguinte maneira:

\begin{lstlisting}[basicstyle=\small]
    Theorem prova_simples : (forall A : Prop, A -> A).
    Proof.
        intros A.
        intros prova_de_A.
        exact prova_de_A.
    Qed.
\end{lstlisting}

Onde \emph{prova\_simples} é o nome do teorema, e \emph{Proof/Qed} delimita a
prova. A ferramenta disponibliza uma maneira de visualizar os estados da prova
durante o processo. Assim, quando terminamos o comando \texttt{intros
prova\_de\_A}, temos o seguinte estado:

\begin{lstlisting}[basicstyle=\small]
   A : Prop
   prova_de_A : A
   ============================
   A
\end{lstlisting}

Todos os elementos acima da barra horizontal são nossas \emph{hipóteses}, e
chamamos este conjunto de \emph{contexto}. Abaixo da barra está nosso objetivo
atual. Esta prova foi concluída com a tática \texttt{exact prova\_de\_A}, para
dizer que a prova é exatamente o elemento \texttt{prova\_de\_A}. 

Usando o comando \texttt{Print prova\_simples}, podemos ver o termo de prova
gerado pelas táticas.

\begin{lstlisting}[basicstyle=\small]
prova_simples = 
    fun (A : Prop) (prova_de_A : A) => prova_de_A
         : forall A : Prop, A -> A

\end{lstlisting}

Para um melhor entendimento do funcionamento da ferramenta, da linguagem e
teoria envolvidos, ver \cite{pierce}.


\section{O cálculo lambda}

\subsection{Visão geral}

No início do século XX, tornou-se necessária, na matemática, uma definição 
precisa para a noção intuitiva do processo computacional. Diversos modelos foram
propostos para resolver este problema. Entre eles está o cálculo lambda,
proposto por Alonzo Church \cite{lambda_first}, em 1936. Inicialmente, o cálculo
fazia parte de um sistema maior, proposto para servir como uma base
formal para o estudo das fundações da matemática. Porém, devido a
inconsistências neste sistema, Alonzo Church foi obrigado a abrir mão de seu
objetivo inicial, separou a parte utilizável do sistema e formou o que hoje
conhecemos como cálculo lambda.

O grande diferencial deste cálculo está em sua expressividade, com poder
computacional equivalente ao da Máquina de Turing, e sua simplicidade,
demonstrada por sua gramática concisa e poucas regras. A ideia central deste
consiste em simular a criação e aplicação de funções. Diferentemente da noção
usual no trabalho matemático, as funções neste sistema são chamadas "anônimas",
pois são definidas tendo em vista somente seus argumentos e o resultado. Como
exemplo, uma função simples como: "$double(x) = 2*x$" \ é definida anonimamente como
"$\lambda x.\ 2*x$". É utilizada também uma notação especial para a aplicação de
funções, denotada como "$ (\lambda x.\ 2*x)$  $3$".

É importante notar que os exemplos acima não são representados exatamente desta
maneira. Como dito anteriormente, o sistema possui uma gramática simples, e
todas as noções, inclusive números e operações, devem ser definidas com base em
abstrações e aplicações. A gramática do cálculo lambda pode ser descrita
sucintamente como:

\[ \tau := x\ |\ \lambda y.\tau\ |\ \tau \tau \]

Onde $\tau$ representa um termo, e \textbf{x} representa uma variável livre. A
variável \textbf{y} no segundo caso é chamada de \textit{variável ligada},
pois suas ocorrências em $\tau$ estão associadas à abstração. 
O conjunto de variáveis livres de \emph{t} é denotado por \emph{fv(t)}. O processo
computacional é simulado no sistema através da regra de $\beta$-redução,
definida como:

\[ (\lambda x.t)\ u \rightarrow_{\beta} t\{x/u\} \]



\begin{table}[h]
\begin{empheq}[box=\fbox]{align*}
    fv(x)\ & \equiv \{x\} \\
    fv(t\ u)\ & \equiv\ fv(t)\ U\ fv(u) \\
    fv(\lambda x. t)\ & \equiv\ fv(t)/ \{x\}
\end{empheq}
    \caption{Definição da função fv}
    \label{table:fv}
\end{table}

Note que \emph{t\{x/u\}} é uma \textit{meta-operação}, definida pela substituição das
ocorrências da variável \textbf{x}, no termo \textbf{t}, pelo termo \textbf{u}.
Abaixo, alguns exemplos de $\lambda$-termos.

\begin{itemize}
    \item A função identidade pode ser representada pelo termo $ (\lambda x. x) $.
        É fácil ver a correspondência na seguinte redução: $ (\lambda x.x) u
    \rightarrow_\beta x \{x/u\} \rightarrow u $. 
    \item A função constante pode ser representada pelo termo $ (\lambda x. M) $,
        onde M é um termo qualquer, tal que $x \notin fv(M)$.
        É fácil ver a correspondência na seguinte redução: $ (\lambda x.M) u
    \rightarrow_\beta M \{x/u\} \rightarrow M $. 
    \item Por último, podemos representar uma função que recebe dois termos e
        retorna o primeiro, como $ (\lambda x. \lambda y. x)$. Sua aplicação é
        reduzida da seguinte maneira: $ ((\lambda x. \lambda y. x)\ M)\ N)
    \rightarrow_\beta ((\lambda y. x) \{x/M\} N) \rightarrow (\lambda y. M) N
    \rightarrow_\beta M \{y/N\} \rightarrow N$, com \textbf{x,y} não ocorrendo 
    livres em M ou N.
\end{itemize}



A partir destas definições, várias propriedades sobre o sistema podem ser
estudadas. Entre elas, é importante ressaltar as noções de \textit{forma
 normal} e de \textit{confluência}:

\begin{description}
    \item[Forma normal:] Um termo \textbf{t} é dito estar em forma
    normal quando não existe \textbf{t'} tal que $ t \rightarrow_\beta t' $. 
    É possível demonstrar que nem todo termo que pode ser expressado no sistema
    possui uma forma normal. Como exemplo, observe que $ (\lambda x.x\ x)\ (\lambda
    x.x\ x) \rightarrow_\beta (\lambda x.x\ x)\ (\lambda x.x\ x) $. Um termo é
    dito normalizável quando existe uma estratégia de redução que leva a uma
    forma normal. O termo é \emph{fortemente normalizável} se toda estratégia
    leva à forma normal. A noção de normalização é especialmente importante,
    pois indica se um termo pode ou não terminar quando for avaliado, o que é de
    grande interesse no estudo computacional.

    \item[Confluência:] Um sistema de reescrita, tomando como exemplo o cálculo
    lambda, é dito confluente se, para todo termo \textbf{t}, se $ t
    \rightarrow_\beta t' $ e $ t \rightarrow_\beta t'' $, então deve existir
    um termo \textbf{u} tal que \textbf{t'} e \textbf{t''} reduzem para ele,
    em zero ou mais passos da $\beta$-redução.  Os termos \textbf{t'} e
    \textbf{t''} são ditos \textit{$\beta$-equivalentes}.  A propriedade de
    confluência pode ser entendida, essencialmente, como uma garantia que a
    ordem que as reduções são feitas dentro de um termo não afetam o
    resultado final do processo. Ou seja, a confluência garante o determinismo
    do processo computacional.
\end{description}

Outro ponto importante a ser mencionado é a noção de $\alpha$-equivalência de
termos. Um termo $(\lambda x. t)$ é dito $\alpha$-equivalente a $(\lambda y. u)$
se $ t\{x/y\} = u $. Esta noção captura a ideia de que a escolha do nome das
variáveis ligadas não importa em geral, sendo o real objeto de interesse a
estrutura do termo. Esta definição é útil para evitar certos problemas, como por
exemplo o de \textit{captura de variáveis livres}. No exemplo:

\[ ((\lambda x.\ x\ y)\ u)\{y/x\},\ x\ \neq\ y,\ y  \notin fv(u) \]

Observe que, reduzindo a aplicação antes de aplicar a substituição, temos:


\[ ((\lambda x.\ x\ y)\ u)\{y/x\} \rightarrow_\beta ((x\ y)\{x/u\}\{y/x\})
\rightarrow (u\ y)\{y/x\} \rightarrow (u\ x) \]

Mas, fazendo a substituição imediatamente, temos:

\[ ((\lambda x.\ x\ y)\ u)\{y/x\} \rightarrow ((\lambda x.\ x\ x) u)
\rightarrow_\beta (x\ x)\{x/u\} \rightarrow (u\ u) \]

Veja que isto muda completamente o funcionamento do termo. A variável \textbf{x}
na substituição não é a mesma da que está ligada na abstração! Isto pode causar
vários problemas inesperados, como, por exemplo, a perda da normalização do
termo. Se o termo \textbf{u} for da forma $(\lambda z. z z)$, ao realizar a
substituição imediatamente, obtemos o termo $(\lambda z. z z)\ (\lambda z. z z)$,
que, como dito anteriormente, não termina.
Para evitar este problema, podemos renomear a variável ligada \textbf{x}, antes
da substituição, por uma variável nova, de maneira a obter um termo
$\alpha$-equivalente e podendo então realizar a substituição sem modificar a
estrutura do termo.

\subsection{Representação de $\lambda$-termos}
\label{sub:int_lnr}

Como dito anteriormente, o cálculo lambda é usado como modelo teórico para
linguagens funcionais. A noção de $\alpha$-equivalência, apesar de útil, pode ser
muito custosa em uma implementação prática. Por este motivo, foram propostas
algumas representações diferentes de termos, de modo a evitar a necessidade do
renomeamento de variáveis. 

Uma das primeiras, e mais importantes, tentativas de resolver o problema da
$\alpha$-conversão é a notação utilizando indices de De Bruijn. Nela, são
utilizados números naturais para representar as variáveis. Cada número
representa a quantidade de abstrações no escopo da ocorrência. Números que
ultrapassam esta quantidade representam variáveis livres. A gramática pode ser
definida como:

\[ \tau := n\ |\ \lambda . \tau\ |\ \tau \tau \]

Onde \textit{n} representa um número natural, a partir do 1. Exemplos de termos
nesta notação incluem a identidade $(\lambda. 1)$, a função constante $(\lambda.
u)$, com \textit{u} não contendo 1 como índice livre, e um termo com índice
livre, como $((\lambda.2)\ u)$. Apesar de sua aparente praticidade de
implementação, esta notação se afasta muito da utilização do cálculo no papel.
Além disto, introduz a necessidade de se manter um contexto externo para
registrar as variáveis livres, junto com uma álgebra para lidar com tal
contexto.

Como exemplo das complexidades introduzidas por esta notação, podemos observar a
representação do termo $(\lambda x.\ x\ a)\ b$, onde \textit{a, b} são
variáveis livres. Este termo é representado na notação de de Bruijn como:

\[ (\lambda.\ 1\ 2)\ 3 \]

Junto com o seguinte contexto $[a,\ b]$, que representa as variáveis livres.
Utilizando esse contexto, vemos que o índice 1 representa a variável \textit{x},
pois temos apenas uma abstração. Após isso, os índices 2 e 3 representam as
variáveis livres, e estão relacionadas em ordem com as variáveis no contexto.
Assim, $(2\ \rightarrow\ a)$ e $(3\ \rightarrow\ b)$. 

Após realizar a $\beta$-redução, o termo original é representado por $(b\ a)$.
Porém, o termo na notação da representação de de Bruijn não é representado por
$(3\ 2)$, mas sim por $(2\ 1)$. Isto porque os novos termos não possuem
abstratores, logo o primeiro índice para representar variáveis livres não é
\textit{2}, mas \textit{1}.

Uma solução para estes problemas é usar a \textit{locally nameless
representation}, usada para representar os $\lambda$-termos neste trabalho.
A gramática do sistema é modificada parcialmente:

\[ \tau := x\ |\ n\ |\ \lambda . \tau\ |\ \tau \tau \]

Observe que a abstração também não carrega uma variável ligada. Ao invés disso,
são usados índices (que podem ser números naturais, como em de Bruijn) para
representar estas variáveis. As variáveis livres, denotadas na gramática por
\textit{x}, devem pertencer a um conjunto infinito e disjunto dos índices que
representa as variáveis ligadas. A meta-substituição deve ser adaptada para
levar em conta estes índices, como no exemplo:

\[ (\lambda.\ 2) \{1/t\} \rightarrow ( \lambda. (2 \{2/t\}) ) \]

Note que o índice a ser substituído foi incrementado quando a substituição entra
na abstração, de modo a corresponder à variável correta a ser substituída. 
A operação de substituição pode ser melhor definida como:

\begin{empheq}[box=\fbox]{align*}
    \{k \rightarrow x\} i\ \ \ \  & \equiv\ \ x,\ se\ (i = k);\ i,\ c.c\\
    \{k \rightarrow x\} y\ \ \ \  & \equiv\ y\\
    \{k \rightarrow x\} (t\ u)\ \ \ \  & \equiv\ (\{k \rightarrow x\}t\ \{k
    \rightarrow x\}u)\\
    \{k \rightarrow x\} (\lambda . t) \ \ \ \  & \equiv\ 
    (\lambda . \{k+1 \rightarrow x\}t)
\end{empheq}



Esta representação foi detalhada por Arthur Charguéraud em \cite{chargueraud},
junto com provas de seu bom funcionamento e um framework para sua utilização em
Coq.  Entraremos agora em alguns detalhes do uso desta notação, descritos em
\cite{chargueraud}, já que ela tem um interesse especial neste trabalho.

Na notação usual, quando queremos estudar o corpo de uma abstração $(\lambda x.
t)$, podemos trabalhar diretamente com o termo \textit{t}. Porém, nesta nova
representação, a abstração tem a forma $(\lambda . t)$, e é necessário que seja
fornecida uma variável \textit{x} para se trabalhar com o corpo. Esta operação é
chamada \emph{abrir o termo} \textit{t} com \textit{x}, e será representada por
$t^{x}$. Mais precisamente, a abertura do termo $(\lambda . t)$ cria uma cópia
de \emph{t} onde todas as ocorrências do índice ligado à abstração mais externa
são trocados pela variável \emph{x}. Como exemplo, abrir a abstração $(\lambda.
(0\ y))$ com \emph{x} nos dá o termo $(x\ y)$. A operação de abrir o termo deve
ajustar o índice a ser mudado à medida que entra no termo. Assim, podemos usar a
já definida operação de substituição, diretamente com a variável \emph{x}, para
realizar a abertura. Temos, então, $t^{x} \equiv t\{0/x\}$.

Similarmente, podemos querer abstrair todas as ocorrências de \emph{x} no termo
\emph{t}, construindo então o termo $(\lambda x. t)$. Com a nova notação, é
necessário definir uma operação que substitui todas as ocorrências de \emph{x}
pelo índice 0, antes de adicionar a abstração. Esse processo é chamado
\emph{fechar o termo}, representado por $ ^{\textbackslash x}t$. Assim, para
construir a abstração de maneira equivalente, fazemos $(\lambda .
^{\textbackslash x}t)$. Podemos definir a operação de fechamento como $
^{\textbackslash x}t \equiv \{0 \leftarrow t\}$, onde esta nova substituição é
definida como:

\begin{empheq}[box=\fbox]{align*}
    \{k \leftarrow x\} i\ \ \ \  & \equiv\ i\\
    \{k \leftarrow x\} y\ \ \ \  & \equiv\ \ 0,\ se\ (x = y);\ y,\ c.c.\\
    \{k \leftarrow x\} (t\ u)\ \ \ \  & \equiv\ (\{k \leftarrow x\}t\ \{k
    \leftarrow x\}u)\\
    \{k \leftarrow x\} (\lambda . t) \ \ \ \  & \equiv\ 
    (\lambda . \{k+1 \leftarrow x\}t)
\end{empheq}

Como dito anteriormente, esta representação possui termos que contém
\emph{índices livres}. Tais termos não possuem correspondentes no sistema
original, pois os índices não estão ligados a nenhuma abstração e não
representam variáveis livres. Assim, precisamos tomar cuidado para evitar
trabalhar com termos com tais índices, já que estes não são nosso objeto de
interesse. Para isto, chamaremos um termo sem índices livres de \emph{localmente
fechado}.

Existem duas maneiras de conferir se um termo é localmente fechado. A primeira
consiste em andar pela estrutura do termo, abrindo cada abstração com uma nova
variável. Desta maneira, se o termo for de fato fechado, nunca encontraremos um
índice. A segunda abordagem consiste em analizar diretamente os índices do termo,
checando, para cada um deles, se o seu valor é menor ou igual ao número de
abstrações que o rodeiam.

A primeira opção dá lugar a uma definição natural de um predicado, denotado por
\emph{lc\ t} indicando que o termo é localmente fechado. Com apenas três regras
de inferência, podemos facilmente fazer uma análise mais formal da propriedade
de ser localmente fechado, sendo bastante útil em provas.

\begin{mathpar} 
    \inferrule*[Right=lc\_fvar]{  }
    {lc(x)}
    \and
    \inferrule*[Right=lc\_app]{lc\ t1 \\ lc\ t2}
    {lc(t1\ t2)}
    \and
    \inferrule*[Right=lc\_abs]{\forall x \notin L,\ lc\ (t^{x})}
    {lc(\lambda. t)}
\end{mathpar}

A premissa $\forall x \notin L$ no caso da abstração captura a ideia de \emph{x}
ser uma variável nova, pois podemos tomar o conjunto L, sempre finito, como
sendo o conjunto de variáveis já usadas e, assim, sempre ter uma escolha nova de
\emph{x}.

A segunda abordagem tem um caráter mais naturalmente computacional, dando lugar
a uma função recursiva para conferir se um termo é localmente fechado. Tal
função vai navegando pelo termo, entrando em seus subtermos, guardando um
contador. Quando entrar em uma abstração, a função incrementa tal contador. Ao
encontrar um índice, basta conferir se ele é menor que o contador. Assim, a
função pode ser definida como a seguir.

\begin{table}[h]
\begin{empheq}[box=\fbox]{align*}
    lc\_at\ k\ (i)\ \ \ &\equiv\ i<k \\ 
    lc\_at\ k\ (x)\ \ \ &\equiv\ True \\ 
    lc\_at\ k\ (t1\ t2)\ \ \ &\equiv\ (lc\_at\ k\ t1)\ \&\ (lc\_at\ k\ t2) \\ 
    lc\_at\ k\ (\lambda. t1)\ \ \ &\equiv\ (lc\_at\ k + 1\ t1)
\end{empheq}
    \caption{Definição da função lc\_at}
    \label{table:lc_at}
\end{table}

Dizemos que o termo \emph{t} é localmente fechado se a função \emph{(lc\_at 0
t)} retorna \textbf{True}.  Não é difícil provar a equivalência de ambas as
definições, ou seja, que vale $lc\ t\ \iff\ (lc\_at\ 0\ t)\ =\ True$.

Ambas as implicações possuem uma demonstração simples. Para o caso em que $lc\ t
\rightarrow (lc\_at\ 0\ t)\ =\ True$, podemos fazer uma indução no predicado
\emph{lc}. O único subcaso que não sai imediatamente é o da abstração, que pode
ser facilmente resolvido se observarmos que um termo $t^{x}$ é fechado a um
nível k se, e somente se, \emph{t} é fechado a nível k + 1.

A volta, ou seja $lc\ t \leftarrow (lc\_at\ 0\ t)\ =\ True$, se faz por uma
indução na estrutura do termo \emph{t}. Novamente, o caso da abstração merece um
cuidado especial, mas ainda sai de maneira simples, escolhendo o conjunto L como
sendo exatamente o conjunto de variáveis livres de \emph{t}.

Vale a pena ressaltar duas outras equivalências, referentes às noções de
abertura e fechamento de termos. Temos:

\begin{empheq}[box=\fbox]{align*}
    ^{\setminus x}(t^{x})\ &=\ t,\ se\ x\ \notin\ fv(t) \\
    (^{\setminus x}t)^{x}\ &=\ t,\ se\ vale\ (lc\ t)
\end{empheq}


Estas definições serão extremamente importantes ao decorrer do trabalho e podem
ser citadas com frequência.

Para mais informações a respeito do cálculo lambda, veja \cite{barendregt}.

%%%%%%%%%%%%%%%%%%%%%%%%%%%%%%%%%%%%%%%%%%%%%%%%%%%%%%%%%%%%%%%%%%%%%%%%%%%%%%%%
%%%%%%%%%%%%%%%%%%%%%%%%%%%%%%%%%%%%%%%%%%%%%%%%%%%%%%%%%%%%%%%%%%%%%%%%%%%%%%%%
%%%%%%%%%%%%%%%%%%%%%%%%%%%%%%%%%%%%%%%%%%%%%%%%%%%%%%%%%%%%%%%%%%%%%%%%%%%%%%%%
\



\section{Substituições explícitas}

\subsection{Motivação e histórico}

Sabemos que a ordem em que as reduções são feitas não altera a forma normal do
termo, mas isto não significa que não existem vantagens em se adotar certas
estratégias na normalização. Como exemplo, observe os seguintes casos:

\[ ((\lambda.\ x) t) \{0/y\} \rightarrow ((\lambda.\ x\{1/y\})\ t\{0/y\}) 
    \rightarrow ((\lambda.\ x) t') \rightarrow (x\{0/t\}) \rightarrow x
\]

\[ ((\lambda.\ x) t) \{0/y\} \rightarrow (x\{0/t\}) \{0/y\} 
    \rightarrow x \{0/y\} \rightarrow x \]

Veja que, na segunda abordagem, evitamos uma operação potencialmente custosa de
realizar uma substituição em um termo \emph{t} que seria imediatamente
descartado. Infelizmente, o uso da meta-operação nos impede de realizar este
tipo de manipulação, pois a substituição não faz parte da sintaxe do sistema.

Na implementação de linguagens de programação, a substituição muitas vezes é
"atrasada", de modo a evitar computações desnecessárias. Para aproximar o modelo
teórico de seu correspondente prático, podemos tentar fracionar a operação de
substituição em partes atômicas, de maneira a permitir uma manipulação simbólica
mais precisa. 

Por este motivo, várias tentativas de se formalizar a noção de substituição,
dando então espaço para o formalismo conhecido como \textit{substituição
explícita}. A princípio, podemos tentar extender a gramática de termos da
seguinte maneira:

\[ \tau := x\ |\ \lambda x.\tau\ |\ \tau \tau\ |\ \tau[x/\tau]\ \]

Podemos então espelhar o funcionamento da meta-substituição através de regras de
redução no novo cálculo, gerando o sistema conhecido como $\lambda$x.

\begin{table}[h]
\begin{empheq}[box=\fbox]{align*}
    (\lambda x.\ t)\ u\ \ \ &\rightarrow\ t[x/u] \\
    x[x/u]\ \ \             &\rightarrow\ u \\
    y[x/u]\ \ \             &\rightarrow\ y,\ se\ x\ \neq\ y \\
    (t\ u)[x/v]\ \ \        &\rightarrow\ t[x/v]\ u[x/v] \\
    (\lambda y.\ u)[x/v]\ \ &\rightarrow\ (\lambda y.\ u[x/v])
\end{empheq}
    \caption{Regras do sistema $\lambda$x}
\end{table}

O sistema $\lambda$x corresponde ao comportamento mínimo que se espera de um
cálculo com substituições explícitas. Porém, existem outras propriedades
interessantes que podem ser adicionadas ao sistema e, em consequência disto,
vários outros modelos foram propostos, como o $\lambda_\sigma$, $\lambda_{ws}$,
$\lambda$lxr, entre outros. 

Um problema que pode acontecer em sistemas com substituição explícita é a perda
da preservação da normalização forte (\textbf{PSN}), que pode ser definida como:

\begin{description}
    \item[PSN:] Se $\lambda$z é um cálculo baseado no cálculo $\lambda$, deve
        valer a seguinte afirmação. Se \emph{t} é um $\lambda$-termo fortemente
        normalizável, então seu correspondente em $\lambda$z também o é neste
        novo sistema.
\end{description}

Este tipo de problema é especialmente comum em cálculos de substituições
explícitas que possuem a propriedade de composição de substituições.
Essencialmente, dado um termo $t[x/u][y/v]$, podemos compor as duas
substituições, de maneira a reduzir a segunda substituição antes da primeira.
Como resultado, teríamos o termo $t[y/v][x/u[y/v]]$. 

Várias estratégias são usadas para se tentar garantir a propriedade \textbf{PSN}
do sistema, como utilização de marcas em termos, restrição de composições ou
reduções dentro de substituições explícitas, definições de classes de
equivalências, entre outros. Para uma visão geral do histórico de cálculos de
substituições explícitas, veja \cite{es_overview}.

A seguir, veremos um sistema que propõe uma maneira de se compor tais
substituições sem perder a propriedade \textbf{PSN}. Este sistema será o foco
deste trabalho, e utilizaremos o assistente de provas Coq para formalizar certas
propriedades essenciais deste.



%%%%%%%%%%%%%%%%%%%%%%%%%%%%%%%%%%%%%%%%%%%%%%%%%%%%%%%%%%%%%%%%%%%%%%%%%%%%%%%%
%%%%%%%%%%%%%%%%%%%%%%%%%%%%%%%%%%%%%%%%%%%%%%%%%%%%%%%%%%%%%%%%%%%%%%%%%%%%%%%%
%%%%%%%%%%%%%%%%%%%%%%%%%%%%%%%%%%%%%%%%%%%%%%%%%%%%%%%%%%%%%%%%%%%%%%%%%%%%%%%%


  \chapter{O sistema $\lambda ex$}

\section{Visão geral}
\label{sec:int_lex}

%\subsection{Descrição do sistema}

Como visto no capítulo anterior, várias extensões do cálculo $\lambda$ foram
propostas com o objetivo de obter um sistema fiel às propriedades deste e onde a
operação de substituição fosse um elemento primitivo da
linguagem.

O sistema proposto em \cite{delia}, chamado $\lambda$ex, é o primeiro sistema
que captura de maneira simples tal noção, enquanto ainda possui a propriedade
\textbf{PSN}, ou seja, a preservação da normalização forte. Este sistema será
repetido aqui, e um maior entendimento pode ser obtido na fonte original. São
introduzidas várias mudanças, a começar pela gramática: 

\[ \tau := x\ |\ \lambda x.\tau\ |\ \tau \tau\ |\ \tau[x/\tau]\ \]
\

A nova construção é chamada \textit{substituição explícita}, e é o que permite as
manipulações sintáticas com substituições no cálculo. Precisamos então redefinir
o conjunto de variáveis livres do novo cálculo.

\begin{definicao}
    Definimos o conjunto de variáveis livres de $t$, denotado por $fv(t)$,
    indutivamente. Na definição abaixo, $t,u$ denotam termos e $x$ denota uma
    variável.
\begin{empheq}{align*}
    fv(x)\ & = \{x\} \\
    fv(t'\ u)\ & =\ fv(t')\ \cup\ fv(u) \\
    fv(\lambda x. t')\ & =\ fv(t')\setminus \{x\} \\
    fv(t'[x/u])\ & =\ (fv(t')\setminus \{x\}) \cup fv(u) 
\end{empheq}
\end{definicao}

\begin{table}[h]
    
\begin{empheq}[box=\fbox]{align*}
    x[x/u]\ \ \             &\rightarrow_{Var}\ u \\
    t[x/u]\ \ \             &\rightarrow_{Gc}\ t    & se\ \emph{x} \notin fv(t)\\
    (t\ u)[x/v]\ \ \        &\rightarrow_{App}\ t[x/v]\ u[x/v] \\
    (\lambda y.\ u)[x/v]\ \ &\rightarrow_{Lamb}\ (\lambda y.\ u[x/v])\\
    t[x/u][y/v]\ \ \        &\rightarrow_{Comp}\ t[y/v][x/u[y/v]] & se\ y\ \in
    fv(u) \\ 
    (\lambda x.\ t)\ u\ \ \ &\rightarrow_B\ t[x/u]
\end{empheq}
    \caption{Regras de redução}
    \label{table:red_lambex}

\end{table}

Como consequência desta
mudança, novas regras de redução são definidas, como mostrado na Tabela
\ref{table:red_lambex}.  As 5 primeiras regras formam a relação $\rightarrow_x$,
e o acréscimo da última forma a relação $\rightarrow_{Bx}$. 

Observe que, devido ao acréscimo das substituições explícitas, é possível
definir termos que possuem uma sintaxe distinta, diferindo apenas na permutação
de substituições independentes, e que constituirem o mesmo
$\lambda$-termo no sistema original. Um exemplo simples pode ser visto a seguir.

\pagebreak
Sejam $t, u, v$ termos e $x, y$ variáveis tais que $y \notin fv(u)$ e $x \notin
fv(v)$. Então temos:

\begin{displaymath}
        \xymatrix{ & ((\lambda x.\ t)\ u)[y/v] \ar[dl]_{B} \ar@{->>}[dr]_x &     \\
                  t[x/u][y/v] &               & (\lambda x.\ t[y/v])\ u \ar[d]_B \\
                              &               &  t[y/v][x/u]                    }
\end{displaymath}

Temos então dois termos sintaticamente distintos, $t[x/u][y/v]$ e $t[y/v][x/u]$,
obtidos a partir do mesmo $\lambda$-termo. Perdemos, assim, a propriedade de
confluência do cálculo.

Para resolver este problema, é necessário tornar o novo sistema equacional,
adicionando uma relação de equivalência:

\[ t[x/u][y/v] =_C t[y/v][x/u] \ \ \ \ \ se\ y \notin fv(u)\ \&\ x \notin fv(v)\] 

A relação de equivalência $=_e$ é formada com a junção de $=_C$ e
$\alpha$-equivalência. Devido à mudança na gramática, precisamos estender a
definição de $\alpha$-equivalência para lidar com substituições explícitas.


\begin{definicao}[$\alpha$-equivalência]
    Um termo $(\lambda x. t)$ é dito $\alpha$-equivalente a\ $(\lambda y. u)$ se
    $ t\{x/y\} = u $.  Um termo $t[x/u]$ é dito $\alpha$-equivalente a\
    $t'[y/u]$ se $ t\{x/y\} = t' $. Mais geralmente, dois termos são ditos
    $\alpha$-equivalentes se um pode ser obtido a partir do outro através de
    renomeamento de variáveis ligadas.
    É fácil ver intuitivamente que esta definição é uma relação de equivalência:
    para a reflexividade, basta fazer um renomeamento trivial. Para a
    transitivadade, basta compor os renomeamentos.  Para a simetria, basta fazer
    o renomeamento contrário, ou seja: $ u\{y/x\} = t $

\end{definicao}



As relações $\rightarrow_{ex}$ e $\rightarrow_{\lambda
    ex}$ são definidas como:

\[t\ \rightarrow_{ex}\ t'\ \iff\ \exists\ s,s'\ t.q.\ t\ =_{e}\ s\
    \rightarrow_x\ s'\ =_e\ t' \]
\[t\ \rightarrow_{\lambda ex}\ t'\ \iff\ \exists\ s,s'\ t.q.\ t\ =_{e}\ s\
    \rightarrow_{Bx}\ s'\ =_e\ t' \]

Estas relações serão amplamente utilizadas ao decorrer deste trabalho, cujo foco
principal é contribuir com o andamento da formalização da propriedade
\textbf{PSN} do sistema.  A ideia da prova é definir uma estratégia de redução
para este sistema, e utilizar esta estratégia para demonstrar que o conjunto dos
termos fortemente normalizáveis do cálculo $\lambda$ está contido no conjunto de
termos fortemente normalizáveis do cálculo $\lambda ex$, garantindo a
propriedade \textbf{PSN}.

Essencial para a prova da propriedade \textbf{PSN} é que a estratégia seja
perpétua, definição que será apresentada posteriormente. Para isto, é necessário
demonstrar a propriedade \textbf{IE} do sistema.

\begin{definicao}[Propriedade IE]
    Sejam $t,\ u$ termos. Seja $SN_{\lambda ex}$ o conjunto de termos fortemente
    normalizáveis do sistema $\lambda ex$. Se $u \in SN_{\lambda ex}$ e
    $t\{x/u\} \in SN_{\lambda ex}$, então $t[x/u] \in SN_{\lambda ex}$.
\end{definicao}

Esta pode ser entendida intuitivamente da seguinte maneira: Dados $t$, $u$ tais
que $u$ e $t\{x/u\}$ são fortemente normalizáveis no sistema, então o
correspondente utilizando substituições explícitas, $t[x/u]$ também será. Ou
seja, a normalização da substituição implícita implica a normalização da
explícita.

Para realizar a prova da propriedade \textbf{IE}, adicionamos um novo formalismo
na gramática, \emph{marcando} algumas substituições explícitas que sabemos que
não introduzem problemas de normalização. Dividiremos, então, a regra principal
deste sistema estendido, a relação $\lambda \underline{ex}$, em duas novas
relações complementares: $\lambda \underline{ex}^i$ e $\lambda
\underline{ex}^e$, chamadas reduções interna e externa, respectivamente. Estas
reduções serão usadas para ter um melhor controle das reduções efetuadas dentro
de substituições marcadas.

Com estes conceitos em mente, podemos definir de maneira mais clara a
contribuição deste trabalho: continuar a formalização iniciada em \cite{initial},
adicionando o formalismo de substituições marcadas, demonstrando que este
preserva as propriedades já provadas para as substituições explícitas e, por
fim, definir as reduções interna e externa acima mencionadas. Ou seja, o
objetivo final é formalizar a equivalência $\lambda \underline{ex} = \lambda
\underline{ex}^i \cup \lambda \underline{ex}^e$, chave para a prova da
propriedade \textbf{IE}, permitindo, então, a conclusão da formalização da
propriedade \textbf{PSN} do sistema $\lambda ex$.

\subsection{A Gramática de pré-termos}
\label{sec:termos}

Nesta seção, será feita um paralelo entre as estruturas básicas do sistema
$\lambda ex$ e sua formalização em Coq.

Como visto na Seção \ref{sec:int_lex}, o sistema possui uma gramática que pode
ser vista como uma extensão do cálculo $\lambda$ original, consistindo de
variáveis, abstrações, aplicações e \emph{substituições explícitas}. Ela pode
ser descrita sucintamente como abaixo.

\[ \tau := x\ |\ \lambda x.\tau\ |\ \tau \tau\ |\ \tau[x/\tau]\ \]

Já encontramos então nossa primeira divergência na formalização. Neste projeto,
será usada a representação chamada \textit{Locally Nameless Representation}
(LNR), introduzida na subseção \ref{sub:int_lnr}. Como visto anteriormente,
existem termos nesta representação que não possuem correspondentes no sistema
original, por conta da possibilidade de existirem índices que não estão ligados
a nenhuma abstração.

Assim, torna-se novamente necessária uma gramática de \emph{pré-termos}, que
consiste de todos as expressões possíveis de serem escritos em LNR. Esta
gramática é análoga à gramática LNR apresentada em  \ref{sub:int_lnr}, porém
agora adicionamos também o construtor de substituições explícitas.

\[ \tau := x\ |\ n\ |\ \lambda \tau\ |\ \tau \tau\ |\ \tau[\tau]\ \]

Esta gramática é formalizada usando um tipo indutivo.

\bigskip
\coqdocnoindent \coqdockw{Inductive} \coqdef{LambdaES
    Defs.pterm}{pterm}{\coqdocinductive{pterm}} : \coqdockw{Set} :=\coqdoceol
\coqdocindent{1.00em} \ensuremath{|} \coqdef{LambdaES Defs.pterm
    bvar}{pterm\_bvar}{\coqdocconstructor{pterm\_bvar}} :
\coqexternalref{nat}{http://coq.inria.fr/distrib/8.4pl4/stdlib/Coq.Init.Datatypes}{\coqdocinductive{nat}}
\ensuremath{\rightarrow} \coqref{LambdaES
    Defs.pterm}{\coqdocinductive{pterm}}\coqdoceol \coqdocindent{1.00em}
\ensuremath{|} \coqdef{LambdaES Defs.pterm
    fvar}{pterm\_fvar}{\coqdocconstructor{pterm\_fvar}} : \coqdocdefinition{var}
\ensuremath{\rightarrow} \coqref{LambdaES
    Defs.pterm}{\coqdocinductive{pterm}}\coqdoceol \coqdocindent{1.00em}
\ensuremath{|} \coqdef{LambdaES Defs.pterm
    app}{pterm\_app}{\coqdocconstructor{pterm\_app}}  : \coqref{LambdaES
    Defs.pterm}{\coqdocinductive{pterm}} \ensuremath{\rightarrow}
\coqref{LambdaES Defs.pterm}{\coqdocinductive{pterm}} \ensuremath{\rightarrow}
\coqref{LambdaES Defs.pterm}{\coqdocinductive{pterm}}\coqdoceol
\coqdocindent{1.00em} \ensuremath{|} \coqdef{LambdaES Defs.pterm
    abs}{pterm\_abs}{\coqdocconstructor{pterm\_abs}}  : \coqref{LambdaES
    Defs.pterm}{\coqdocinductive{pterm}} \ensuremath{\rightarrow}
\coqref{LambdaES Defs.pterm}{\coqdocinductive{pterm}}\coqdoceol
\coqdocindent{1.00em} \ensuremath{|} \coqdef{LambdaES Defs.pterm
    sub}{pterm\_sub}{\coqdocconstructor{pterm\_sub}} : \coqref{LambdaES
    Defs.pterm}{\coqdocinductive{pterm}} \ensuremath{\rightarrow}
\coqref{LambdaES Defs.pterm}{\coqdocinductive{pterm}} \ensuremath{\rightarrow}
\coqref{LambdaES Defs.pterm}{\coqdocinductive{pterm}} \coqdoceol
%\coqdocindent{1.00em} \ensuremath{|} \coqdef{LambdaES Defs.pterm
    %lsub}{pterm\_lsub}{\coqdocconstructor{pterm\_lsub}} : \coqref{LambdaES
    %Defs.pterm}{\coqdocinductive{pterm}} \ensuremath{\rightarrow}
%\coqref{LambdaES Defs.pterm}{\coqdocinductive{pterm}} \ensuremath{\rightarrow}
%\coqref{LambdaES Defs.pterm}{\coqdocinductive{pterm}}.\coqdoceol 
\bigskip

Veja que variáveis livres e ligadas possuem construtores distintos. Para as
ligadas, o construtor correto é \texttt{pterm\_bvar}, que recebe um natural
representando um índice. As variáveis livres são construídas com
\texttt{pterm\_fvar}, recebendo um elemento do tipo \texttt{var}, definido no
framework de Charguéraud.
As aplicações, abstrações e substituições são representadas através do
construtores \texttt{pterm\_app}, \texttt{pterm\_abs} e \texttt{pterm\_sub},
respectivamente. 

Devemos também estender as operações de substituição e fechamento de pré-termos
para se adequar à nova gramática, adicionando os seguintes casos:

\[    \{k \rightarrow x\} (t[u]) \ \ \ \  & =\ \{k+1 \rightarrow x\}t[\{k
    \rightarrow x\}u]  \]

\[    \{k \leftarrow x\} (t[u]) \ \ \ \  & =\ \{k+1 \leftarrow x\}t[\{k
    \leftarrow x\}u] \]

\section{Termos e Relações}
\subsection{Termos bem formados}

Devido à correspondência assimétrica entre os termos do sistema $\lambda ex$ e
em LNR, torna-se necessário redefinir os predicados de boa formação de termos.

Como pré-requisito para a redefinição destes predicados, precisamos implementar
as noções de abertura e fechamento de termos, como visto na subseção anterior.
Relembrando, a operação de abertura foi definida como $t^{x} = \{0 \rightarrow
x\}t$.

Para implementar tal operação, precisaremos de duas funções, \texttt{open\_rec}
e \texttt{open}.

\bigskip
\coqdockw{Fixpoint} \coqdef{LambdaES Defs.open
    rec}{open\_rec}{\coqdocdefinition{open\_rec}} (\coqdocvar{k} :
\coqexternalref{nat}{http://coq.inria.fr/distrib/8.4pl4/stdlib/Coq.Init.Datatypes}{\coqdocinductive{nat}})
(\coqdocvar{u} : \coqref{LambdaES Defs.pterm}{\coqdocinductive{pterm}})
(\coqdocvar{t} : \coqref{LambdaES Defs.pterm}{\coqdocinductive{pterm}})
\{\coqdockw{struct} \coqdocvar{t}\} : \coqref{LambdaES
    Defs.pterm}{\coqdocinductive{pterm}} := ...\coqdoceol

\coqdockw{Definition} \coqdef{LambdaES Defs.open}{open}{\coqdocdefinition{open}}
\coqdocvar{t} \coqdocvar{u} := \coqref{LambdaES Defs.open
    rec}{\coqdocdefinition{open\_rec}} 0 \coqdocvariable{u}
\coqdocvariable{t}.\coqdoceol

\coqdockw{Notation} " \{k {\raise.17ex\hbox{$\scriptstyle\mathtt{\sim}$}}{}> u\}  t" := (\coqref{LambdaES Defs.open rec}{\coqdocdefinition{open\_rec}} \coqdocvar{k} \coqdocvar{u} \coqdocvar{t}) (\coqdoctac{at} \coqdockw{level} 67).\coqdoceol
\coqdockw{Notation} "t \^{} x" := (\coqref{LambdaES Defs.open}{\coqdocdefinition{open}} \coqdocvar{t} (\coqref{LambdaES Defs.pterm fvar}{\coqdocconstructor{pterm\_fvar}} \coqdocvar{x})).\coqdoceol
\coqdocemptyline

\bigskip

A primeira adentra o termo \texttt{t} recursivamente, procurando pelo índice
\texttt{k} e substituindo pelo termo \texttt{u}. Ao encontrar uma abstração ou
substituição, o índice \texttt{k} é incrementado.  A segunda, que é a chamada da
operação de fato, apenas chama \texttt{open\_rec} com $k = 0$.

Similarmente, a operação de fechamento foi definida como $ ^{\textbackslash x}t
\equiv \{0 \leftarrow t\}$. Esta operação é definida através da função
\texttt{close}.

\bigskip

\coqdockw{Fixpoint} \coqdef{LambdaES Defs.close
    rec}{close\_rec}{\coqdocdefinition{close\_rec}}  (\coqdocvar{k} :
\coqexternalref{nat}{http://coq.inria.fr/distrib/8.4pl4/stdlib/Coq.Init.Datatypes}{\coqdocinductive{nat}})
(\coqdocvar{x} : \coqdocdefinition{var}) (\coqdocvar{t} : \coqref{LambdaES
    Defs.pterm}{\coqdocinductive{pterm}}) \{\coqdockw{struct} \coqdocvar{t}\} :
\coqref{LambdaES Defs.pterm}{\coqdocinductive{pterm}} := ...\coqdoceol

\coqdockw{Definition} \coqdef{LambdaES
    Defs.close}{close}{\coqdocdefinition{close}} \coqdocvar{t} \coqdocvar{x} :=
\coqref{LambdaES Defs.close rec}{\coqdocdefinition{close\_rec}} 0
\coqdocvariable{x} \coqdocvariable{t}.\coqdoceol

\bigskip

\begin{definicao}\label{def_term}
Sejam $t, u$ pré-termos; $L$ um conjunto finito e $x$ uma variável.
Definimos o predicado $term$, que caracteriza termos bem formados, com base nas
seguintes regras de inferência:

\begin{mathpar} 
    \inferrule*[Right=term\_var]{  }
    {term(x)}
    \and
    \inferrule*[Right=term\_app]{term\ t \\ term\ u}
    {term(t\ u)}
    \\
    \inferrule*[Right=term\_abs]{\forall x \notin L,\ term\ (t^{x})}
    {lc(\lambda t)}
    \and
    \inferrule*[Right=term\_sub]{\forall x \notin L,\ term\ (t^{x}) \\ term\ u}
    {term(t[u])}
\end{mathpar}
\end{definicao}

O predicado $term$ é o análogo do predicado $lc$ (Definição \ref{def_lc})
neste sistema. Ele recebe um pré-termo e indica que este elemento é bem
formado, ou seja, não possui índices livres. Para cada construtor de pré-termo,
temos um construtor diferente do predicado. Observe que não existe regra de
inferência para índices livres, como desejado.

Em alguns casos, como dito na subseção \ref{sub:int_lnr}, pode ser mais
vantajoso verificar se um termo está bem formado com uma função recursiva.
Tal verificação é feita através do predicado \texttt{term'}, que deve funcionar de
maneira equivalente ao predicado \texttt{term}. Este novo predicado é definido
com base na definição recursiva \texttt{lc\_at}, que verifica se o termo
\emph{t} está \textit{fechado} a um nível \emph{k}, ou seja, se não existe
índice livre de valor maior ou igual a \emph{k}. Sua implementação é feita com
base na Definição \ref{def_lc_at}, adicionando apenas o caso a seguir, para
substituições explícitas.

\[  lc\_at\ k\ (t1[t2])\ \ \ \equiv\ (lc\_at\ (S k)\ t1)\ \&\ (lc\_at\ k\ t2) \]

\bigskip
\coqdockw{Fixpoint} \coqdocvar{lc\_at} (\coqdocvar{k}:\coqdocvar{nat})
(\coqdocvar{t}:\coqdocvar{pterm}) \{\coqdockw{struct} \coqdocvar{t}\} :
\coqdockw{Prop} :=\coqdoceol

\coqdockw{Definition} \coqdocvar{term'} \coqdocvar{t} := \coqdocvar{lc\_at} 0 \coqdocvar{t}.\coqdoceol
\bigskip

Assim, essa equivalência também teve que ser formalizada, utilizando uma série
de lemas auxiliares. 

Primeiramente, foram provados dois resultados de enfraquecimento para o
\texttt{lc\_at}.

\bigskip
\coqnoindent \coqdockw{Lemma} \coqdocvar{lc\_rec\_open\_var\_rec} :
\coqdockw{\ensuremath{\forall}} \coqdocvar{x} \coqdocvar{t}
\coqdocvar{k}, \coqdocvar{lc\_at} \coqdocvar{k}
(\coqdocvar{open\_rec} \coqdocvar{k} \coqdocvar{x} \coqdocvar{t})
\ensuremath{\rightarrow} \coqdocvar{lc\_at} (\coqdocvar{S} \coqdocvar{k})
\coqdocvar{t}.\coqdoceol

\bigskip

Este lema garante que, se um termo \emph{t}, aberto com uma variável a um
nível \emph{k}, é fechado a este mesmo nível \emph{k}, então o termo \emph{t}
sem a abertura é fechado a nível \emph{k + 1}.

\bigskip

\coqnoindent \coqdockw{Lemma} \coqdocvar{lc\_at\_open\_var\_rec} :
\coqdockw{\ensuremath{\forall}} \coqdocvar{x} \coqdocvar{t}
\coqdocvar{k}, \coqdocvar{lc\_at} (\coqdocvar{S}
\coqdocvar{k}) \coqdocvar{t} \ensuremath{\rightarrow} \coqdocvar{lc\_at}
\coqdocvar{k} (\coqdocvar{open\_rec} \coqdocvar{k} (\coqdocvar{pterm\_fvar}
\coqdocvar{x}) \coqdocvar{t}).\coqdoceol
\bigskip

O segundo é exatamente o contrário: Se um termo \emph{t} é fechado a um nível
\emph{k + 1}, então este mesmo termo, aberto com uma variável a nível \emph{k},
é fechado a nível \emph{k}.

Já podemos agora provar a equivalência entre as duas definições de termos bem
formados. Este resultado é o análogo em nossa formalização do Teorema
\ref{teo:lc_lc_at}.

\bigskip
\coqdockw{Lemma} \coqdocvar{term\_eq\_term'} : \coqdockw{\ensuremath{\forall}}
\coqdocvar{t}, \coqdocvar{term} \coqdocvar{t} \ensuremath{\leftrightarrow}
\coqdocvar{term'} \coqdocvar{t}.\coqdoceol
\bigskip

\begin{proof}
    $ $\par\nobreak\ignorespaces
\begin{itemize}
    \item[($\rightarrow$)] Este caso é bem direto, fazendo uma indução no
        predicado $term\ t$. As hipóteses de indução resolvem os casos
        diretamente, exceto para abstração e substituição. Nesses, a hipótese é
        dada para $t^x$, quando precisamos provar $lc\_at\ 1\ t$. Felizmente os
        lemas de enfraquecimento acima resolvem, bastando usar o lema
        $lc\_rec\_open\_var\_rec$. Observando este lema, vemos que podemos
        concluir $lc\_at\ 1\ t$ se tivermos $lc\_at\ 0 \ (open\_rec\ 0\ x\ t)$.
        Ou seja, se tivermos $lc\_at\ 0\ t^x$, que é nossa hipótese de indução.

    \item[($\leftarrow$)] O segundo caso não é tão imediato, pois fazemos uma
        indução no \emph{tamanho do termo}, para auxiliar na prova. A hipótese
        de indução se refere a termos com tamanho igual ao do termo sobre o qual
        queremos provar a propriedade. Isto é útil pois o tamanho de um termo ao
        ser aberto com uma variável não muda. Novamente, no caso da abstração e
        substituição, precisamos fazer um ajuste utilizando o lema
        $lc\_at\_open\_var\_rec$. Queremos provar $term\ (pterm\_abs\ t1)$. Temos
        como hipótese que vale $lc\_at\ 1\ t1$ e que, para todo termo $t2$, de
        mesmo tamanho que o termo $t1$, e toda variável $x$ que não ocorre livre
        em $t2$, se vale $lc\_at\ 0\ t2^x$, vale $term\ (t2^x)$. Assim,
        analisando o construtor de $term$ para abstração e escolhendo o conjunto
        $L$ como $fv(t1)$, precisamos provar que, para qualquer $x \notin
        fv(t1)$, vale $term\ (t1^x)$. Assim, podemos usar a hipótese de indução,
        bastando provar agora que vale $lc\_at\ 0\ t1^x$. Observando o lema
        $lc\_at\_open\_var\_rec$, vemos que basta mostrar que vale $lc\_at\ 1\
        t1$, que é dado como hipótese. A prova é similar para o caso da
        substituição.
\end{itemize}
\end{proof}

Com esta prova, fica mais evidente as dificuldades que surgem no processo de
formalização. Enquanto a prova do Teorema \ref{teo:lc_lc_at} é bem direta,
a nossa versão tem que lidar com vários detalhes explicitamente, e em um dos
casos tem que adotar até mesmo outra estratégia de prova, fazendo indução no
tamanho do termo. 

\subsection{Equivalência de termos}
\label{sub:equival_ncia_de_termos}

Como dito anteriormente, o acréscimo de substituições explícitas pode nos levar
à perda de confluência do cálculo, como no exemplo abaixo, já mostrado:

\begin{displaymath}
        \xymatrix{ & ((\lambda x.\ t)\ u)[y/v] \ar[dl]_{B} \ar@{->>}[dr]_x &     \\
                  t[x/u][y/v] &               & (\lambda x.\ t[y/v])\ u \ar[d]_B \\
                              &               &  t[y/v][x/u]                    }
\end{displaymath}

Uma solução para este problema é tornar o novo sistema \emph{equacional}, ou
seja, tornar os termos $t[x/u][y/v]$ e $t[y/v][x/u]$ equivalentes, segundo
alguma relação.

Para isto, observemos o lema da substituição, no cálculo $\lambda$:

\[ t\{x/u\}\{y/v\} = t\{y/v\}\{x/u\},\ \ x \notin fv(v),\ y \notin fv(u) \]

Basicamente, se as substituições são \emph{independentes}, podemos permutá-las
sem alterar seu resultado. Queremos que o análogo para substituições explícitas
também seja válido, ou seja, que $t[x/u][y/v] \equiv t[y/v][x/u]$, se $x \notin
fv(v)$ e $y \notin fv(u)$.  Porém, os dois termos são sintaticamente distintos.
Para resolver este problema, adicionamos no sistema uma regra de
\emph{permutação de substituições independentes}.

\[ t[x/u][y/v] =_C t[y/v][x/u] \ \ \ \ \ se\ y \notin fv(u)\ \&\ x \notin fv(v)\] 

Dizemos que ambas as representações são \emph{equivalentes}.

Precisamos, então, formalizar essa definição no sistema.

\bigskip
\coqdocnoindent
\coqdockw{Inductive} \coqdef{Equation C.eqc}{eqc}{\coqdocinductive{eqc}} :
\coqdocinductive{pterm} \ensuremath{\rightarrow} \coqdocinductive{pterm}
\ensuremath{\rightarrow} \coqdockw{Prop} := \coqdoceol \coqdocindent{1.00em}
\ensuremath{|} \coqdef{Equation C.eqc
    def}{eqc\_def}{\coqdocconstructor{eqc\_def}}:
\coqdockw{\ensuremath{\forall}} \coqdocvar{t} \coqdocvar{u} \coqdocvar{v},
\coqdocdefinition{lc\_at} 2 \coqdocvariable{t} \ensuremath{\rightarrow}
\coqdocinductive{term} \coqdocvariable{u} \ensuremath{\rightarrow}
\coqdocinductive{term} \coqdocvariable{v} \ensuremath{\rightarrow}
\coqref{Equation C.eqc}{\coqdocinductive{eqc}}
(\coqdocvariable{t}\coqdocnotation{[}\coqdocvariable{u}\coqdocnotation{][}\coqdocvariable{v}\coqdocnotation{]})
(\coqdocnotation{(}\coqdocnotation{\&}
\coqdocvariable{t}\coqdocnotation{)[}\coqdocvariable{v}\coqdocnotation{][}\coqdocvariable{u}\coqdocnotation{]}).\coqdoceol
\bigskip

%As exigências que \emph{u} e \emph{v} sejam termos são puramente técnicas, pois
%queremos trabalhar apenas com termos sem índices livres no sistema. 

Veja que exigimos que \emph{u} e \emph{v} sejam termos, e que deva valer
\texttt{lc\_at 2 t}. Isto é necessário para que eqc seja equivalente à equação
$=_C$. Na definição original, queremos que $y \notin fv(u)$ e $x \notin fv(v)$,
para que as substituições sejam independentes. Aqui as exigências \emph{term u}
e \emph{term v} cumprem esse papel, mas são, também, mais fortes: queremos
trabalhar apenas com termos sem índices livres. Este fato também justifica a
exigência de que \texttt{lc\_at 2 t}.

O único outro detalhe que muda na formalização é ocorrência do operador $\&$,
que troca todas as ocorrências de índices zero e um, para que eles continuem se
referindo à mesma substituição.

Note que esta definição ainda possui muitas limitações. Por exemplo, se um termo
possui uma lista de substituições, só podemos trocar as duas últimas. Também não
é possível realizar permutações em subtermos, ou várias permutações seguidas.
Assim, precisamos criar \emph{fechos} em cima dessa definição, para ajustar às
nossas necessidades. 

\begin{definicao}[Fecho contextual]\label{def_fecho}
    Sejam $t,t',u$ pré-termos; $x$ uma variável e $R$ uma relação binária entre
    pré-termos. As regras a seguir definem o fecho contextual de R.

\begin{mathpar} 
    \inferrule*[Right=ES\_redex]{ (R\ t\ t') }
    {(ES\_contextual\_closure\ R)\ t\ t'}
    \\
    \inferrule*[Right=ES\_app\_left]{((ES\_contextual\_closure\ R)\ t\ t'),\
        (term\ u)}
    {((ES\_contextual\_closure\ R)\ (t\ u)\ (t'\ u))}
    \\
    \inferrule*[Right=ES\_app\_right]{((ES\_contextual\_closure\ R)\ t\ t'),\
        (term\ u)}
    {((ES\_contextual\_closure\ R)\ (u\ t)\ (u\ t'))}
    \\
    \inferrule*[Right=ES\_abs\_in]{(\forall\ x,\ x \notin L \rightarrow
        ((ES\_contextual\_closure\ R)\ t^x\ t'^x))}
    {((ES\_contextual\_closure\ R)\ (\lambda . t)\ (\lambda . t'))}
    \\
    \inferrule*[Right=ES\_subst\_left]{(\forall\ x,\ x \notin L \rightarrow
        ((ES\_contextual\_closure\ R)\ t^x\ t'^x)),\ (term\ u)}
    {((ES\_contextual\_closure\ R)\ (t[u])\ (t'[u]))}
    \\
    \inferrule*[Right=ES\_subst\_right]{((ES\_contextual\_closure\ R)\ u\ u'),\
        (body\ t)}
    {((ES\_contextual\_closure\ R)\ (t[u])\ (t[u']))}
\end{mathpar}

\end{definicao}

\bigskip
\coqdockw{Definition} \coqdef{Equation C.eqc ctx}{eqc\_ctx}{\coqdocdefinition{eqc\_ctx}} (\coqdocvar{t} \coqdocvar{u}: \coqdocinductive{pterm}) := \coqdocinductive{ES\_contextual\_closure} \coqref{Equation C.eqc}{\coqdocinductive{eqc}} \coqdocvariable{t} \coqdocvariable{u}.\coqdoceol
\coqdockw{Notation} "t =c u" := (\coqref{Equation C.eqc ctx}{\coqdocdefinition{eqc\_ctx}} \coqdocvar{t} \coqdocvar{u}) (\coqdoctac{at} \coqdockw{level} 66).\coqdoceol
\bigskip

A definição \texttt{ES\_contextual\_closure} é um dos chamados \emph{fechos
    contextuais}. A ideia é que, se vale $t \rightarrow_R t'$, então, para
qualquer termo construído a partir de $t$, vale a redução pelo fecho contextual de
R, para o mesmo termo, mas construído a partir de $t'$. Como exemplo:

\[(t \rightarrow_R t') \Rightarrow ((pterm\_app\ t\ u)
    \rightarrow_{ES\_Contextual\_Closure\ R} (pterm\_app\ t'\ u)) \]

%Definimos também os fechos \emph{transitivo} e \emph{transitivo-reflexivo}.
%Essencialmente, se temos $t \rightarrow_R t'$ e $t' \rightarrow_R t''$, então
%vale o fecho transitivo $t \rightarrow_{R^+} t''$. Isto é representado pelo
%construtor \texttt{transitive\_reduction}. Note que isto vale para
%apenas um passo também: $t \rightarrow_R t'$ implica $t \rightarrow_{R^+} t'$.
%Este é representado pelo construtor \texttt{one\_step\_reduction}.

%\begin{definicao}[Fecho transitivo]
    %Sejam $t, t', u$ pré-termos e $R$ uma relação entre pré-termos. As regras
    %abaixo definem o fecho transitivo de $R$.
    
%\begin{mathpar} 
    %\inferrule*[Right=one\_step\_reduction]{ (R\ t\ t') }
    %{(trans\_closure\ R)\ t\ t'}
    %\\
    %\inferrule*[Right=transitive\_reduction]{((trans\_closure\ R)\ t\ u),\
        %((trans\_closure\ R)\ u\ t')}
    %{((trans\_closure\ R)\ (t\ t'))}
%\end{mathpar}
%\end{definicao}

%O fecho \emph{transitivo-reflexivo} é uma extensão do transitivo, onde vale $t
%\rightarrow_{R^*} t$, sempre. A extensão é presentada por dois construtores:
%\texttt{reflexive\_reduction}, que garante \texttt{star\_closure R t t}, e
%\texttt{star\_trans\_reduction}, que constrói \texttt{star\_closure R t u}, se
%vale \texttt{trans\_closure R t u}.

%\begin{definicao}[Fecho transitivo-reflexivo]
    %Sejam $t, t', u$ pré-termos e $R$ uma relação entre pré-termos. As regras
    %abaixo definem o fecho transitivo-reflexivo de $R$.
%\begin{mathpar} 
    %\inferrule*[Right=reflexive\_reduction]{  }
    %{(star\_closure\ R)\ t\ t}
    %\\
    %\inferrule*[Right=star\_trans\_reduction]{((trans\_closure\ R)\ t\ u)}
    %{((star_trans\_closure\ R)\ (t\ u))}
%\end{mathpar}
%\end{definicao}

Definimos então o análogo da equação $=_C$, com base no fecho
transitivo-reflexivo da relação \texttt{eqc\_ctx}

\bigskip
\coqnoindent\coqdockw{Definition} \coqdef{Equation C.eqc
    trans}{eqc\_trans}{\coqdocdefinition{eqc\_trans}} (\coqdocvar{t}
\coqdocvar{u}: \coqdocinductive{pterm}) := \coqdocinductive{trans\_closure}
\coqref{Equation C.eqc ctx}{\coqdocdefinition{eqc\_ctx}} \coqdocvariable{t}
\coqdocvariable{u}.\coqdoceol \coqdockw{Notation} "t =c+ u" := (\coqref{Equation
    C.eqc trans}{\coqdocdefinition{eqc\_trans}} \coqdocvar{t} \coqdocvar{u})
(\coqdoctac{at} \coqdockw{level} 66).\coqdoceol

\coqdockw{Definition} \coqdef{Equation C.eqC}{eqC}{\coqdocdefinition{eqC}} (\coqdocvar{t} : \coqdocinductive{pterm}) (\coqdocvar{u} : \coqdocinductive{pterm}) := \coqdocinductive{star\_closure} \coqref{Equation C.eqc ctx}{\coqdocdefinition{eqc\_ctx}} \coqdocvariable{t} \coqdocvariable{u}.\coqdoceol
\coqdockw{Notation} "t =e u" := (\coqref{Equation C.eqC}{\coqdocdefinition{eqC}} \coqdocvar{t} \coqdocvar{u}) (\coqdoctac{at} \coqdockw{level} 66).\coqdoceol
\bigskip

Assim, a relação de equivalência que é interessante ser estudada é a \texttt{=e}, que
permite vários passos de permutações, além de permutações no interior dos
termos. Uma importante diferença de provas usando equivalências no papel e numa
formalização é que resultados que são intuitivos e admitidos num ambiente
informal devem ser provados minuciosamente.

Como exemplo, podemos querer mostrar a compatibilidade da igualdade com a
função \texttt{lc\_at}. Isto é intuitivo, pois permutar duas substituições não
irá criar índices livres. Porém, é preciso entrar em detalhes na prova formal. 

Observe que precisamos provar três resultados. Não basta mostrar que o predicado
\texttt{eqc} é compatível com \texttt{lc\_at}, pois ainda usaremos seus fechos.

\bigskip 
\coqnoindent \coqdockw{Lemma} \coqdef{Equation C.lc at
    eqc}{lc\_at\_eqc}{\coqdoclemma{lc\_at\_eqc}} :
\coqdockw{\ensuremath{\forall}} \coqdocvar{n} \coqdocvar{t} \coqdocvar{u},
\coqref{Equation C.eqc}{\coqdocinductive{eqc}} \coqdocvariable{t}
\coqdocvariable{u}  \ensuremath{\rightarrow} (\coqdocdefinition{lc\_at}
\coqdocvariable{n} \coqdocvariable{t} \coqexternalref{:type scope:x '<->'
    x}{http://coq.inria.fr/distrib/8.4pl4/stdlib/Coq.Init.Logic}{\coqdocnotation{\ensuremath{\leftrightarrow}}}
\coqdocdefinition{lc\_at} \coqdocvariable{n} \coqdocvariable{u}).\coqdoceol
\bigskip 

A prova de \texttt{lc\_at\_eqc} já apresenta certos detalhes. Ela é feita
através de uma análise de casos do predicado \texttt{eqc}. É preciso realizar
certos ajustes com índices devido ao uso do operador \texttt{\&} na definição da
equação. Usamos também a equivalência entre \texttt{lc\_at} e o
predicado \texttt{term} para usar a hipótese que as expressões dentro das
substituições são termos, além de regras de enfraquecimento para o
\texttt{lc\_at}.

\bigskip 
\coqnoindent \coqdockw{Lemma} \coqdef{Equation C.lc at ES ctx
    eqc}{lc\_at\_ES\_ctx\_eqc}{\coqdoclemma{lc\_at\_ES\_ctx\_eqc}} :
\coqdockw{\ensuremath{\forall}} \coqdocvar{n} \coqdocvar{t} \coqdocvar{u},
(\coqdocinductive{ES\_contextual\_closure} \coqref{Equation
    C.eqc}{\coqdocinductive{eqc}}) \coqdocvariable{t} \coqdocvariable{u}
\ensuremath{\rightarrow} (\coqdocdefinition{lc\_at} \coqdocvariable{n}
\coqdocvariable{t} \coqexternalref{:type scope:x '<->'
    x}{http://coq.inria.fr/distrib/8.4pl4/stdlib/Coq.Init.Logic}{\coqdocnotation{\ensuremath{\leftrightarrow}}}
\coqdocdefinition{lc\_at} \coqdocvariable{n} \coqdocvariable{u}).\coqdoceol
\bigskip 

No caso do fecho contextual, fazemos indução no próprio fecho. As hipóteses de
indução resolvem os casos mais simples. Os casos de abstração e substituição se
tornam um pouco mais complexos, pois temos que lidar com o corpo dos pré-termos.
A hipótese de indução é dada da seguinte maneira: dado um conjunto L, para toda
variável $x$ que não está em L, e todo natural $n$, vale $lc\_at\ n\ (t^x)
\iff lc\_at\ n\ (t'^x)$. Precisamos, porém, provar que vale $lc\_at\ (S\ n)\ t
\iff lc\_at\ (S\ n)\ t'$. Para resolver este problema tomamos uma variável qualquer
que não esteja em L, e abrimos os subtermos $t$ e $t'$ com ela, a um nível zero.
Podemos fazer isto sem afetar a validade do predicado $lc\_at$, pois não
estamos adicionando nenhum novo índice. O objetivo então se torna provar
$lc\_at\ (S\ n)\ (t^x) \iff lc\_at\ (S\ n)\ (t'^x)$ Basta então aplicar a hipótese de
indução, concluindo, assim, a prova.

\bigskip 
\coqnoindent \coqdockw{Lemma} \coqdef{Equation C.lc at
    eqC}{lc\_at\_eqC}{\coqdoclemma{lc\_at\_eqC}} :
\coqdockw{\ensuremath{\forall}} \coqdocvar{n} \coqdocvar{t} \coqdocvar{t'},
\coqdocvariable{t} \coqref{Equation C.::x '=e'
    x}{\coqdocnotation{=}}\coqref{Equation C.::x '=e' x}{\coqdocnotation{e}}
\coqdocvariable{t'} \ensuremath{\rightarrow} (\coqdocdefinition{lc\_at}
\coqdocvariable{n} \coqdocvariable{t} \coqexternalref{:type scope:x '<->'
    x}{http://coq.inria.fr/distrib/8.4pl4/stdlib/Coq.Init.Logic}{\coqdocnotation{\ensuremath{\leftrightarrow}}}
\coqdocdefinition{lc\_at} \coqdocvariable{n} \coqdocvariable{t'}).\coqdoceol
\bigskip

Esta prova é bem simples, fazendo indução no fecho transitivo-reflexivo. O caso
reflexivo é trivial. No passo indutivo, fechamos facilmente usando a hipótese de
indução, pela transitividade da relação $\leftrightarrow$.


Esta é uma parte do trabalho especialmente extensa e detalhada. É preciso
mostrar que vários predicados importantes da teoria são preservados pelas
classes de equivâlencia entre termos. Várias vezes durante as provas
trabalharemos com termos equivalentes módulo $=_e$ e, sem uma base bem
construída de resultados sobre a relação \texttt{eqC}, fica impossível
concluí-las.

Os principais resultados que foram provados nesta parte incluem:

\begin{itemize}
    \item Preservação da estrutura de termos pela equivalência. Construtores de
        termos, abertura de termos, substituições e renomeamento de variáveis
        devem funcionar de maneira análoga para dois termos equivalentes.
    \item Boa formação de termos e corpos de abstrações, conjunto de variáveis
        livres e reduções dentro de fechos contextuais também devem ser
        preservadas pela equivalência.
\end{itemize}

\subsection{Reduções do Sistema}
\label{sub:redu_es_do_sistema}

Com a estrutura de termos e as regras de equivalência bem estabelecidas, podemos
iniciar a formalização das reduções do sistema $\lambda ex$.
As regras de redução listadas em  \ref{table:red_lambex} serão formalizadas nos
tipos indutivos \texttt{sys\_x} e \texttt{rule\_b}.

\bigskip
\coqdocnoindent \coqdockw{Inductive} \coqdef{Lambda Ex.rule
    b}{rule\_b}{\coqdocinductive{rule\_b}} : \coqdocinductive{pterm}
\ensuremath{\rightarrow} \coqdocinductive{pterm} \ensuremath{\rightarrow}
\coqdockw{Prop} :=\coqdoceol \coqdocindent{1.50em} \coqdef{Lambda Ex.reg rule
    b}{reg\_rule\_b}{\coqdocconstructor{reg\_rule\_b}} :
\coqdockw{\ensuremath{\forall}} (\coqdocvar{t}
\coqdocvar{u} : \coqdocinductive{pterm}),  \coqdocdefinition{body}
\coqdocvariable{t} \ensuremath{\rightarrow} \coqdocinductive{term}
\coqdocvariable{u} \ensuremath{\rightarrow}  \coqdoceol \coqdocindent{2.50em}
\coqref{Lambda Ex.rule b}{\coqdocinductive{rule\_b}}
(\coqdocconstructor{pterm\_app} (\coqdocconstructor{pterm\_abs}
\coqdocvariable{t}) \coqdocvariable{u})
(\coqdocvariable{t}\coqdocnotation{[}\coqdocvariable{u}\coqdocnotation{]}).\coqdoceol
\coqdocnoindent
\coqdockw{Notation} "t ->\_B u" :=
(\coqref{Lambda Ex.rule b}{\coqdocinductive{rule\_b}} \coqdocvar{t}
\coqdocvar{u}) (\coqdoctac{at} \coqdockw{level} 66).\coqdoceol

\medskip

\coqdocnoindent \coqdockw{Inductive} \coqdef{Lambda Ex.sys
    x}{sys\_x}{\coqdocinductive{sys\_x}} : \coqdocinductive{pterm}
\ensuremath{\rightarrow} \coqdocinductive{pterm} \ensuremath{\rightarrow}
\coqdockw{Prop} :=\coqdoceol \coqdocnoindent \ensuremath{|} \coqdef{Lambda
    Ex.reg rule var}{reg\_rule\_var}{\coqdocconstructor{reg\_rule\_var}} :
\coqdockw{\ensuremath{\forall}} \coqdocvar{t}, \coqdocinductive{term}
\coqdocvariable{t} \ensuremath{\rightarrow} \coqref{Lambda Ex.sys
    x}{\coqdocinductive{sys\_x}} (\coqdocconstructor{pterm\_bvar} 0
\coqdocnotation{[}\coqdocvariable{t}\coqdocnotation{]})
\coqdocvariable{t}\coqdoceol \coqdocnoindent \ensuremath{|} \coqdef{Lambda
    Ex.reg rule gc}{reg\_rule\_gc}{\coqdocconstructor{reg\_rule\_gc}} :
\coqdockw{\ensuremath{\forall}} \coqdocvar{t} \coqdocvar{u},
\coqdocinductive{term} \coqdocvariable{t} \ensuremath{\rightarrow}
\coqdocinductive{term} \coqdocvariable{u} \ensuremath{\rightarrow}
\coqref{Lambda Ex.sys x}{\coqdocinductive{sys\_x}}
(\coqdocvariable{t}\coqdocnotation{[}\coqdocvariable{u}\coqdocnotation{]})
\coqdocvariable{t}\coqdoceol \coqdocnoindent \ensuremath{|} \coqdef{Lambda
    Ex.reg rule app}{reg\_rule\_app}{\coqdocconstructor{reg\_rule\_app}} :
\coqdockw{\ensuremath{\forall}} \coqdocvar{t1} \coqdocvar{t2} \coqdocvar{u},
\coqdocdefinition{body} \coqdocvariable{t1} \ensuremath{\rightarrow}
\coqdocdefinition{body} \coqdocvariable{t2} \ensuremath{\rightarrow}
\coqdocinductive{term} \coqdocvariable{u} \ensuremath{\rightarrow} \coqdoceol
\coqdocindent{1.00em} \coqref{Lambda Ex.sys x}{\coqdocinductive{sys\_x}}
(\coqdocnotation{(}\coqdocconstructor{pterm\_app} \coqdocvariable{t1}
\coqdocvariable{t2}\coqdocnotation{)[}\coqdocvariable{u}\coqdocnotation{]})
(\coqdocconstructor{pterm\_app}
(\coqdocvariable{t1}\coqdocnotation{[}\coqdocvariable{u}\coqdocnotation{]})
(\coqdocvariable{t2}\coqdocnotation{[}\coqdocvariable{u}\coqdocnotation{]}))\coqdoceol
\coqdocnoindent \ensuremath{|} \coqdef{Lambda Ex.reg rule
    lamb}{reg\_rule\_lamb}{\coqdocconstructor{reg\_rule\_lamb}} :
\coqdockw{\ensuremath{\forall}} \coqdocvar{t} \coqdocvar{u},
\coqdocdefinition{body} (\coqdocconstructor{pterm\_abs} \coqdocvariable{t})
\ensuremath{\rightarrow} \coqdocinductive{term} \coqdocvariable{u}
\ensuremath{\rightarrow} \coqdoceol \coqdocindent{1.00em} \coqref{Lambda Ex.sys
    x}{\coqdocinductive{sys\_x}}
(\coqdocnotation{(}\coqdocconstructor{pterm\_abs}
\coqdocvariable{t}\coqdocnotation{)[}\coqdocvariable{u}\coqdocnotation{]})
(\coqdocconstructor{pterm\_abs} (\coqdocnotation{(}\coqdocnotation{\&}
\coqdocvariable{t}\coqdocnotation{)[}\coqdocvariable{u}\coqdocnotation{]}))\coqdoceol
\coqdocnoindent \ensuremath{|} \coqdef{Lambda Ex.reg rule
    comp}{reg\_rule\_comp}{\coqdocconstructor{reg\_rule\_comp}} :
\coqdockw{\ensuremath{\forall}} \coqdocvar{t} \coqdocvar{u} \coqdocvar{v},
\coqdocdefinition{body}
(\coqdocvariable{t}\coqdocnotation{[}\coqdocvariable{u}\coqdocnotation{]})
\ensuremath{\rightarrow} \coqexternalref{:type scope:'x7E'
    x}{http://coq.inria.fr/distrib/8.4pl4/stdlib/Coq.Init.Logic}{\coqdocnotation{\ensuremath{\lnot}}}
\coqdocinductive{term} \coqdocvariable{u} \ensuremath{\rightarrow}
\coqdocinductive{term} \coqdocvariable{v} \ensuremath{\rightarrow} \coqdoceol
\coqdocindent{1.00em} \coqref{Lambda Ex.sys x}{\coqdocinductive{sys\_x}}
(\coqdocvariable{t}\coqdocnotation{[}\coqdocvariable{u}\coqdocnotation{][}\coqdocvariable{v}\coqdocnotation{]})
(\coqdocnotation{((}\coqdocnotation{\&}
\coqdocvariable{t}\coqdocnotation{)[}\coqdocvariable{v}\coqdocnotation{])[}
\coqdocvariable{u}\coqdocnotation{[} \coqdocvariable{v} \coqdocnotation{]}
\coqdocnotation{]}).\coqdoceol \coqdocemptyline

\coqdocnoindent
\coqdockw{Notation} "t ->\_x u" :=


(\coqref{Lambda Ex.sys x}{\coqdocinductive{sys\_x}} \coqdocvar{t} \coqdocvar{u})
(\coqdoctac{at} \coqdockw{level} 59, \coqdoctac{left}
\coqdockw{associativity}).\coqdoceol
\bigskip

A relação $\rightarrow_x$ é definida como sendo exatamente o predicado
\texttt{sys\_x}. Algumas exigências técnicas de \emph{term} e \emph{body} são
adicionadas para auxiliar nas provas, já que queremos sempre trabalhar com
termos bem formados.

O predicado \texttt{rule\_b} é o que formaliza a regra $\rightarrow_B$, sendo
esta a regra que reduz uma aplicação a uma substituição explícita. Podemos, a
partir destes dois predicados, definir a regra de redução principal do sistema
$\lambda ex$.

\bigskip
\coqdocnoindent \coqdockw{Inductive} \coqdocvar{sys\_Bx}: \coqdocinductive{pterm}
\ensuremath{\rightarrow} \coqdocinductive{pterm} \ensuremath{\rightarrow}
\coqdockw{Prop} := \coqdoceol
\coqdocnoindent \ensuremath{|} \coqdocvar{B\_lx} : \coqdockw{\ensuremath{\forall}} \coqdocvar{t}
\coqdocvar{u}, \coqdocvariable{t} \coqref{Lambda Ex.::x '-> B'
    x}{\coqdocnotation{\ensuremath{\rightarrow}}}\coqref{Lambda Ex.::x '-> B'
    x}{\coqdocnotation{\_B}} \coqdocvariable{u} \ensuremath{\rightarrow}
\coqref{Lambda Ex.sys Bx}{\coqdocinductive{sys\_Bx}} \coqdocvariable{t}
\coqdocvariable{u}\coqdoceol \coqdocnoindent \ensuremath{|} \coqdef{sys\_x\_lx}
: \coqdockw{\ensuremath{\forall}} \coqdocvar{t} \coqdocvar{u},
\coqdocvariable{t} \coqref{Lambda Ex.::x '-> x'
    x}{\coqdocnotation{\ensuremath{\rightarrow}}}\coqref{Lambda Ex.::x '-> x'
    x}{\coqdocnotation{\_x}} \coqdocvariable{u} \ensuremath{\rightarrow}
\coqref{Lambda Ex.sys Bx}{\coqdocinductive{sys\_Bx}} \coqdocvariable{t}
\coqdocvariable{u}.\coqdoceol

\smallskip
\coqdocnoindent \coqdockw{Notation} "t ->\_Bx u" := (\coqref{Lambda Ex.sys Bx}{\coqdocinductive{sys\_Bx}} \coqdocvar{t} \coqdocvar{u}) (\coqdoctac{at} \coqdockw{level} 59, \coqdoctac{left} \coqdockw{associativity}).\coqdoceol

\smallskip
\coqdocnoindent \coqdockw{Definition} \coqdef{Equation C.red ctx mod eqC}{red\_ctx\_mod\_eqC}{\coqdocdefinition{red\_ctx\_mod\_eqC}} (\coqdocvar{R}: \coqdocinductive{pterm} \ensuremath{\rightarrow} \coqdocinductive{pterm} \ensuremath{\rightarrow} \coqdockw{Prop}) (\coqdocvar{t}: \coqdocinductive{pterm}) (\coqdocvar{u} : \coqdocinductive{pterm}) := \coqdoceol

\coqdocindent{5.50em}
\coqexternalref{:type scope:'exists' x '..' x ','
    x}{http://coq.inria.fr/distrib/8.4pl4/stdlib/Coq.Init.Logic}{\coqdocnotation{\ensuremath{\exists}}}
\coqdocvar{t'}\coqexternalref{:type scope:'exists' x '..' x ','
    x}{http://coq.inria.fr/distrib/8.4pl4/stdlib/Coq.Init.Logic}{\coqdocnotation{,}}
\coqexternalref{:type scope:'exists' x '..' x ','
    x}{http://coq.inria.fr/distrib/8.4pl4/stdlib/Coq.Init.Logic}{\coqdocnotation{\ensuremath{\exists}}}
\coqdocvar{u'}\coqexternalref{:type scope:'exists' x '..' x ','
    x}{http://coq.inria.fr/distrib/8.4pl4/stdlib/Coq.Init.Logic}{\coqdocnotation{,}}
\coqexternalref{:type scope:x '/x5C'
    x}{http://coq.inria.fr/distrib/8.4pl4/stdlib/Coq.Init.Logic}{\coqdocnotation{(}}\coqdocvariable{t}
\coqref{Equation C.::x '=e' x}{\coqdocnotation{=}}\coqref{Equation C.::x '=e'
    x}{\coqdocnotation{e}} \coqdocvariable{t'}\coqexternalref{:type scope:x
    '/x5C'
    x}{http://coq.inria.fr/distrib/8.4pl4/stdlib/Coq.Init.Logic}{\coqdocnotation{)/\symbol{92}(}}\coqdocinductive{ES\_contextual\_closure}
\coqdocvariable{R} \coqdocvariable{t'} \coqdocvariable{u'}\coqexternalref{:type
    scope:x '/x5C'
    x}{http://coq.inria.fr/distrib/8.4pl4/stdlib/Coq.Init.Logic}{\coqdocnotation{)/\symbol{92}(}}\coqdocvariable{u'}
\coqref{Equation C.::x '=e' x}{\coqdocnotation{=}}\coqref{Equation C.::x '=e'
    x}{\coqdocnotation{e}} \coqdocvariable{u}\coqexternalref{:type scope:x
    '/x5C'
    x}{http://coq.inria.fr/distrib/8.4pl4/stdlib/Coq.Init.Logic}{\coqdocnotation{)}}.\coqdoceol

\medskip
\coqdocnoindent \coqdockw{Definition} \coqdef{Lambda Ex.lex}{lex}{\coqdocdefinition{lex}} \coqdocvar{t} \coqdocvar{u} :=  \coqdocdefinition{red\_ctx\_mod\_eqC} \coqref{Lambda Ex.sys Bx}{\coqdocinductive{sys\_Bx}} \coqdocvariable{t} \coqdocvariable{u}.\coqdoceol

\coqdocnoindent \coqdockw{Notation} "t -->lex u" := (\coqref{Lambda Ex.lex}{\coqdocdefinition{lex}} \coqdocvar{t} \coqdocvar{u}) (\coqdoctac{at} \coqdockw{level} 66).\coqdoceol

\smallskip

\coqdocnoindent
\coqdockw{Definition} \coqdef{Lambda Ex.lex trs}{lex\_trs}{\coqdocdefinition{lex\_trs}} \coqdocvar{t} \coqdocvar{u} := \coqdocinductive{trans\_closure} \coqref{Lambda Ex.lex}{\coqdocdefinition{lex}} \coqdocvariable{t} \coqdocvariable{u}.\coqdoceol
\coqdocnoindent
\coqdockw{Notation} "t -->lex+ u" := (\coqref{Lambda Ex.lex trs}{\coqdocdefinition{lex\_trs}} \coqdocvar{t} \coqdocvar{u}) (\coqdoctac{at} \coqdockw{level} 66).\coqdoceol
\coqdocemptyline
\coqdocnoindent
\coqdockw{Definition} \coqdef{Lambda Ex.lex str}{lex\_str}{\coqdocdefinition{lex\_str}} \coqdocvar{t} \coqdocvar{u} := \coqdocinductive{star\_closure} \coqref{Lambda Ex.lex}{\coqdocdefinition{lex}} \coqdocvariable{t} \coqdocvariable{u}.\coqdoceol
\coqdocnoindent
\coqdockw{Notation} "t -->lex* u" := (\coqref{Lambda Ex.lex str}{\coqdocdefinition{lex\_str}} \coqdocvar{t} \coqdocvar{u}) (\coqdoctac{at} \coqdockw{level} 66).\coqdoceol

\bigskip

O análogo da regra $\rightarrow_{\lambda ex}^*$ no sistema será a relação
"\texttt{-->lex*}", construída a partir do fecho transitivo-reflexivo,
contextual e equacional da relação $\rightarrow_{Bx}$.

Para auxiliar nas provas, definimos alguns predicados que garantem que estamos
trabalhando sempre com termos, e que representam algumas noções intuitivas das
reduções.

\begin{definicao}[Regularidade]
    Sejam $t, u$ pré-termos e $R$ uma relação binária entre pré-termos.
    Dizemos que $R$ é regular se, sempre que vale $R\ t\ u$, então $t$ e $u$ são
    termos bem formados. A noção de regularidade é formalizada pelo predicado
    $red\_regular$, definido como base na regra de inferência abaixo.
    
\begin{mathpar} 
    \inferrule*{(red\_regular\ R),\ (R\ t\ u)}
    {term\ t\ \wedge\ term\ u}
\end{mathpar}
\end{definicao}


A propriedade de regularidade é interessante para facilitar diversas provas,
já que desejamos evitar trabalhar com termos que possuem índices livres.
Basicamente, se fizemos a redução $R$ entre dois pré-termos,
ambos são termos bem formados.

Algumas relações interessantes não são capazes de garantir a regularidade. Ainda
assim, queremos continuar lidando com termos bem formados. Definimos então uma
noção mais fraca de regularidade, que apenas preserva a boa formação de termos.

\begin{definicao}[Regularidade fraca]
    Sejam $t, u$ pré-termos e $R$ uma relação binária entre pré-termos.
    Dizemos que $R$ satisfaz a regularidade fraca se, sempre que vale $R\ t\ u$,
    vale $(term\ t) \iff (term\ u)$. A noção de regularidade fraca é formalizada
    pelo predicado $red\_regular'$, definido como base na regra de inferência
    abaixo.
    
\begin{mathpar} 
    \inferrule*{(red\_regular'\ R),\ (R\ t\ u)}
    {term\ t\ \iff term\ u}
\end{mathpar}
\end{definicao}

%\begin{table}[h]
%\begin{mathpar} 
    %\inferrule*{(red\_out\ R),\ (R\ t\ u)}
    %{R\ ([x \rightarrow u]t)\ ([x \rightarrow u]t')}
%\end{mathpar}
    %\caption{Redução fora da meta-substituição}
%\end{table}


%A propriedade $red\_out$ garante que a redução pode ser feita em um
%termo afetado por uma meta-substituição, sendo $[x\ \mapsto\ u]$ a substituição que
%troca todas as ocorrências de $x$ pelo termo $u$. Ela é útil pois
%esta meta-substituição pode ser vista como uma generalização da abertura de um
%termo.

\begin{definicao}[Renomeamento]
    Sejam $t, u$ pré-termos; $x,y$ variáveis e $R$ uma relação binária entre pré-termos.
    Dizemos que a relação $R$ é compatível com renomeamento de variáveis, ou
    seja, vale $red\_rename\ R$, se $R$ satisfaz a seguinte regra de inferência:    

\begin{mathpar} 
    \inferrule*{(red\_rename\ R),\ (\forall x,\ x \notin fv(t) \rightarrow R\
        (t^x) \ (u^x))}
    {R\ (t^y)\ (u^y)}
\end{mathpar}
\end{definicao}

Basicamente, se um termo $t$ aberto com uma variável $x \notin fv(t)$ se reduz a
um $t'$, também aberto com $x$, então este $x$ pode ser trocado por outra
variável $y \notin fv(t)$, preservando a redução $\rightarrow_{lex}$.

Estes resultados auxiliam muito em diversas provas. Em especial, os resultados
envolvendo abertura de termos e índices livres são necessários para provas em
que fazemos indução na estrutura do termo, pois, em geral, no caso da abstração
as hipóteses se referem ao sub-termo aberto com uma variável.



%------------------------------------------------------------------



\section{Preservação da normalização forte}
\label{sec:psn}

Nesta seção, queremos dar uma visão geral da propriedade PSN e descrever a
formalização da propriedade IE, foco deste trabalho. 

A propriedade PSN é a que garante que, se um termo $t$ é fortemente
normalizável no cálculo original, ou seja, toda cadeia de reduções a partir dele
termina, então ele também é fortemente normalizável no sistema $\lambda ex$.

A prova da PSN é feita primeiramente definindo uma estratégia de redução
\emph{perpétua} para o sistema.

\begin{definicao}[Estratégia de redução perpétua]
    Uma estratégia perpétua fornece uma cadeia de reduções \emph{infinita} para
    um termo, se uma existe. Caso não exista, fornece uma cadeia que termina em
    um termo em forma normal.
\end{definicao}

Observe que, se um termo $t$ não é fortemente normalizável, então a estratégia o
reduzirá para um termo $t'$ que também não o é, e assim por diante, para poder
criar a cadeia de redução infinita. Em outras palavras: Se $t \rightarrow t'$
por uma estratégia perpétua, e se $t$ não é fortemente normalizável, então $t'$
também não será fortemente normalizável.  Pela contrapositiva, se $t$ se reduz,
por esta estratégia, a um termo $t'$ que é fortemente normalizável, então $t$
também o deve ser.


%definida da seguinte maneira: se um termo
%$t$ se reduz a um termo $t'$ por esta estratégia, e se $t'$ é
%fortemente normalizável, então $t$ também o é. Pela contrapositiva, vemos
%que se o termo $t$ não for fortemente normalizável, então a estratégia terá
%que fazer a redução para um termo $t'$ que também não é fortemente
%normalizável, e assim por diante, formando uma cadeia de redução infinita.

Para a prova da propriedade \textbf{PSN}, observamos o uso desta estratégia para
um caso particular: se um termo $t[x/u]$ é reduzido, por uma estratégia
perpétua, para um termo $t\{x/u\}$, sendo este fortemente normalizável, então o
termo original também será fortemente normalizável. Em outras palavras, a
normalização da substituição \emph{implícita} implica na normalização da
substituição \emph{explícita}. Esta é a chamada propriedade \textbf{IE}.

Para a prova da propriedade \textbf{IE}, é adicionado mais uma estrutura no
sistema, chamada de \emph{substituição marcada}.

\subsection{Substituições marcadas}
\label{sub:subst_marc}

A ideia é controlar as reduções feitas envolvendo substituições explícitas. Para
isto, adicionamos um novo construtor na gramática.

\[ \tau := x\ |\ \lambda x.\tau\ |\ \tau \tau\ |\ \tau[x/\tau]\ |\
    \tau[\![x/u]\!]\ \]



Na nova substituição, não podemos colocar qualquer qualquer termo no
lugar de $u$. Restringimos o termo $u$ a apenas termos da gramática
original, ou seja, sem substituições marcadas. Além disso, é necessário que o
termo seja fortemente normalizável. Para fazer esta verificação, usaremos o
predicado \texttt{SN}, definido com base no predicado \texttt{SN\_ind}.

 %| SN_intro : (forall t', R t t' -> exists k, k < n /\ SN_ind k R t') -> SN_ind n R t.
\begin{definicao}
    Seja $t$ um pré-termo e $R$ uma relação binária entre pré-termos. Dizemos
    que $t$ é fortemente normalizável pela redução $R$, ou seja, vale (SN R t)
    se existe $n \in \mathbb{N}$ satisfazendo o predicado SN\_ind, definido com
    base na regra de inferência abaixo.

\begin{mathpar} 
    \inferrule*[Right=SN\_ind]{(\forall t', t \rightarrow_{R} t' \Rightarrow
        \exists k \in \mathbb{N}, k < n, (SN\_ind\ k\ t') )}
    {SN\_ind\ n\ t}
\end{mathpar}

\end{definicao}

Essencialmente, $(SN\ R\ t)$ garante que existe $n \in \mathbb{N}$ tal que toda
cadeia de reduções a partir de $t$ tem comprimento no máximo $n$, ou seja, é
finita.

Precisamos adicionar a nova substituição na nossa formalização. Adicionamos um
novo construtor ao tipo \texttt{pterm}. 

\bigskip
\coqdocnoindent \coqdockw{Inductive} \coqdef{LambdaES
    Defs.pterm}{pterm}{\coqdocinductive{pterm}} : \coqdockw{Set} :=\coqdoceol
\coqdocindent{1.00em} \ensuremath{|} ...\coqdoceol 
\coqdocindent{1.00em} \ensuremath{|} \coqdef{LambdaES Defs.pterm
    lsub}{pterm\_lsub}{\coqdocconstructor{pterm\_lsub}} : \coqref{LambdaES
    Defs.pterm}{\coqdocinductive{pterm}} \ensuremath{\rightarrow}
\coqref{LambdaES Defs.pterm}{\coqdocinductive{pterm}} \ensuremath{\rightarrow}
\coqref{LambdaES Defs.pterm}{\coqdocinductive{pterm}}.\coqdoceol 
\bigskip

Assim como precisamos de um predicado para verificar se um termo comum está bem
formado, vamos criar um outro predicado, chamado \texttt{lab\_term}. 

\begin{definicao}[Termo marcado]
    Este predicado é uma extensão do predicado $term$, visto na Definição
    \ref{def_term}. Adicionamos apenas um caso para lidar com as substituições
    marcadas, sendo todos os outros casos análogos ao anterior.
        
\begin{mathpar} 
    \inferrule*[Right=lab\_term\_sub']{\forall x \notin L,\ lab\_term(t^{x}) \\
        term(u) \\ (SN\ lex)\ u}
    {lc(t[\![u]\!])}
\end{mathpar}
\end{definicao}

Observe que no caso da substituição marcada, temos a exigência $(SN\ lex\ u)$,
que indica que $u$ é fortemente normalizável.

Também devemos estender a noção equivalente de ser localmente fechado, para
manter a equivalência definida no caso do sistema sem as substituições marcadas.
Assim, definimos uma nova função \texttt{lc\_at'}, análoga à do caso não
marcado.  Sua implementação é também feita com base na Definição
\ref{def_lc_at}, com apenas uma extensão para os casos das substituição marcadas
e explícitas.

\begin{definicao}
    A função $lc\_at'$ é definida como uma extensão de $lc\_at$, adicionando os
    seguintes dois casos:

\begin{empheq}{align*}
    lc\_at'\ k\ (t[u])\ \ \ &\equiv\ (lc\_at'\ (S k)\ t)\ \&\ (lc\_at'\ k\ u) \\ 
    lc\_at'\ k\ (t[\![u]\!])\ \ \ &\equiv\ (lc\_at'\ (S k)\ t)\ \&\ (lc\_at\ k\ u)\
    \&\ (SN\ lex\ u) \\ 
\end{empheq}
\end{definicao}

\bigskip

\coqdocnoindent \coqdockw{Fixpoint} \coqdocvar{lc\_at'}
(\coqdocvar{k}:\coqdocvar{nat}) (\coqdocvar{t}:\coqdocvar{pterm})
\{\coqdockw{struct} \coqdocvar{t}\} : \coqdockw{Prop} := ...\coqdoceol

\coqdocnoindent\coqdockw{Definition} \coqdocvar{term'{}'} \coqdocvar{t} := \coqdocvar{lc\_at'} 0 \coqdocvar{t}.\coqdoceol
\bigskip

Como no capítulo anterior, precisamos mostrar a equivalência entre
\texttt{lab\_term} e \texttt{lc\_at'}. Para isto, precisamos mostrar também os
lemas que falam sobre a interação entre \texttt{lc\_at'} e a operação de
substituição.  As provas seguem de maneira análoga às suas versões originais.
Devemos apenas tomar cuidado com as exigências de normalização nos casos das
substituições marcadas. Para resolvê-los tivemos que assumir alguns resultados
envolvendo a relação entre o predicado $SN$ e índices livres. 

Como no caso do sistema simples, queremos definir classes de equivalências
de termos, para trabalhar módulo permutação de substituições.
Para isso, precisamos definir novos fechos contextuais para os termos com
substituições marcadas.

O fecho contextual para termos marcados é definido como uma extensão do
fecho para termos comuns, visto na definição \ref{def_fecho}, com a ressalva
que os predicados de $term$ e $body$ são substituídos por seus análogos,
$lab\_term$ e $lab\_body$, respectivamente. Os casos que lidam com
substituições marcadas são definidos pelas regras de inferência a seguir.
{
    \small
    \begin{mathpar}
        \hspace{ -1.2in } 
        \inferrule*[Right=lab\_subst'\_left]{(\forall\ x,\ x \notin L \rightarrow
            ((lab\_contextual\_closure\ R)\ t^x\ t'^x)),\ (term\ u),\ (SN\ lex\ u)}
        {((lab\_contextual\_closure\ R)\ (t[\![u]\!])\ (t'[\![u]\!]))}
        \\
        \hspace{ -1.2in } 
        \inferrule*[Right=lab\_subst'\_right]{(R\ u\ u'),\
            (lab\_body\ t)}
        {((lab\_contextual\_closure\ R)\ (t[\![u]\!])\ (t[\![u']\!]))}
    \end{mathpar}
}

\bigskip

São definidos também dois outros fechos análogos,
\texttt{simpl\_lab\_contextual\_closure} e
\texttt{ext\_lab\_contextual\_closure}. No primeiro, a diferença é que não é
feita a exigência $(SN\ lex\ u)$ no caso da redução à esquerda da
substituição marcada. Ou seja, temos:

{
    \small
    \begin{mathpar}
        %\hspace{ -0.8in } 
        \inferrule*[right=simpl\_lab\_subst'\_left]{(\forall\ x,\ x \notin l \rightarrow
            ((simpl\_lab\_contextual\_closure\ R)\ t^x\ t'^x)),\ (term\ u) }
        {((simpl\_lab\_contextual\_closure\ R)\ (t[\![u]\!])\ (t'[\![u]\!]))}
        \\
        \hspace{ -1.2in } 
        \inferrule*[right=simpl\_lab\_subst'\_right]{(r\ u\ u'),\
            (lab\_body\ t)}
        {((simpl\_lab\_contextual\_closure\ R)\ (t[\![u]\!])\ (t[\![u']\!]))}
    \end{mathpar}
}

No segundo, queremos reduzir apenas fora de substituições
marcadas. Assim, o fecho não possui um análogo ao caso
\texttt{SIMPL\_LAB\_SUBST'\_RIGHT}.

{
    \small
    \begin{mathpar}
        \hspace{ -1.2in } 
        \inferrule*[Right=ext\_lab\_subst'\_left]{(\forall\ x,\ x \notin L \rightarrow
            ((ext\_lab\_contextual\_closure\ R)\ t^x\ t'^x)),\ (term\ u)}
        {((ext\_lab\_contextual\_closure\ R)\ (t[\![u]\!])\ (t'[\![u]\!]))}
    \end{mathpar}
}


\smallskip

Podemos então definir a relação equacional para termos marcados.

\bigskip

\coqdocnoindent \coqdockw{Inductive} \coqdocvar{lab\_eqc}  : \coqdocvar{pterm}
\ensuremath{\rightarrow} \coqdocvar{pterm} \ensuremath{\rightarrow}
\coqdockw{Prop} := \coqdoceol \coqdocnoindent 
\coqdocindent{1.00em}\ensuremath{|} \coqdocvar{lab\_eqc\_rx1} : \coqdockw{\ensuremath{\forall}}
\coqdocvar{t} \coqdocvar{u} \coqdocvar{v}, \coqdoceol \coqdocindent{2.00em}
\coqdocvar{lab\_term} \coqdocvar{u} \ensuremath{\rightarrow} \coqdocvar{term}
\coqdocvar{v} \ensuremath{\rightarrow} \coqdocvar{lab\_eqc}
(\coqdocvar{t}[\coqdocvar{u}][[\coqdocvar{v}]]) ((\&
\coqdocvar{t})[[\coqdocvar{v}]][\coqdocvar{u}]) \coqdoceol \coqdocnoindent
\coqdocindent{1.00em}\ensuremath{|} \coqdocvar{lab\_eqc\_rx2} : \coqdockw{\ensuremath{\forall}}
\coqdocvar{t} \coqdocvar{u} \coqdocvar{v}, \coqdoceol \coqdocindent{2.00em}
\coqdocvar{term} \coqdocvar{u} \ensuremath{\rightarrow} \coqdocvar{lab\_term}
\coqdocvar{v} \ensuremath{\rightarrow} \coqdocvar{lab\_eqc}
(\coqdocvar{t}[[\coqdocvar{u}]][\coqdocvar{v}]) ((\&
\coqdocvar{t})[\coqdocvar{v}][[\coqdocvar{u}]]) \coqdoceol \coqdocnoindent
\coqdocindent{1.00em}\ensuremath{|} \coqdocvar{lab\_eqc\_rx3} : \coqdockw{\ensuremath{\forall}}
\coqdocvar{t} \coqdocvar{u} \coqdocvar{v}, \coqdoceol \coqdocindent{2.00em}
\coqdocvar{term} \coqdocvar{u} \ensuremath{\rightarrow} \coqdocvar{term}
\coqdocvar{v} \ensuremath{\rightarrow} \coqdocvar{lab\_eqc}
(\coqdocvar{t}[[\coqdocvar{u}]][[\coqdocvar{v}]]) ((\&
\coqdocvar{t})[[\coqdocvar{v}]][[\coqdocvar{u}]]).\coqdoceol

\bigskip

Os construtores basicamente definem como permutar duas substituições, desde que
uma delas seja marcada. Como requisito, é necessário garantir a propriedade
\texttt{lab\_term} ou \texttt{term}, dependendo da substituição. A ideia é
permitir que as substituições marcadas sejam permutadas ``para dentro''\  do termo,
passando por uma substituição comum ou marcada. A substituição normal não pode
atravessar a marcada.

%Para uso futuro, definimos um lema de simetria da relação. Sua prova é feita por
%análise de casos simples, apenas usando o lema \texttt{bswap\_idemp} para
%reduzir permutações de índices que não alteram a semântica do termo.

%\bigskip
%\coqdockw{Lemma} \coqdocvar{bswap\_idemp} : \coqdockw{\ensuremath{\forall}} \coqdocvar{t}, (\& (\& \coqdocvar{t})) = \coqdocvar{t}.\coqdoceol

%\coqdockw{Lemma} \coqdocvar{lab\_eqc\_sym} : \coqdockw{\ensuremath{\forall}}
%\coqdocvar{t} \coqdocvar{u}, \coqdocvar{lab\_eqc} \coqdocvar{t} \coqdocvar{u}
%\ensuremath{\rightarrow} \coqdocvar{lab\_eqc} \coqdocvar{u}
%\coqdocvar{t}.\coqdoceol
%\bigskip

A equação principal utilizada nos termos marcados será a relação 
$=_{\underline{e}}$, formalizada como o fecho contextual e transitivo do
predicado \texttt{lab\_eqc}.

\bigskip
\coqdocnoindent
\coqdockw{Definition} \coqdocvar{lab\_eqC} (\coqdocvar{t}: \coqdocvar{pterm})
(\coqdocvar{u} : \coqdocvar{pterm}) :=  \coqdocvar{trans\_closure}
(\coqdocvar{lab\_contextual\_closure} \coqdocvar{lab\_eqc}) \coqdocvar{t}
\coqdocvar{u} .\coqdoceol \coqdocnoindent \coqdockw{Notation} "t =\~{}e u" :=
(\coqdocvar{lab\_eqC} \coqdocvar{t} \coqdocvar{u}) (\coqdoctac{at}
\coqdockw{level} 66).\coqdoceol
\bigskip


Também é preciso definir um predicado análogo de regularidade para termos
marcados.

\begin{definicao}[Regularidade para termos marcados]
    Sejam $t, u$ pré-termos e $R$ uma relação binária entre pré-termos.
    Dizemos que $R$ é regular se, sempre que vale $R\ t\ u$, então $t$ e $u$ são
    termos marcados bem formados. A noção de regularidade para termos marcados é
    formalizada pelo predicado $red\_lab\_regular$, definido como base na regra
    de inferência abaixo.

\begin{mathpar} 
    \inferrule*{(red\_lab\_regular\ R),\ (R\ t\ u)}
    {lab\_term\ t\ \wedge\ lab\_term\ u}
\end{mathpar}
\end{definicao}

\begin{definicao}[Regularidade fraca para termos marcados]
    Sejam $t, u$ pré-termos e $R$ uma relação binária entre pré-termos.
    Dizemos que $R$ satisfaz a regularidade fraca se, sempre que vale $R\ t\ u$,
    vale $(lab\_term\ t) \iff (lab\_term\ u)$. A noção de regularidade fraca é
    formalizada pelo predicado $red\_lab\_regular'$, definido como base na regra de
    inferência abaixo.
    
\begin{mathpar} 
    \inferrule*{(red\_lab\_regular'\ R),\ (R\ t\ u)}
    {lab\_term\ t\ \iff lab\_term\ u}
\end{mathpar}
\end{definicao}


Com todas as estruturas e propriedades para termos marcados bem definidos,
podemos seguir com a prova da propriedade IE.


\subsection{Equivalência de reduções com o sistema original}
\label{sub:equiv_red}

Queremos utilizar esse sistema estendido com as substituições marcadas para
estudar o sistema original. Para isto, precisamos estender a regra de redução do
sistema para lidar com as novas substituições. Definimos então a redução
$\rightarrow_{\underline{x}}$, como na tabela \ref{table:red_label_x}.

\begin{table}[h]
\begin{empheq}[box=\fbox]{align*}
    x[\![x/u]\!]\ \ \             &\rightarrow_{Var}\ u \\
    t[\![x/u]\!]\ \ \             &\rightarrow_{Gc}\ t    & se\ \emph{x} \notin fv(t)\\
    (t\ u)[\![x/v]\!]\ \ \        &\rightarrow_{App}\ t[\![x/v]\!]\ u[\![x/v]\!] \\
    (\lambda y.\ u)[\![x/v]\!]\ \ &\rightarrow_{Lamb}\ (\lambda y.\ u[\![x/v]\!])\\
    t[\![x/u]\!][\![y/v]\!]\ \ \        &\rightarrow_{Comp}\ t[\![y/v]\!][\![x/u[\![y/v]\!]]\!] & se\ y\ \in
    fv(u)  
\end{empheq}
    \caption{A redução $\rightarrow$$_{\underline{x}}$ }
    \label{table:red_label_x}

\end{table}

Assim, a relação $\rightarrow_{\lambda \underline{ex}}$ é definida como a união
das reduções $\rightarrow_{Bx}$ e $\rightarrow_{\underline{x}}$, módulo
$=_{\alpha}$, $=_e$ e $=_{\underline{e}}$:

\[ t \rightarrow_{\lambda \underline{ex}} t' \iff \exists s,\ s';\ t =_{e \cup
        \underline{e} \cup \alpha} s \rightarrow_{Bx \cup \underline{x}} s' =_{e \cup
        \underline{e} \cup \alpha} t' \] 

Para provar a propriedade PSN, será necessário relacionar a redução
$\rightarrow_{\lambda \underline{ex}}$ com a redução original,
$\rightarrow_{\lambda ex}$. Para isto, iremos decompor a redução em termos
marcados em duas novas reduções, $\rightarrow_{\lambda \underline{ex}^i}$ e
$\rightarrow_{\lambda \underline{ex}^e}$, que também agem em termos marcados.

\begin{definicao}[Redução interna]
    A relação $\lambda \underline{ex}^i$, chamada de \emph{redução interna}, é
    definida adicionando à redução $\rightarrow_{\underline{ex}}$ a redução
    $\rightarrow_{\lambda ex}$ no corpo das substituições marcadas.
    Formalmente, a relação $\rightarrow_{\lambda \underline{ex}^i}$ é definida
    como a seguinte redução, $\rightarrow_{\lambda \underline{x}^i}$, módulo
    $=_{\alpha}$, $=_e$ e $=_{\underline{e}}$:

\begin{itemize}
    \item Se $u \rightarrow_{Bx} u'$ e $u,\ u'$ são termos, então $t[\![x/u]\!]
        \rightarrow_{\lambda \underline{x}^i} t[\![x/u']\!]$ 
    \item Se $t
        \rightarrow_{\underline{x}} t'$, então $t \rightarrow_{\lambda
            \underline{x}^i} t'$
    \item Se $t \rightarrow_{\lambda \underline{x}^i} t'$, então vale 
        $t\ u \rightarrow_{\lambda \underline{x}^i} t'\ u$,  
        $u\ t \rightarrow_{\lambda \underline{x}^i} u\ t'$, 
        $\lambda x. t \rightarrow_{\lambda \underline{x}^i} \lambda x. t'$, 
        $t[x/u] \rightarrow_{\lambda \underline{x}^i} t'[x/u]$, 
        $u[x/t] \rightarrow_{\lambda \underline{x}^i} u[x/t']$ e 
        $t[\![x/u]\!] \rightarrow_{\lambda \underline{x}^i} t'[\![x/u]\!]$.
\end{itemize}
\end{definicao}


\begin{definicao}[Redução externa]
    A relação $\lambda \underline{ex}^e$, chamada de \emph{redução externa}, é
    definida como a redução $\lambda ex$ em todos os lugares de um termo,
    \emph{exceto} no corpo das substituições marcadas.  Formalmente, a relação
    $\rightarrow_{\lambda \underline{ex}^e}$ é definida como a seguinte redução,
    $\rightarrow_{\lambda \underline{x}^e}$, módulo $=_{\alpha}$, $=_e$ e
    $=_{\underline{e}}$:

\begin{itemize}
    \item Se $t \rightarrow_{Bx} t'$ ocorre fora de uma substituição marcada, então 
        $t \rightarrow_{\lambda \underline{x}^e} t$ 
    \item Se $t \rightarrow_{\lambda \underline{x}^e} t'$, então vale 
        $t\ u \rightarrow_{\lambda \underline{x}^e} t'\ u$,  
        $u\ t \rightarrow_{\lambda \underline{x}^e} u\ t'$, 
        $\lambda x. t \rightarrow_{\lambda \underline{x}^e} \lambda x. t'$, 
        $t[x/u] \rightarrow_{\lambda \underline{x}^e} t'[x/u]$, 
        $u[x/t] \rightarrow_{\lambda \underline{x}^e} u[x/t']$ e 
        $t[\![x/u]\!] \rightarrow_{\lambda \underline{x}^e} t'[\![x/u]\!]$.
\end{itemize}
\end{definicao}

O objetivo principal deste trabalho será a formalização destas duas reduções e a
prova de equivalências da união destas com a redução ``\texttt{-{}->[lex]}'', que é a
redução $\rightarrow_{\lambda \underline{ex}}$ formalizada. Em outras palavras, queremos
provar o seguinte Teorema:

\begin{teorema}[Equivalência entre as reduções]\label{teo:main}
    Seja $t$ um termo bem formado, podendo possuir substituições marcadas. Então
    vale que $t \rightarrow_{\lambda \underline{ex}} t' \iff t
    \rightarrow_{\lambda \underline{ex}^i \cup \lambda \underline{ex}^e} t'$. 
\end{teorema}

Para iniciar a formalização, definimos mais uma equação, $=_{EE}$, que serve
como união de ambas as equações anteriores, $=_e e =_{\underline{e}}$:

\bigskip

\coqdocnoindent \coqdockw{Definition} \coqdocvar{eqcc} \coqdocvar{t}
\coqdocvar{t'} := \coqdocvar{eqc} \coqdocvar{t} \coqdocvar{t'} \ensuremath{\lor}
\coqdocvar{lab\_eqc} \coqdocvar{t} \coqdocvar{t'}.\coqdoceol \coqdocnoindent
\coqdockw{Notation} "t =ee t'" := (\coqdocvar{eqcc} \coqdocvar{t}
\coqdocvar{t'}) (\coqdoctac{at} \coqdockw{level} 66).\coqdoceol \coqdocemptyline


\coqdocnoindent \coqdockw{Definition} \coqdocvar{star\_ctx\_eqcc}
(\coqdocvar{t}: \coqdocvar{pterm}) (\coqdocvar{u} : \coqdocvar{pterm}) :=
\coqdocvar{star\_closure} (\coqdocvar{simpl\_lab\_contextual\_closure}
\coqdocvar{eqcc}) \coqdocvar{t} \coqdocvar{u} .\coqdoceol \coqdocnoindent
\coqdockw{Notation} "t =EE u" := (\coqdocvar{star\_ctx\_eqcc} \coqdocvar{t}
\coqdocvar{u}) (\coqdoctac{at} \coqdockw{level} 66).\coqdoceol \coqdocemptyline


Como primeiro passo para a formalização deste teorema, devemos definir análogos
para as reduções $\lambda \underline{ex}$, $\lambda \underline{ex}^i$, $\lambda
\underline{ex}^e$.

Iniciamos, então, definindo a relação $\rightarrow_{\underline{x}}$, dada pela
tabela \ref{table:red_label_x}, como o tipo \texttt{lab\_sys\_x}:

\bigskip

\coqdocnoindent \coqdockw{Inductive} \coqdocvar{lab\_sys\_x} : \coqdocvar{pterm}
\ensuremath{\rightarrow} \coqdocvar{pterm} \ensuremath{\rightarrow}
\coqdockw{Prop} :=\coqdoceol \coqdocnoindent \ensuremath{|}
\coqdocvar{lab\_reg\_rule\_var} : \coqdockw{\ensuremath{\forall}} \coqdocvar{t},
\coqdocvar{lab\_term} (\coqdocvar{pterm\_bvar} 0 [[\coqdocvar{t}]])
\ensuremath{\rightarrow} \coqdocvar{lab\_sys\_x} (\coqdocvar{pterm\_bvar} 0
[[\coqdocvar{t}]]) \coqdocvar{t}\coqdoceol \coqdocnoindent 
\coqdocnoindent \ensuremath{|} \coqdocvar{lab\_reg\_rule\_gc} :
\coqdockw{\ensuremath{\forall}} \coqdocvar{t} \coqdocvar{u},
\coqdocvar{lab\_term} \coqdocvar{t} \ensuremath{\rightarrow}
\coqdocvar{lab\_term} (\coqdocvar{t}[[\coqdocvar{u}]]) \ensuremath{\rightarrow}
\coqdocvar{lab\_sys\_x} (\coqdocvar{t}[[\coqdocvar{u}]]) \coqdocvar{t}\coqdoceol
\coqdocnoindent \coqdocnoindent \ensuremath{|}
\coqdocvar{lab\_reg\_rule\_app} : \coqdockw{\ensuremath{\forall}} \coqdocvar{t1}
\coqdocvar{t2} \coqdocvar{u}, \coqdocvar{lab\_term}
(\coqdocvar{t1}[[\coqdocvar{u}]]) \ensuremath{\rightarrow} \coqdocvar{lab\_term}
(\coqdocvar{t2}[[\coqdocvar{u}]]) \ensuremath{\rightarrow}\coqdoceol
\coqdocindent{1.00em} \coqdocvar{lab\_sys\_x} ((\coqdocvar{pterm\_app}
\coqdocvar{t1} \coqdocvar{t2})[[\coqdocvar{u}]]) (\coqdocvar{pterm\_app}
(\coqdocvar{t1}[[\coqdocvar{u}]]) (\coqdocvar{t2}[[\coqdocvar{u}]]))
\coqdocnoindent \coqdoceol \coqdocnoindent \ensuremath{|}
\coqdocvar{lab\_reg\_rule\_lamb} : \coqdockw{\ensuremath{\forall}} \coqdocvar{t}
\coqdocvar{u}, \coqdocvar{lab\_term} ((\coqdocvar{pterm\_abs}
\coqdocvar{t})[[\coqdocvar{u}]]) \ensuremath{\rightarrow} \coqdoceol
\coqdocindent{1.00em} \coqdocvar{lab\_sys\_x} ((\coqdocvar{pterm\_abs}
\coqdocvar{t})[[\coqdocvar{u}]]) (\coqdocvar{pterm\_abs} ((\&
\coqdocvar{t})[[\coqdocvar{u}]]))\coqdoceol \coqdocnoindent 
\coqdocnoindent \ensuremath{|} \coqdocvar{lab\_reg\_rule\_comp} :
\coqdockw{\ensuremath{\forall}} \coqdocvar{t} \coqdocvar{u} \coqdocvar{v},
\coqdocvar{lab\_term} ((\coqdocvar{t}[\coqdocvar{u}])[[\coqdocvar{v}]])
\ensuremath{\rightarrow} \ensuremath{\lnot} \coqdocvar{term} \coqdocvar{u}
\ensuremath{\rightarrow} \coqdoceol \coqdocindent{1.00em}
\coqdocvar{lab\_sys\_x} (\coqdocvar{t}[\coqdocvar{u}][[\coqdocvar{v}]]) (((\&
\coqdocvar{t})[[\coqdocvar{v}]])[\coqdocvar{u}[[\coqdocvar{v}]]]).\coqdoceol
\coqdoceol
\coqdocnoindent \coqdockw{Notation} "t ->\_lab\_x u" := (\coqdocvar{lab\_sys\_x}
\coqdocvar{t} \coqdocvar{u}) (\coqdoctac{at} \coqdockw{level} 59,
\coqdoctac{left} \coqdockw{associativity}).\coqdoceol \coqdocemptyline

\bigskip

Agora já podemos definir o análogo da relação $\lambda \underline{ex}$.
Para isto, começamos com a formalização da relação $\rightarrow_{Bx \cup
    \underline{x}}$, no tipo indutivo \texttt{lab\_sys\_lx}:

\bigskip

\coqdocnoindent \coqdockw{Inductive} \coqdocvar{lab\_sys\_lx}: \coqdocvar{pterm}
\ensuremath{\rightarrow} \coqdocvar{pterm} \ensuremath{\rightarrow}
\coqdockw{Prop} :=\coqdoceol \coqdocnoindent \ensuremath{|} \coqdocvar{B\_lx} :
\coqdockw{\ensuremath{\forall}} \coqdocvar{t} \coqdocvar{u}, \coqdocvar{t}
\ensuremath{\rightarrow}\coqdocvar{\_B} \coqdocvar{u} \ensuremath{\rightarrow}
\coqdocvar{lab\_sys\_lx} \coqdocvar{t} \coqdocvar{u}\coqdoceol \coqdocnoindent
\ensuremath{|} \coqdocvar{sys\_x\_lx} : \coqdockw{\ensuremath{\forall}}
\coqdocvar{t} \coqdocvar{u}, \coqdocvar{t}
\ensuremath{\rightarrow}\coqdocvar{\_x} \coqdocvar{u} \ensuremath{\rightarrow}
\coqdocvar{lab\_sys\_lx} \coqdocvar{t} \coqdocvar{u}\coqdoceol \coqdocnoindent
\ensuremath{|} \coqdocvar{sys\_x\_lab\_lx} : \coqdockw{\ensuremath{\forall}}
\coqdocvar{t} \coqdocvar{u}, \coqdocvar{t}
\ensuremath{\rightarrow}\coqdocvar{\_lab\_x} \coqdocvar{u}
\ensuremath{\rightarrow} \coqdocvar{lab\_sys\_lx} \coqdocvar{t}
\coqdocvar{u}.\coqdoceol \coqdocemptyline

\bigskip

Assim, o análogo do sistema da relação $\lambda \underline{ex}$ será
"\texttt{-{}->[lex]}", dado por:

\bigskip

\coqdocnoindent \coqdockw{Definition} \coqdocvar{lab\_lex} (\coqdocvar{t}:
\coqdocvar{pterm}) (\coqdocvar{u} : \coqdocvar{pterm}) := \coqdoceol
\coqdocindent{2.00em} \coqdoctac{\ensuremath{\exists}} \coqdocvar{t'}
\coqdocvar{u'}, (\coqdocvar{t} =\coqdocvar{EE}
\coqdocvar{t'})/\symbol{92}(\coqdocvar{lab\_contextual\_closure}
\coqdocvar{lab\_sys\_lx} \coqdocvar{t'}
\coqdocvar{u'})/\symbol{92}(\coqdocvar{u'} =\coqdocvar{EE}
\coqdocvar{u}).\coqdoceol \coqdocemptyline

\coqdocnoindent
\coqdockw{Notation} "t -->[lex] u" := (\coqdocvar{lab\_lex} \coqdocvar{t} \coqdocvar{u}) (\coqdoctac{at} \coqdockw{level} 59, \coqdoctac{left} \coqdockw{associativity}).\coqdoceol

\bigskip

Em seguida, iremos definir a redução externa. Ela consistirá apenas da aplicação
da regra do sistema original, $\rightarrow_{Bx}$, em qualquer parte de um termo
que seja fora de uma substituição explícita. Em outras palavras, ela será o
fecho contextual externo da relação $\rightarrow_{Bx}$, módulo $=_{EE}$.

\bigskip
\coqdocnoindent \coqdockw{Notation} "t -->[lx\_e] u" :=
(\coqdocvar{ext\_lab\_EE\_ctx\_red} \coqdocvar{sys\_Bx} \coqdocvar{t} \coqdocvar{u}) .
\coqdoceol
\bigskip

O fecho \texttt{ext\_lab\_EE\_ctx\_red} é o que realiza uma relação em qualquer
ponto de um termo, \textbf{exceto} dentro de uma substituição marcada, permitindo
a permutação de substituições tanto antes quanto depois da redução ser feita.
Ele realiza isto utilizando o fecho \texttt{ext\_lab\_contextual\_closure}
e aplicando a equação \texttt{=EE}.

\bigskip

\coqdocnoindent
\coqdockw{Definition} \coqdocvar{ext\_lab\_EE\_ctx\_red} (\coqdocvar{R}:
\coqdocvar{pterm} \ensuremath{\rightarrow} \coqdocvar{pterm}
\ensuremath{\rightarrow} \coqdockw{Prop}) (\coqdocvar{t}: \coqdocvar{pterm})
(\coqdocvar{u} : \coqdocvar{pterm}) := 
\coqdocindent{4.00em}
\coqdoctac{\ensuremath{\exists}} \coqdocvar{t'} \coqdocvar{u'}, (\coqdocvar{t}
=\coqdocvar{EE}
\coqdocvar{t'})/\symbol{92}(\coqdocvar{ext\_lab\_contextual\_closure}
\coqdocvar{R} \coqdocvar{t'} \coqdocvar{u'})/\symbol{92}(\coqdocvar{u'}
=\coqdocvar{EE} \coqdocvar{u}).\coqdoceol \coqdocemptyline \coqdocnoindent

\bigskip


Para definir a redução interna, precisamos do predicado
\texttt{lab\_x\_i}, que irá permitir aplicações de $\rightarrow_{Bx}$ apenas
dentro de substituições marcadas, e aplicações de $\rightarrow_{\underline{x}}$
no resto do corpo do termo.

\bigskip

\coqdocnoindent\coqdockw{Inductive} \coqdocvar{lab\_x\_i}: \coqdocvar{pterm}
\ensuremath{\rightarrow} \coqdocvar{pterm} \ensuremath{\rightarrow}
\coqdockw{Prop} :=\coqdoceol \coqdocnoindent \ensuremath{|}
\coqdocvar{xi\_from\_bx\_in\_les}: \coqdockw{\ensuremath{\forall}}
\coqdocvar{t1} \coqdocvar{t2} \coqdocvar{t2'}, \coqdoceol \coqdocindent{11.50em}
\coqdocvar{lab\_term} (\coqdocvar{t1} [[ \coqdocvar{t2} ]])
\ensuremath{\rightarrow}\coqdoceol \coqdocindent{11.50em} (\coqdocvar{sys\_Bx}
\coqdocvar{t2} \coqdocvar{t2'}) \ensuremath{\rightarrow}\coqdoceol
\coqdocindent{11.50em} \coqdocvar{lab\_x\_i} (\coqdocvar{t1} [[ \coqdocvar{t2}
]]) (\coqdocvar{t1} [[ \coqdocvar{t2'} ]])\coqdoceol \coqdocnoindent
\ensuremath{|} \coqdocvar{xi\_from\_x} : \coqdockw{\ensuremath{\forall}}
\coqdocvar{t} \coqdocvar{t'}, \coqdoceol \coqdocindent{8.00em}
\coqdocvar{lab\_term} \coqdocvar{t} \ensuremath{\rightarrow}\coqdoceol
\coqdocindent{8.00em} \coqdocvar{lab\_sys\_x} \coqdocvar{t} \coqdocvar{t'}
\ensuremath{\rightarrow} \coqdoceol \coqdocindent{8.00em} \coqdocvar{lab\_x\_i}
\coqdocvar{t} \coqdocvar{t'}.\coqdoceol 

\bigskip


Podemos agora formalizar a redução interna:

\bigskip

\coqdocemptyline \coqdocnoindent \coqdockw{Notation} "t -->[lx\_i] u" :=
(\coqdocvar{ext\_lab\_EE\_ctx\_red} \coqdocvar{lab\_x\_i} \coqdocvar{t} \coqdocvar{u}) .

\bigskip

Assim, $\lambda \underline{ex}^i$ é formalizada como a redução "t -{}->[lx\_i]
u". Por outro lado, a relação $\lambda \underline{ex}^e$ formalizada como a
redução "t -{}->[lx\_e] u".

\coqdocemptyline 
%\coqdocnoindent
%\coqdockw{Definition} \coqdocvar{lab\_EE\_ctx\_red} (\coqdocvar{R}:
%\coqdocvar{pterm} \ensuremath{\rightarrow} \coqdocvar{pterm}
%\ensuremath{\rightarrow} \coqdockw{Prop}) (\coqdocvar{t}: \coqdocvar{pterm})
%(\coqdocvar{u} : \coqdocvar{pterm}) := \coqdoceol \coqdocindent{4.00em}
%\coqdoctac{\ensuremath{\exists}} \coqdocvar{t'} \coqdocvar{u'}, (\coqdocvar{t}
%=\coqdocvar{EE} \coqdocvar{t'})/\symbol{92}(\coqdocvar{lab\_contextual\_closure}
%\coqdocvar{R} \coqdocvar{t'} \coqdocvar{u'})/\symbol{92}(\coqdocvar{u'}
%=\coqdocvar{EE} \coqdocvar{u}).\coqdoceol \coqdocemptyline \coqdocnoindent


Como consequência dos diversos fechos e equações utilizadas, a manipulação dos
construtores da redução se torna trabalhosa, pois precisamos lidar com
quantificadores existenciais, conjunções, fechos transitivos, fechos
contextuais, etc.

Uma estratégia utilizada para facilitar as provas foi abstrair, em lemas
auxiliares, vários problemas destas que se repetem. Podemos então tratar tais
problemas com um contexto limpo, facilitando as provas por indução. 

%\bigskip
%\coqdockw{Lemma} \coqdocvar{star\_lab\_closure\_app\_left}:
%\coqdockw{\ensuremath{\forall}} \coqdocvar{R} \coqdocvar{t} \coqdocvar{t'}
%\coqdocvar{u}, \coqdocvar{lab\_term} \coqdocvar{u} \ensuremath{\rightarrow} \\
%\coqdocvar{star\_closure} (\coqdocvar{simpl\_lab\_contextual\_closure}
%\coqdocvar{R}) \coqdocvar{t} \coqdocvar{t'} \ensuremath{\rightarrow}
%\coqdocvar{star\_closure} (\coqdocvar{simpl\_lab\_contextual\_closure}
%\coqdocvar{R}) (\coqdocvar{pterm\_app} \coqdocvar{t} \coqdocvar{u})
%(\coqdocvar{pterm\_app} \coqdocvar{t'} \coqdocvar{u}).\coqdoceol

%\smallskip

%\coqdockw{Lemma} \coqdocvar{EE\_clos\_app\_left}:
%\coqdockw{\ensuremath{\forall}} \coqdocvar{R} \coqdocvar{t} \coqdocvar{t'}
%\coqdocvar{u}, \coqdocvar{lab\_term} \coqdocvar{u} \ensuremath{\rightarrow}
%((\coqdocvar{lab\_EE\_ctx\_red} \coqdocvar{R}) \coqdocvar{t} \coqdocvar{t'})
%\ensuremath{\rightarrow} ((\coqdocvar{lab\_EE\_ctx\_red} \coqdocvar{R})
%(\coqdocvar{pterm\_app} \coqdocvar{t} \coqdocvar{u}) (\coqdocvar{pterm\_app}
%\coqdocvar{t'} \coqdocvar{u})).\coqdoceol

%\smallskip

%\coqdockw{Lemma} \coqdocvar{EE\_ext\_clos\_app\_left}:
%\coqdockw{\ensuremath{\forall}} \coqdocvar{R} \coqdocvar{t} \coqdocvar{t'}
%\coqdocvar{u}, \coqdocvar{lab\_term} \coqdocvar{u} \ensuremath{\rightarrow}
%((\coqdocvar{ext\_lab\_EE\_ctx\_red} \coqdocvar{R}) \coqdocvar{t}
%\coqdocvar{t'}) \ensuremath{\rightarrow} ((\coqdocvar{ext\_lab\_EE\_ctx\_red}
%\coqdocvar{R}) (\coqdocvar{pterm\_app} \coqdocvar{t} \coqdocvar{u})
%(\coqdocvar{pterm\_app} \coqdocvar{t'} \coqdocvar{u})).\coqdoceol
%\bigskip

Para cada fecho, definimos um lema para se realizar a redução em um termo maior,
a partir da redução em um subtermo. Estes lemas são úteis para se evitar que
tenhamos que destrinchar as reduções dentro da prova principal, facilitando
muito o processo.  Para cada construtor de termo, temos um lema para associar o
construtor a um fecho, mais especificamente os fechos
\texttt{lab\_EE\_ctx\_red}, \texttt{ext\_lab\_EE\_ctx\_red} e também para o
fecho transitivo-reflexivo de \texttt{lab\_contextual\_closure}. 

A maioria destes lemas são simples de se resolver. Como de costume, os casos que
lidam com abstrações e substituições precisam de um cuidado especial. Neles,
precisamos assumir que valem os predicados \texttt{red\_rename R} e
\texttt{red\_lab\_regular' R} para a relação \texttt{R} sobre o qual estamos
realizando o fecho. Isto porque iniciamos as provas escolhendo uma variável
"nova", que não ocorre em nenhum dos termos do contexto. Isto auxilia a
lidar com a quantificação necessária nas hipóteses de indução e construtores, em
que precisamos de uma variável nova em um conjunto $L$. Precisamos então
utilizar as propriedades de renomeamento e regularidade para poder substituir
essa variável.

Além disso, alguns lemas para lidar com a relação entre as reduções e equações
foram necessários:

\bigskip
\coqdockw{Lemma} \coqdocvar{EE\_presv\_ie}: \coqdockw{\ensuremath{\forall}}
\coqdocvar{t} \coqdocvar{t'} \coqdocvar{u} \coqdocvar{u'}, \coqdocvar{t}
=\coqdocvar{EE} \coqdocvar{u} \ensuremath{\rightarrow} \coqdocvar{u'}
=\coqdocvar{EE} \coqdocvar{t'} \ensuremath{\rightarrow} ((\coqdocvar{u}
-->[\coqdocvar{lx\_i}] \coqdocvar{u'} \ensuremath{\lor} \coqdocvar{u}
-->[\coqdocvar{lx\_e}] \coqdocvar{u'}) \ensuremath{\rightarrow} (\coqdocvar{t}
-->[\coqdocvar{lx\_i}] \coqdocvar{t'} \ensuremath{\lor} \coqdocvar{t}
-->[\coqdocvar{lx\_e}] \coqdocvar{t'})).\coqdoceol

\smallskip

\coqdockw{Lemma} \coqdocvar{EE\_presv\_lab\_lex}:
\coqdockw{\ensuremath{\forall}} \coqdocvar{t} \coqdocvar{t'} \coqdocvar{u}
\coqdocvar{u'}, \coqdocvar{t} =\coqdocvar{EE} \coqdocvar{u}
\ensuremath{\rightarrow} \coqdocvar{u'} =\coqdocvar{EE} \coqdocvar{t'}
\ensuremath{\rightarrow} ((\coqdocvar{u} -->[\coqdocvar{lex}] \coqdocvar{u'})
\ensuremath{\rightarrow} (\coqdocvar{t} -->[\coqdocvar{lex}]
\coqdocvar{t'})).\coqdoceol
\bigskip


Novamente, esses lemas evitam que precisemos adentrar na definição das
equações, reduzindo o tamanho das provas principais, e são facilmente provados
observando a transitividade da relação de equivalência.

\bigskip

Podemos então partir para o resultado principal deste trabalho, ou seja, a prova
do Teorema \ref{teo:main}, formalizado como o teorema
\texttt{lab\_ex\_eq\_i\_e}.

\bigskip

\coqdockw{Theorem} \coqdocvar{lab\_ex\_eq\_i\_e}:
\coqdockw{\ensuremath{\forall}} \coqdocvar{t} \coqdocvar{t'},
\coqdocvar{lab\_term} \coqdocvar{t} \ensuremath{\rightarrow} (\coqdocvar{t}
-->[\coqdocvar{lex}] \coqdocvar{t'} \ensuremath{\leftrightarrow} (\coqdocvar{t}
-->[\coqdocvar{lx\_i}] \coqdocvar{t'} \ensuremath{\lor} \coqdocvar{t}
-->[\coqdocvar{lx\_e}] \coqdocvar{t'})).\coqdoceol

\bigskip

Dividimos a prova em dois lemas, cada um representado uma direção da
equivalência.

\bigskip

\coqdocnoindent \coqdockw{Lemma} \coqdocvar{lab\_ex\_impl\_i\_e}:
\coqdockw{\ensuremath{\forall}} \coqdocvar{t} \coqdocvar{t'},
\coqdocvar{lab\_term} \coqdocvar{t} \ensuremath{\rightarrow} \coqdocvar{t}
-->[\coqdocvar{lex}] \coqdocvar{t'} \ensuremath{\rightarrow} (\coqdocvar{t}
-->[\coqdocvar{lx\_i}] \coqdocvar{t'} \ensuremath{\lor} \coqdocvar{t}
-->[\coqdocvar{lx\_e}] \coqdocvar{t'}).\coqdoceol

\begin{addmargin}[1em]{2em}
\begin{proof}
Escolhemos, para cada redução possível feita pela relação
\texttt{-{}->[lex]}, a redução apropriada entre a interna e a externa.  Para
isto, abrimos a definição da relação \texttt{-{}->[lex]} e fazemos indução no
fecho contextual, ou seja, fazemos indução no predicado
\texttt{lab\_contextual\_closure lab\_sys\_lx t t'}. O caso base é tratado no
lema auxiliar \texttt{lab\_sys\_x\_i\_e}, que relaciona a relação
\texttt{lab\_sys\_lx} com as relações interna e externa, e é feito com análise
de casos simples nos construtores da relação \texttt{lab\_sys\_lx}. 

Nos passos indutivos, utilizamos o lema \texttt{EE\_presv\_ie} para adequar o
objetivo à hipótese de indução, substituindo os termos dados pelos termos
equivalentes, obtidos pela definição da redução \texttt{-{}->[lex]}, podendo assim
aplicar a hipótese. No caso em que a redução é feita dentro da substituição
marcada, devemos obrigatoriamente realizar a redução interna. Em todos os outros
casos, realizamos a prova tanto para a redução interna quanto para a externa.
Nos casos em que lidamos com abstrações e substituições, são necessários os
lemas de renomeamento mencionados na subseção \ref{sub:subst_marc}.
\end{proof}

\end{addmargin}

\bigskip

\coqdocnoindent \coqdockw{Lemma} \coqdocvar{lab\_ie\_impl\_ex}:
\coqdockw{\ensuremath{\forall}} \coqdocvar{t} \coqdocvar{t'},
\coqdocvar{lab\_term} \coqdocvar{t} \ensuremath{\rightarrow} (\coqdocvar{t}
-{}->[\coqdocvar{lx\_i}] \coqdocvar{t'} \ensuremath{\lor} \coqdocvar{t}
-{}->[\coqdocvar{lx\_e}] \coqdocvar{t'}) \ensuremath{\rightarrow} \coqdocvar{t}
-{}->[\coqdocvar{lex}] \coqdocvar{t'}.\coqdoceol

\begin{addmargin}[1em]{2em}
\begin{proof}
A prova deste lema é dividida em duas partes: quando a redução realizada é a
interna, e quando é a externa. 

No caso da interna, a indução é feita no fecho do predicado interno, ou seja, é
feita no predicado \texttt{ext\_lab\_contextual\_closure lab\_x\_i t t'}. O caso
base é feito apenas analisando os construtores da relação \texttt{lab\_x\_i}, e
casando com o construtor adequado de lab\_sys\_lx. Nos passos indutivos, fazemos
de maneira análoga ao lema anterior: utilizamos agora o predicado
\texttt{EE\_presv\_lab\_lex} para ajustar o objetivo à hipótese, e utilizamos os
lemas relacionando as reduções e fechos aos construtores, mencionados acima,
para reduzir o objetivo à redução dada como hipótese. Novamente, no caso de
abstrações e substituições, precisamos dos lemas de renomeamento.

No caso da externa, o processo é exatamente o mesmo. A diferença é que no passo
base, fazemos a análise de casos na redução \texttt{sys\_Bx} e, além disso, não
temos que lidar com o caso da redução dentro de substituição marcada.
\end{proof}

\end{addmargin}

\bigskip

Com ambos os lemas completos, a prova do teorema \texttt{lab\_ex\_eq\_i\_e} se
reduz a apenas aplicá-los, como visto a seguir. 

\bigskip \coqdockw{Theorem} \coqdocvar{lab\_ex\_eq\_i\_e}:
\coqdockw{\ensuremath{\forall}} \coqdocvar{t} \coqdocvar{t'},
\coqdocvar{lab\_term} \coqdocvar{t} \ensuremath{\rightarrow} (\coqdocvar{t}
-->[\coqdocvar{lex}] \coqdocvar{t'} \ensuremath{\leftrightarrow} (\coqdocvar{t}
-->[\coqdocvar{lx\_i}] \coqdocvar{t'} \ensuremath{\lor} \coqdocvar{t}
-->[\coqdocvar{lx\_e}] \coqdocvar{t'})).\coqdoceol \coqdocnoindent
\coqdockw{Proof}.\coqdoceol \coqdocindent{2.00em} \coqdoctac{split}.\coqdoceol
\coqdocindent{2.00em} \coqdoctac{intros}; \coqdoctac{apply}
\coqdocvar{lab\_ex\_impl\_i\_e}; \coqdoctac{auto}.\coqdoceol
\coqdocindent{2.00em} \coqdoctac{intros}; \coqdoctac{apply}
\coqdocvar{lab\_ie\_impl\_ex}; \coqdoctac{auto}.\coqdoceol \coqdocnoindent
\coqdockw{Qed}.\coqdoceol \bigskip

Assim, terminamos a formalização da prova de que a redução do sistema com
substituições marcadas, $\lambda \underline{ex}$, é equivalente à união das reduções
interna e externa, $\lambda \underline{ex}^i$ e $\lambda \underline{ex}^e$.

  \chapter{Conclusão}

Cálculos de substituição explícita são de interesse prático por servirem como um
framework formal para o estudo de propriedades de sistemas reais, como
implementações de linguagens funcionais e assistentes de prova \cite{levy1999}.
Desta forma, uma formalização se torna interessante pois fornece uma maior
garantia na confiabilidade destes sistemas. Estes também são importantes no
estudo de propriedades do próprio cálculo $\lambda$ \cite{ben_cbv, ben_beta},
pois algumas destas se tornam mais fáceis de serem provadas em um sistema com
substituições explícitas, bastando assim mostrar a preservação da propriedade
entre os cálculos.

Neste trabalho, foi continuada a formalização \footnote{ A versão atual desta
    formalização está disponível em
    \url{https://github.com/Lucas1993/LambdaEX_TCC}} do cálculo $\lambda ex$,
iniciada em \cite{initial}. Em um primeiro momento, foi dado enfoque na
continuação da construção da teoria, em especial realizando provas de
preservação de propriedades do cálculo pelas classes de equivalência. Após isto,
definimos, dentro do sistema, todas as definições e propriedades relacionadas às
substituições marcadas, e provamos suas características principais. Iniciamos,
então, a prova da propriedade IE, realizando a prova formal da equivalência
entre a redução do sistema estendida para substituições marcadas, $\lambda
\underline{ex}$, e a união das reduções interna e externa, $\lambda
\underline{ex}^i$ e $\lambda \underline{ex}^e$.

Como trabalho futuro, queremos estudar propriedades de normalização do sistema,
procurando uma definição mais intuitiva para o conceito de \textbf{normalização
forte}. Com isto, poderemos concluir a prova da propriedade IE dentro do
sistema, finalizando, assim, a formalização do cálculo $\lambda ex$.


  % ...

  \postextual
  \bibliographystyle{abbrv}
  \bibliography{bibliografia}

\end{document}

%%%%%%%%%%%%%%%%%%%%%%%%%%%%%%%%%%%%%%%%
% Classe do documento
%%%%%%%%%%%%%%%%%%%%%%%%%%%%%%%%%%%%%%%%

% Nós usamos a classe "unb-cic".  Deixe apenas uma das linhas
% abaixo não-comentada, dependendo se você for do bacharelado ou
% da licenciatura.

\documentclass[bacharelado]{unb-cic}
%\documentclass[licenciatura]{unb-cic}

\nonstopmode

%%%%%%%%%%%%%%%%%%%%%%%%%%%%%%%%%%%%%%%%
% Pacotes importados
%%%%%%%%%%%%%%%%%%%%%%%%%%%%%%%%%%%%%%%%

\usepackage[brazil,american]{babel}
\usepackage[T1]{fontenc}
\usepackage{indentfirst}
\usepackage{natbib}
\usepackage{xcolor,graphicx,url}
\usepackage{amsmath}
\usepackage{amssymb}
\usepackage{amsthm}
\usepackage{empheq}
\usepackage{mathpartir}
\usepackage{listings}
\usepackage{scrextend}
\usepackage[all]{xy}
\usepackage[color]{coqdoc}
\usepackage[utf8]{inputenc}


\lstset{frame = single, aboveskip=20pt, belowskip=20pt }

%%%%%%%%%%%%%%%%%%%%%%%%%%%%%%%%%%%%%%%%
% Definições
%%%%%%%%%%%%%%%%%%%%%%%%%%%%%%%%%%%%%%%%

\newtheorem{teorema}{Teorema}
\newtheorem{lema}{Lema}[chapter]
\newtheorem{definicao}{Definição}[chapter]

%%%%%%%%%%%%%%%%%%%%%%%%%%%%%%%%%%%%%%%%
% Cores dos links
%%%%%%%%%%%%%%%%%%%%%%%%%%%%%%%%%%%%%%%%

% Veja o arquivos cores.tex se quiser ver que outras cores estão
% pré-definidas.  Utilizando o comando \hypersetup abaixo nós
% evitamos aquelas caixas vermelhas feias em volta dos links.

\input{cores}
\hypersetup{
  colorlinks=true,
  linkcolor=DarkScarletRed,
  citecolor=DarkScarletRed,
  filecolor=DarkScarletRed,
  urlcolor= DarkScarletRed
}



%%%%%%%%%%%%%%%%%%%%%%%%%%%%%%%%%%%%%%%%
% Informações sobre a monografia
%%%%%%%%%%%%%%%%%%%%%%%%%%%%%%%%%%%%%%%%

\title{Em direção à formalização das propriedades de normalização do sistema
    $\lambda ex$}%

\orientador{\prof \dr Flávio L. C. de Moura}{CIC/UnB}%
\coordenador{\prof \dr Rodrigo Bonifácio de Almeida}{Universidade de Brasília}%
\diamesano{09}{Dezembro}{2016}%

\membrobanca{\prof \dr Mauricio Ayala-Ríncon}{MAT/CIC/UnB}%
\membrobanca{\prof \dr Daniel Lima Ventura}{INF/UFG}%

\autor{Lucas de Moura}{Amaral}%

\CDU{004.4}

\palavraschave{cálculo lambda, verificação formal, substituições explícitas, Coq}%
\keywords{lambda calculus, formal verification, explicit substitutions, Coq}%



%%%%%%%%%%%%%%%%%%%%%%%%%%%%%%%%%%%%%%%%
% Texto
%%%%%%%%%%%%%%%%%%%%%%%%%%%%%%%%%%%%%%%%

\begin{document}
  \maketitle
  \pretextual

  \begin{dedicatoria}
      Dedico este trabalho à memória do meu pai, Gesley, que sempre fez de tudo
      para me dar uma boa educação e me incentivou a escolher o curso de Ciência
      de Computação. Também à minha família, que sempre me ajuda em tudo que
      preciso e não deixa de demonstrar interesse em minha vida: à minha
      mãe, Esther; meu padastro, Márcio; e a meus irmãos Thiago, Davi e
      Benjamin. Agradeço em especial à minha mãe, meu tio Donney e minha avó
      Nicinha, por nunca deixar faltar nada para que eu pudesse me concentrar em
      minha formação.

      Dedico também aos meus amigos, tanto da escola quanto da faculdade. Em
      especial, agradeço ao Calil, à Dani e à minha prima, Tephinha, por
      aguentarem minhas reclamações e me tranquilizarem nos momentos de
      ansiedade e cansaço.
  \end{dedicatoria}

  \begin{agradecimentos}
      Agradeço ao Departamento de Ciência da Computação e aos meus professores
      pela boa base técnica dada. Em especial, agradeço ao meu orientador,
      Flávio, pela disponibilidade e paciência para me ajudar no trabalho.
  \end{agradecimentos}

  \begin{resumo}
    O cálculo $\lambda$ é um sistema formal, capaz de expressar o processo
    computacional.  Pela sua simplicidade e expressividade, este cálculo é usado
    como modelo teórico para o paradigma de programação funcional. Em consequência
    disto, uma grande quantidade de extensões do cálculo foi proposta, com o
    objetivo de obter um sistema formal intermediário entre o cálculo $\lambda$ e suas
    implementações.  O objeto de estudo deste trabalho é uma destas variantes,
    chamada $\lambda$ex, um cálculo com substituições explicitas proposto por Delia
    Kesner, em \cite{delia}.  Em resumo, o objetivo é continuar o trabalho de
    formalização deste cálculo, no assistente de prova Coq, iniciado em
    \cite{initial}, e que tem por objetivo fornecer uma prova mecância e
    construtiva da propriedade de formalização forte para o cálculo $\lambda
    ex$.
    \end{resumo}

  \selectlanguage{american}
  \begin{abstract}
    The $\lambda$-calculus is a formal system, capable of expressing the
    computational process.  Because of it's simplicity and expressiveness, this
    calculus is used as a theorical model for the paradigm of functional
    programming. Consequently, a great variety of extensions was proposed, with
    the objective of obtaining an intermidiate formal system between the
    $\lambda$-calculus and its implementations.  The object of study of this
    work is one of these variants, called $\lambda$ex, a calculus with explicit
    subsititutions, proposed by Delia Kesner, on \cite{delia}.  In short, the
    goal is to continue the work in the formalization of this calculus, in the
    Coq proof assistant, initiated in \cite{initial}, with the goal of providing
    a mechanical and constructive proof of the strong normalization property for
    the $\lambda ex$ calculus.
\end{abstract}
  \selectlanguage{brazil}

  \tableofcontents
  \listoffigures
  \listoftables

  \textual
  \chapter{Introdução}

%%%%%%%%%%%%%%%%%%%%%%%%%%%%%%%%%%%%%%%%%%%%%%%%%%%%%%%%%%%%%%%%%%%%%%%%%%%%%%%%
%%%%%%%%%%%%%%%%%%%%%%%%%%%%%%%%%%%%%%%%%%%%%%%%%%%%%%%%%%%%%%%%%%%%%%%%%%%%%%%%
%%%%%%%%%%%%%%%%%%%%%%%%%%%%%%%%%%%%%%%%%%%%%%%%%%%%%%%%%%%%%%%%%%%%%%%%%%%%%%%%
\section{Assistentes de Prova} 
\subsection{Motivação}

Assistentes de provas são sistemas computacionais que permitem aos usuários
especificar teorias e provar propriedades destas em um computador. Neles, o
usuário pode construir toda sua teoria matemática em uma linguagem em que o
sistema seja capaz de verificar sua correção. Ou seja, a principal motivação por
trás de um assistente de prova é verificar formalmente as propriedades de uma
teoria. Apesar de já existir um processo humano de verificação, muitas vezes
ocorrem erros neste processo, e provas que foram aceitas numa primeira avaliação
são descobertas problemáticas algum tempo depois. Como exemplo, podemos citar o
teorema das quatro cores \cite{four_colour}, que desafiou matemáticos por anos e
foram apresentadas falsas provas diversas vezes. Este teorema foi provado em
2008, no assistente de prova  Coq, estabelecendo, assim, a sua correção.
\cite{4colproof, gonthieril:_four_colour_theor}

Mas, então, o que exatamente significa uma prova? Uma prova normalmente é
definida como o processo de se estabelecer a validade de alguma afirmação. Na
matemática, as provas costumam exigir uma clareza e rigor mais extremo, de modo
a se tornar indiscutível quando analizada com cuidado. Porém, matemáticos são
humanos e, infelizmente, cometem erros. Com isto em mente, foi definida uma
noção ainda mais forte de prova, chamada \emph{prova formal}.

Uma prova formal é uma árvore finita cujos nós são marcados com fórmulas
lógicas. As folhas desta árvore são marcadas com axiomas, enquanto que os nós
filhos são marcados com fórmulas obtidas a partir dos nós pais via regras de
inferência. A vantagem do rigor das provas formais é que conferí-la se torna um
processo muito mais simples, sendo necessário apenas confirmar de onde vem cada
uma das sentenças.

Por este motivo, provas formais são comumente construídas e verificadas pelos
assistentes de prova. Ao utilizar a linguagem do assistente, ele aos poucos
constrói a sequência de sentenças e simultaneamente checa sua validade. Porém,
isto gera outra dúvida: Por que confiar nos assistentes de prova?


\begin{description}
    \item[Lógica do assistente:] Os assistentes de prova em geral possuem uma
        teoria forte no qual são baseados. Em geral, existe um sistema
        matemático independente de implementação que pode ser estudado e
        verificado anteriormente.
    \item[Checagem do assitente:] O assistente em si é, também, um programa. Assim,
        podemos analizar seus algoritmos, demonstrar que só é possível provar
        teoremas derivados no sistema lógico interno e testar seu funcionamento
        como um programa normal.
    \item[Critério de De Bruijn:] O critério de de Bruijn afirma que a correção
        de um sistema deve ser garantida por um verificador \emph{pequeno}
        \cite{freek}. Em outras palavras, deve haver um \emph{kernel} pelo qual
        todas as fórmulas passam. Assim, se torna mais fácil estabelecer a
        confiabilidade do sistema, pois podemos verificar o kernel
        separadamente. Nem todos os assistentes de prova passam neste critério,
        como visto na Tabela \ref{table:dbj_crit}.
\end{description}

\begin{table}[h]
\centering
\caption{Relação entre assistentes de prova e o critério de de Bruijn \cite{freek}}
\label{table:dbj_crit}
\begin{tabular}{l|c}
              & \multicolumn{1}{l}{Critério de de Bruijn} \\ \hline
HOL           & $\bullet$                                 \\
Mizar         & \multicolumn{1}{l}{}                      \\
PVS           & \multicolumn{1}{l}{}                      \\ 
Coq           & $\bullet$                                 \\
Otter/Ivy     & $\bullet$                                 \\
Isabelle/Isar & $\bullet$                                 \\
Alfa/Agda     & $\bullet$                                 \\
ACL2          &                                           \\
PhoX          & $\bullet$                                 \\
IMPS          &                                           \\
Metamath      & $\bullet$                                 \\
Theorema      &                                           \\
Lego          & $\bullet$                                 \\
NuPRL         &                                           \\
$\Omega$mega  & $\bullet$                                        
\end{tabular}
\end{table}


Existe um grande número de matemáticos interessados em assistentes de prova,
buscando construir uma teoria consistente para o uso destes e produzindo o
software necessário para facilitar seu uso. Em especial, um dos assistentes com
maior uso é o chamado Coq, a ser apresentado a seguir.

Para uma visão geral sobre o histórico e uso de assistentes de prova, veja
\cite{proof_assist}. 

\subsection{O assistente de prova Coq}

Neste trabalho, usaremos o Coq, um assistente de provas que é desenvolvido desde
1983, pelo INRIA (Institut-National de Recherche en Informatique et en
Automatique). O Coq provê um rico ambiente para o desenvolvimento de um
raciocínio formal checado automaticamente. O núcleo do sistema é um checador de
provas simples que garante que apenas passos válidos de dedução são efetuados.
Além deste núcleo, o ambiente provê diversas táticas para facilitar a
construções de provas, junto com uma vasta biblioteca de definições e teoremas
comuns.

A ferramenta vem acompanhada de uma linguagem de programação funcional, com
tipos dependentes. É através desta linguagem que podemos criar as definições e
construir as provas das propriedades de nossa teoria. Ela é baseada no Cálculo
de Construções Indutivas \cite{coquand}, uma extensão do cálculo $\lambda$ que
serve como modelo teórico para o sistema. O processo de se verificar a correção
de uma prova em Coq se reduz ao problema de \emph{checagem de tipos}. A seguir,
será feita uma introdução à sintaxe e ao funcionamento da ferramenta, baseada em
tutoriais disponibilizados na página oficial do sistema, em \cite{coq} e
\cite{coq2}.


Os objetos de Coq podem ser divididos em duas categorias, \emph{Prop} e
\emph{Type}. A categoria \emph{Prop} é a das proposições bem formadas. Um
exemplo de proposição na linguagem seria:

\bigskip
\coqdockw{\ensuremath{\forall}} \coqdocvar{A} \coqdocvar{B} : \coqdockw{Prop},
\coqdocvar{A} \ensuremath{\rightarrow} (\coqdocvar{A} \ensuremath{\lor}
\coqdocvar{B}).\coqdoceol
\bigskip

Predicados podem ser definidos indutivamente. Na definição a seguir,
\texttt{even} é um predicado que indica que um natural é par, e \texttt{odd}
indica que um natural é ímpar.

\bigskip \coqdockw{Inductive} \coqdocvar{even} : \coqdocvar{nat}
\ensuremath{\rightarrow} \coqdockw{Prop} :=\coqdoceol 
\coqdocindent{2.00em}\ensuremath{|} \coqdocvar{even\_0} : \coqdocvar{even} 0\coqdoceol
\coqdocindent{2.00em}\ensuremath{|} \coqdocvar{even\_S} \coqdocvar{n} :
\coqdocvar{odd} \coqdocvar{n} \ensuremath{\rightarrow} \coqdocvar{even}
(\coqdocvar{n} + 1)\coqdoceol \coqdockw{with} \coqdocvar{odd} :
\coqdocvar{nat} \ensuremath{\rightarrow} \coqdockw{Prop} :=\coqdoceol
\coqdocindent{2.00em}\ensuremath{|} \coqdocvar{odd\_S} \coqdocvar{n} :
\coqdocvar{even} \coqdocvar{n} \ensuremath{\rightarrow} \coqdocvar{odd}
(\coqdocvar{n} + 1).\coqdoceol \bigskip

Predicados também podem ser feitos como definições diretas, como:

\bigskip
\coqdockw{Definition} \coqdocvar{sqr} (\coqdocvar{x} : \coqdocvar{N}) :=
\coqdoctac{\ensuremath{\exists}} \coqdocvar{z}, \coqdocvar{z}
\ensuremath{\times} \coqdocvar{z} = \coqdocvar{x}.\coqdoceol
\bigskip

Assim, podemos utilizar estes predicados como propriedades de algum objeto,
provando algo como \emph{even(2)} ou \emph{sqr(4)}.

\emph{Type} é a categoria de estruturas matemáticas e estruturas de dados.
Alguns exemplos de tipos são:

\bigskip
\coqdocvar{$\mathbb{Z}$} \ensuremath{\times} \coqdocvar{$\mathbb{Z}$}
\ensuremath{\rightarrow} \coqdocvar{$\mathbb{Z}$}
\bigskip

Este é um tipo funcional.  \coqdocvar{$\mathbb{Z}$} \ensuremath{\times}
\coqdocvar{$\mathbb{Z}$} é o tipo de pares de números inteiros. O tipo completo,
\coqdocvar{$\mathbb{Z}$} \ensuremath{\times} \coqdocvar{$\mathbb{Z}$}
\ensuremath{\rightarrow} \coqdocvar{$\mathbb{Z}$}, consiste de funções de pares
de inteiros que retornam inteiros.

Tipos também podem ser definidos indutivamente:

\bigskip
\coqdockw{Inductive} \coqdocvar{nat} : \coqdockw{Set} :=\coqdoceol
\coqdocindent{2.00em} \ensuremath{|} 0 : \coqdocvar{nat}\coqdoceol 
\coqdocindent{2.00em} \ensuremath{|} \coqdocvar{S} : \coqdocvar{nat} \ensuremath{\rightarrow}
\coqdocvar{nat}.\coqdoceol
\bigskip

Neste caso, elementos do tipo \emph{nat} são: 0, S ( 0 ), S ( S ( 0 ) ), etc.

O desenvolvimento de provas em Coq é feito através de uma linguagem de provas,
que permite um processo guiado pelo usuário. Ao utilizar uma tática, o usuário
está construindo os objetos de prova. Por exemplo, a tática
\texttt{intro n}, onde \texttt{n} é do tipo \texttt{nat}, constrói o termo (com um
buraco):

\bigskip
\coqdockw{fun} (\coqdocvar{n}:\coqdocvar{nat}) \ensuremath{\Rightarrow} \coqdocvar{\_}\coqdoceol
\bigskip

Onde \_\ representa um termo que irá ser construído futuramente, utilizando
outras táticas. 

Neste trabalho, precisaremos também definir algumas funções recursivas. Para
garantir a consistência do sistema, a ferramenta exige que todas as funções
sejam terminantes. Um exemplo da sintaxe usada para definir funções no trabalho
pode ser visto abaixo.

\bigskip
\coqdocnoindent \coqdockw{Fixpoint} \coqdocvar{mult\_5} (\coqdocvar{n} :
\coqdocvar{nat}) : \coqdocvar{nat} :=\coqdoceol \coqdocindent{2.00em}
\coqdockw{match} \coqdocvar{n} \coqdockw{with}\coqdoceol \coqdocindent{2.00em}
\ensuremath{|} \coqdocvar{S} \coqdocvar{k} \ensuremath{\Rightarrow} 5 +
(\coqdocvar{mult\_5} \coqdocvar{k})\coqdoceol \coqdocindent{2.00em}
\ensuremath{|} 0   \ensuremath{\Rightarrow} 0\coqdoceol \coqdocindent{2.00em}
\coqdockw{end}.\coqdoceol \coqdocemptyline
\bigskip

A função é chamada \texttt{mult\_5} e recebe um natural. Ela realiza uma
análise de casos no argumento, observando seus construtores. Caso o número seja
zero, a função retorna zero. Se for o sucessor de algum natural \texttt{k}, ela
soma 5 ao resultado de \texttt{mult\_5 k}. O sistema consegue deduzir a
terminação desta função, pois a chamada recursiva é realizada em um subtermo do
argumento inicial.

Como exemplo de uma prova simples a ser realizada no sistema, podemos comparar
o funcionamento da nossa função com a multiplicação padrão do Coq.

\bigskip
\coqdocnoindent
\coqdockw{Theorem} \coqdocvar{mult\_5\_is\_correct}:
\coqdockw{\ensuremath{\forall}} \coqdocvar{n} : \coqdocvar{nat},
(\coqdocvar{mult\_5} \coqdocvar{n}) = 5*\coqdocvar{n}.\coqdoceol \coqdocnoindent
\coqdockw{Proof}.\coqdoceol \coqdocindent{2.00em} \coqdoctac{intros}.\coqdoceol
\coqdocindent{2.00em} \coqdoctac{induction} \coqdocvar{n}.\coqdoceol
\coqdocindent{2.00em} \coqdoctac{simpl}. \coqdoctac{reflexivity}.\coqdoceol
\coqdocindent{2.00em} \coqdoctac{simpl} \coqdocvar{mult\_5}.\coqdoceol
\coqdocindent{2.00em} \coqdoctac{rewrite} \coqdocvar{mult\_comm}.\coqdoceol
\coqdocindent{2.00em} \coqdoctac{simpl}.\coqdoceol \coqdocindent{2.00em}
\coqdoctac{rewrite} \coqdocvar{IHn}.\coqdoceol \coqdocindent{2.00em}
\coqdoctac{rewrite} \coqdocvar{mult\_comm}.\coqdoceol \coqdocindent{2.00em}
\coqdoctac{reflexivity}.\coqdoceol \coqdocnoindent \coqdockw{Qed}.\coqdoceol
\bigskip

No exemplo acima, \texttt{mult\_5\_is\_correct} é o nome do teorema, e
\emph{Proof/Qed} delimita a prova.  A ferramenta disponibliza uma maneira de
visualizar os estados da prova durante o processo. Introduzimos um natural
arbitrário para começar a prova, com o comando \texttt{intros n}. Após isto,
realizamos uma prova por indução com o comando \texttt{induction n}, que divide
a prova em dois subcasos. O primeiro corresponde à base de indução, e é dado
por:

\begin{lstlisting}[basicstyle=\small]
    ======================== ( 1 / 2 )
    mult_5 0 = 5 * 0

\end{lstlisting}

Usamos o comando \texttt{simpl} para computar os valores de \texttt{mult\_5 0} e
\texttt{5 * 0}, tendo assim a igualdade \texttt{0 = 0}. A validade disso vem da
reflexividade da igualdade, concluindo este caso com o comando
\texttt{reflexivity}. No passo indutivo, temos o seguinte estado:


\begin{lstlisting}[basicstyle=\small]
    n : nat
    IHn : mult_5 n = 5 * n

    ======================== ( 1 / 1 )
    mult_5 (S n) = 5 * S n
\end{lstlisting}

Onde \texttt{IHn} é nossa hipótese de indução. Começamos a prova computando o
valor de \texttt{mult\_5 (S n)}, com o comando \texttt{simpl mult\_5}.
Observando a definição desta função, vemos que o a expressão será avaliada para
\texttt{S (S (S (S (S (mult\_5 n)))))}, refletido no estado da prova. Queremos
então avaliar também o resultado de \texttt{5 * S n}. Devido à definição do
operador *, a prova se torna mais fácil se comutarmos os termos \texttt{5} e
\texttt{S n} nesta expressão. Fazemos isto utilizando o lema
\texttt{mult\_comm}, pertencente às bibliotecas do sistema, com o comando
\texttt{rewrite mult\_comm}. Após isto, computamos o valor de \texttt{(S n) *
    5}, com o comando \texttt{simpl}. Temos, então, o seguinte estado:

\begin{lstlisting}[basicstyle=\small]
    n : nat
    IHn : mult_5 n = 5 * n

    ======================== ( 1 / 1 )
    S (S (S (S (S (mult_5 n))))) = S (S (S (S (S (n * 5)))))
\end{lstlisting}

Podemos finalmente usar nossa hipótese de indução! Reescrevemos o termo
\texttt{mult\_5 n} para \texttt{5 * n}, com o comando \texttt{rewrite IHn}.
Utilizamos novamente a comutatividade do operador *, tendo finalmente o seguinte
estado:

\begin{lstlisting}[basicstyle=\small]
    n : nat
    IHn : mult_5 n = 5 * n

    ======================== ( 1 / 1 )
    S (S (S (S (S (n * 5))))) = S (S (S (S (S (n * 5)))))
\end{lstlisting}

Agora temos o mesmo termo em ambos os lados da igualdade, podendo concluir a
prova com \texttt{reflexivity}.

Os conceitos apresentados nesta seção serão amplamente utilizados no trabalho.
Para um melhor entendimento do funcionamento da ferramenta, da linguagem e
teoria envolvidas, veja \cite{pierce}.


\section{O cálculo $\lambda$}

\subsection{Visão geral}

Um dos grandes desafios do início do século XX era obter uma definição 
precisa para a noção intuitiva do processo computacional. Diversos modelos foram
propostos para resolver este problema. Entre eles está o cálculo $\lambda$,
proposto por Alonzo Church \cite{lambda_first}, em 1936. Inicialmente, o cálculo
fazia parte de um sistema maior, proposto para servir como uma base
formal para o estudo das fundações da matemática. Porém, devido a
inconsistências neste sistema, Alonzo Church foi obrigado a abrir mão de seu
objetivo inicial, separou a parte utilizável do sistema e formou o que hoje
conhecemos como cálculo $\lambda$.

O grande diferencial deste cálculo está em sua expressividade, com poder
computacional equivalente ao da Máquina de Turing, e sua simplicidade,
demonstrada por sua gramática concisa e poucas regras. A ideia central deste
consiste em simular a criação e aplicação de funções. Diferentemente da noção
usual no trabalho matemático, as funções neste sistema são chamadas "anônimas",
pois são definidas tendo em vista somente seus argumentos e o resultado. Como
exemplo, uma função simples como: "$double(x) = 2*x$" \ é definida anonimamente como
"$\lambda x.\ 2*x$". É utilizada também uma notação especial para a aplicação de
funções, denotada como "$ (\lambda x.\ 2*x)$  $3$".

É importante notar que os exemplos acima não são representados exatamente desta
maneira. Como dito anteriormente, o sistema possui uma gramática simples, e
todas as noções, inclusive números e operações, devem ser definidas com base em
abstrações e aplicações. A gramática do cálculo $\lambda$ pode ser descrita
sucintamente como:

\[ \tau := x\ |\ \lambda x.\tau\ |\ \tau \tau \]

Onde $\tau$ representa um termo, e $x$ representa uma variável. Assim, um termo
do cálculo pode ser, respectivamente, uma variável, uma abstração de um termo
$\tau$ por uma variável $x$, e uma aplicação de dois termos.  

\begin{definicao}[Variável Livre]
    Dizemos que uma variável $x$ é livre em um termo $\tau$ se não está contida
    em nenhuma abstração cujo argumento é $x$. 
\end{definicao}

Como exemplo, a variável $x$ é livre em $(\lambda y.\ y)\ x$, enquanto a
variável $y$ não é.  O conjunto de variáveis livres de \emph{t} é denotado por
\emph{fv(t)}. 

\begin{definicao}
    Definimos o conjunto de variáveis livres de $t$, denotado por $fv(t)$,
    indutivamente. Na definição abaixo, $t,u$ denotam termos e $x$ denota uma
    variável.
\begin{empheq}{align*}
    fv(x)\ & = \{x\} \\
    fv(t\ u)\ & = fv(t)\ \cup\ fv(u) \\
    fv(\lambda x. t)\ & = fv(t)\setminus \{x\}
\end{empheq}
\end{definicao}

\bigskip

O processo computacional é simulado no sistema através da regra de
$\beta$-redução, definida como:

\[ (\lambda x.t)\ u \rightarrow_{\beta} t\{x/u\} \]

Note que $t\{x/u\}$ é uma \textit{meta-operação}, definida pela substituição das
ocorrências da variável $x$ no termo $t$ pelo termo $u$. Esta operação será
definida formalmente em breve.

Para facilitar o entendimento, vejamos alguns exemplos de $\lambda$-termos:

\begin{itemize}
    \item A função identidade pode ser representada pelo termo $ (\lambda x. x) $.
        É fácil ver a correspondência na seguinte redução: $ (\lambda x.x) u
    \rightarrow_\beta x \{x/u\} = u $. 
    \item A função constante pode ser representada pelo termo $ (\lambda x. M) $,
        onde M é um termo qualquer, tal que $x \notin fv(M)$.
        É fácil ver a correspondência na seguinte redução: $ (\lambda x.M) u
    \rightarrow_\beta M \{x/u\} = M $. 
    \item Por último, podemos representar uma função que recebe dois termos e
        retorna o primeiro, como $ (\lambda x. \lambda y. x)$. Sua aplicação é
        reduzida da seguinte maneira: $ ((\lambda x. \lambda y. x)\ M)\ N)
    \rightarrow_\beta ((\lambda y. x) \{x/M\} N) = (\lambda y. M) N
    \rightarrow_\beta M \{y/N\} = N$, com $x$,$y$ não ocorrendo 
    livres em M ou N.
\end{itemize}


A partir destas definições, várias propriedades sobre o sistema podem ser
estudadas. Entre elas, é importante ressaltar as noções de \textit{forma
 normal} e de \textit{confluência}:

\begin{definicao}[Forma normal]\label{def_normal}
    Um termo $t$ é dito estar em forma normal quando não existe $t'$ tal que $ t
    \rightarrow t' $.  Um termo é dito \textbf{fracamente normalizável} quando
    existe uma estratégia de redução que leva a uma forma normal. O termo é
    \textbf{fortemente normalizável} se toda estratégia leva à forma normal.
\end{definicao}

É possível demonstrar que nem todo termo possui uma forma normal. Como exemplo,
observe que $ (\lambda x.x\ x)\ (\lambda x.x\ x) \rightarrow_\beta (\lambda x.x\
x)\ (\lambda x.x\ x) $. A noção de normalização é especialmente importante, pois
indica se um termo pode ou não terminar quando for avaliado, o que é de grande
interesse no estudo de propriedades computacionais.

\begin{definicao}[Confluência]
    Um sistema de reescrita, tomando como exemplo o cálculo $\lambda$, é dito
    fracamente confluente se, para todo termo $t$, com $ t \rightarrow_\beta t' $
    e $ t \rightarrow_\beta t'' $, então deve existir um termo $u$ tal que $t'
    \rightarrow_\beta^* u$ e $t'' \rightarrow_\beta^* u$, onde
    $\rightarrow_\beta^*$ representa o fecho transitivo-reflexivo da
    $\beta$-redução.  O sistema é dito fortemente confluente se, para todo termo
    $t$ tal que $ t \rightarrow_\beta^* t' $ e $ t \rightarrow_\beta^* t'' $,
    existe um termo $u$ tal que $t' \rightarrow_\beta^* u$ e $t''
    \rightarrow_\beta^* u$. 
\end{definicao}

\begin{definicao}[$\beta$-equivalência]
    A relação de $\beta$-equivalência $\equiv_{\beta}$ é definida como a menor
    relação de equivalência que contém a relação $\rightarrow_{\beta}$. Em
    outras palavras, dizemos que dois termos $t$ e $t'$ são $\beta$-equivalentes
    se vale $t \rightarrow_{\beta}^* t'$ ou $t' \rightarrow_{\beta}^* t$.
    Podemos ver que esta definição é de fato uma equivalência, pois é transitiva
    e reflexiva, pela definição de $\rightarrow_{\beta}^*$, e simétrica, pela
    definição aqui dada.
\end{definicao}

A propriedade de confluência pode ser entendida, essencialmente, como uma
garantia que a ordem que as reduções são feitas dentro de um termo não afetam o
resultado final do processo. Ou seja, a confluência garante o determinismo do
processo computacional.


Outro ponto importante a ser mencionado é a noção de $\alpha$-equivalência de
termos. 

\begin{definicao}[$\alpha$-equivalência]
    Um termo $(\lambda x. t)$ é dito $\alpha$-equivalente a\ $(\lambda y. u)$ se
    $ t\{x/y\} = u $. Mais geralmente, dois termos são ditos
    $\alpha$-equivalentes se um pode ser obtido a partir do outro através de
    renomeamento de variáveis ligadas. É fácil ver intuitivamente que esta
    definição é uma relação de equivalência: para a reflexividade, basta fazer
    um renomeamento trivial. Para a transitivadade, basta compor os
    renomeamentos.  Para a simetria, basta fazer o renomeamento contrário, ou
    seja: $ u\{y/x\} = t $.
\end{definicao}

Esta noção captura a ideia de que a escolha do nome das
variáveis ligadas não importa em geral, sendo o real objeto de interesse a
estrutura do termo. Esta definição é útil para evitar certos problemas, como por
exemplo o de \textit{captura de variáveis livres}. No exemplo:

\[ (\lambda x.\ y)\{y/x\},\ x\ \neq\ y \]

Efetuando a substituição sem tomar o devido cuidado, teremos o termo $\lambda
x.\ x$, que representa a função identidade. Note que isto mudou completamente a
semântica do termo! O termo anterior, $\lambda x.\ y$, representava uma função
constante, que sempre é avaliada para $y$, independente do termo dado como
argumento. Aplicando a substituição, seria razoável esperar que teríamos uma
função que é sempre avaliada para a variável $x$, o que não é o caso. 

Veja que a variável $x$ na substituição não é a mesma da que está ligada na
abstração. Isto pode causar vários problemas inesperados, como, por exemplo, a
perda da normalização do termo.  Como exemplo, considere o termo $((\lambda x.\
y\ y)\ (\lambda x.\ y\ y))$. É fácil ver que este termo deveria avaliar para
$(y\ y)$. Se queremos substituir $y$ por $x$, deveríamos obter o termo $(x\ x)$.
Porém, aplicando a substituição no termo original, temos $((\lambda x.\ y\ y)\
(\lambda x.\ y\ y))\{y/x\}$, que é avaliado para $((\lambda x.\ x\ x)\ (\lambda
x.\ x\ x))$, que não possui forma normal!

Para evitar este problema, podemos renomear a variável ligada $x$, antes da
substituição, por uma variável nova, de maneira a obter um termo
$\alpha$-equivalente e podendo então realizar a substituição sem modificar a
semântica do termo. Nos exemplos anteriores:

\[ (\lambda x.\ y)\{y/x\}\ =_{\alpha} (\lambda z.\ y)\{y/x\} = (\lambda z.\ x) \]

Observe que chegamos ao resultado esperado, ou seja, a função constante que é
sempre avaliada para $x$.

\[ ((\lambda x.\ y\ y)\ (\lambda x.\ y\ y))\{y/x\} =_{\alpha} ((\lambda z.\ y\
    y)\ (\lambda z.\ y\ y))\{y/x\} \\
    = ((\lambda z.\ x\ x)\ (\lambda z.\ x\ x)) \rightarrow_{\beta} (x\ x)\]

Conseguimos, também, evitar a perda da normalização do termo.

Assim, podemos definir formalmente a operação de substituição, da seguinte
maneira:

\begin{definicao}\label{classic_subst}
    Sejam $t, u$ termos e $x$ uma variável. A substituição de $x$ por $u$ em
    $t$ é definida indutivamente na estrutura de $t$ como:
\smallskip
\begin{empheq}{align*}
    \ y\{x/u\}\ \ \ \  & =\ u,\ se\ (x = y);\ y,\ c.c \\
    \ (t'\ v)\{x/u\} \ \ \ \  & =\ (t'\{x/u\}\ v\{x/u\}) \\
    \ (\lambda y. t') \{x/u\}  \ \ \ \  & =\ (\lambda y. t'), se (x = y);  
        (\lambda y. t\{x/u\}),\ c.c. \\
\end{empheq}
\end{definicao}

Note que, no caso da abstração, se $y \in fv(u)$, devemos realizar a
$\alpha$-equivalência para preservar a semântica do termo.


\subsection{Representação de $\lambda$-termos}
\label{sub:int_lnr}

Como dito anteriormente, o cálculo $\lambda$ é usado como modelo teórico para
linguagens funcionais. A noção de $\alpha$-equivalência, apesar de útil, pode ser
muito custosa em uma implementação prática. Por este motivo, foram propostas
algumas representações diferentes de termos, de modo a evitar a necessidade do
renomeamento de variáveis. 

Uma das primeiras, e mais importantes, tentativas de resolver o problema da
$\alpha$-conversão é a notação utilizando índices de De Bruijn \cite{debruijn72}. Nela, são
utilizados números naturais para representar as variáveis. Cada número
representa a quantidade de abstrações no escopo da ocorrência da variável.
Números que ultrapassam esta quantidade representam variáveis livres. A
gramática pode ser definida como:

\[ \tau := n\ |\ \lambda \tau\ |\ \tau \tau \]

Onde $n$ representa um número natural, a partir do 1. Exemplos de termos
nesta notação incluem a identidade $(\lambda 1)$, a função constante $(\lambda
u)$, com $u$ não contendo 1 como índice livre, e um termo com índice
livre, como $((\lambda 2)\ u)$. Apesar de sua aparente praticidade de
implementação, esta notação se afasta muito da utilização do cálculo no papel.
Além disto, introduz a necessidade de se manter um contexto externo para
registrar as variáveis livres, junto com uma álgebra para lidar com tal
contexto. Para lidar com substituições utilizando a notação de de Bruijn,
devemos introduzir a noção de $i$-elevação.

\begin{definicao}[$i$-elevação \cite{ayala}] Seja $t$ um termo da gramática de de Bruijn. A
    $i$-elevação de $t$, denotada por $t^{+i}$, é definida indutivamente como:
    \begin{itemize}
        \item $(t_1\ t_2)^{+i} = (t_1^{+i}\ t_2^{+i})$
        \item $(\lambda t_1)^{+i} = \lambda (t_1^{+(i+1)})$
        \item $n^{+i} = \begin{cases}
                            n + 1   & \text{se } n > i \\
                            n       & \text{se } n \leq i
                        \end{cases}$
    \end{itemize}
    A elevação de um termo $t$ é sua $0$-elevação, e é denotada por $t^+$.
    
\end{definicao}

A definição de substituição deve ser adaptada para lidar com os índices de de
Bruijn.

\begin{definicao}[\cite{ayala}] A substituição de um termo $t$, no nível $n -
    1$, pelo termo $u$,
    denotada por $t\{n/u\}$, é definida indutivamente como:
    \begin{itemize}
        \item $(t_1\ t_2)\{n/u\} = (t_1\{n/u\}\ t_2\{n/u\})$
        \item $(\lambda t_1)\{n/u\} = \lambda t_1\{n+1/u^+\}$
        \item $m\{n/u\} = \begin{cases}
                             m - 1 & \text{se } m > n \\
                             u     & \text{se } m = n \\
                             m     & \text{se } m \leq n
                          \end{cases}$
    \end{itemize}
\end{definicao}


\smallskip
Temos então que a relação de $\beta$-redução é definida por $(\lambda t_1)\ t_2
\rightarrow_{\beta} t_1\{1/t_2\}$.
\smallskip

Como exemplo das complexidades introduzidas por esta notação, podemos observar a
representação do termo $(\lambda x.\ x\ a)\ b$, onde $a,b$ são
variáveis livres. Este termo é representado na notação de de Bruijn como:

\[ (\lambda\ 1\ 2)\ 2 \]

Junto com o contexto $[a,\ b]$, que representa as variáveis livres.  Utilizando
esse contexto, vemos que o índice 1 representa a variável $x$, pois temos apenas
uma abstração. Após isso, cada ocorrência do índice 2 representa uma variável
livre distinta que estão relacionadas, em ordem, com as variáveis no contexto.
Assim, a primeira ocorrência do índice $2$ deve ser substituída por $a$, e a
segunda deve ser substituída por $b$.

Após realizar a $\beta$-redução, o termo original é representado por $(b\ a)$.
Porém, o termo na notação da representação de de Bruijn não é representado por
$(2\ 2)$, mas sim por $(2\ 1)$, devido às manipulações efetuadas pela
substituição. Isto é necessário porque o novo termo não possui abstratores, logo
o primeiro índice para representar variáveis livres não é 2, mas 1.

Temos então, dois problemas: utilizar a notação padrão do cálculo $\lambda$,
junto com $\alpha$-equivalência, é algo custoso do ponto de vista computacional,
pois devemos sempre tomar o cuidado com captura de variáveis livres e fazer o
renomeamento destas. Por outro lado, utilizar a notação de de Bruijn também
exige cuidados, devido à necessidade de se manter um contexto externo para
representar variáveis livres e de se utilizar uma álgebra mais complexa ao
realizar substituições.

Uma solução para estes problemas é usar a \textit{locally nameless
representation}, usada para representar os $\lambda$-termos neste trabalho. Esta
representação tenta capturar o melhor de ambos os casos: não temos a necessidade
de realizar renomeamento constante de variáveis, nem precisamos lidar com um
contexto externo.

O conjunto $\Lambda_{lnr}$ de expressões na notação LNR é definido a partir da
gramática a seguir:

\[ \tau := x\ |\ n\ |\ \lambda \tau\ |\ \tau \tau \]

Como nos casos anteriores, $x$ representa uma variável; $n$ é um índice,
representado por um natural, a partir de 1; $\lambda \tau$ representa uma
abstração e $\tau \tau$ representa uma aplicação.

Os índices $n$ representam as variáveis ligadas da expressão. O valor do índice
representa a abstração a qual ele se refere, ou seja, um índice $k$ está ligado
à $k$-ésima abstração que o contém. 

Dizemos que uma expressão é um pré-termo se pertence ao conjunto $\Lambda_{lnr}$
gerado pela gramática acima. Observe que, como os índices representam as
variáveis ligadas, não é interessante a ocorrência de índices soltos, ou seja,
sem uma abstração correspondente. Por conta disto, pré-termos que possuírem
índices soltos não possuirão correspondentes na representação padrão do cálculo
$\lambda$. Definimos então a noção de termos, que serão as expressões sem
índices livres, foco de interesse no trabalho.

\begin{definicao}[Termos]
    Dizemos que um pré-termo é um termo (bem formado) se toda ocorrência de um
    índice pertence a um número correspondente de abstrações. Em outras palavras,
    um termo não possui índices soltos.
\end{definicao}

A meta-substituição deve ser adaptada para levar em conta os índices, como no
exemplo:

\[ \{1 \rightarrow t\}(\lambda\ 2) = ( \lambda (\{2 \rightarrow t\} 2) ) \]

Note que o índice a ser substituído foi incrementado quando a substituição entra
na abstração, de modo a corresponder à variável correta a ser substituída.
Precisamos, então, definir formalmente a operação.


\begin{definicao}
    Sejam $t,u$ termos; $x,y$ variáveis e $k,i$ índices.
    A operação de substituição é definida indutivamente como:
\smallskip
\begin{empheq}{align*}
    \{k \rightarrow x\} i\ \ \ \  & =\ x,\ se\ (i = k);\ i,\ c.c\\
    \{k \rightarrow x\} y\ \ \ \  & =\ y\\
    \{k \rightarrow x\} (t\ u)\ \ \ \  & =\ (\{k \rightarrow x\}t\ \{k
    \rightarrow x\}u)\\
    \{k \rightarrow x\} (\lambda  t) \ \ \ \  & =\ 
    (\lambda  \{k+1 \rightarrow x\}t)
\end{empheq}
\end{definicao}

\medskip


Esta representação foi detalhada por Arthur Charguéraud em \cite{chargueraud},
junto com provas de seu bom funcionamento e um framework para sua utilização em
Coq.  Entraremos agora em alguns detalhes do uso desta notação, descritos em
\cite{chargueraud}, já que ela tem um interesse especial neste trabalho.

Na notação usual, quando queremos estudar o corpo de uma abstração $(\lambda x.
t')$, podemos trabalhar diretamente com o termo $t'$. Porém, nesta nova
representação, a abstração tem a forma $(\lambda\ t')$, e é necessário que seja
fornecida uma variável $x$, já que $t'$ não é um termo, em geral. Esta operação
é chamada \emph{abrir o termo} $t'$ com $x$ e será representada por $t'^{x}$. Mais
precisamente, a abertura do termo $(\lambda\ t')$ cria uma cópia de $t'$ onde todas
as ocorrências do índice ligado à abstração mais externa são trocados pela
variável $x$. Como exemplo, abrir a abstração $(\lambda (0\ y))$ com $x$ nos dá
o termo $(x\ y)$. A operação de abrir o termo deve ajustar o índice a ser mudado
à medida que entra no termo. Assim, podemos usar a já definida operação de
substituição, diretamente com a variável $x$, para realizar a abertura. Temos,
então, $t'^{x} = t'\{0 \rightarrow x\}$.

Similarmente, podemos abstrair todas as ocorrências de uma variável $x$ no termo
$t'$, construindo então o termo $(\lambda x. t')$. Com a nova notação, é
necessário definir uma operação que substitui todas as ocorrências de $x$ pelo
índice 0, antes de adicionar a abstração. Esse processo é chamado \emph{fechar o
    termo}, representado por $ ^{\textbackslash x}t'$. Assim, para construir a
abstração de maneira equivalente, fazemos $(\lambda ^{\textbackslash x}t')$. 

\begin{definicao}
    Sejam $t$ um termo, $x$ uma variável e $k$ um índice.
    Definimos indutivamente a substituição das ocorrências da variável $x$ em
    $t$ pelo índice $k$, denotada por $\{k \leftarrow x\}t$, como a seguir:
    
\begin{empheq}{align*}
    \{k \leftarrow x\} i\ \ \ \  & =\ i\\
    \{k \leftarrow x\} y\ \ \ \  & =\ \ k,\ se\ (x = y);\ y,\ c.c.\\
    \{k \leftarrow x\} (t\ u)\ \ \ \  & =\ (\{k \leftarrow x\}t\ \{k
    \leftarrow x\}u)\\
    \{k \leftarrow x\} (\lambda t) \ \ \ \  & =\ 
    (\lambda \{k+1 \leftarrow x\}t)
\end{empheq}
\end{definicao}

Podemos, então, definir a operação de fechamento como $ ^{\textbackslash x}t
= \{0 \leftarrow x\}t$.

Como dito anteriormente, esta representação possui termos que contém
\emph{índices livres}, que não possuem correspondentes no sistema original, e
não são nosso objeto de interesse. A seguir, chamaremos um termo sem índices
livres, ou seja, um termo bem formado, de \emph{localmente fechado}.

Existem duas maneiras de conferir se um termo é localmente fechado. A primeira
consiste em percorrer a estrutura do termo, abrindo cada abstração com uma nova
variável. Desta maneira, se o termo for de fato fechado, nunca encontraremos um
índice. A segunda abordagem consiste em analizar diretamente os índices do termo,
checando, para cada um deles, se o seu valor é menor ou igual ao número de
abstrações que o contêm.

A primeira opção dá lugar a uma definição natural de um predicado, denotado por
$lc\ t$ indicando que o termo é localmente fechado. Com apenas três regras
de inferência, podemos facilmente fazer uma análise mais formal da propriedade
de ser localmente fechado, sendo bastante útil em provas.

\begin{definicao}[Localmente fechado]\label{def_lc}
    Sejam $t, t1, t2$ termos, $x$ uma variável e $L$ um conjunto finito de
    variáveis. Dizemos que um termo é localmente fechado se não possui índices
    livres. Formalmente, isto é definido através do predicado $lc$, definido
    pelas regras de inferência abaixo.

\begin{mathpar} 
    \inferrule*[Right=lc\_fvar]{  }
    {lc(x)}
    \and
    \inferrule*[Right=lc\_app]{lc\ t1 \\ lc\ t2}
    {lc(t1\ t2)}
    \and
    \inferrule*[Right=lc\_abs]{\forall x \notin L,\ lc\ (t^{x})}
    {lc(\lambda. t)}
\end{mathpar}
\end{definicao}

A premissa $\forall x \notin L$ no caso da abstração captura a ideia de $x$
ser uma variável nova, pois podemos tomar o conjunto L, sempre finito, como
sendo o conjunto de variáveis já usadas e, assim, sempre ter uma escolha nova de
$x$.

A segunda abordagem tem um caráter mais naturalmente computacional, dando lugar
a uma função recursiva para conferir se um termo é localmente fechado. Tal
função vai navegando pelo termo, entrando em seus subtermos, guardando um
contador. Quando entrar em uma abstração, a função incrementa tal contador. Ao
encontrar um índice, basta conferir se ele é menor que o contador. Assim, a
função pode ser definida como a seguir.

\begin{definicao}\label{def_lc_at}
    O predicado binário $lc\_at$ que recebe um índice $k$ e um pré-termo $t$
    como argumentos é definido indutivamente na estrutura de $t$, como segue:
    
\begin{empheq}{align*}
    lc\_at\ k\ (i)\ \ \ &=\ i<k \\ 
    lc\_at\ k\ (x)\ \ \ &=\ \textbf{True} \\ 
    lc\_at\ k\ (t1\ t2)\ \ \ &=\ (lc\_at\ k\ t1)\ \&\ (lc\_at\ k\ t2) \\ 
    lc\_at\ k\ (\lambda t1)\ \ \ &=\ (lc\_at\ k + 1\ t1)
\end{empheq}
\end{definicao}

\medskip

Dizemos que o termo $t$ é localmente fechado se a função $lc\_at\ 0\ 
t$ retorna \textbf{True}.  Não é difícil provar a equivalência de ambas as
definições, ou seja, que vale $lc\ t\ \iff\ (lc\_at\ 0\ t\ =\ \textbf{True})$.
Para isto, precisaremos do seguinte lema, que é bem intuitivo:

\begin{lema}\label{lema_lc_open}
    Sejam \ $t$ um pré-termo e $x$ uma variável. Então vale 
    $(lc\_at\ k\ \{k \rightarrow x\}t) \iff (lc\_at\ (S\ k)\ t)$, para todo $k
    \in \mathbb{N}$.
\end{lema}

\begin{teorema} \label{teo:lc_lc_at}
    Seja $t$ um pré-termo. Vale ($lc\ t$) se, e somente se $(lc\_at\
    0\ t\ =\ \textbf{True})$.
\end{teorema}

\begin{proof} 
    $ $\par\nobreak\ignorespaces
\begin{itemize}
    \item[($\Rightarrow$)] Indução no predicado $lc$. O único
        subcaso que não sai imediatamente é o da abstração, que pode ser
        facilmente resolvido se observarmos que um termo $\{k \rightarrow x\}t$
        é fechado a um nível k se, e somente se, $t$ é fechado a nível k + 1,
        como visto no Lema \ref{lema_lc_open}.
    \item[($\Leftarrow$)] Indução na estrutura do termo $t$.
        Novamente, o caso da abstração merece um cuidado especial, mas ainda sai
        de maneira simples, escolhendo o conjunto L referenciado no construtor
        de $lc$ como sendo exatamente o conjunto de variáveis livres de $t$.
        Temos então, como hipótese, que vale $lc\_at\ 1\ t'$, onde $t'$ é o
        corpo da abstração, e queremos provar $(lc\ t'^x)$, para algum $x \notin
        fv(t')$. Basta então usar novamente o Lema \ref{lema_lc_open} para
        concluir a prova.
\end{itemize}
\end{proof}

Estas definições serão extremamente importantes ao decorrer do trabalho e podem
ser citadas com frequência. Em especial, a definição de localmente fechado,
correspondente à noção de termo bem formado, é essencial, pois são estes os
termos que possuem correspondentes no cálculo $\lambda$.

Vale a pena ressaltar duas outras equivalências, referentes às noções de
abertura e fechamento de termos. Temos:

\begin{lema}
    Seja $t$ um termo e $x$ uma variável. Então $^{\setminus x}(t^{x})\ =\ t,\ se\ x\
    \notin\ fv(t) $
\end{lema}
\begin{lema}
    Seja $t$ um termo e $x$ uma variável. Então $(^{\setminus x}t)^{x}\ =\ t,\ se\
    vale\ (lc\ t)$ 
\end{lema}

Para mais informações a respeito do cálculo $\lambda$, veja \cite{barendregt}.

%%%%%%%%%%%%%%%%%%%%%%%%%%%%%%%%%%%%%%%%%%%%%%%%%%%%%%%%%%%%%%%%%%%%%%%%%%%%%%%%
%%%%%%%%%%%%%%%%%%%%%%%%%%%%%%%%%%%%%%%%%%%%%%%%%%%%%%%%%%%%%%%%%%%%%%%%%%%%%%%%
%%%%%%%%%%%%%%%%%%%%%%%%%%%%%%%%%%%%%%%%%%%%%%%%%%%%%%%%%%%%%%%%%%%%%%%%%%%%%%%%
\



\section{Substituições explícitas}

\subsection{Motivação e histórico}

Sabemos que a ordem em que as reduções são feitas não altera a forma normal do
termo, mas isto não significa que não existem vantagens em se adotar certas
estratégias na normalização. 

Na implementação de linguagens de programação, a substituição muitas vezes é
"atrasada", de modo a evitar computações desnecessárias. Para aproximar o modelo
teórico de seu correspondente prático, podemos tentar fracionar a operação de
substituição em partes mais simples, de maneira a permitir uma manipulação simbólica
mais precisa. \cite{levy1999}

Além disto, separar a operação de substituição em partes mais simples pode auxiliar
no estudo de propriedades do próprio cálculo $\lambda$. Uma estratégia comum
\cite{ben_cbv, ben_beta} no estudo de propriedades do cálculo, referentes às
substituições, é analisar o problema em uma versão estendida do cálculo
$\lambda$, que possui um formalismo sintático representando as substituições,
permitindo um controle mais preciso. Demonstrada a propriedade no novo
sistema estendido, basta mostrar que ela é preservada entre os cálculos.

Por este motivo, existem várias tentativas de se formalizar a noção de
substituição, dando então espaço para o formalismo conhecido como
\textit{substituição explícita}. A princípio, podemos tentar estender a
gramática de termos da seguinte maneira:

\[ \tau := x\ |\ \lambda x.\tau\ |\ \tau \tau\ |\ \tau[x/\tau]\ \]

Podemos então espelhar o funcionamento da meta-substituição através de regras de
redução no novo cálculo, gerando o sistema conhecido como $\lambda$x.
\cite{lins86, rose92, bloo95}

\begin{table}[h]
\begin{empheq}[box=\fbox]{align*}
    (\lambda x.\ t)\ u\ \ \ &\rightarrow\ t[x/u] \\
    x[x/u]\ \ \             &\rightarrow\ u \\
    y[x/u]\ \ \             &\rightarrow\ y,\ se\ x\ \neq\ y \\
    (t\ u)[x/v]\ \ \        &\rightarrow\ t[x/v]\ u[x/v] \\
    (\lambda y.\ u)[x/v]\ \ &\rightarrow\ (\lambda y.\ u[x/v])
\end{empheq}
    \caption{Regras do sistema $\lambda$x}
\end{table}

O sistema $\lambda$x corresponde ao comportamento mínimo que se espera de um
cálculo com substituições explícitas. Porém, existem outras propriedades
interessantes que podem ser adicionadas ao sistema e, em consequência disto,
vários outros modelos foram propostos, como o $\lambda_\sigma$ \cite{hl89},
$\lambda_{ws}$ \cite{dg01},
$\lambda$lxr \cite{kes07}, entre outros. 

Um problema que pode acontecer em sistemas com substituição explícita é a perda
da preservação da normalização forte (\textbf{PSN}). \cite{mellies, Gu99}

\begin{definicao}[PSN]
    Seja $\lambda$z uma extensão do cálculo $\lambda$. Dizemos que $\lambda z$
    preserva a normalização forte se, para todo $\lambda$-termo fortemente
    normalizável (Definição \ref{def_normal}), seu correspondente em
    $\lambda$z também é fortemente normalizável nesta extensão.
\end{definicao}

Este tipo de problema é especialmente comum em cálculos com substituições
explícitas que possuem a propriedade de composição de substituições.
Essencialmente, dado um termo $t[x/u][y/v]$, podemos compor as duas
substituições, de maneira a reduzir a segunda substituição antes da primeira.
Como resultado, teríamos o termo $t[y/v][x/u[y/v]]$. 

Várias estratégias são usadas para se tentar garantir a propriedade \textbf{PSN}
do sistema, como utilização de marcas em termos, restrição de composições ou
reduções dentro de substituições explícitas, definições de classes de
equivalências, entre outros. Para uma visão geral do histórico de cálculos de
substituições explícitas, veja \cite{delia}.

%A seguir, veremos um sistema que propõe uma maneira de se compor tais
%substituições sem perder a propriedade \textbf{PSN}. Este sistema será o foco
%deste trabalho, e utilizaremos o assistente de provas Coq para formalizar certas
%propriedades essenciais deste.

A seguir, veremos um sistema que propõe uma maneira de se compor tais
substituições sem perder a propriedade \textbf{PSN}, o chamado $\lambda ex$. O
foco deste trabalho será iniciar a formalização da propriedade \textbf{IE},
chave para a prova da propriedade \textbf{PSN} deste sistema.  Para isto,
\emph{marcaremos} algumas substituições explícitas e iremos definir duas outras
reduções, de modo a controlar melhor as interações de termos com substituições
explícitas. Por fim, mostraremos a equivalência destas novas reduções com a
regra principal do sistema $\lambda ex$.


%%%%%%%%%%%%%%%%%%%%%%%%%%%%%%%%%%%%%%%%%%%%%%%%%%%%%%%%%%%%%%%%%%%%%%%%%%%%%%%%
%%%%%%%%%%%%%%%%%%%%%%%%%%%%%%%%%%%%%%%%%%%%%%%%%%%%%%%%%%%%%%%%%%%%%%%%%%%%%%%%
%%%%%%%%%%%%%%%%%%%%%%%%%%%%%%%%%%%%%%%%%%%%%%%%%%%%%%%%%%%%%%%%%%%%%%%%%%%%%%%%


  \chapter{O sistema $\lambda ex$}

\section{Visão geral}
\label{sec:int_lex}

%\subsection{Descrição do sistema}

Como visto no capítulo anterior, várias extensões do cálculo lambda foram
propostas com o objetivo de expressar de maneira concisa a noção de substituição
explícita.

O sistema proposto em \cite{delia}, chamado $\lambda$ex, é o primeiro sistema
que captura de maneira simples tal noção, enquanto ainda possui a propridade
\textbf{PSN}, ou seja, a preservação da normalização forte. Este sistema será
repetido aqui, e um maior entendimento pode ser obtido na fonte original. São
introduzidas várias mudanças, a começar pela gramática: 

\[ \tau := x\ |\ \lambda x.\tau\ |\ \tau \tau\ |\ \tau[x/\tau]\ \]
\

\begin{table}[h]
    
\begin{empheq}[box=\fbox]{align*}
    x[x/u]\ \ \             &\rightarrow_{Var}\ u \\
    t[x/u]\ \ \             &\rightarrow_{Gc}\ t    & se\ \emph{x} \notin fv(t)\\
    (t\ u)[x/v]\ \ \        &\rightarrow_{App}\ t[x/v]\ u[x/v] \\
    (\lambda y.\ u)[x/v]\ \ &\rightarrow_{Lamb}\ (\lambda y.\ u[x/v])\\
    t[x/u][y/v]\ \ \        &\rightarrow_{Comp}\ t[y/v][x/u[y/v]] & se\ y\ \in
    fv(u) \\ 
    (\lambda x.\ t)\ u\ \ \ &\rightarrow_B\ t[x/u]
\end{empheq}
    \caption{Regras de redução}
    \label{table:red_lambex}

\end{table}

A nova construção é chamada \textit{substituição explícita}, e é o que permite as
manipulações sintáticas com substituições no cálculo. Como consequência desta
mudança, novas regras de redução são definidas, como mostrado na tabela
\ref{table:red_lambex}.  As 5 primeiras regras formam a relação $\rightarrow_x$,
e o acréscimo da última forma a relação $\rightarrow_{Bx}$. 

Observe que, devido ao acréscimo das substituições explícitas, é possível
definir termos que possuem uma sintaxe distinta, diferindo apenas na permutação
de substituições independentes, e que constiturem o mesmo
$\lambda$-termo no sistema original. Um exemplo simples pode ser visto a seguir.

\[ (0\ 1) \{0/x\}\{1/y\} = (0\{0/x\}\ 1\{1/y\}) = (x\ y) =  (0\ 1) \{1/y\}\{0/x\} \]
\[ (0\ 1) [0/x][1/y] \neq  (0\ 1) [1/y][0/x] \]

Onde a noção de igualdade usada é a sintática.

Assim, é necessário tornar o novo sistema
equacional, adicionando uma relação de equivalência:

\[ t[x/u][y/v] =_C t[y/v][x/u] \ \ \ \ \ se\ y \notin fv(u)\ \&\ x \notin fv(v)\] 

A relação de equivalência $=_e$ é formada com a junção de $=_C$ e
$\alpha$-equivalência. As relações $\rightarrow_{ex}$ e $\rightarrow_{\lambda
    ex}$ são definidas como:

\[t\ \rightarrow_{ex}\ t'\ \iff\ \exists\ s,s'\ t.q.\ t\ =_{e}\ s\
    \rightarrow_x\ s'\ =_e\ t' \]
\[t\ \rightarrow_{\lambda ex}\ t'\ \iff\ \exists\ s,s'\ t.q.\ t\ =_{e}\ s\
    \rightarrow_{Bx}\ s'\ =_e\ t' \]

Estas relações serão amplamente utilizadas ao decorrer deste trabalho. 
O foco principal é a formalização da propriedade \textbf{IE} do sistema, que
pode ser entendida intuitivamente da seguinte maneira: Dados \textbf{t},
\textbf{u} tais que \emph{u} e \emph{t\{x/u\}} são fortemente normalizáveis
no sistema, então o correspondente utilizando substituições explícitas,
\emph{t[x/u]} também será. Ou seja, a normalização da substituição implícita
implica a normalização da explícita.

\subsection{A Gramática de pré-termos}
\label{sec:termos}

Nesta seção, será feita um paralelo entre as estruturas básicas do sistema
$\lambda ex$ e sua formalização em Coq.

Como visto em \ref{sub:int_lex}, o sistema possui uma gramática que pode ser vista como
uma extensão do cálculo lambda original, consistindo de variáveis, abstrações,
aplicações e \emph{substituições explícitas}. Ela pode ser descrita sucintamente
como abaixo.

\[ \tau := x\ |\ \lambda x.\tau\ |\ \tau \tau\ |\ \tau[x/\tau]\ \]

Já encontramos então nossa primeira divergência na formalização. Neste projeto,
será usada a representação chamada \textit{Locally Nameless Representation}
(LNR), onde as variáveis ligadas são representadas por índices. Por conta disso,
existem termos nesta nova gramática que não possuem correspondentes no sistema
original, por conta de índices que não estão ligados a nenhuma abstração.

Assim, torna-se necessária uma gramática de \emph{pré-termos}, que consiste de
todos os termos possíveis de serem escritos em LNR. Esta gramática é formalizada
usando um tipo indutivo.

\begin{lstlisting}[basicstyle=\small]
Inductive pterm : Set :=
  | pterm_bvar : nat -> pterm
  | pterm_fvar : var -> pterm
  | pterm_app  : pterm -> pterm -> pterm
  | pterm_abs  : pterm -> pterm
  | pterm_sub  : pterm -> pterm -> pterm 
  | pterm_lsub : pterm -> pterm -> pterm.
\end{lstlisting}

Veja que variáveis livres e ligadas possuem construtores distintos. Para as
livres, o construtor correto é \texttt{pterm\_bvar}, que recebe um natural
representando um índice. As variáveis livres são construídas com
\texttt{pterm\_fvar}, recebendo um elemento do tipo \texttt{var}, definido no
framework de Charguéraud.
As aplicações, abstrações e substituições são representadas através do
construtores \texttt{pterm\_app}, \texttt{pterm\_abs} e \texttt{pterm\_sub},
respectivamente. Todos recebem um ou dois pré-termos, que correspondem aos
termos internos das estruturas.

O último construtor, \texttt{pterm\_lsub}, será usado para representar
\emph{substituições marcadas}, um formalismo necessário para as provas de
correspondência com a meta-substituição.

\section{Termos e Relações}
\subsection{Termos bem formados}

Devido à correspondência assimétrica entre os termos do sistema $\lambda ex$ e
em LNR, torna-se necessário definir os predicados de boa formação de termos.

Como pré-requisito para a definição destes predicados, precisamos implementar as
noções de abertura e fechamento de termos, como visto em \ref{sub:int_lnr}. 
Relembrando, a operação de abertura foi definida como $t^{x} \equiv t\{0/x\}$.

Para implementar tal operação, precisaremos de duas funções, \texttt{open\_rec}
e \texttt{open}.

\begin{lstlisting}[basicstyle=\small]
Fixpoint open_rec (k : nat) (u : pterm) (t : pterm) {struct t} 
        : pterm := ...

Definition open t u := open_rec 0 u t.
\end{lstlisting}

A primeira adentra o termo \texttt{t} recursivamente, procurando pelo índice
\texttt{k} e substituindo pelo termo \texttt{u}. Ao encontrar uma abstração ou
substituição, o índice \texttt{k} é incrementado.  A segunda, que é a chamada da
operação de fato, apenas chama \texttt{open\_rec} com $k = 0$.

Similarmente, a operação de fechamento foi definida como $ ^{\textbackslash x}t
\equiv \{0 \leftarrow t\}$. Esta operação é definida através da função
\texttt{close}.

\begin{lstlisting}[basicstyle=\small]
Fixpoint close_rec  (k : nat) (x : var) (t : pterm) {struct t} 
    : pterm := ...

Definition close t x := close_rec 0 x t.
\end{lstlisting}

Podemos agora definir o predicado \texttt{term}, representando termos bem
formados, com base nas seguintes regras de inferência.

\begin{mathpar} 
    \inferrule*[Right=term\_var]{  }
    {lc(x)}
    \and
    \inferrule*[Right=term\_app]{lc\ t1 \\ lc\ t2}
    {lc(t1\ t2)}
    \and
    \inferrule*[Right=term\_abs]{\forall x \notin L,\ lc\ (t^{x})}
    {lc(\lambda. t)}
    \\
    \inferrule*[Right=term\_sub]{\forall x \notin L,\ lc\ (t^{x}) \\ lc\ u}
    {lc(t[u])}
\end{mathpar}

O predicado \texttt{term} recebe um pré-termo, e indica que este elemento é bem
formado, ou seja, não possui índices livres. Para cada tipo de pré-termo, temos
um construtor diferente do predicado. Observe que não existe regra de inferência
para índices livres, como desejado.

Em alguns casos, como dito na subseção \ref{sub:int_lnr}, pode ser mais
vantajoso verificar se um termo está bem formado com uma função recursiva.
Tal verificação é feita através do predicado \texttt{term'}, que deve funcionar de
maneira equivalente ao predicado \texttt{term}. Este novo predicado é definido
com base na definição recursiva \texttt{lc\_at}. A função \texttt{lc\_at}
verifica se o termo \emph{t} está \textit{fechado} a um nível \emph{k}, ou seja,
se não existe índice livre de valor maior ou igual a \emph{k}. Sua implementação
é feita com base na definição da tabela \ref{table:lc_at}, adicionando apenas o
caso a seguir, para substituições explícitas.

\begin{empheq}[box=\fbox]{align*}
    lc\_at\ k\ (t1[t2])\ \ \ &\equiv\ (lc\_at\ (S k)\ t1)\ \&\ (lc\_at\ k\ t2) \\ 
\end{empheq}

\begin{lstlisting}[basicstyle=\small]
Fixpoint lc_at (k:nat) (t:pterm) {struct t} : Prop := ...

Definition term' t := lc_at 0 t.
\end{lstlisting}

Assim, essa equivalência também teve que ser formalizada, utilizando uma série
de lemas auxiliares. 

Primeiramente, foram provados dois resultados de enfraquecimento para o
\texttt{lc\_at}.

\begin{gather}
 lc\_rec\_open\_var\_rec:\ \forall\ x\ t\ k,\ (lc\_at\ k\ (open\_rec\ k\ x\
    t))\ \rightarrow\ (lc\_at\ (S\ k)\ t) \nonumber \\
 lc\_at\_open\_var\_rec:\ \forall\ x\ t\ k,\ (lc\_at\ (S\ k)\ t)\ \rightarrow\
    (lc\_at\ k\ (open\_rec\ k\ x\ t)) \nonumber 
\end{gather}


O primeiro lema garante que, se um termo \emph{t}, aberto com uma variável a um
nível \emph{k}, é fechado a este mesmo nível \emph{k}, então o termo \emph{t}
sem a abertura é fechado a nível \emph{k + 1}.

O segundo é exatamente o contrário: Se um termo \emph{t} é fechado a um nível
\emph{k + 1}, então este mesmo termo, aberto com uma variável a nível \emph{k},
é fechado a nível \emph{k}.

Já podemos agora provar a equivalência entre as duas definições de termos bem
formados.

\begin{gather}
 term\_eq\_term':\ \forall\ t,\ (term\ t)\ \iff\ (term'\ t) \nonumber 
\end{gather}

\begin{itemize}
    \item[($\Rightarrow$)] Este caso é bem direto, fazendo uma indução no
        predicado \texttt{term t}. As hipóteses de indução resolvem os casos
        diretamente, exceto para abstração e substituição. Nesses, a hipótese é
        dada para $t^x$, quando precisamos provar $lc\_at\ 1\ t$. Felizmente os
        lemas de enfraquecimento acima resolvem, bastando usar o lema
        lc\_rec\_open\_var\_rec.
    \item[($\Leftarrow$)] O segundo caso não é tão imediato, pois fazemos uma
        indução no \emph{tamanho do termo}, para auxiliar na prova. A hipótese
        de indução se refere a termos com tamanho igual ao do termo sobre qual
        queremos provar a propriedade. Isto é útil pois o tamanho de um termo ao
        ser aberto não muda. Novamente, no caso da abstração e substituição,
        precisamos fazer um ajuste utilizando o lema lc\_at\_open\_var\_rec.
\end{itemize}

\subsection{Equivalência de termos}
\label{sub:equival_ncia_de_termos}

Como dito anteriormente, o acréscimo de substituições explícitas pode nos levar
a casos em que um mesmo termo no cálculo original tenha duas representações no
sistema. Observe o exemplo abaixo:

\[ t\{x/u\}\{y/v\},\ \ x \notin fv(v),\ y \notin fv(u) \]
\[ t[x/u][y/v]\ \ \ t[y/v][x/u]\]

Como as substituições são independentes, ambas as representações são válidas.
Para resolver este problema, foi adicionado no sistema uma regra de
\emph{permutação de substituições independentes}.

\[ t[x/u][y/v] =_C t[y/v][x/u] \ \ \ \ \ se\ y \notin fv(u)\ \&\ x \notin fv(v)\] 

Dizemos que ambas as representações são \emph{equivalentes}.

Precisamos, então, formalizar essa definição no sistema.

\begin{lstlisting}[basicstyle=\small]
Inductive eqc : pterm -> pterm -> Prop := 
  | eqc_def: forall t u v, term u -> term v -> 
          eqc (t[u][v]) ((& t)[v][u]).
\end{lstlisting}

As exigências que \emph{u} e \emph{v} sejam termos são puramente técnicas, pois
queremos trabalhar apenas com termos sem índices livres no sistema. O único
outro detalhe que muda na formalização é ocorrência do operador $\&$, que
troca todas as ocorrências de índices zero e um, para que eles continuem se
referindo à mesma substituição.

Note que esta definição ainda possui muitas limitações. Por exemplo, se um termo
possui uma lista de substituições, só podemos trocar as duas últimas. Também não
é possível realizar permutações em subtermos, ou várias permutações seguidas.
Assim, precisamos criar \emph{fechos} em cima dessa definição, para ajustar às
nossas necessidades. Nas regras a seguir, \emph{R} indica uma relação entre dois
termos. 

\begin{mathpar} 
    \inferrule*[Right=ES\_redex]{ (R\ t\ s) }
    {(ES\_contextual\_closure\ R)\ t\ s}
    \\
    \inferrule*[Right=ES\_app\_left]{((ES\_contextual\_closure\ R)\ t\ t'),\
        (term\ u)}
    {((ES\_contextual\_closure\ R)\ (t\ u)\ (t'\ u))}
    \\
    \inferrule*[Right=ES\_app\_right]{((ES\_contextual\_closure\ R)\ t\ t'),\
        (term\ u)}
    {((ES\_contextual\_closure\ R)\ (u\ t)\ (u\ t'))}
    \\
    \inferrule*[Right=ES\_abs\_in]{(\forall\ x,\ x \notin L \rightarrow
        ((ES\_contextual\_closure\ R)\ t^x\ t'^x))}
    {((ES\_contextual\_closure\ R)\ (\lambda . t)\ (\lambda . t'))}
    \\
    \inferrule*[Right=ES\_subst\_left]{(\forall\ x,\ x \notin L \rightarrow
        ((ES\_contextual\_closure\ R)\ t^x\ t'^x)),\ (term\ u)}
    {((ES\_contextual\_closure\ R)\ (t[u])\ (t'[u]))}
    \\
    \inferrule*[Right=ES\_subst\_right]{((ES\_contextual\_closure\ R)\ u\ u'),\
        (body\ t)}
    {((ES\_contextual\_closure\ R)\ (t[u])\ (t[u']))}
\end{mathpar}


\begin{lstlisting}[basicstyle=\small]
Definition eqc_ctx (t u: pterm) := ES_contextual_closure eqc t u.
Notation "t =c u" := (eqc_ctx t u) (at level 66). 
\end{lstlisting}

A definição \texttt{ES\_contextual\_closure} é um dos chamados \emph{fechos
    contextuais}. A ideia é que, se vale $t \rightarrow_R t'$, então, para
qualquer termo construído a partir de \emph{t}, vale a redução pelo fecho contextual de
R, para o mesmo termo, mas construído a partir de \emph{t'}. Como exemplo:

\[(t \rightarrow_R t') \Rightarrow ((pterm\_app\ t\ u)
    \rightarrow_{ES\_Contextual\_Closure\ R} (pterm\_app\ t'\ u)) \]

Definimos também os fechos \emph{transitivo} e \emph{transitivo-reflexivo}.
Essencialmente, se temos $t \rightarrow_R t'$ e $t' \rightarrow_R t''$, então
vale o fecho transitivo $t \rightarrow_{R^+} t''$. Isto é representado pelo
construtor \texttt{transitive\_reduction}. Note que isto vale para
apenas um passo também: $t \rightarrow_R t'$ implica $t \rightarrow_{R^+} t'$.
Este é representado pelo construtor \texttt{one\_step\_reduction}.

\begin{table}
\begin{mathpar} 
    \inferrule*[Right=one\_step\_reduction]{ (R\ t\ s) }
    {(trans\_closure\ R)\ t\ s}
    \\
    \inferrule*[Right=transitive\_reduction]{((trans\_closure\ R)\ t\ u),\
        ((trans\_closure\ R)\ u\ t')}
    {((trans\_closure\ R)\ (t\ t'))}
\end{mathpar}
    \caption{Fecho transitivo}
\end{table}

O fecho \emph{transitivo-reflexivo} é uma extensão do transitivo, onde vale $t
\rightarrow_{R^*} t$, sempre. A extensão é presentada por dois construtores:
\texttt{reflexive\_reduction}, que garante \texttt{star\_closure R t t}, e
\texttt{star\_trans\_reduction}, que constrói \texttt{star\_closure R t u}, se
vale \texttt{trans\_closure R t u}.

\begin{table}
\begin{mathpar} 
    \inferrule*[Right=reflexive\_reduction]{  }
    {(star\_closure\ R)\ t\ t}
    \\
    \inferrule*[Right=star\_trans\_reduction]{((trans\_closure\ R)\ t\ u)}
    {((star_trans\_closure\ R)\ (t\ u))}
\end{mathpar}
    \caption{Fecho transitivo-reflexivo}
\end{table}

\begin{lstlisting}[basicstyle=\small]
Definition eqc_trans (t u: pterm) := (trans_closure eqc_ctx) t u.
Notation "t =c+ u" := (eqc_trans t u) (at level 66). 

Definition eqC (t : pterm) (u : pterm) := (star_closure eqc_ctx) t u.
Notation "t =e u" := (eqC t u) (at level 66). 
\end{lstlisting}


Assim, a relação de equivalência que é interessante ser estudada é a $=_e$, que
permite vários passos de permutações, além de permutações no interior dos
termos. Uma importante diferença de provas usando equivalências no papel e numa
formalização é que resultados que são intuitivos e admitidos num ambiente
informal devem ser provados minuciosamente.

Como exemplo, podemos querer mostrar a compatibilidade da igualdade com a
função \texttt{lc\_at}. Isto é intuitivo, pois permutar duas substituições não
irá criar índices livres. Porém, é preciso entrar em detalhes na prova formal. 

\begin{lstlisting}[basicstyle=\small]
Lemma lc_at_eqc : forall n t u, eqc t u  -> (lc_at n t <-> lc_at n u).
Lemma lc_at_ES_ctx_eqc: forall n t u, (ES_contextual_closure eqc) t u  
                            -> (lc_at n t <-> lc_at n u).
Lemma lc_at_eqC : forall n t t', t =e t' -> 
                            (lc_at n t <-> lc_at n t').
\end{lstlisting}

Observe que precisamos provar três resultados. Não basta mostrar que o predicado
\texttt{eqc} é compatível com \texttt{lc\_at}, pois ainda usaremos seus fechos.
A prova de \texttt{lc\_at\_eqc} já apresenta certos detalhes. Ela é feita
através de uma análise de casos do predicado \texttt{eqc}. É preciso realizar
certos ajustes com índices e usar a equivalência entre \texttt{lc\_at} e o
predicado \texttt{term}, além de regras de enfraquecimento para o \texttt{lc\_at}.

Os casos dos fechos em geral saem mais rapidamente, fazendo indução no próprio
fecho. No caso do fecho contextual, os casos de abstração e substituição se
tornam um pouco mais longos, mas são facilmente resolvidos usando a propriedade
\texttt{lc\_at\_open'}, que garante que se um termo está fechado a um nível
\emph{n}, abrí-lo a um nível $k < n$ não gera índices livres.

Esta é uma parte do trabalho especialmente extensa e detalhada. É preciso
mostrar que vários predicados importantes da teoria são preservados pelas
classes de equivâlencia entre termos. Várias vezes durante as provas
trabalharemos com termos equivalentes módulo $=_e$ e, sem uma base bem
construída de resultados sobre a relação de \texttt{eqC}, fica impossível
concluir diversas provas.

Os principais resultados a serem provados nesta parte incluem:

\begin{itemize}
    \item Preservação da estrutura de termos pela equivalência. Construtores de
        termos, abertura de termos, substituições e renomeamento de variáveis
        devem funcionar de maneira análoga para dois termos equivalentes.
    \item Boa formação de termos e corpos de abstrações, conjunto de variáveis
        livres e reduções dentro de fechos contextuais também devem ser
        preservadas pela equivalência.
\end{itemize}

\subsection{Reduções do Sistema}
\label{sub:redu_es_do_sistema}

Com a estrutura de termos e as regras de equivalência bem estabelecidas, podemos
iniciar a formalização das reduções do sistema $\lambda ex$.
As regras de redução listadas em  \ref{table:red_lambex} serão formalizadas nos
tipos indutivos \texttt{sys\_x} e \texttt{rule\_b}.

\begin{lstlisting}[basicstyle=\small]
Inductive rule_b : pterm -> pterm -> Prop := ...
Notation "t ->_B u" := (rule_b t u) (at level 66).

Inductive sys_x : pterm -> pterm -> Prop := ...
Notation "t ->_x u" := (sys_x t u) (at level 59, left associativity).
\end{lstlisting}

A relação $\rightarrow_x$ é definida como sendo exatamente o predicado
\texttt{sys\_x}. Algumas exigências técnicas de \emph{term} e \emph{body} são
adicionadas para auxiliar nas provas, já que queremos sempre trabalhar com
termos bem formados.

O predicado \texttt{rule\_b} é o que formaliza a regra $\rightarrow_B$, sendo
esta a regra que reduz uma aplicação a uma substituição explícita. Podemos, a
partir destes dois predicados, definir a regra de redução principal do sistema
$\lambda ex$.

\begin{lstlisting}[basicstyle=\small]
Inductive sys_Bx: pterm -> pterm -> Prop :=
| B_lx : forall t u, t ->_B u -> sys_Bx t u
| sys_x_lx : forall t u, t ->_x u -> sys_Bx t u.

Notation "t ->_Bx u" := (sys_Bx t u).

red_ctx_mod_eqC = 
fun (R : pterm -> pterm -> Prop) (t u : pterm) =>
exist t' u', t =e t' /\ ES_contextual_closure R t' u' /\ u' =e u
     : (pterm -> pterm -> Prop) -> pterm -> pterm -> Prop

Definition lex t u :=  red_ctx_mod_eqC sys_Bx t u.
Notation "t -->lex u" := (lex t u) (at level 66).

Definition lex_trs t u := trans_closure lex t u.
Notation "t -->lex+ u" := (lex_trs t u) (at level 66).

Definition lex_str t u := star_closure lex t u.
Notation "t -->lex* u" := (lex_str t u) (at level 66).
\end{lstlisting}

O análogo da regra $\rightarrow_{\lambda ex}$ no sistema será a relação
\texttt{''-->lex*"}, construída a partir do fecho transitivo-reflexivo,
contextual e equacional da relação $\rightarrow_{Bx}$.

\vfill
Para auxiliar nas provas, definimos alguns predicados que garantem que estamos
trabalhando sempre com termos, e que representam algumas noções intuitivas das
reduções.

\begin{table}[h]
\begin{mathpar} 
    \inferrule*{(red\_regular\ R),\ (R\ t\ u)}
    {term\ t\ \and term\ u}
\end{mathpar}
    \caption{Noção de regularidade}
\end{table}


A propriedade de regularidade é interessante para facilitar em diversas provas,
já que desejamos evitar trabalhar com termos que possuem índices livres.
Basicamente, se fizemos a redução \emph{R} entre dois pré-termos,
ambos são termos bem formados.


\begin{table}[h]
\begin{mathpar} 
    \inferrule*{(red\_out\ R),\ (R\ t\ u)}
    {R\ ([x \rightarrow u]t)\ ([x \rightarrow u]t')}
\end{mathpar}
    \caption{Redução fora da meta-substituição}
\end{table}


A propriedade \texttt{red\_out} garante que a redução pode ser feita em um
termo afetado por uma meta-substituição, sendo $[x\ \mapsto\ u]$ a substituição que
troca todas as ocorrências de \texttt{x} pelo termo \texttt{u}. Ela é útil pois
esta meta-substituição pode ser vista como uma generalização da abertura de um
termo.

\begin{table}[h]
\begin{mathpar} 
    \inferrule*{(red\_rename\ R),\ (\forall x,\ x \notin fv(t) \rightarrow R\
        (t^x) \ (u^x))}
    {R\ (t^y)\ (u^y)}
\end{mathpar}
    \caption{Renomeamento em abertura com variável}
\end{table}

A propriedade \emph{red\_out} pode ser comparada à $\alpha$-equivalência de termos no cálculo
original. Basicamente, se um termo \textit{t} aberto com uma variável $x \notin
fv(t)$ se reduz a um \textit{t'}, também aberto com \textit{x}, então este
\textit{x} pode ser trocado por outra variável $y \notin fv(t)$, preservando a
redução $\rightarrow_{lex}$.

Estes resultados auxiliam muito em diversas provas. Em especial, os resultados
envolvendo abertura de termos e índices livres são necessários para provas em
que fazemos indução na estrutura do termo, pois, em geral, no caso da abstração
as hipóteses se referem ao sub-termo aberto com uma variável.



%------------------------------------------------------------------



\section{Preservação da normalização forte}
\label{sec:psn}

Nesta seção, queremos dar uma visão geral da propriedade PSN e descrever a
formalização da propriedade IE, foco deste trabalho. 

A propriedade PSN é a que garante que, se um termo \emph{t} é fortemente
normalizável no cálculo original, ou seja, toda cadeia de reduções a partir dele
termina, então ele também é fortemente normalizável no sistema $\lambda ex$.

A prova da PSN é feita primeiramente definindo uma estratégia de redução
\emph{perpétua} para o sistema, definida da seguinte maneira: se um termo
\emph{t} se reduz a um termo \emph{t'} por esta estratégia, e se \emph{t'} é
fortemente normalizável, então \emph{t} também o é. Pela contrapositiva, vemos
que se o termo \emph{t} não for fortemente normalizável, então a estratégia terá
que fazer a redução para um termo \emph{t'} que também não é fortemente
normalizável, e assim por diante, formando uma cadeia de redução infinita.

Em particular, se um termo $t[x/u]$ é reduzido, por esta estratégia, para um
termo $t\{x/u\}$, sendo este fortemente normalizável, então o termo original
também será fortemente normalizável. Em outras palavras, a normalização da
substituição \emph{implícita} implica na normalização da substituição
\emph{explícita}. Esta é a chamada propriedade \textbf{IE}.

Para a prova da propriedade IE, é adicionado mais uma estrutura no sistema,
chamada de \emph{substituição marcada}.

\subsection{Substituições marcadas}
\label{sub:subst_marc}

A ideia é controlar as reduções feitas envolvendo substituições explícitas. Para
isto, adicionamos uma nova substituição na gramática.

\[ \tau := x\ |\ \lambda x.\tau\ |\ \tau \tau\ |\ \tau[x/\tau]\ |\
    \tau[[x/u]]\ \]

Note que na nova substituição, não podemos colocar qualquer qualquer termo no
lugar de \textit{u}. Restringimos o termo \textit{u} a apenas termos da gramática
original, ou seja, sem substituições marcadas. Além disso, é necessário que o
termo seja fortementa normalizável.

Na nossa gramática de pré-termos, este formalismo já está adicionado, sendo
representado pelo construtor \texttt{pterm\_lsub}. Porém, assim como precisamos
de um predicado para verificar se um termo comum está bem formado, vamos criar
um outro predicado, chamado \texttt{lab\_term}. Este predicado terá a mesma
funcionalidade, mas agora extendido para termos com substituições marcadas. 

\begin{mathpar} 
    \inferrule*[Right=lab\_term\_var]{  }
    {lc(x)}
    \\
    \inferrule*[Right=lab\_term\_app]{lab\_term(t1) \\ lab\_term(t2)}
    {lc(t1\ t2)}
    \\
    \inferrule*[Right=lab\_term\_abs]{\forall x \notin L,\ lab\_term(t^{x})}
    {lc(\lambda. t)}
    \\
    \inferrule*[Right=lab\_term\_sub]{\forall x \notin L,\ lab\_term(t^{x}) \\
        lab\_term(u)}
    {lc(t[u])}
    \\
    \inferrule*[Right=lab\_term\_sub']{\forall x \notin L,\ lab\_term(t^{x}) \\
        term(u) \\ (SN\ lex)\ u}
    {lc(t[[u]])}
\end{mathpar}

Observe que no caso da substituição marcada, temos a exigência \texttt{(SN lex
    t2)}, que indica que \texttt{t2} é fortemente normalizável.

Também devemos estender a noção equivalente de ser localmente fechado, para
manter a equivalência definida no caso do sistema sem as substituições marcadas.
Assim, definimos uma nova função \texttt{lc\_at'}, análoga à do caso não
marcado.  Sua implementação é também feita com base na definição da tabela
\ref{table:lc_at}, com apenas uma extensão para os casos das substituição
marcadas e explícitas.

\begin{table}[h]
\begin{empheq}[box=\fbox]{align*}
    lc\_at'\ k\ (t1[t2])\ \ \ &\equiv\ (lc\_at'\ (S k)\ t1)\ \&\ (lc\_at'\ k\ t2) \\ 
    lc\_at'\ k\ (t1[[t2]])\ \ \ &\equiv\ (lc\_at'\ (S k)\ t1)\ \&\ (lc\_at'\ k\ t2)\
    \&\ (SN\ lex\ t2) \\ 
\end{empheq}
    \caption{Definição da função lc\_at'}
    \label{table:lab_lc_at}
\end{table}


\begin{lstlisting}[basicstyle=\small]
Fixpoint lc_at' (k:nat) (t:pterm) {struct t} : Prop := ...

Definition term'' t := lc_at' 0 t.
\end{lstlisting}


Podemos seguir a prova da equivalência entre \texttt{lab\_term} e
\texttt{lc\_at'} de maneira similar a prova do capítulo anterior, com a ajuda
dos seguintes lemas auxiliares.

\begin{gather}
 lc\_rec\_open\_var\_rec':\ \forall\ x\ t\ k,\ (lc\_at'\ k\ (open\_rec\ k\ x\
    t))\ \rightarrow\ (lc\_at'\ (S\ k)\ t) \nonumber \\
 lc\_at\_open\_var\_rec:\ \forall\ x\ t\ k,\ (lc\_at'\ (S\ k)\ t)\ \rightarrow\
    (lc\_at'\ k\ (open\_rec\ k\ x\ t)) \nonumber \\
 term\_impl\_lab\_term:\ \forall\ t,\ (term\ t)\ \rightarrow\ (lab\_term\ t)
 \nonumber \\
 term\_eq\_term':\ \forall\ t,\ (term\ t)\ \iff\ (term'\ t) \nonumber
\end{gather}

Como no caso do sistema simples, queremos definir classes de equivalências
de termos, para trabalhar módulo permutação de substituições.
Para isso, precisamos definir novos fechos contextuais para os termos com
substituições marcadas.

\begin{mathpar} 
    \inferrule*[Right=lab\_redex]{ (R\ t\ s) }
    {(lab\_contextual\_closure\ R)\ t\ s}
    \\
    \inferrule*[Right=lab\_app\_left]{((lab\_contextual\_closure\ R)\ t\ t'),\
        (lab\_term\ u)}
    {((lab\_contextual\_closure\ R)\ (t\ u)\ (t'\ u))}
    \\
    \inferrule*[Right=lab\_app\_right]{((lab\_contextual\_closure\ R)\ t\ t'),\
        (lab\_term\ u)}
    {((lab\_contextual\_closure\ R)\ (u\ t)\ (u\ t'))}
    \\
    \inferrule*[Right=lab\_abs\_in]{(\forall\ x,\ x \notin L \rightarrow
        ((lab\_contextual\_closure\ R)\ t^x\ t'^x))}
    {((lab\_contextual\_closure\ R)\ (\lambda . t)\ (\lambda . t'))}
    \\
    \inferrule*[Right=lab\_subst\_left]{(\forall\ x,\ x \notin L \rightarrow
        ((lab\_contextual\_closure\ R)\ t^x\ t'^x)),\ (lab\_term\ u)}
    {((lab\_contextual\_closure\ R)\ (t[u])\ (t'[u]))}
    \\
    \inferrule*[Right=lab\_subst\_right]{((lab\_contextual\_closure\ R)\ u\ u'),\
        (lab\_body\ t)}
    {((lab\_contextual\_closure\ R)\ (t[u])\ (t[u']))}
    \\
    \inferrule*[Right=lab\_subst'\_left]{(\forall\ x,\ x \notin L \rightarrow
        ((lab\_contextual\_closure\ R)\ t^x\ t'^x)),\ (term\ u),\ (SN\ lex\ u)}
    {((lab\_contextual\_closure\ R)\ (t[[u]])\ (t'[[u]]))}
    \\
    \inferrule*[Right=lab\_subst'\_right]{(R\ u\ u'),\
        (lab\_body\ t)}
    {((lab\_contextual\_closure\ R)\ (t[[u]])\ (t[[u']]))}
\end{mathpar}

Em sua maioria, os construtores são análogos aos definidos no caso não marcado.
Vale notar que, ao invés de se necessitar \texttt{term t} como hipótese, é
necessário que o termo seja marcado, ou seja, que consigamos demonstrar
\texttt{lab\_term t}.  Adicionamos também construtores pras substituições
marcadas.  Para uma redução fora da substituição marcada, o processo é análogo
ao da comum.  Para dentro desta, não podemos construir um fecho a partir de
outro: é necessário que o termo \texttt{u} se reduza para \texttt{u'}
diretamente.

São definidos também dois outros fechos análogos,
\texttt{simpl\_lab\_contextual\_closure} e
\texttt{ext\_lab\_contextual\_closure}. No primeiro, a diferença é que não é
feita a exigência \texttt{(SN lex u)} no caso da redução à esquerda da
substituição marcada. No segundo, queremos reduzir apenas fora de substituições
marcadas. Assim, o fecho não possui este caso.


Podemos então definir a relação equacional para termos marcados.

\begin{lstlisting}[basicstyle=\small]
    
Inductive lab_eqc  : pterm -> pterm -> Prop := 
| lab_eqc_rx1 : forall t u v, 
                  lab_term u -> term v -> 
                  lab_eqc (t[u][[v]]) ((& t)[[v]][u]) 
| lab_eqc_rx2 : forall t u v, 
                  term u -> lab_term v -> 
                  lab_eqc (t[[u]][v]) ((& t)[v][[u]]) 
| lab_eqc_rx3 : forall t u v, 
                  term u -> term v -> 
                  lab_eqc (t[[u]][[v]]) ((& t)[[v]][[u]]).
\end{lstlisting}

Os construtores basicamente definem como permutar duas substituições, desde que
uma delas seja marcada. Como requisito, é necessário garantir a propriedade
\texttt{lab\_term} ou \texttt{term}, dependendo da substituição. A ideia é
permitir que as substituições marcadas sejam permutadas ``para dentro" do termo,
passando por uma substuição comum ou marcada. A substituição normal não pode
atravessar a marcada.

Para uso futuro, definimos um lema de simetria da relação. Sua prova é feita por
análise de casos simples, apenas usando o lema \texttt{bswap\_idemp} para
reduzir permutações de índices que não alteram a semântica do termo.

\begin{lstlisting}[basicstyle=\small]
Lemma lab_eqc_sym : forall t u, lab_eqc t u -> lab_eqc u t.
\end{lstlisting}

A equação principal utilizada nos termos marcados será a relação 
$=_{\underline{e}}$, formalizada como o fecho contextual e transitivo do
predicado \texttt{lab\_eqc}.

\begin{lstlisting}[basicstyle=\small]
Definition lab_eqC (t: pterm) (u : pterm) :=  
    trans_closure (simpl_lab_contextual_closure lab_eqc) t u . 
Notation "t =~e u" := (lab_eqC t u) (at level 66).
\end{lstlisting}


Também é preciso definir um predicado análogo de regularidade para termos
marcados.

\begin{table}[h]
\begin{mathpar} 
    \inferrule*{(red\_lab\_regular\ R),\ (R\ t\ u)}
    {lab\_term\ t\ \and lab\_term\ u}
\end{mathpar}
    \caption{Noção de regularidade para termos marcados}
\end{table}

Com todas as estruturas e propriedades para termos marcados bem definidos,
podemos seguir com a prova da propriedade IE.


\subsection{Equivalência de reduções com o sistema original}
\label{sub:equiv_red}

Queremos utilizar esse sistema extendido com as substituições marcadas para
estudar o sistema original. Para isto, precisamos estender a regra de redução do
sistema para lidar com as novas substituições. Definimos então a redução
$\rightarrow_{\underline{x}}$.

\begin{table}[h]
\begin{empheq}[box=\fbox]{align*}
    x[\![x/u]\!]\ \ \             &\rightarrow_{Var}\ u \\
    t[\![x/u]\!]\ \ \             &\rightarrow_{Gc}\ t    & se\ \emph{x} \notin fv(t)\\
    (t\ u)[\![x/v]\!]\ \ \        &\rightarrow_{App}\ t[\![x/v]\!]\ u[\![x/v]\!] \\
    (\lambda y.\ u)[\![x/v]\!]\ \ &\rightarrow_{Lamb}\ (\lambda y.\ u[\![x/v]\!])\\
    t[\![x/u]\!][\![y/v]\!]\ \ \        &\rightarrow_{Comp}\ t[\![y/v]\!][\![x/u[\![y/v]\!]]\!] & se\ y\ \in
    fv(u)  
\end{empheq}
    \caption{A redução $\rightarrow_{x}$ }
    \label{table:red_label_x}

\end{table}

Assim, a relação $\rightarrow_{\lambda \underline{ex}}$ é definida como a união
das reduções $\rightarrow_{Bx}$ e $\rightarrow_{\underline{x}}$, módulo
$=_{\alpha}$, $=_e$ e $=_{\underline{e}}$:

\[ t \rightarrow_{\lambda \underline{ex}} t' \iff \exists s,\ s';\ t =_{e \cup
        \underline{e} \cup \alpha} s \rightarrow_{Bx \cup \underline{x}} s' =_{e \cup
        \underline{e} \cup \alpha} t' \] 

Para provar a propriedade PSN, será necessário relacionar a redução
$\rightarrow_{\lambda \underline{ex}}$ com a redução original,
$\rightarrow_{\lambda ex}$. Para isto, iremos decompor a redução em termos
marcados em duas novas reduções, $\rightarrow_{\lambda \underline{ex}^i}$ e
$\rightarrow_{\lambda \underline{ex}^e}$, que também agem em termos marcados.

A relação $\lambda \underline{ex}^i$, chamada de \emph{redução interna}, é
definida adicionando à redução $\rightarrow_{\underline{ex}}$ a redução
$\rightarrow_{\lambda ex}$ no corpo das substituições marcadas.
Formalmente, a relação $\rightarrow_{\lambda \underline{ex}^i}$ é definida como a
seguinte redução, $\rightarrow_{\lambda \underline{x}^i}$, módulo $=_{\alpha}$,
$=_e$ e $=_{\underline{e}}$:

\begin{itemize}
    \item Se $u \rightarrow_{Bx} u'$ e $u,\ u'$ são termos, então $t[\![x/u]\!]
        \rightarrow_{\lambda \underline{x}^i} t[\![x/u']\!]$ 
    \item Se $t
        \rightarrow_{\underline{x}} t'$, então $t \rightarrow_{\lambda
            \underline{x}^i} t'$
    \item Se $t \rightarrow_{\lambda \underline{x}^i} t'$, então vale 
        $t\ u \rightarrow_{\lambda \underline{x}^i} t'\ u$,  
        $u\ t \rightarrow_{\lambda \underline{x}^i} u\ t'$, 
        $\lambda x. t \rightarrow_{\lambda \underline{x}^i} \lambda x. t'$, 
        $t[x/u] \rightarrow_{\lambda \underline{x}^i} t'[x/u]$, 
        $u[x/t] \rightarrow_{\lambda \underline{x}^i} u[x/t']$ e 
        $t[\![x/u]\!] \rightarrow_{\lambda \underline{x}^i} t'[\![x/u]\!]$.
\end{itemize}


A relação $\lambda \underline{ex}^e$, chamada de \emph{redução externa}, é
definida como a redução $\lambda ex$ em todos os lugares de um
termo, \emph{exceto} no corpo das substituições marcadas.
Formalmente, a relação $\rightarrow_{\lambda \underline{ex}^e}$ é definida como a
seguinte redução, $\rightarrow_{\lambda \underline{x}^e}$, módulo $=_{\alpha}$,
$=_e$ e $=_{\underline{e}}$:

\begin{itemize}
    \item Se $t \rightarrow_{Bx} t'$ ocorre fora de uma substituição marcada, então 
        $t \rightarrow_{\lambda \underline{x}^e} t$ 
    \item Se $t \rightarrow_{\lambda \underline{x}^e} t'$, então vale 
        $t\ u \rightarrow_{\lambda \underline{x}^e} t'\ u$,  
        $u\ t \rightarrow_{\lambda \underline{x}^e} u\ t'$, 
        $\lambda x. t \rightarrow_{\lambda \underline{x}^e} \lambda x. t'$, 
        $t[x/u] \rightarrow_{\lambda \underline{x}^e} t'[x/u]$, 
        $u[x/t] \rightarrow_{\lambda \underline{x}^e} u[x/t']$ e 
        $t[\![x/u]\!] \rightarrow_{\lambda \underline{x}^e} t'[\![x/u]\!]$.
\end{itemize}

O objetivo principal deste trabalho será a formalização destas duas reduções e a
prova de equivalências da união destas com a redução ``\texttt{-{}->[lex]}'', que é a
redução $\lambda \underline{ex}$ formalizada.

Para a formalização das reduções $\lambda \underline{ex}^i$ e $\lambda
\underline{ex}^e$, definimos os seguintes fechos e relações:

\begin{lstlisting}[basicstyle=\small]
Inductive lab_x_i: pterm -> pterm -> Prop :=
| xi_from_bx_in_les: forall t1 t2 t2', 
                       lab_term (t1 [[ t2 ]]) ->
                       (sys_Bx t2 t2') ->
                       lab_x_i (t1 [[ t2 ]]) (t1 [[ t2' ]])
| xi_from_x : forall t t', 
                lab_term t ->
                lab_sys_x t t' -> 
                lab_x_i t t'. 

Definition lab_EE_ctx_red 
(R: pterm -> pterm -> Prop) (t: pterm) (u : pterm) := 
exists t' u', (t =EE t')/\(lab_contextual_closure R t' u')/\(u' =EE u).

Definition ext_lab_EE_ctx_red 
(R: pterm -> pterm -> Prop) (t: pterm) (u : pterm) := 
exists t' u', (t =EE t')/\(ext_lab_contextual_closure R t' u')/\(u' =EE u).

Definition lab_x_i_eq := ext_lab_EE_ctx_red lab_x_i.
Definition lab_x_e_eq := ext_lab_EE_ctx_red sys_Bx.

Notation "t -->[lx_i] u" := (lab_x_i_eq t u) (at level 59, left associativity).
Notation "t -->[lx_e] u" := (lab_x_e_eq t u) (at level 59, left associativity).
\end{lstlisting}

A relação \texttt{lab\_x\_i} formaliza a relação $\rightarrow_{\underline{x}}$.
O fecho \texttt{ext\_lab\_EE\_ctx\_red} realiza uma relação em qualquer ponto de
um termo, \textbf{exceto} dentro de uma substituição marcada. Ele permite
permutação de substituições tanto antes quanto depois da redução ser feita. Este
fecho é utilizado então para formalizar as reduções de interesse. Assim,
$\lambda \underline{ex}^i$ é formalizada como a redução "t -{}->[lx\_i] u",
definida por \texttt{(ext\_lab\_EE\_ctx\_red lab\_x\_i) t u}. Por outro lado, a
relação $\lambda \underline{ex}^e$ formalizada como a redução "t -{}->[lx\_e] u",
definida por \texttt{(ext\_lab\_EE\_ctx\_red lab\_x\_e) t u}. 

Uma estratégia utilizada para facilitar as provas foi abstrair, em lemas
auxiliares, vários problemas destas que se repetem. Podemos então tratar tais
problemas com um contexto limpo, facilitando as provas por indução. 

\begin{lstlisting}[basicstyle=\small]
Lemma star_lab_closure_app_left: forall R t t' u, lab_term u -> 
star_closure (simpl_lab_contextual_closure R) t t' -> 
star_closure (simpl_lab_contextual_closure R) (pterm_app t u) (pterm_app t' u).


Lemma EE_clos_app_left: forall R t t' u, lab_term u -> 
((lab_EE_ctx_red R) t t') -> 
((lab_EE_ctx_red R) (pterm_app t u) (pterm_app t' u)).


Lemma EE_ext_clos_app_left: forall R t t' u, lab_term u -> 
((ext_lab_EE_ctx_red R) t t') -> 
((ext_lab_EE_ctx_red R) (pterm_app t u) (pterm_app t' u)).
\end{lstlisting}

Os lemas acima exemplificam uma das principais simplificações feita: para cada
fecho, definimos um lema para se realizar a redução em um termo maior, a partir
da redução em um subtermo. Estes lemas são úteis para se evitar que tenhamos que
destrinchar as reduções dentro da prova principal, facilitando muito o processo.
Para cada construtor de termo temos um lema análogo a cada um dos três acima
exemplificados, \texttt{star\_lab\_closure\_app\_left},
\texttt{EE\_clos\_app\_left} e \texttt{EE\_ext\_clos\_app\_left}.

Além disso, alguns lemas para lidar com a relação entre as reduções e equações
foram precisos:

\bigskip
\coqdockw{Lemma} \coqdocvar{EE\_presv\_ie}: \coqdockw{\ensuremath{\forall}}
\coqdocvar{t} \coqdocvar{t'} \coqdocvar{u} \coqdocvar{u'}, \coqdocvar{t}
=\coqdocvar{EE} \coqdocvar{u} \ensuremath{\rightarrow} \coqdocvar{u'}
=\coqdocvar{EE} \coqdocvar{t'} \ensuremath{\rightarrow} ((\coqdocvar{u}
-->[\coqdocvar{lx\_i}] \coqdocvar{u'} \ensuremath{\lor} \coqdocvar{u}
-->[\coqdocvar{lx\_e}] \coqdocvar{u'}) \ensuremath{\rightarrow} (\coqdocvar{t}
-->[\coqdocvar{lx\_i}] \coqdocvar{t'} \ensuremath{\lor} \coqdocvar{t}
-->[\coqdocvar{lx\_e}] \coqdocvar{t'})).\coqdoceol

\smallskip

\coqdockw{Lemma} \coqdocvar{EE\_presv\_lab\_lex}:
\coqdockw{\ensuremath{\forall}} \coqdocvar{t} \coqdocvar{t'} \coqdocvar{u}
\coqdocvar{u'}, \coqdocvar{t} =\coqdocvar{EE} \coqdocvar{u}
\ensuremath{\rightarrow} \coqdocvar{u'} =\coqdocvar{EE} \coqdocvar{t'}
\ensuremath{\rightarrow} ((\coqdocvar{u} -->[\coqdocvar{lex}] \coqdocvar{u'})
\ensuremath{\rightarrow} (\coqdocvar{t} -->[\coqdocvar{lex}]
\coqdocvar{t'})).\coqdoceol
\bigskip


Novamente, esses lemas evitam que precisemos adentrar na definição das
equações, reduzindo o tamanho das provas principais. Podemos então partir para
o resultado principal deste trabalho.

\bigskip

\coqdockw{Theorem} \coqdocvar{lab\_ex\_eq\_i\_e}:
\coqdockw{\ensuremath{\forall}} \coqdocvar{t} \coqdocvar{t'},
\coqdocvar{lab\_term} \coqdocvar{t} \ensuremath{\rightarrow} (\coqdocvar{t}
-->[\coqdocvar{lex}] \coqdocvar{t'} \ensuremath{\leftrightarrow} (\coqdocvar{t}
-->[\coqdocvar{lx\_i}] \coqdocvar{t'} \ensuremath{\lor} \coqdocvar{t}
-->[\coqdocvar{lx\_e}] \coqdocvar{t'})).\coqdoceol

\bigskip

Dividimos a prova de que $\lambda \underline{ex} =
\lambda \underline{ex}^i \cup \lambda \underline{ex}^e$, formalizada no teorema
\texttt{lab\_ex\_eq\_i\_e}, em dois lemas, cada um representado uma direção da
equivalência.

\bigskip

\coqdocnoindent \coqdockw{Lemma} \coqdocvar{lab\_ex\_impl\_i\_e}:
\coqdockw{\ensuremath{\forall}} \coqdocvar{t} \coqdocvar{t'},
\coqdocvar{lab\_term} \coqdocvar{t} \ensuremath{\rightarrow} \coqdocvar{t}
-->[\coqdocvar{lex}] \coqdocvar{t'} \ensuremath{\rightarrow} (\coqdocvar{t}
-->[\coqdocvar{lx\_i}] \coqdocvar{t'} \ensuremath{\lor} \coqdocvar{t}
-->[\coqdocvar{lx\_e}] \coqdocvar{t'}).\coqdoceol

\begin{addmargin}[1em]{2em}
\textbf{Prova:} Escolhemos, para cada redução possível feita pela relação
\texttt{-{}->[lex]}, a redução apropriada entre a interna e a externa.  Para
isto, abrimos a definição da relação \texttt{-{}->[lex]} e fazemos indução no
fecho contextual, ou seja, fazemos indução no predicado
\texttt{lab\_contextual\_closure lab\_sys\_lx t t'}. O caso base é tratado no
lema auxiliar \texttt{lab\_sys\_x\_i\_e}, feito com análise de casos simples nos
construtores da relação \texttt{lab\_sys\_lx}. 

Nos passos indutivos, utilizamos o lema \texttt{EE\_presv\_ie} para adequar o
objetivo à hipótese de indução, substituindo os termos dados pelos termos
equivalentes, obtidos pela definição da redução \texttt{-{}->[lex]}, podendo assim
aplicar a hipótese. No caso em que a redução é feita dentro da substituição
marcada, devemos obrigatoriamente realizar a redução interna. Em todos os outros
casos, realizamos a prova tanto para a redução interna quanto para a externa.
Nos casos em que lidamos com abstrações e substituições, são necessários os
lemas de renomeamento mencionados na subseção \ref{sub:subst_marc}.
\end{addmargin}

\bigskip

\coqdocnoindent \coqdockw{Lemma} \coqdocvar{lab\_ie\_impl\_ex}:
\coqdockw{\ensuremath{\forall}} \coqdocvar{t} \coqdocvar{t'},
\coqdocvar{lab\_term} \coqdocvar{t} \ensuremath{\rightarrow} (\coqdocvar{t}
-{}->[\coqdocvar{lx\_i}] \coqdocvar{t'} \ensuremath{\lor} \coqdocvar{t}
-{}->[\coqdocvar{lx\_e}] \coqdocvar{t'}) \ensuremath{\rightarrow} \coqdocvar{t}
-{}->[\coqdocvar{lex}] \coqdocvar{t'}.\coqdoceol

\begin{addmargin}[1em]{2em}
\textbf{Prova:} A prova deste lema é dividida em duas partes: quando a redução realizada é a
interna, e quando é a externa. 

No caso da interna, a indução é feita no fecho do predicado interno, ou seja, é
feita no predicado \texttt{ext\_lab\_contextual\_closure lab\_x\_i t t'}. O caso
base é feito apenas analisando os construtores da relação \texttt{lab\_x\_i}, e
casando com o construtor adequado de lab\_sys\_lx. Nos passos indutivos, fazemos
de maneira análoga ao lema anterior: utilizamos agora o predicado
\texttt{EE\_presv\_lab\_lex} para ajustar o objetivo à hipótese, e utilizamos os
lemas relacionando as reduções aos construtores, como
\texttt{EE\_clos\_app\_left}, para reduzir o objetivo à redução dada como
hipótese. Novamente, no caso de abstrações e substituições, precisamos dos lemas
de renomeamento.

No caso da externa, o processo é exatamente o mesmo. A diferença é que no passo
base, fazemos a análise de casos na redução \texttt{sys\_Bx} e, além disso, não
temos que lidar com o caso da redução dentro de substituição marcada.
\end{addmargin}

\bigskip

Com ambos os lemas completos, a prova do teorema \texttt{lab\_ex\_eq\_i\_e} se
reduz a apenas aplicá-los. Assim, terminamos a prova de que a redução marcada
formalizada, \texttt{-{}->[lex]} é equivalente à união das reduções interna
e externa, \texttt{-{}->[lx\_i]} e \texttt{-{}->[lx\_e]}.

  \chapter{Conclusão}

Cálculos de substituição explícita são importantes por servirem como um
framework formal para o estudo de propriedades de sistemas reais, como
implementações de linguagens funcionais e assistentes de prova. Desta forma, uma
formalização se torna interessante pois fornece uma maior garantia na
confiabilidade destes sistemas.

Neste trabalho, foi continuada a formalização \footnote{ Esta formalização está
    disponível em \url{https://github.com/Lucas1993/LambdaEX_TCC}} do cálculo
$\lambda ex$, iniciada em \cite{initial}. Definimos, dentro do sistema, todas as
definições e propriedades relacionadas às substituições marcadas, e provamos
suas características principais. Iniciamos, então, a prova da propriedade IE,
realizando a prova formal da equivalência entre a redução do sistema estendida
para substituições marcadas, $\lambda \underline{ex}$, e a união das reduções
interna e externa, $\lambda \underline{ex}^i$ e $\lambda \underline{ex}^e$.

Como trabalho futuro, queremos estudar propriedades de normalização do sistema,
procurando uma definição mais intuitiva para o conceito de \textbf{normalização
forte}. Com isto, poderemos concluir a prova da propriedade IE dentro do
sistema, finalizando, assim, a formalização do cálculo $\lambda ex$.


  % ...

  \postextual
  \bibliographystyle{abbrv}
  \bibliography{bibliografia}

\end{document}

\chapter{Conclusão}

Cálculos de substituição explícita são importantes por servirem como um
framework formal para o estudo de propriedades de sistemas reais, como
implementações de linguagens funcionais e assistentes de prova. Desta forma, uma
formalização se torna interessante pois fornece uma maior garantia na
confiabilidade destes sistemas.

Neste trabalho, foi continuada a formalização \footnote{ Esta formalização está
    disponível em \url{https://github.com/Lucas1993/LambdaEX_TCC}} do cálculo
$\lambda ex$, iniciada em \cite{initial}. Definimos, dentro do sistema, todas as
definições e propriedades relacionadas às substituições marcadas, e provamos
suas características principais. Iniciamos, então, a prova da propriedade IE,
realizando a prova formal da equivalência entre a redução do sistema estendida
para substituições marcadas, $\lambda \underline{ex}$, e a união das reduções
interna e externa, $\lambda \underline{ex}^i$ e $\lambda \underline{ex}^e$.

Como trabalho futuro, queremos estudar propriedades de normalização do sistema,
procurando uma definição mais intuitiva para o conceito de \textbf{normalização
forte}. Com isto, poderemos concluir a prova da propriedade IE dentro do
sistema, finalizando, assim, a formalização do cálculo $\lambda ex$.


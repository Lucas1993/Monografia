\chapter{O sistema $\lambda ex$}

\section{Visão geral}
\label{sec:int_lex}

%\subsection{Descrição do sistema}

Como visto no capítulo anterior, várias extensões do cálculo $\lambda$ foram
propostas com o objetivo de expressar de maneira concisa a noção de substituição
explícita.

O sistema proposto em \cite{delia}, chamado $\lambda$ex, é o primeiro sistema
que captura de maneira simples tal noção, enquanto ainda possui a propridade
\textbf{PSN}, ou seja, a preservação da normalização forte. Este sistema será
repetido aqui, e um maior entendimento pode ser obtido na fonte original. São
introduzidas várias mudanças, a começar pela gramática: 

\[ \tau := x\ |\ \lambda x.\tau\ |\ \tau \tau\ |\ \tau[x/\tau]\ \]
\

\begin{table}[h]
    
\begin{empheq}[box=\fbox]{align*}
    x[x/u]\ \ \             &\rightarrow_{Var}\ u \\
    t[x/u]\ \ \             &\rightarrow_{Gc}\ t    & se\ \emph{x} \notin fv(t)\\
    (t\ u)[x/v]\ \ \        &\rightarrow_{App}\ t[x/v]\ u[x/v] \\
    (\lambda y.\ u)[x/v]\ \ &\rightarrow_{Lamb}\ (\lambda y.\ u[x/v])\\
    t[x/u][y/v]\ \ \        &\rightarrow_{Comp}\ t[y/v][x/u[y/v]] & se\ y\ \in
    fv(u) \\ 
    (\lambda x.\ t)\ u\ \ \ &\rightarrow_B\ t[x/u]
\end{empheq}
    \caption{Regras de redução}
    \label{table:red_lambex}

\end{table}

A nova construção é chamada \textit{substituição explícita}, e é o que permite as
manipulações sintáticas com substituições no cálculo. Como consequência desta
mudança, novas regras de redução são definidas, como mostrado na tabela
\ref{table:red_lambex}.  As 5 primeiras regras formam a relação $\rightarrow_x$,
e o acréscimo da última forma a relação $\rightarrow_{Bx}$. 

Observe que, devido ao acréscimo das substituições explícitas, é possível
definir termos que possuem uma sintaxe distinta, diferindo apenas na permutação
de substituições independentes, e que constiturem o mesmo
$\lambda$-termo no sistema original. Um exemplo simples pode ser visto a seguir.

\[ (0\ 1) \{0/x\}\{1/y\} = (0\{0/x\}\ 1\{1/y\}) = (x\ y) =  (0\ 1) \{1/y\}\{0/x\} \]
\[ (0\ 1) [0/x][1/y] \neq  (0\ 1) [1/y][0/x] \]

Onde a noção de igualdade usada é a sintática.

Assim, é necessário tornar o novo sistema
equacional, adicionando uma relação de equivalência:

\[ t[x/u][y/v] =_C t[y/v][x/u] \ \ \ \ \ se\ y \notin fv(u)\ \&\ x \notin fv(v)\] 

A relação de equivalência $=_e$ é formada com a junção de $=_C$ e
$\alpha$-equivalência. As relações $\rightarrow_{ex}$ e $\rightarrow_{\lambda
    ex}$ são definidas como:

\[t\ \rightarrow_{ex}\ t'\ \iff\ \exists\ s,s'\ t.q.\ t\ =_{e}\ s\
    \rightarrow_x\ s'\ =_e\ t' \]
\[t\ \rightarrow_{\lambda ex}\ t'\ \iff\ \exists\ s,s'\ t.q.\ t\ =_{e}\ s\
    \rightarrow_{Bx}\ s'\ =_e\ t' \]

Estas relações serão amplamente utilizadas ao decorrer deste trabalho. 
O foco principal é a formalização da propriedade \textbf{IE} do sistema, que
pode ser entendida intuitivamente da seguinte maneira: Dados $t$,
$u$ tais que $u$ e $t\{x/u\}$ são fortemente normalizáveis
no sistema, então o correspondente utilizando substituições explícitas,
$t[x/u]$ também será. Ou seja, a normalização da substituição implícita
implica a normalização da explícita.

\subsection{A Gramática de pré-termos}
\label{sec:termos}

Nesta seção, será feita um paralelo entre as estruturas básicas do sistema
$\lambda ex$ e sua formalização em Coq.

Como visto em \ref{sec:int_lex}, o sistema possui uma gramática que pode ser vista como
uma extensão do cálculo $\lambda$ original, consistindo de variáveis, abstrações,
aplicações e \emph{substituições explícitas}. Ela pode ser descrita sucintamente
como abaixo.

\[ \tau := x\ |\ \lambda x.\tau\ |\ \tau \tau\ |\ \tau[x/\tau]\ \]

Já encontramos então nossa primeira divergência na formalização. Neste projeto,
será usada a representação chamada \textit{Locally Nameless Representation}
(LNR), onde as variáveis ligadas são representadas por índices. Por conta disso,
existem termos nesta nova gramática que não possuem correspondentes no sistema
original, por conta de índices que não estão ligados a nenhuma abstração.

Assim, torna-se necessária uma gramática de \emph{pré-termos}, que consiste de
todos os termos possíveis de serem escritos em LNR. Esta gramática é formalizada
usando um tipo indutivo.

\bigskip
\coqdocnoindent \coqdockw{Inductive} \coqdef{LambdaES
    Defs.pterm}{pterm}{\coqdocinductive{pterm}} : \coqdockw{Set} :=\coqdoceol
\coqdocindent{1.00em} \ensuremath{|} \coqdef{LambdaES Defs.pterm
    bvar}{pterm\_bvar}{\coqdocconstructor{pterm\_bvar}} :
\coqexternalref{nat}{http://coq.inria.fr/distrib/8.4pl4/stdlib/Coq.Init.Datatypes}{\coqdocinductive{nat}}
\ensuremath{\rightarrow} \coqref{LambdaES
    Defs.pterm}{\coqdocinductive{pterm}}\coqdoceol \coqdocindent{1.00em}
\ensuremath{|} \coqdef{LambdaES Defs.pterm
    fvar}{pterm\_fvar}{\coqdocconstructor{pterm\_fvar}} : \coqdocdefinition{var}
\ensuremath{\rightarrow} \coqref{LambdaES
    Defs.pterm}{\coqdocinductive{pterm}}\coqdoceol \coqdocindent{1.00em}
\ensuremath{|} \coqdef{LambdaES Defs.pterm
    app}{pterm\_app}{\coqdocconstructor{pterm\_app}}  : \coqref{LambdaES
    Defs.pterm}{\coqdocinductive{pterm}} \ensuremath{\rightarrow}
\coqref{LambdaES Defs.pterm}{\coqdocinductive{pterm}} \ensuremath{\rightarrow}
\coqref{LambdaES Defs.pterm}{\coqdocinductive{pterm}}\coqdoceol
\coqdocindent{1.00em} \ensuremath{|} \coqdef{LambdaES Defs.pterm
    abs}{pterm\_abs}{\coqdocconstructor{pterm\_abs}}  : \coqref{LambdaES
    Defs.pterm}{\coqdocinductive{pterm}} \ensuremath{\rightarrow}
\coqref{LambdaES Defs.pterm}{\coqdocinductive{pterm}}\coqdoceol
\coqdocindent{1.00em} \ensuremath{|} \coqdef{LambdaES Defs.pterm
    sub}{pterm\_sub}{\coqdocconstructor{pterm\_sub}} : \coqref{LambdaES
    Defs.pterm}{\coqdocinductive{pterm}} \ensuremath{\rightarrow}
\coqref{LambdaES Defs.pterm}{\coqdocinductive{pterm}} \ensuremath{\rightarrow}
\coqref{LambdaES Defs.pterm}{\coqdocinductive{pterm}} \coqdoceol
\coqdocindent{1.00em} \ensuremath{|} \coqdef{LambdaES Defs.pterm
    lsub}{pterm\_lsub}{\coqdocconstructor{pterm\_lsub}} : \coqref{LambdaES
    Defs.pterm}{\coqdocinductive{pterm}} \ensuremath{\rightarrow}
\coqref{LambdaES Defs.pterm}{\coqdocinductive{pterm}} \ensuremath{\rightarrow}
\coqref{LambdaES Defs.pterm}{\coqdocinductive{pterm}}.\coqdoceol 
\bigskip

Veja que variáveis livres e ligadas possuem construtores distintos. Para as
livres, o construtor correto é \texttt{pterm\_bvar}, que recebe um natural
representando um índice. As variáveis livres são construídas com
\texttt{pterm\_fvar}, recebendo um elemento do tipo \texttt{var}, definido no
framework de Charguéraud.
As aplicações, abstrações e substituições são representadas através do
construtores \texttt{pterm\_app}, \texttt{pterm\_abs} e \texttt{pterm\_sub},
respectivamente. Todos recebem um ou dois pré-termos, que correspondem aos
termos internos das estruturas.

O último construtor, \texttt{pterm\_lsub}, será usado para representar
\emph{substituições marcadas}, um formalismo necessário para as provas de
correspondência com a meta-substituição.

\section{Termos e Relações}
\subsection{Termos bem formados}

Devido à correspondência assimétrica entre os termos do sistema $\lambda ex$ e
em LNR, torna-se necessário definir os predicados de boa formação de termos.

Como pré-requisito para a definição destes predicados, precisamos implementar as
noções de abertura e fechamento de termos, como visto em \ref{sub:int_lnr}. 
Relembrando, a operação de abertura foi definida como $t^{x} \equiv t\{0/x\}$.

Para implementar tal operação, precisaremos de duas funções, \texttt{open\_rec}
e \texttt{open}.

\bigskip
\coqdockw{Fixpoint} \coqdef{LambdaES Defs.open
    rec}{open\_rec}{\coqdocdefinition{open\_rec}} (\coqdocvar{k} :
\coqexternalref{nat}{http://coq.inria.fr/distrib/8.4pl4/stdlib/Coq.Init.Datatypes}{\coqdocinductive{nat}})
(\coqdocvar{u} : \coqref{LambdaES Defs.pterm}{\coqdocinductive{pterm}})
(\coqdocvar{t} : \coqref{LambdaES Defs.pterm}{\coqdocinductive{pterm}})
\{\coqdockw{struct} \coqdocvar{t}\} : \coqref{LambdaES
    Defs.pterm}{\coqdocinductive{pterm}} := ...\coqdoceol

\coqdockw{Definition} \coqdef{LambdaES Defs.open}{open}{\coqdocdefinition{open}}
\coqdocvar{t} \coqdocvar{u} := \coqref{LambdaES Defs.open
    rec}{\coqdocdefinition{open\_rec}} 0 \coqdocvariable{u}
\coqdocvariable{t}.\coqdoceol
\bigskip

A primeira adentra o termo \texttt{t} recursivamente, procurando pelo índice
\texttt{k} e substituindo pelo termo \texttt{u}. Ao encontrar uma abstração ou
substituição, o índice \texttt{k} é incrementado.  A segunda, que é a chamada da
operação de fato, apenas chama \texttt{open\_rec} com $k = 0$.

Similarmente, a operação de fechamento foi definida como $ ^{\textbackslash x}t
\equiv \{0 \leftarrow t\}$. Esta operação é definida através da função
\texttt{close}.

\bigskip

\coqdockw{Fixpoint} \coqdef{LambdaES Defs.close
    rec}{close\_rec}{\coqdocdefinition{close\_rec}}  (\coqdocvar{k} :
\coqexternalref{nat}{http://coq.inria.fr/distrib/8.4pl4/stdlib/Coq.Init.Datatypes}{\coqdocinductive{nat}})
(\coqdocvar{x} : \coqdocdefinition{var}) (\coqdocvar{t} : \coqref{LambdaES
    Defs.pterm}{\coqdocinductive{pterm}}) \{\coqdockw{struct} \coqdocvar{t}\} :
\coqref{LambdaES Defs.pterm}{\coqdocinductive{pterm}} := ...\coqdoceol

\coqdockw{Definition} \coqdef{LambdaES
    Defs.close}{close}{\coqdocdefinition{close}} \coqdocvar{t} \coqdocvar{x} :=
\coqref{LambdaES Defs.close rec}{\coqdocdefinition{close\_rec}} 0
\coqdocvariable{x} \coqdocvariable{t}.\coqdoceol

\bigskip

\begin{definicao}\label{def_term}
Sejam $t, t1, t2$ pré-termos; $L$ um conjunto finito de variáveis e $x$ uma variável.
Definimos o predicado \texttt{term}, representando termos bem
formados, com base nas seguintes regras de inferência.

\begin{mathpar} 
    \inferrule*[Right=term\_var]{  }
    {term(x)}
    \and
    \inferrule*[Right=term\_app]{term\ t1 \\ term\ t2}
    {term(t1\ t2)}
    \\
    \inferrule*[Right=term\_abs]{\forall x \notin L,\ term\ (t^{x})}
    {lc(\lambda. t)}
    \and
    \inferrule*[Right=term\_sub]{\forall x \notin L,\ term\ (t^{x}) \\ term\ u}
    {term(t[u])}
\end{mathpar}
\end{definicao}

O predicado \texttt{term} recebe um pré-termo, e indica que este elemento é bem
formado, ou seja, não possui índices livres. Para cada tipo de pré-termo, temos
um construtor diferente do predicado. Observe que não existe regra de inferência
para índices livres, como desejado.

Em alguns casos, como dito na subseção \ref{sub:int_lnr}, pode ser mais
vantajoso verificar se um termo está bem formado com uma função recursiva.
Tal verificação é feita através do predicado \texttt{term'}, que deve funcionar de
maneira equivalente ao predicado \texttt{term}. Este novo predicado é definido
com base na definição recursiva \texttt{lc\_at}. A função \texttt{lc\_at}
verifica se o termo \emph{t} está \textit{fechado} a um nível \emph{k}, ou seja,
se não existe índice livre de valor maior ou igual a \emph{k}. Sua implementação
é feita com base na Definição \ref{def_lc_at}, adicionando apenas o caso a
seguir, para substituições explícitas.

\[  lc\_at\ k\ (t1[t2])\ \ \ \equiv\ (lc\_at\ (S k)\ t1)\ \&\ (lc\_at\ k\ t2) \]

\bigskip
\coqdockw{Fixpoint} \coqdocvar{lc\_at} (\coqdocvar{k}:\coqdocvar{nat})
(\coqdocvar{t}:\coqdocvar{pterm}) \{\coqdockw{struct} \coqdocvar{t}\} :
\coqdockw{Prop} :=\coqdoceol

\coqdockw{Definition} \coqdocvar{term'} \coqdocvar{t} := \coqdocvar{lc\_at} 0 \coqdocvar{t}.\coqdoceol
\bigskip

Assim, essa equivalência também teve que ser formalizada, utilizando uma série
de lemas auxiliares. 

Primeiramente, foram provados dois resultados de enfraquecimento para o
\texttt{lc\_at}.

\bigskip
\coqnoindent \coqdockw{Lemma} \coqdocvar{lc\_rec\_open\_var\_rec} :
\coqdockw{\ensuremath{\forall}} \coqdocvar{x} \coqdocvar{t}
\coqdocvar{k}, \coqdocvar{lc\_at} \coqdocvar{k}
(\coqdocvar{open\_rec} \coqdocvar{k} \coqdocvar{x} \coqdocvar{t})
\ensuremath{\rightarrow} \coqdocvar{lc\_at} (\coqdocvar{S} \coqdocvar{k})
\coqdocvar{t}.\coqdoceol

\coqnoindent \coqdockw{Lemma} \coqdocvar{lc\_at\_open\_var\_rec} :
\coqdockw{\ensuremath{\forall}} \coqdocvar{x} \coqdocvar{t}
\coqdocvar{k}, \coqdocvar{lc\_at} (\coqdocvar{S}
\coqdocvar{k}) \coqdocvar{t} \ensuremath{\rightarrow} \coqdocvar{lc\_at}
\coqdocvar{k} (\coqdocvar{open\_rec} \coqdocvar{k} (\coqdocvar{pterm\_fvar}
\coqdocvar{x}) \coqdocvar{t}).\coqdoceol
\bigskip

O primeiro lema garante que, se um termo \emph{t}, aberto com uma variável a um
nível \emph{k}, é fechado a este mesmo nível \emph{k}, então o termo \emph{t}
sem a abertura é fechado a nível \emph{k + 1}.

O segundo é exatamente o contrário: Se um termo \emph{t} é fechado a um nível
\emph{k + 1}, então este mesmo termo, aberto com uma variável a nível \emph{k},
é fechado a nível \emph{k}.

Já podemos agora provar a equivalência entre as duas definições de termos bem
formados.

\bigskip
\coqdockw{Lemma} \coqdocvar{term\_eq\_term'} : \coqdockw{\ensuremath{\forall}}
\coqdocvar{t}, \coqdocvar{term} \coqdocvar{t} \ensuremath{\leftrightarrow}
\coqdocvar{term'} \coqdocvar{t}.\coqdoceol
\bigskip

\begin{proof}
\begin{itemize}
    \item[($\rightarrow$)] Este caso é bem direto, fazendo uma indução no
        predicado \texttt{term t}. As hipóteses de indução resolvem os casos
        diretamente, exceto para abstração e substituição. Nesses, a hipótese é
        dada para $t^x$, quando precisamos provar $lc\_at\ 1\ t$. Felizmente os
        lemas de enfraquecimento acima resolvem, bastando usar o lema
        lc\_rec\_open\_var\_rec.
    \item[($\leftarrow$)] O segundo caso não é tão imediato, pois fazemos uma
        indução no \emph{tamanho do termo}, para auxiliar na prova. A hipótese
        de indução se refere a termos com tamanho igual ao do termo sobre qual
        queremos provar a propriedade. Isto é útil pois o tamanho de um termo ao
        ser aberto não muda. Novamente, no caso da abstração e substituição,
        precisamos fazer um ajuste utilizando o lema lc\_at\_open\_var\_rec.
\end{itemize}
\end{proof}

\subsection{Equivalência de termos}
\label{sub:equival_ncia_de_termos}

Como dito anteriormente, o acréscimo de substituições explícitas pode nos levar
a casos em que um mesmo termo no cálculo original tenha duas representações no
sistema. Observe o exemplo abaixo:

\[ t\{x/u\}\{y/v\},\ \ x \notin fv(v),\ y \notin fv(u) \]
\[ t[x/u][y/v]\ \ \ t[y/v][x/u]\]

Como as substituições são independentes, ambas as representações são válidas.
Para resolver este problema, foi adicionado no sistema uma regra de
\emph{permutação de substituições independentes}.

\[ t[x/u][y/v] =_C t[y/v][x/u] \ \ \ \ \ se\ y \notin fv(u)\ \&\ x \notin fv(v)\] 

Dizemos que ambas as representações são \emph{equivalentes}.

Precisamos, então, formalizar essa definição no sistema.

\bigskip
\coqdocnoindent
\coqdockw{Inductive} \coqdef{Equation C.eqc}{eqc}{\coqdocinductive{eqc}} :
\coqdocinductive{pterm} \ensuremath{\rightarrow} \coqdocinductive{pterm}
\ensuremath{\rightarrow} \coqdockw{Prop} := \coqdoceol \coqdocindent{1.00em}
\ensuremath{|} \coqdef{Equation C.eqc
    def}{eqc\_def}{\coqdocconstructor{eqc\_def}}:
\coqdockw{\ensuremath{\forall}} \coqdocvar{t} \coqdocvar{u} \coqdocvar{v},
\coqdocdefinition{lc\_at} 2 \coqdocvariable{t} \ensuremath{\rightarrow}
\coqdocinductive{term} \coqdocvariable{u} \ensuremath{\rightarrow}
\coqdocinductive{term} \coqdocvariable{v} \ensuremath{\rightarrow}
\coqref{Equation C.eqc}{\coqdocinductive{eqc}}
(\coqdocvariable{t}\coqdocnotation{[}\coqdocvariable{u}\coqdocnotation{][}\coqdocvariable{v}\coqdocnotation{]})
(\coqdocnotation{(}\coqdocnotation{\&}
\coqdocvariable{t}\coqdocnotation{)[}\coqdocvariable{v}\coqdocnotation{][}\coqdocvariable{u}\coqdocnotation{]}).\coqdoceol
\bigskip

As exigências que \emph{u} e \emph{v} sejam termos são puramente técnicas, pois
queremos trabalhar apenas com termos sem índices livres no sistema. O único
outro detalhe que muda na formalização é ocorrência do operador $\&$, que
troca todas as ocorrências de índices zero e um, para que eles continuem se
referindo à mesma substituição.

Note que esta definição ainda possui muitas limitações. Por exemplo, se um termo
possui uma lista de substituições, só podemos trocar as duas últimas. Também não
é possível realizar permutações em subtermos, ou várias permutações seguidas.
Assim, precisamos criar \emph{fechos} em cima dessa definição, para ajustar às
nossas necessidades. 

\begin{definicao}[Fecho contextual]\label{def_fecho}
    Sejam $t,t',u$ pré-termos; $x$ uma variável e $R$ uma relação entre
    pré-termos. As regras a seguir definem o fecho contextual de R.

\begin{mathpar} 
    \inferrule*[Right=ES\_redex]{ (R\ t\ t') }
    {(ES\_contextual\_closure\ R)\ t\ t'}
    \\
    \inferrule*[Right=ES\_app\_left]{((ES\_contextual\_closure\ R)\ t\ t'),\
        (term\ u)}
    {((ES\_contextual\_closure\ R)\ (t\ u)\ (t'\ u))}
    \\
    \inferrule*[Right=ES\_app\_right]{((ES\_contextual\_closure\ R)\ t\ t'),\
        (term\ u)}
    {((ES\_contextual\_closure\ R)\ (u\ t)\ (u\ t'))}
    \\
    \inferrule*[Right=ES\_abs\_in]{(\forall\ x,\ x \notin L \rightarrow
        ((ES\_contextual\_closure\ R)\ t^x\ t'^x))}
    {((ES\_contextual\_closure\ R)\ (\lambda . t)\ (\lambda . t'))}
    \\
    \inferrule*[Right=ES\_subst\_left]{(\forall\ x,\ x \notin L \rightarrow
        ((ES\_contextual\_closure\ R)\ t^x\ t'^x)),\ (term\ u)}
    {((ES\_contextual\_closure\ R)\ (t[u])\ (t'[u]))}
    \\
    \inferrule*[Right=ES\_subst\_right]{((ES\_contextual\_closure\ R)\ u\ u'),\
        (body\ t)}
    {((ES\_contextual\_closure\ R)\ (t[u])\ (t[u']))}
\end{mathpar}

\end{definicao}

\bigskip
\coqdockw{Definition} \coqdef{Equation C.eqc ctx}{eqc\_ctx}{\coqdocdefinition{eqc\_ctx}} (\coqdocvar{t} \coqdocvar{u}: \coqdocinductive{pterm}) := \coqdocinductive{ES\_contextual\_closure} \coqref{Equation C.eqc}{\coqdocinductive{eqc}} \coqdocvariable{t} \coqdocvariable{u}.\coqdoceol
\coqdockw{Notation} "t =c u" := (\coqref{Equation C.eqc ctx}{\coqdocdefinition{eqc\_ctx}} \coqdocvar{t} \coqdocvar{u}) (\coqdoctac{at} \coqdockw{level} 66).\coqdoceol
\bigskip

A definição \texttt{ES\_contextual\_closure} é um dos chamados \emph{fechos
    contextuais}. A ideia é que, se vale $t \rightarrow_R t'$, então, para
qualquer termo construído a partir de $t$, vale a redução pelo fecho contextual de
R, para o mesmo termo, mas construído a partir de $t'$. Como exemplo:

\[(t \rightarrow_R t') \Rightarrow ((pterm\_app\ t\ u)
    \rightarrow_{ES\_Contextual\_Closure\ R} (pterm\_app\ t'\ u)) \]

Definimos também os fechos \emph{transitivo} e \emph{transitivo-reflexivo}.
Essencialmente, se temos $t \rightarrow_R t'$ e $t' \rightarrow_R t''$, então
vale o fecho transitivo $t \rightarrow_{R^+} t''$. Isto é representado pelo
construtor \texttt{transitive\_reduction}. Note que isto vale para
apenas um passo também: $t \rightarrow_R t'$ implica $t \rightarrow_{R^+} t'$.
Este é representado pelo construtor \texttt{one\_step\_reduction}.

\begin{definicao}[Fecho transitivo]
    Sejam $t, t', u$ pré-termos e $R$ uma relação entre pré-termos. As regras
    abaixo definem o fecho transitivo de $R$.
    
\begin{mathpar} 
    \inferrule*[Right=one\_step\_reduction]{ (R\ t\ t') }
    {(trans\_closure\ R)\ t\ t'}
    \\
    \inferrule*[Right=transitive\_reduction]{((trans\_closure\ R)\ t\ u),\
        ((trans\_closure\ R)\ u\ t')}
    {((trans\_closure\ R)\ (t\ t'))}
\end{mathpar}
\end{definicao}

O fecho \emph{transitivo-reflexivo} é uma extensão do transitivo, onde vale $t
\rightarrow_{R^*} t$, sempre. A extensão é presentada por dois construtores:
\texttt{reflexive\_reduction}, que garante \texttt{star\_closure R t t}, e
\texttt{star\_trans\_reduction}, que constrói \texttt{star\_closure R t u}, se
vale \texttt{trans\_closure R t u}.

\begin{definicao}[Fecho transitivo-reflexivo]
    Sejam $t, t', u$ pré-termos e $R$ uma relação entre pré-termos. As regras
    abaixo definem o fecho transitivo-reflexivo de $R$.
\begin{mathpar} 
    \inferrule*[Right=reflexive\_reduction]{  }
    {(star\_closure\ R)\ t\ t}
    \\
    \inferrule*[Right=star\_trans\_reduction]{((trans\_closure\ R)\ t\ u)}
    {((star_trans\_closure\ R)\ (t\ u))}
\end{mathpar}
\end{definicao}

\bigskip
\coqnoindent\coqdockw{Definition} \coqdef{Equation C.eqc
    trans}{eqc\_trans}{\coqdocdefinition{eqc\_trans}} (\coqdocvar{t}
\coqdocvar{u}: \coqdocinductive{pterm}) := \coqdocinductive{trans\_closure}
\coqref{Equation C.eqc ctx}{\coqdocdefinition{eqc\_ctx}} \coqdocvariable{t}
\coqdocvariable{u}.\coqdoceol \coqdockw{Notation} "t =c+ u" := (\coqref{Equation
    C.eqc trans}{\coqdocdefinition{eqc\_trans}} \coqdocvar{t} \coqdocvar{u})
(\coqdoctac{at} \coqdockw{level} 66).\coqdoceol

\coqdockw{Definition} \coqdef{Equation C.eqC}{eqC}{\coqdocdefinition{eqC}} (\coqdocvar{t} : \coqdocinductive{pterm}) (\coqdocvar{u} : \coqdocinductive{pterm}) := \coqdocinductive{star\_closure} \coqref{Equation C.eqc ctx}{\coqdocdefinition{eqc\_ctx}} \coqdocvariable{t} \coqdocvariable{u}.\coqdoceol
\coqdockw{Notation} "t =e u" := (\coqref{Equation C.eqC}{\coqdocdefinition{eqC}} \coqdocvar{t} \coqdocvar{u}) (\coqdoctac{at} \coqdockw{level} 66).\coqdoceol
\bigskip

Assim, a relação de equivalência que é interessante ser estudada é a $=_e$, que
permite vários passos de permutações, além de permutações no interior dos
termos. Uma importante diferença de provas usando equivalências no papel e numa
formalização é que resultados que são intuitivos e admitidos num ambiente
informal devem ser provados minuciosamente.

Como exemplo, podemos querer mostrar a compatibilidade da igualdade com a
função \texttt{lc\_at}. Isto é intuitivo, pois permutar duas substituições não
irá criar índices livres. Porém, é preciso entrar em detalhes na prova formal. 

\bigskip
\coqnoindent \coqdockw{Lemma} \coqdef{Equation C.lc at eqc}{lc\_at\_eqc}{\coqdoclemma{lc\_at\_eqc}} : \coqdockw{\ensuremath{\forall}} \coqdocvar{n} \coqdocvar{t} \coqdocvar{u}, \coqref{Equation C.eqc}{\coqdocinductive{eqc}} \coqdocvariable{t} \coqdocvariable{u}  \ensuremath{\rightarrow} (\coqdocdefinition{lc\_at} \coqdocvariable{n} \coqdocvariable{t} \coqexternalref{:type scope:x '<->' x}{http://coq.inria.fr/distrib/8.4pl4/stdlib/Coq.Init.Logic}{\coqdocnotation{\ensuremath{\leftrightarrow}}} \coqdocdefinition{lc\_at} \coqdocvariable{n} \coqdocvariable{u}).\coqdoceol

\coqnoindent \coqdockw{Lemma} \coqdef{Equation C.lc at ES ctx eqc}{lc\_at\_ES\_ctx\_eqc}{\coqdoclemma{lc\_at\_ES\_ctx\_eqc}} : \coqdockw{\ensuremath{\forall}} \coqdocvar{n} \coqdocvar{t} \coqdocvar{u}, (\coqdocinductive{ES\_contextual\_closure} \coqref{Equation C.eqc}{\coqdocinductive{eqc}}) \coqdocvariable{t} \coqdocvariable{u}  \ensuremath{\rightarrow} (\coqdocdefinition{lc\_at} \coqdocvariable{n} \coqdocvariable{t} \coqexternalref{:type scope:x '<->' x}{http://coq.inria.fr/distrib/8.4pl4/stdlib/Coq.Init.Logic}{\coqdocnotation{\ensuremath{\leftrightarrow}}} \coqdocdefinition{lc\_at} \coqdocvariable{n} \coqdocvariable{u}).\coqdoceol

\coqnoindent \coqdockw{Lemma} \coqdef{Equation C.lc at eqC}{lc\_at\_eqC}{\coqdoclemma{lc\_at\_eqC}} : \coqdockw{\ensuremath{\forall}} \coqdocvar{n} \coqdocvar{t} \coqdocvar{t'}, \coqdocvariable{t} \coqref{Equation C.::x '=e' x}{\coqdocnotation{=}}\coqref{Equation C.::x '=e' x}{\coqdocnotation{e}} \coqdocvariable{t'} \ensuremath{\rightarrow} (\coqdocdefinition{lc\_at} \coqdocvariable{n} \coqdocvariable{t} \coqexternalref{:type scope:x '<->' x}{http://coq.inria.fr/distrib/8.4pl4/stdlib/Coq.Init.Logic}{\coqdocnotation{\ensuremath{\leftrightarrow}}} \coqdocdefinition{lc\_at} \coqdocvariable{n} \coqdocvariable{t'}).\coqdoceol

\bigskip

Observe que precisamos provar três resultados. Não basta mostrar que o predicado
\texttt{eqc} é compatível com \texttt{lc\_at}, pois ainda usaremos seus fechos.
A prova de \texttt{lc\_at\_eqc} já apresenta certos detalhes. Ela é feita
através de uma análise de casos do predicado \texttt{eqc}. É preciso realizar
certos ajustes com índices e usar a equivalência entre \texttt{lc\_at} e o
predicado \texttt{term}, além de regras de enfraquecimento para o \texttt{lc\_at}.

Os casos dos fechos em geral saem mais rapidamente, fazendo indução no próprio
fecho. No caso do fecho contextual, os casos de abstração e substituição se
tornam um pouco mais longos, mas são facilmente resolvidos usando a propriedade
\texttt{lc\_at\_open'}, que garante que se um termo está fechado a um nível
\emph{n}, abrí-lo a um nível $k < n$ não gera índices livres.

Esta é uma parte do trabalho especialmente extensa e detalhada. É preciso
mostrar que vários predicados importantes da teoria são preservados pelas
classes de equivâlencia entre termos. Várias vezes durante as provas
trabalharemos com termos equivalentes módulo $=_e$ e, sem uma base bem
construída de resultados sobre a relação de \texttt{eqC}, fica impossível
concluir diversas provas.

Os principais resultados a serem provados nesta parte incluem:

\begin{itemize}
    \item Preservação da estrutura de termos pela equivalência. Construtores de
        termos, abertura de termos, substituições e renomeamento de variáveis
        devem funcionar de maneira análoga para dois termos equivalentes.
    \item Boa formação de termos e corpos de abstrações, conjunto de variáveis
        livres e reduções dentro de fechos contextuais também devem ser
        preservadas pela equivalência.
\end{itemize}

\subsection{Reduções do Sistema}
\label{sub:redu_es_do_sistema}

Com a estrutura de termos e as regras de equivalência bem estabelecidas, podemos
iniciar a formalização das reduções do sistema $\lambda ex$.
As regras de redução listadas em  \ref{table:red_lambex} serão formalizadas nos
tipos indutivos \texttt{sys\_x} e \texttt{rule\_b}.

\bigskip
\coqdocnoindent \coqdockw{Inductive} \coqdocvar{rule\_b} : \coqdocinductive{pterm}
\ensuremath{\rightarrow} \coqdocinductive{pterm} \ensuremath{\rightarrow}
\coqdockw{Prop} := ...\coqdoceol

\coqdocnoindent
\coqdockw{Notation} "t ->\_B u" :=
(\coqref{Lambda Ex.rule b}{\coqdocinductive{rule\_b}} \coqdocvar{t}
\coqdocvar{u}) (\coqdoctac{at} \coqdockw{level} 66).\coqdoceol

\coqdocnoindent
\coqdockw{Inductive} \coqdocvar{sys\_x}
: \coqdocinductive{pterm} \ensuremath{\rightarrow} \coqdocinductive{pterm}
\ensuremath{\rightarrow} \coqdockw{Prop} := ...\coqdoceol

\coqdocnoindent
\coqdockw{Notation} "t ->\_x u" :=
(\coqref{Lambda Ex.sys x}{\coqdocinductive{sys\_x}} \coqdocvar{t} \coqdocvar{u})
(\coqdoctac{at} \coqdockw{level} 59, \coqdoctac{left}
\coqdockw{associativity}).\coqdoceol
\bigskip

A relação $\rightarrow_x$ é definida como sendo exatamente o predicado
\texttt{sys\_x}. Algumas exigências técnicas de \emph{term} e \emph{body} são
adicionadas para auxiliar nas provas, já que queremos sempre trabalhar com
termos bem formados.

O predicado \texttt{rule\_b} é o que formaliza a regra $\rightarrow_B$, sendo
esta a regra que reduz uma aplicação a uma substituição explícita. Podemos, a
partir destes dois predicados, definir a regra de redução principal do sistema
$\lambda ex$.

\bigskip
\coqdocnoindent \coqdockw{Inductive} \coqdocvar{sys\_Bx}: \coqdocinductive{pterm}
\ensuremath{\rightarrow} \coqdocinductive{pterm} \ensuremath{\rightarrow}
\coqdockw{Prop} := \coqdoceol
\coqdocnoindent \ensuremath{|} \coqdocvar{B\_lx} : \coqdockw{\ensuremath{\forall}} \coqdocvar{t}
\coqdocvar{u}, \coqdocvariable{t} \coqref{Lambda Ex.::x '-> B'
    x}{\coqdocnotation{\ensuremath{\rightarrow}}}\coqref{Lambda Ex.::x '-> B'
    x}{\coqdocnotation{\_B}} \coqdocvariable{u} \ensuremath{\rightarrow}
\coqref{Lambda Ex.sys Bx}{\coqdocinductive{sys\_Bx}} \coqdocvariable{t}
\coqdocvariable{u}\coqdoceol \coqdocnoindent \ensuremath{|} \coqdef{sys\_x\_lx}
: \coqdockw{\ensuremath{\forall}} \coqdocvar{t} \coqdocvar{u},
\coqdocvariable{t} \coqref{Lambda Ex.::x '-> x'
    x}{\coqdocnotation{\ensuremath{\rightarrow}}}\coqref{Lambda Ex.::x '-> x'
    x}{\coqdocnotation{\_x}} \coqdocvariable{u} \ensuremath{\rightarrow}
\coqref{Lambda Ex.sys Bx}{\coqdocinductive{sys\_Bx}} \coqdocvariable{t}
\coqdocvariable{u}.\coqdoceol

\smallskip
\coqdocnoindent \coqdockw{Notation} "t ->\_Bx u" := (\coqref{Lambda Ex.sys Bx}{\coqdocinductive{sys\_Bx}} \coqdocvar{t} \coqdocvar{u}) (\coqdoctac{at} \coqdockw{level} 59, \coqdoctac{left} \coqdockw{associativity}).\coqdoceol

\smallskip
\coqdocnoindent \coqdockw{Definition} \coqdef{Equation C.red ctx mod eqC}{red\_ctx\_mod\_eqC}{\coqdocdefinition{red\_ctx\_mod\_eqC}} (\coqdocvar{R}: \coqdocinductive{pterm} \ensuremath{\rightarrow} \coqdocinductive{pterm} \ensuremath{\rightarrow} \coqdockw{Prop}) (\coqdocvar{t}: \coqdocinductive{pterm}) (\coqdocvar{u} : \coqdocinductive{pterm}) := \coqdoceol

\coqdocindent{5.50em}
\coqexternalref{:type scope:'exists' x '..' x ','
    x}{http://coq.inria.fr/distrib/8.4pl4/stdlib/Coq.Init.Logic}{\coqdocnotation{\ensuremath{\exists}}}
\coqdocvar{t'}\coqexternalref{:type scope:'exists' x '..' x ','
    x}{http://coq.inria.fr/distrib/8.4pl4/stdlib/Coq.Init.Logic}{\coqdocnotation{,}}
\coqexternalref{:type scope:'exists' x '..' x ','
    x}{http://coq.inria.fr/distrib/8.4pl4/stdlib/Coq.Init.Logic}{\coqdocnotation{\ensuremath{\exists}}}
\coqdocvar{u'}\coqexternalref{:type scope:'exists' x '..' x ','
    x}{http://coq.inria.fr/distrib/8.4pl4/stdlib/Coq.Init.Logic}{\coqdocnotation{,}}
\coqexternalref{:type scope:x '/x5C'
    x}{http://coq.inria.fr/distrib/8.4pl4/stdlib/Coq.Init.Logic}{\coqdocnotation{(}}\coqdocvariable{t}
\coqref{Equation C.::x '=e' x}{\coqdocnotation{=}}\coqref{Equation C.::x '=e'
    x}{\coqdocnotation{e}} \coqdocvariable{t'}\coqexternalref{:type scope:x
    '/x5C'
    x}{http://coq.inria.fr/distrib/8.4pl4/stdlib/Coq.Init.Logic}{\coqdocnotation{)/\symbol{92}(}}\coqdocinductive{ES\_contextual\_closure}
\coqdocvariable{R} \coqdocvariable{t'} \coqdocvariable{u'}\coqexternalref{:type
    scope:x '/x5C'
    x}{http://coq.inria.fr/distrib/8.4pl4/stdlib/Coq.Init.Logic}{\coqdocnotation{)/\symbol{92}(}}\coqdocvariable{u'}
\coqref{Equation C.::x '=e' x}{\coqdocnotation{=}}\coqref{Equation C.::x '=e'
    x}{\coqdocnotation{e}} \coqdocvariable{u}\coqexternalref{:type scope:x
    '/x5C'
    x}{http://coq.inria.fr/distrib/8.4pl4/stdlib/Coq.Init.Logic}{\coqdocnotation{)}}.\coqdoceol

\medskip
\coqdocnoindent \coqdockw{Definition} \coqdef{Lambda Ex.lex}{lex}{\coqdocdefinition{lex}} \coqdocvar{t} \coqdocvar{u} :=  \coqdocdefinition{red\_ctx\_mod\_eqC} \coqref{Lambda Ex.sys Bx}{\coqdocinductive{sys\_Bx}} \coqdocvariable{t} \coqdocvariable{u}.\coqdoceol

\coqdocnoindent \coqdockw{Notation} "t -->lex u" := (\coqref{Lambda Ex.lex}{\coqdocdefinition{lex}} \coqdocvar{t} \coqdocvar{u}) (\coqdoctac{at} \coqdockw{level} 66).\coqdoceol

\smallskip

\coqdocnoindent
\coqdockw{Definition} \coqdef{Lambda Ex.lex trs}{lex\_trs}{\coqdocdefinition{lex\_trs}} \coqdocvar{t} \coqdocvar{u} := \coqdocinductive{trans\_closure} \coqref{Lambda Ex.lex}{\coqdocdefinition{lex}} \coqdocvariable{t} \coqdocvariable{u}.\coqdoceol
\coqdocnoindent
\coqdockw{Notation} "t -->lex+ u" := (\coqref{Lambda Ex.lex trs}{\coqdocdefinition{lex\_trs}} \coqdocvar{t} \coqdocvar{u}) (\coqdoctac{at} \coqdockw{level} 66).\coqdoceol
\coqdocemptyline
\coqdocnoindent
\coqdockw{Definition} \coqdef{Lambda Ex.lex str}{lex\_str}{\coqdocdefinition{lex\_str}} \coqdocvar{t} \coqdocvar{u} := \coqdocinductive{star\_closure} \coqref{Lambda Ex.lex}{\coqdocdefinition{lex}} \coqdocvariable{t} \coqdocvariable{u}.\coqdoceol
\coqdocnoindent
\coqdockw{Notation} "t -->lex* u" := (\coqref{Lambda Ex.lex str}{\coqdocdefinition{lex\_str}} \coqdocvar{t} \coqdocvar{u}) (\coqdoctac{at} \coqdockw{level} 66).\coqdoceol

\bigskip

O análogo da regra $\rightarrow_{\lambda ex}$ no sistema será a relação
"\texttt{-->lex*}", construída a partir do fecho transitivo-reflexivo,
contextual e equacional da relação $\rightarrow_{Bx}$.

Para auxiliar nas provas, definimos alguns predicados que garantem que estamos
trabalhando sempre com termos, e que representam algumas noções intuitivas das
reduções.

\begin{definicao}[Regularidade]
    Sejam $t, u$ pré-termos e $R$ uma relação entre pré-termos.
    Dizemos que $R$ é regular se, sempre que vale $t\ R\ u$, então $t$ e $u$ são
    termos bem formados. A noção de regularidade é formalizada pelo predicado
    $red\_regular$, definido como base na regra de inferência abaixo.
    
\begin{mathpar} 
    \inferrule*{(red\_regular\ R),\ (R\ t\ u)}
    {term\ t\ \and term\ u}
\end{mathpar}
\end{definicao}


A propriedade de regularidade é interessante para facilitar diversas provas,
já que desejamos evitar trabalhar com termos que possuem índices livres.
Basicamente, se fizemos a redução $R$ entre dois pré-termos,
ambos são termos bem formados.

Algumas relações interessantes não são capazes de garantir a regularidade. Ainda
assim, queremos continuar lidando com termos bem formados. Definimos então uma
noção mais fraca de regularidade, que apenas preserva a boa formação de termos.

\begin{definicao}[Regularidade fraca]
    Sejam $t, u$ pré-termos e $R$ uma relação entre pré-termos.
    Dizemos que $R$ satisfaz a regularidade fraca se, sempre que vale $t\ R\ u$,
    vale $(term\ t) \iff (term\ u)$. A noção de regularidade fraca é formalizada
    pelo predicado $red\_regular'$, definido como base na regra de inferência
    abaixo.
    
\begin{mathpar} 
    \inferrule*{(red\_regular'\ R),\ (R\ t\ u)}
    {term\ t\ \iff term\ u}
\end{mathpar}
\end{definicao}

%\begin{table}[h]
%\begin{mathpar} 
    %\inferrule*{(red\_out\ R),\ (R\ t\ u)}
    %{R\ ([x \rightarrow u]t)\ ([x \rightarrow u]t')}
%\end{mathpar}
    %\caption{Redução fora da meta-substituição}
%\end{table}


%A propriedade $red\_out$ garante que a redução pode ser feita em um
%termo afetado por uma meta-substituição, sendo $[x\ \mapsto\ u]$ a substituição que
%troca todas as ocorrências de $x$ pelo termo $u$. Ela é útil pois
%esta meta-substituição pode ser vista como uma generalização da abertura de um
%termo.

\begin{definicao}[Renomeamento]
    Sejam $t, u$ pré-termos; $x,y$ variáveis e $R$ uma relação entre pré-termos.
    Dizemos que a relação $R$ é compatível com renomeamento de variáveis, ou
    seja, vale $red\_rename\ R$, se $R$ satisfaz a seguinte regra de inferência:    

\begin{mathpar} 
    \inferrule*{(red\_rename\ R),\ (\forall x,\ x \notin fv(t) \rightarrow R\
        (t^x) \ (u^x))}
    {R\ (t^y)\ (u^y)}
\end{mathpar}
\end{definicao}

A propriedade \emph{red\_rename} pode ser comparada à $\alpha$-equivalência de termos no cálculo
original. Basicamente, se um termo $t$ aberto com uma variável $x \notin
fv(t)$ se reduz a um $t'$, também aberto com $x$, então este
$x$ pode ser trocado por outra variável $y \notin fv(t)$, preservando a
redução $\rightarrow_{lex}$.

Estes resultados auxiliam muito em diversas provas. Em especial, os resultados
envolvendo abertura de termos e índices livres são necessários para provas em
que fazemos indução na estrutura do termo, pois, em geral, no caso da abstração
as hipóteses se referem ao sub-termo aberto com uma variável.



%------------------------------------------------------------------



\section{Preservação da normalização forte}
\label{sec:psn}

Nesta seção, queremos dar uma visão geral da propriedade PSN e descrever a
formalização da propriedade IE, foco deste trabalho. 

A propriedade PSN é a que garante que, se um termo $t$ é fortemente
normalizável no cálculo original, ou seja, toda cadeia de reduções a partir dele
termina, então ele também é fortemente normalizável no sistema $\lambda ex$.

A prova da PSN é feita primeiramente definindo uma estratégia de redução
\emph{perpétua} para o sistema, definida da seguinte maneira: se um termo
$t$ se reduz a um termo $t'$ por esta estratégia, e se $t'$ é
fortemente normalizável, então $t$ também o é. Pela contrapositiva, vemos
que se o termo $t$ não for fortemente normalizável, então a estratégia terá
que fazer a redução para um termo $t'$ que também não é fortemente
normalizável, e assim por diante, formando uma cadeia de redução infinita.

Em particular, se um termo $t[x/u]$ é reduzido, por esta estratégia, para um
termo $t\{x/u\}$, sendo este fortemente normalizável, então o termo original
também será fortemente normalizável. Em outras palavras, a normalização da
substituição \emph{implícita} implica na normalização da substituição
\emph{explícita}. Esta é a chamada propriedade \textbf{IE}.

Para a prova da propriedade IE, é adicionado mais uma estrutura no sistema,
chamada de \emph{substituição marcada}.

\subsection{Substituições marcadas}
\label{sub:subst_marc}

A ideia é controlar as reduções feitas envolvendo substituições explícitas. Para
isto, adicionamos uma nova substituição na gramática.

\[ \tau := x\ |\ \lambda x.\tau\ |\ \tau \tau\ |\ \tau[x/\tau]\ |\
    \tau[[x/u]]\ \]

Note que na nova substituição, não podemos colocar qualquer qualquer termo no
lugar de $u$. Restringimos o termo $u$ a apenas termos da gramática
original, ou seja, sem substituições marcadas. Além disso, é necessário que o
termo seja fortementa normalizável.

Na nossa gramática de pré-termos, este formalismo já está adicionado, sendo
representado pelo construtor \texttt{pterm\_lsub}. Porém, assim como precisamos
de um predicado para verificar se um termo comum está bem formado, vamos criar
um outro predicado, chamado \texttt{lab\_term}. 


\begin{definicao}[Termo marcado]
    Este predicado é uma extensão do predicado $term$, visto na Definição
    \ref{def_term}. Adicionamos apenas um caso para lidar com as substituições
    marcadas, sendo todos os outros casos análogos ao anterior.
        
\begin{mathpar} 
    \inferrule*[Right=lab\_term\_sub']{\forall x \notin L,\ lab\_term(t^{x}) \\
        term(u) \\ (SN\ lex)\ u}
    {lc(t[[u]])}
\end{mathpar}
\end{definicao}

Observe que no caso da substituição marcada, temos a exigência $(SN\ lex\ u)$,
que indica que $u$ é fortemente normalizável.

Também devemos estender a noção equivalente de ser localmente fechado, para
manter a equivalência definida no caso do sistema sem as substituições marcadas.
Assim, definimos uma nova função \texttt{lc\_at'}, análoga à do caso não
marcado.  Sua implementação é também feita com base na Definição
\ref{def_lc_at}, com apenas uma extensão para os casos das substituição marcadas
e explícitas.

\begin{definicao}
    A função $lc\_at'$ é definida como uma extensão de $lc\_at$, adicionando os
    seguintes dois casos:

\begin{empheq}{align*}
    lc\_at'\ k\ (t[u])\ \ \ &\equiv\ (lc\_at'\ (S k)\ t)\ \&\ (lc\_at'\ k\ u) \\ 
    lc\_at'\ k\ (t[[u]])\ \ \ &\equiv\ (lc\_at'\ (S k)\ t)\ \&\ (lc\_at'\ k\ u)\
    \&\ (SN\ lex\ u) \\ 
\end{empheq}
\end{definicao}

\bigskip

\coqdocnoindent \coqdockw{Fixpoint} \coqdocvar{lc\_at'}
(\coqdocvar{k}:\coqdocvar{nat}) (\coqdocvar{t}:\coqdocvar{pterm})
\{\coqdockw{struct} \coqdocvar{t}\} : \coqdockw{Prop} := ...\coqdoceol

\coqdocnoindent\coqdockw{Definition} \coqdocvar{term'{}'} \coqdocvar{t} := \coqdocvar{lc\_at'} 0 \coqdocvar{t}.\coqdoceol
\bigskip

Podemos seguir a prova da equivalência entre \texttt{lab\_term} e
\texttt{lc\_at'} de maneira similar a prova do capítulo anterior, com a ajuda
dos seguintes lemas auxiliares.

\bigskip \coqdockw{Lemma} \coqdocvar{lc\_rec\_open\_var\_rec'} :
\coqdockw{\ensuremath{\forall}} \coqdocvar{x} \coqdocvar{t}
\coqdocvar{k},\coqdoceol \coqdocindent{2.00em} \coqdocvar{lc\_at'} \coqdocvar{k}
(\coqdocvar{open\_rec} \coqdocvar{k} \coqdocvar{x} \coqdocvar{t})
\ensuremath{\rightarrow} \coqdocvar{lc\_at'} (\coqdocvar{S} \coqdocvar{k})
\coqdocvar{t}.\coqdoceol

\smallskip
\coqdockw{Lemma} \coqdocvar{lc\_at\_open\_var\_rec'} :
\coqdockw{\ensuremath{\forall}} \coqdocvar{x} \coqdocvar{t}
\coqdocvar{k},\coqdoceol \coqdocindent{2.00em} \coqdocvar{lc\_at'}
(\coqdocvar{S} \coqdocvar{k}) \coqdocvar{t} \ensuremath{\rightarrow}
\coqdocvar{lc\_at'} \coqdocvar{k} (\coqdocvar{open\_rec} \coqdocvar{k}
(\coqdocvar{pterm\_fvar} \coqdocvar{x}) \coqdocvar{t}).\coqdoceol

\smallskip
\coqdockw{Lemma} \coqdocvar{term\_impl\_lab\_term}:
\coqdockw{\ensuremath{\forall}} \coqdocvar{t}, \coqdocvar{term} \coqdocvar{t}
\ensuremath{\rightarrow} \coqdocvar{lab\_term} \coqdocvar{t}.\coqdoceol

\smallskip
\coqdockw{Lemma} \coqdocvar{lab\_term\_eq\_term'{}'} :
\coqdockw{\ensuremath{\forall}} \coqdocvar{t}, \coqdocvar{lab\_term}
\coqdocvar{t} \ensuremath{\leftrightarrow} \coqdocvar{term'{}'}
\coqdocvar{t}.\coqdoceol 
\bigskip

Como no caso do sistema simples, queremos definir classes de equivalências
de termos, para trabalhar módulo permutação de substituições.
Para isso, precisamos definir novos fechos contextuais para os termos com
substituições marcadas.

\begin{definicao}[Fecho contextual para termos marcados]
    O fecho contextual para termos marcados é definido como uma extensão do
    fecho para termos comuns, visto na definição \ref{def_fecho}, com a ressalva
    que os predicados de $term$ e $body$ são substituídos por seus análogos,
    $lab\_term$ e $lab\_body$, respectivamente. Os casos que lidam com
    substituições marcadas são definidos pelas regras de inferência a seguir.
    
\begin{mathpar} 
    \inferrule*[Right=lab\_subst'\_left]{(\forall\ x,\ x \notin L \rightarrow
        ((lab\_contextual\_closure\ R)\ t^x\ t'^x)),\ (term\ u),\ (SN\ lex\ u)}
    {((lab\_contextual\_closure\ R)\ (t[[u]])\ (t'[[u]]))}
    \\
    \inferrule*[Right=lab\_subst'\_right]{(R\ u\ u'),\
        (lab\_body\ t)}
    {((lab\_contextual\_closure\ R)\ (t[[u]])\ (t[[u']]))}
\end{mathpar}
\end{definicao}

\bigskip

São definidos também dois outros fechos análogos,
\texttt{simpl\_lab\_contextual\_closure} e
\texttt{ext\_lab\_contextual\_closure}. No primeiro, a diferença é que não é
feita a exigência $(SN\ lex\ u)$ no caso da redução à esquerda da
substituição marcada. No segundo, queremos reduzir apenas fora de substituições
marcadas. Assim, o fecho não possui um análogo ao caso
\texttt{LAB\_SUBST'\_RIGHT}.

\smallskip

Podemos então definir a relação equacional para termos marcados.

\coqdocnoindent \coqdockw{Inductive} \coqdocvar{lab\_eqc}  : \coqdocvar{pterm}
\ensuremath{\rightarrow} \coqdocvar{pterm} \ensuremath{\rightarrow}
\coqdockw{Prop} := \coqdoceol \coqdocnoindent 
\coqdocindent{1.00em}\ensuremath{|} \coqdocvar{lab\_eqc\_rx1} : \coqdockw{\ensuremath{\forall}}
\coqdocvar{t} \coqdocvar{u} \coqdocvar{v}, \coqdoceol \coqdocindent{2.00em}
\coqdocvar{lab\_term} \coqdocvar{u} \ensuremath{\rightarrow} \coqdocvar{term}
\coqdocvar{v} \ensuremath{\rightarrow} \coqdocvar{lab\_eqc}
(\coqdocvar{t}[\coqdocvar{u}][[\coqdocvar{v}]]) ((\&
\coqdocvar{t})[[\coqdocvar{v}]][\coqdocvar{u}]) \coqdoceol \coqdocnoindent
\coqdocindent{1.00em}\ensuremath{|} \coqdocvar{lab\_eqc\_rx2} : \coqdockw{\ensuremath{\forall}}
\coqdocvar{t} \coqdocvar{u} \coqdocvar{v}, \coqdoceol \coqdocindent{2.00em}
\coqdocvar{term} \coqdocvar{u} \ensuremath{\rightarrow} \coqdocvar{lab\_term}
\coqdocvar{v} \ensuremath{\rightarrow} \coqdocvar{lab\_eqc}
(\coqdocvar{t}[[\coqdocvar{u}]][\coqdocvar{v}]) ((\&
\coqdocvar{t})[\coqdocvar{v}][[\coqdocvar{u}]]) \coqdoceol \coqdocnoindent
\coqdocindent{1.00em}\ensuremath{|} \coqdocvar{lab\_eqc\_rx3} : \coqdockw{\ensuremath{\forall}}
\coqdocvar{t} \coqdocvar{u} \coqdocvar{v}, \coqdoceol \coqdocindent{2.00em}
\coqdocvar{term} \coqdocvar{u} \ensuremath{\rightarrow} \coqdocvar{term}
\coqdocvar{v} \ensuremath{\rightarrow} \coqdocvar{lab\_eqc}
(\coqdocvar{t}[[\coqdocvar{u}]][[\coqdocvar{v}]]) ((\&
\coqdocvar{t})[[\coqdocvar{v}]][[\coqdocvar{u}]]).\coqdoceol

\bigskip

Os construtores basicamente definem como permutar duas substituições, desde que
uma delas seja marcada. Como requisito, é necessário garantir a propriedade
\texttt{lab\_term} ou \texttt{term}, dependendo da substituição. A ideia é
permitir que as substituições marcadas sejam permutadas ``para dentro" do termo,
passando por uma substuição comum ou marcada. A substituição normal não pode
atravessar a marcada.

Para uso futuro, definimos um lema de simetria da relação. Sua prova é feita por
análise de casos simples, apenas usando o lema \texttt{bswap\_idemp} para
reduzir permutações de índices que não alteram a semântica do termo.

\bigskip
\coqdockw{Lemma} \coqdocvar{lab\_eqc\_sym} : \coqdockw{\ensuremath{\forall}}
\coqdocvar{t} \coqdocvar{u}, \coqdocvar{lab\_eqc} \coqdocvar{t} \coqdocvar{u}
\ensuremath{\rightarrow} \coqdocvar{lab\_eqc} \coqdocvar{u}
\coqdocvar{t}.\coqdoceol
\bigskip

A equação principal utilizada nos termos marcados será a relação 
$=_{\underline{e}}$, formalizada como o fecho contextual e transitivo do
predicado \texttt{lab\_eqc}.

\bigskip
\coqdocnoindent
\coqdockw{Definition} \coqdocvar{lab\_eqC} (\coqdocvar{t}: \coqdocvar{pterm})
(\coqdocvar{u} : \coqdocvar{pterm}) :=  \coqdocvar{trans\_closure}
(\coqdocvar{lab\_contextual\_closure} \coqdocvar{lab\_eqc}) \coqdocvar{t}
\coqdocvar{u} .\coqdoceol \coqdocnoindent \coqdockw{Notation} "t =\~{}e u" :=
(\coqdocvar{lab\_eqC} \coqdocvar{t} \coqdocvar{u}) (\coqdoctac{at}
\coqdockw{level} 66).\coqdoceol
\bigskip


Também é preciso definir um predicado análogo de regularidade para termos
marcados.

\begin{definicao}[Regularidade para termos marcados]
    Sejam $t, u$ pré-termos e $R$ uma relação entre pré-termos.
    Dizemos que $R$ é regular se, sempre que vale $t\ R\ u$, então $t$ e $u$ são
    termos marcados bem formados. A noção de regularidade para termos marcados é
    formalizada pelo predicado $red\_lab\_regular$, definido como base na regra
    de inferência abaixo.

\begin{mathpar} 
    \inferrule*{(red\_lab\_regular\ R),\ (R\ t\ u)}
    {lab\_term\ t\ \and lab\_term\ u}
\end{mathpar}
\end{definicao}

\begin{definicao}[Regularidade fraca para termos marcados]
    Sejam $t, u$ pré-termos e $R$ uma relação entre pré-termos.
    Dizemos que $R$ satisfaz a regularidade fraca se, sempre que vale $t\ R\ u$,
    vale $(lab\_term\ t) \iff (lab\_term\ u)$. A noção de regularidade fraca é
    formalizada pelo predicado $red\_lab\_regular'$, definido como base na regra de
    inferência abaixo.
    
\begin{mathpar} 
    \inferrule*{(red\_lab\_regular'\ R),\ (R\ t\ u)}
    {lab\_term\ t\ \iff lab\_term\ u}
\end{mathpar}
\end{definicao}


Com todas as estruturas e propriedades para termos marcados bem definidos,
podemos seguir com a prova da propriedade IE.


\subsection{Equivalência de reduções com o sistema original}
\label{sub:equiv_red}

Queremos utilizar esse sistema extendido com as substituições marcadas para
estudar o sistema original. Para isto, precisamos estender a regra de redução do
sistema para lidar com as novas substituições. Definimos então a redução
$\rightarrow_{\underline{x}}$.

\begin{table}[h]
\begin{empheq}[box=\fbox]{align*}
    x[\![x/u]\!]\ \ \             &\rightarrow_{Var}\ u \\
    t[\![x/u]\!]\ \ \             &\rightarrow_{Gc}\ t    & se\ \emph{x} \notin fv(t)\\
    (t\ u)[\![x/v]\!]\ \ \        &\rightarrow_{App}\ t[\![x/v]\!]\ u[\![x/v]\!] \\
    (\lambda y.\ u)[\![x/v]\!]\ \ &\rightarrow_{Lamb}\ (\lambda y.\ u[\![x/v]\!])\\
    t[\![x/u]\!][\![y/v]\!]\ \ \        &\rightarrow_{Comp}\ t[\![y/v]\!][\![x/u[\![y/v]\!]]\!] & se\ y\ \in
    fv(u)  
\end{empheq}
    \caption{A redução $\rightarrow$$_{\underline{x}}$ }
    \label{table:red_label_x}

\end{table}

Assim, a relação $\rightarrow_{\lambda \underline{ex}}$ é definida como a união
das reduções $\rightarrow_{Bx}$ e $\rightarrow_{\underline{x}}$, módulo
$=_{\alpha}$, $=_e$ e $=_{\underline{e}}$:

\[ t \rightarrow_{\lambda \underline{ex}} t' \iff \exists s,\ s';\ t =_{e \cup
        \underline{e} \cup \alpha} s \rightarrow_{Bx \cup \underline{x}} s' =_{e \cup
        \underline{e} \cup \alpha} t' \] 

Para provar a propriedade PSN, será necessário relacionar a redução
$\rightarrow_{\lambda \underline{ex}}$ com a redução original,
$\rightarrow_{\lambda ex}$. Para isto, iremos decompor a redução em termos
marcados em duas novas reduções, $\rightarrow_{\lambda \underline{ex}^i}$ e
$\rightarrow_{\lambda \underline{ex}^e}$, que também agem em termos marcados.

A relação $\lambda \underline{ex}^i$, chamada de \emph{redução interna}, é
definida adicionando à redução $\rightarrow_{\underline{ex}}$ a redução
$\rightarrow_{\lambda ex}$ no corpo das substituições marcadas.
Formalmente, a relação $\rightarrow_{\lambda \underline{ex}^i}$ é definida como a
seguinte redução, $\rightarrow_{\lambda \underline{x}^i}$, módulo $=_{\alpha}$,
$=_e$ e $=_{\underline{e}}$:

\begin{itemize}
    \item Se $u \rightarrow_{Bx} u'$ e $u,\ u'$ são termos, então $t[\![x/u]\!]
        \rightarrow_{\lambda \underline{x}^i} t[\![x/u']\!]$ 
    \item Se $t
        \rightarrow_{\underline{x}} t'$, então $t \rightarrow_{\lambda
            \underline{x}^i} t'$
    \item Se $t \rightarrow_{\lambda \underline{x}^i} t'$, então vale 
        $t\ u \rightarrow_{\lambda \underline{x}^i} t'\ u$,  
        $u\ t \rightarrow_{\lambda \underline{x}^i} u\ t'$, 
        $\lambda x. t \rightarrow_{\lambda \underline{x}^i} \lambda x. t'$, 
        $t[x/u] \rightarrow_{\lambda \underline{x}^i} t'[x/u]$, 
        $u[x/t] \rightarrow_{\lambda \underline{x}^i} u[x/t']$ e 
        $t[\![x/u]\!] \rightarrow_{\lambda \underline{x}^i} t'[\![x/u]\!]$.
\end{itemize}


A relação $\lambda \underline{ex}^e$, chamada de \emph{redução externa}, é
definida como a redução $\lambda ex$ em todos os lugares de um
termo, \emph{exceto} no corpo das substituições marcadas.
Formalmente, a relação $\rightarrow_{\lambda \underline{ex}^e}$ é definida como a
seguinte redução, $\rightarrow_{\lambda \underline{x}^e}$, módulo $=_{\alpha}$,
$=_e$ e $=_{\underline{e}}$:

\begin{itemize}
    \item Se $t \rightarrow_{Bx} t'$ ocorre fora de uma substituição marcada, então 
        $t \rightarrow_{\lambda \underline{x}^e} t$ 
    \item Se $t \rightarrow_{\lambda \underline{x}^e} t'$, então vale 
        $t\ u \rightarrow_{\lambda \underline{x}^e} t'\ u$,  
        $u\ t \rightarrow_{\lambda \underline{x}^e} u\ t'$, 
        $\lambda x. t \rightarrow_{\lambda \underline{x}^e} \lambda x. t'$, 
        $t[x/u] \rightarrow_{\lambda \underline{x}^e} t'[x/u]$, 
        $u[x/t] \rightarrow_{\lambda \underline{x}^e} u[x/t']$ e 
        $t[\![x/u]\!] \rightarrow_{\lambda \underline{x}^e} t'[\![x/u]\!]$.
\end{itemize}

O objetivo principal deste trabalho será a formalização destas duas reduções e a
prova de equivalências da união destas com a redução ``\texttt{-{}->[lex]}'', que é a
redução $\lambda \underline{ex}$ formalizada.

Para a formalização das reduções $\lambda \underline{ex}^i$ e $\lambda
\underline{ex}^e$, definimos os seguintes fechos e relações:


\bigskip

\coqdocnoindent\coqdockw{Inductive} \coqdocvar{lab\_x\_i}: \coqdocvar{pterm}
\ensuremath{\rightarrow} \coqdocvar{pterm} \ensuremath{\rightarrow}
\coqdockw{Prop} :=\coqdoceol \coqdocnoindent \ensuremath{|}
\coqdocvar{xi\_from\_bx\_in\_les}: \coqdockw{\ensuremath{\forall}}
\coqdocvar{t1} \coqdocvar{t2} \coqdocvar{t2'}, \coqdoceol \coqdocindent{11.50em}
\coqdocvar{lab\_term} (\coqdocvar{t1} [[ \coqdocvar{t2} ]])
\ensuremath{\rightarrow}\coqdoceol \coqdocindent{11.50em} (\coqdocvar{sys\_Bx}
\coqdocvar{t2} \coqdocvar{t2'}) \ensuremath{\rightarrow}\coqdoceol
\coqdocindent{11.50em} \coqdocvar{lab\_x\_i} (\coqdocvar{t1} [[ \coqdocvar{t2}
]]) (\coqdocvar{t1} [[ \coqdocvar{t2'} ]])\coqdoceol \coqdocnoindent
\ensuremath{|} \coqdocvar{xi\_from\_x} : \coqdockw{\ensuremath{\forall}}
\coqdocvar{t} \coqdocvar{t'}, \coqdoceol \coqdocindent{8.00em}
\coqdocvar{lab\_term} \coqdocvar{t} \ensuremath{\rightarrow}\coqdoceol
\coqdocindent{8.00em} \coqdocvar{lab\_sys\_x} \coqdocvar{t} \coqdocvar{t'}
\ensuremath{\rightarrow} \coqdoceol \coqdocindent{8.00em} \coqdocvar{lab\_x\_i}
\coqdocvar{t} \coqdocvar{t'}.\coqdoceol \coqdocemptyline \coqdocnoindent
\coqdockw{Definition} \coqdocvar{lab\_EE\_ctx\_red} (\coqdocvar{R}:
\coqdocvar{pterm} \ensuremath{\rightarrow} \coqdocvar{pterm}
\ensuremath{\rightarrow} \coqdockw{Prop}) (\coqdocvar{t}: \coqdocvar{pterm})
(\coqdocvar{u} : \coqdocvar{pterm}) := \coqdoceol \coqdocindent{4.00em}
\coqdoctac{\ensuremath{\exists}} \coqdocvar{t'} \coqdocvar{u'}, (\coqdocvar{t}
=\coqdocvar{EE} \coqdocvar{t'})/\symbol{92}(\coqdocvar{lab\_contextual\_closure}
\coqdocvar{R} \coqdocvar{t'} \coqdocvar{u'})/\symbol{92}(\coqdocvar{u'}
=\coqdocvar{EE} \coqdocvar{u}).\coqdoceol \coqdocemptyline \coqdocnoindent
\coqdockw{Definition} \coqdocvar{ext\_lab\_EE\_ctx\_red} (\coqdocvar{R}:
\coqdocvar{pterm} \ensuremath{\rightarrow} \coqdocvar{pterm}
\ensuremath{\rightarrow} \coqdockw{Prop}) (\coqdocvar{t}: \coqdocvar{pterm})
(\coqdocvar{u} : \coqdocvar{pterm}) := \coqdoceol \coqdocindent{2.00em}
\coqdoctac{\ensuremath{\exists}} \coqdocvar{t'} \coqdocvar{u'}, (\coqdocvar{t}
=\coqdocvar{EE}
\coqdocvar{t'})/\symbol{92}(\coqdocvar{ext\_lab\_contextual\_closure}
\coqdocvar{R} \coqdocvar{t'} \coqdocvar{u'})/\symbol{92}(\coqdocvar{u'}
=\coqdocvar{EE} \coqdocvar{u}).\coqdoceol \coqdocemptyline \coqdocnoindent
\coqdockw{Definition} \coqdocvar{lab\_x\_i\_eq} :=
\coqdocvar{ext\_lab\_EE\_ctx\_red} \coqdocvar{lab\_x\_i}.\coqdoceol
\coqdocemptyline \coqdocnoindent \coqdockw{Definition} \coqdocvar{lab\_x\_e\_eq}
:= \coqdocvar{ext\_lab\_EE\_ctx\_red} \coqdocvar{sys\_Bx}.\coqdoceol
\coqdocemptyline \coqdocnoindent \coqdockw{Notation} "t -->[lx\_i] u" :=
(\coqdocvar{lab\_x\_i\_eq} \coqdocvar{t} \coqdocvar{u}) (\coqdoctac{at}
\coqdockw{level} 59, \coqdoctac{left} \coqdockw{associativity}).\coqdoceol
\coqdocnoindent \coqdockw{Notation} "t -->[lx\_e] u" :=
(\coqdocvar{lab\_x\_e\_eq} \coqdocvar{t} \coqdocvar{u}) (\coqdoctac{at}
\coqdockw{level} 59, \coqdoctac{left} \coqdockw{associativity}).\coqdoceol

\bigskip

A relação \texttt{lab\_x\_i} formaliza a relação $\rightarrow_{\underline{x}}$.
O fecho \texttt{ext\_lab\_EE\_ctx\_red} realiza uma relação em qualquer ponto de
um termo, \textbf{exceto} dentro de uma substituição marcada. Ele permite
permutação de substituições tanto antes quanto depois da redução ser feita. Este
fecho é utilizado então para formalizar as reduções de interesse. Assim,
$\lambda \underline{ex}^i$ é formalizada como a redução "t -{}->[lx\_i] u",
definida por \texttt{(ext\_lab\_EE\_ctx\_red lab\_x\_i) t u}. Por outro lado, a
relação $\lambda \underline{ex}^e$ formalizada como a redução "t -{}->[lx\_e] u",
definida por \texttt{(ext\_lab\_EE\_ctx\_red lab\_x\_e) t u}. 

Uma estratégia utilizada para facilitar as provas foi abstrair, em lemas
auxiliares, vários problemas destas que se repetem. Podemos então tratar tais
problemas com um contexto limpo, facilitando as provas por indução. 

\bigskip
\coqdockw{Lemma} \coqdocvar{star\_lab\_closure\_app\_left}:
\coqdockw{\ensuremath{\forall}} \coqdocvar{R} \coqdocvar{t} \coqdocvar{t'}
\coqdocvar{u}, \coqdocvar{lab\_term} \coqdocvar{u} \ensuremath{\rightarrow} \\
\coqdocvar{star\_closure} (\coqdocvar{simpl\_lab\_contextual\_closure}
\coqdocvar{R}) \coqdocvar{t} \coqdocvar{t'} \ensuremath{\rightarrow}
\coqdocvar{star\_closure} (\coqdocvar{simpl\_lab\_contextual\_closure}
\coqdocvar{R}) (\coqdocvar{pterm\_app} \coqdocvar{t} \coqdocvar{u})
(\coqdocvar{pterm\_app} \coqdocvar{t'} \coqdocvar{u}).\coqdoceol

\smallskip

\coqdockw{Lemma} \coqdocvar{EE\_clos\_app\_left}:
\coqdockw{\ensuremath{\forall}} \coqdocvar{R} \coqdocvar{t} \coqdocvar{t'}
\coqdocvar{u}, \coqdocvar{lab\_term} \coqdocvar{u} \ensuremath{\rightarrow}
((\coqdocvar{lab\_EE\_ctx\_red} \coqdocvar{R}) \coqdocvar{t} \coqdocvar{t'})
\ensuremath{\rightarrow} ((\coqdocvar{lab\_EE\_ctx\_red} \coqdocvar{R})
(\coqdocvar{pterm\_app} \coqdocvar{t} \coqdocvar{u}) (\coqdocvar{pterm\_app}
\coqdocvar{t'} \coqdocvar{u})).\coqdoceol

\smallskip

\coqdockw{Lemma} \coqdocvar{EE\_ext\_clos\_app\_left}:
\coqdockw{\ensuremath{\forall}} \coqdocvar{R} \coqdocvar{t} \coqdocvar{t'}
\coqdocvar{u}, \coqdocvar{lab\_term} \coqdocvar{u} \ensuremath{\rightarrow}
((\coqdocvar{ext\_lab\_EE\_ctx\_red} \coqdocvar{R}) \coqdocvar{t}
\coqdocvar{t'}) \ensuremath{\rightarrow} ((\coqdocvar{ext\_lab\_EE\_ctx\_red}
\coqdocvar{R}) (\coqdocvar{pterm\_app} \coqdocvar{t} \coqdocvar{u})
(\coqdocvar{pterm\_app} \coqdocvar{t'} \coqdocvar{u})).\coqdoceol
\bigskip

Os lemas acima exemplificam uma das principais simplificações feita: para cada
fecho, definimos um lema para se realizar a redução em um termo maior, a partir
da redução em um subtermo. Estes lemas são úteis para se evitar que tenhamos que
destrinchar as reduções dentro da prova principal, facilitando muito o processo.
Para cada construtor de termo temos um lema análogo a cada um dos três acima
exemplificados, \texttt{star\_lab\_closure\_app\_left},
\texttt{EE\_clos\_app\_left} e \texttt{EE\_ext\_clos\_app\_left}.

Além disso, alguns lemas para lidar com a relação entre as reduções e equações
foram precisos:

\bigskip
\coqdockw{Lemma} \coqdocvar{EE\_presv\_ie}: \coqdockw{\ensuremath{\forall}}
\coqdocvar{t} \coqdocvar{t'} \coqdocvar{u} \coqdocvar{u'}, \coqdocvar{t}
=\coqdocvar{EE} \coqdocvar{u} \ensuremath{\rightarrow} \coqdocvar{u'}
=\coqdocvar{EE} \coqdocvar{t'} \ensuremath{\rightarrow} ((\coqdocvar{u}
-->[\coqdocvar{lx\_i}] \coqdocvar{u'} \ensuremath{\lor} \coqdocvar{u}
-->[\coqdocvar{lx\_e}] \coqdocvar{u'}) \ensuremath{\rightarrow} (\coqdocvar{t}
-->[\coqdocvar{lx\_i}] \coqdocvar{t'} \ensuremath{\lor} \coqdocvar{t}
-->[\coqdocvar{lx\_e}] \coqdocvar{t'})).\coqdoceol

\smallskip

\coqdockw{Lemma} \coqdocvar{EE\_presv\_lab\_lex}:
\coqdockw{\ensuremath{\forall}} \coqdocvar{t} \coqdocvar{t'} \coqdocvar{u}
\coqdocvar{u'}, \coqdocvar{t} =\coqdocvar{EE} \coqdocvar{u}
\ensuremath{\rightarrow} \coqdocvar{u'} =\coqdocvar{EE} \coqdocvar{t'}
\ensuremath{\rightarrow} ((\coqdocvar{u} -->[\coqdocvar{lex}] \coqdocvar{u'})
\ensuremath{\rightarrow} (\coqdocvar{t} -->[\coqdocvar{lex}]
\coqdocvar{t'})).\coqdoceol
\bigskip


Novamente, esses lemas evitam que precisemos adentrar na definição das
equações, reduzindo o tamanho das provas principais. Podemos então partir para
o resultado principal deste trabalho.

\bigskip

\coqdockw{Theorem} \coqdocvar{lab\_ex\_eq\_i\_e}:
\coqdockw{\ensuremath{\forall}} \coqdocvar{t} \coqdocvar{t'},
\coqdocvar{lab\_term} \coqdocvar{t} \ensuremath{\rightarrow} (\coqdocvar{t}
-->[\coqdocvar{lex}] \coqdocvar{t'} \ensuremath{\leftrightarrow} (\coqdocvar{t}
-->[\coqdocvar{lx\_i}] \coqdocvar{t'} \ensuremath{\lor} \coqdocvar{t}
-->[\coqdocvar{lx\_e}] \coqdocvar{t'})).\coqdoceol

\bigskip

Dividimos a prova de que $\lambda \underline{ex} =
\lambda \underline{ex}^i \cup \lambda \underline{ex}^e$, formalizada no teorema
\texttt{lab\_ex\_eq\_i\_e}, em dois lemas, cada um representado uma direção da
equivalência.

\bigskip

\coqdocnoindent \coqdockw{Lemma} \coqdocvar{lab\_ex\_impl\_i\_e}:
\coqdockw{\ensuremath{\forall}} \coqdocvar{t} \coqdocvar{t'},
\coqdocvar{lab\_term} \coqdocvar{t} \ensuremath{\rightarrow} \coqdocvar{t}
-->[\coqdocvar{lex}] \coqdocvar{t'} \ensuremath{\rightarrow} (\coqdocvar{t}
-->[\coqdocvar{lx\_i}] \coqdocvar{t'} \ensuremath{\lor} \coqdocvar{t}
-->[\coqdocvar{lx\_e}] \coqdocvar{t'}).\coqdoceol

\begin{addmargin}[1em]{2em}
\begin{proof}
Escolhemos, para cada redução possível feita pela relação
\texttt{-{}->[lex]}, a redução apropriada entre a interna e a externa.  Para
isto, abrimos a definição da relação \texttt{-{}->[lex]} e fazemos indução no
fecho contextual, ou seja, fazemos indução no predicado
\texttt{lab\_contextual\_closure lab\_sys\_lx t t'}. O caso base é tratado no
lema auxiliar \texttt{lab\_sys\_x\_i\_e}, feito com análise de casos simples nos
construtores da relação \texttt{lab\_sys\_lx}. 

Nos passos indutivos, utilizamos o lema \texttt{EE\_presv\_ie} para adequar o
objetivo à hipótese de indução, substituindo os termos dados pelos termos
equivalentes, obtidos pela definição da redução \texttt{-{}->[lex]}, podendo assim
aplicar a hipótese. No caso em que a redução é feita dentro da substituição
marcada, devemos obrigatoriamente realizar a redução interna. Em todos os outros
casos, realizamos a prova tanto para a redução interna quanto para a externa.
Nos casos em que lidamos com abstrações e substituições, são necessários os
lemas de renomeamento mencionados na subseção \ref{sub:subst_marc}.
\end{proof}

\end{addmargin}

\bigskip

\coqdocnoindent \coqdockw{Lemma} \coqdocvar{lab\_ie\_impl\_ex}:
\coqdockw{\ensuremath{\forall}} \coqdocvar{t} \coqdocvar{t'},
\coqdocvar{lab\_term} \coqdocvar{t} \ensuremath{\rightarrow} (\coqdocvar{t}
-{}->[\coqdocvar{lx\_i}] \coqdocvar{t'} \ensuremath{\lor} \coqdocvar{t}
-{}->[\coqdocvar{lx\_e}] \coqdocvar{t'}) \ensuremath{\rightarrow} \coqdocvar{t}
-{}->[\coqdocvar{lex}] \coqdocvar{t'}.\coqdoceol

\begin{addmargin}[1em]{2em}
\begin{proof}
A prova deste lema é dividida em duas partes: quando a redução realizada é a
interna, e quando é a externa. 

No caso da interna, a indução é feita no fecho do predicado interno, ou seja, é
feita no predicado \texttt{ext\_lab\_contextual\_closure lab\_x\_i t t'}. O caso
base é feito apenas analisando os construtores da relação \texttt{lab\_x\_i}, e
casando com o construtor adequado de lab\_sys\_lx. Nos passos indutivos, fazemos
de maneira análoga ao lema anterior: utilizamos agora o predicado
\texttt{EE\_presv\_lab\_lex} para ajustar o objetivo à hipótese, e utilizamos os
lemas relacionando as reduções aos construtores, como
\texttt{EE\_clos\_app\_left}, para reduzir o objetivo à redução dada como
hipótese. Novamente, no caso de abstrações e substituições, precisamos dos lemas
de renomeamento.

No caso da externa, o processo é exatamente o mesmo. A diferença é que no passo
base, fazemos a análise de casos na redução \texttt{sys\_Bx} e, além disso, não
temos que lidar com o caso da redução dentro de substituição marcada.
\end{proof}

\end{addmargin}

\bigskip

Com ambos os lemas completos, a prova do teorema \texttt{lab\_ex\_eq\_i\_e} se
reduz a apenas aplicá-los. Assim, terminamos a prova de que a redução marcada
formalizada, \texttt{-{}->[lex]} é equivalente à união das reduções interna
e externa, \texttt{-{}->[lx\_i]} e \texttt{-{}->[lx\_e]}.
